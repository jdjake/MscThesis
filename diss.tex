%% The ubcdiss class provides several options:
%%   gpscopy (aka fogscopy)
%%       set parameters to exactly how GPS specifies
%%         * single-sided
%%         * page-numbering starts from title page
%%         * the lists of figures and tables have each entry prefixed
%%           with 'Figure' or 'Table'
%%       This can be tested by `\ifgpscopy ... \else ... \fi'
%%   10pt, 11pt, 12pt
%%       set default font size
%%   oneside, twoside
%%       whether to format for single-sided or double-sided printing
%%   balanced
%%       when double-sided, ensure page content is centred
%%       rather than slightly offset (the default)
%%   singlespacing, onehalfspacing, doublespacing
%%       set default inter-line text spacing; the ubcdiss class
%%       provides \textspacing to revert to this configured spacing
%%   draft
%%       disable more intensive processing, such as including
%%       graphics, etc.

% For submission to GPS
\documentclass[gpscopy,onehalfspacing,12pt]{ubcdiss}

% For your own copies (looks nicer)
%\documentclass[balanced,twoside,12pt]{ubcdiss}

%%
%% FONTS:
%% 
%% The defaults below configures Times Roman for the serif font,
%% Helvetica for the sans serif font, and Courier for the
%% typewriter-style font.

%% NFSS font specification (New Font Selection Scheme)
\usepackage{times,mathptmx,courier}
\usepackage[scaled=.92]{helvet}

\usepackage{amsmath, amsfonts, amssymb, accents, mdwlist}

\usepackage{amsthm}
\theoremstyle{plain}
\newtheorem{lemma}{Lemma}
\newtheorem*{example}{Example}
\newtheorem*{fact}{Fact}
\newtheorem*{corollary}{Corollary}
\newtheorem{theorem}{Theorem}
\newtheorem*{remark}{Remark}

\usepackage{algorithm}
\usepackage[noend]{algpseudocode}

\algblockdefx{MRepeat}{EndRepeat}{\textbf{Repeat}}{}
\algnotext{EndRepeat}

\algblockdefx{MForAll}{EndForAll}{\textbf{For all}}{}
\algnotext{EndForAll}

\usepackage{checkend} % better error messages on left-open environments
\usepackage{graphicx} % for incorporating external images
\usepackage{booktabs} % Provides commands for typesetting tables.
\usepackage{comment} % Comment environment.

\usepackage{listings} % Supports for source doe listings.
\lstset{basicstyle=\sffamily\scriptsize,showstringspaces=false,fontadjust}

\usepackage[printonlyused,nohyperlinks]{acronym} % Acronyms and Glossaries
% Typeset acronyms in small-caps, as recommended by Bringhurst.
\renewcommand{\acsfont}[1]{{\scshape \MakeTextLowercase{#1}}}

%% For an excellent set of examples, see Tufte's "Visual Display of
%% Quantitative Information" or "Envisioning Information".
\usepackage{color, xcolor}
\definecolor{greytext}{gray}{0.5}
\definecolor{crimsonred}{RGB}{132,22,23}
\definecolor{darkblue}{RGB}{72,61,139}

%% Provides citing commands such as \citeauthor{} to provide author names
%% \citet{} to produce author-and-reference citation,
%% \citep{} to produce parenthetical citations.
%% We use \citeeg{} to provide examples
\usepackage[numbers,sort&compress]{natbib}
\newcommand{\citeeg}[1]{\citep[e.g.,][]{#1}}

%% The titlesec package provides commands to vary how chapter and
%% section titles are typeset.  The following uses more compact
%% spacings above and below the title.  The titleformat that follow
%% ensure chapter/section titles are set in singlespace.
\usepackage[compact]{titlesec}
\titleformat*{\section}{\singlespacing\raggedright\bfseries\Large}
\titleformat*{\subsection}{\singlespacing\raggedright\bfseries\large}
\titleformat*{\subsubsection}{\singlespacing\raggedright\bfseries}
\titleformat*{\paragraph}{\singlespacing\raggedright\itshape}

%% The caption package provides support for varying how table and
%% figure captions are typeset.
\usepackage[format=hang,indention=-1cm,labelfont={bf},margin=1em]{caption}

%% url: for typesetting URLs and smart(er) hyphenation.
%% \url{http://...} 
\usepackage{url}
\urlstyle{sf}	% typeset urls in sans-serif

\usepackage{pdfpages}	% insert pages from other PDF files
\usepackage{longtable}	% provide tables spanning multiple pages
\usepackage{chngpage}	% for changing the page widths on demand
\usepackage{tabularx}   % enhanced tabular environment
\usepackage{subfig}     % include subfigures in figures.

\usepackage{enumitem}   % allows pausing and resuming enumerate environments.
\setlist[enumerate]{label*={\normalfont(\Alph*)},ref=(\Alph*)}

%% ragged2e: provides several new new commands \Centering, \RaggedLeft,
%% \RaggedRight and \justifying and new environments Center, FlushLeft,
%% FlushRight and justify, which set ragged text and are easily
%% configurable to allow hyphenation.
%\usepackage{ragged2e}

%% The ulem package provides a \sout{} for striking out text and
%% \xout for crossing out text.  The normalem and normalbf are
%% necessary as the package messes with the emphasis and bold fonts
%% otherwise.
%\usepackage[normalem,normalbf]{ulem}    % for \sout

%%%%%%%%%%%%%%%%%%%%%%%%%%%%%%%%%%%%%%%%%%%%%%%%%%%%%%%%%%%%%%%%%%%%%%
%% HYPERREF:
%% The hyperref package provides for embedding hyperlinks into your
%% document.  By default the table of contents, references, citations,
%% and footnotes are hyperlinked.
%%
%% Hyperref provides a very handy command for doing cross-references:
%% \autoref{}.  This is similar to \ref{} and \pageref{} except that
%% it automagically puts in the *type* of reference.  For example,
%% referencing a figure's label will put the text `Figure 3.4'.
%% And the text will be hyperlinked to the appropriate place in the
%% document.
%%
%% Generally hyperref should appear after most other packages

%% The following puts hyperlinks in very faint grey boxes.
%% The `pagebackref' causes the references in the bibliography to have
%% back-references to the citing page; `backref' puts the citing section
%% number.  See further below for other examples of using hyperref.
%% 2009/12/09: now use `linktocpage' (Jacek Kisynski): GPS now prefers
%%   that the ToC, LoF, LoT place the hyperlink on the page number,
%%   rather than the entry text.
\usepackage[bookmarks,bookmarksnumbered,%
%    allbordercolors={0.8 0.8 0.8},%
    pagebackref,linktocpage%
    ]{hyperref}
\hypersetup{
    colorlinks=true,
    citecolor=black,
    filecolor=black,
    linkcolor=black,
    urlcolor=black,
    linktoc=all
}
%% The following change how the the back-references text is typeset in a
%% bibliography when `backref' or `pagebackref' are used
%%
%% Change \nocitations if you'd like some text shown where there
%% are no citations found (e.g., pulled in with \nocite{xxx})
\newcommand{\nocitations}{\relax}
%%\newcommand{\nocitations}{No citations}
%%
%\renewcommand*{\backref}[1]{}% necessary for backref < 1.33
\renewcommand*{\backrefsep}{,~}%
\renewcommand*{\backreftwosep}{,~}% ', and~'
\renewcommand*{\backreflastsep}{,~}% ' and~'
\renewcommand*{\backrefalt}[4]{%
\textcolor{greytext}{\ifcase #1%
\nocitations%
\or
\(\rightarrow\) page #2%
\else
\(\rightarrow\) pages #2%
\fi}}


%% The following uses most defaults, which causes hyperlinks to be
%% surrounded by colourful boxes; the colours are only visible in
%% PDFs and don't show up when printed:
%\usepackage[bookmarks,bookmarksnumbered]{hyperref}

%% The following disables the colourful boxes around hyperlinks.
%\usepackage[bookmarks,bookmarksnumbered,pdfborder={0 0 0}]{hyperref}

%% The following disables all hyperlinking, but still enabled use of
%% \autoref{}
%\usepackage[draft]{hyperref}

%% The following commands causes chapter and section references to
%% uppercase the part name.
\renewcommand{\chapterautorefname}{Chapter}
\renewcommand{\sectionautorefname}{Section}
\renewcommand{\subsectionautorefname}{Section}
\renewcommand{\subsubsectionautorefname}{Section}

%% If you have long page numbers (e.g., roman numbers in the 
%% preliminary pages for page 28 = xxviii), you might need to
%% uncomment the following and tweak the \@pnumwidth length
%% (default: 1.55em).  See the tocloft documentation at
%% http://www.ctan.org/tex-archive/macros/latex/contrib/tocloft/
% \makeatletter
% \renewcommand{\@pnumwidth}{3em}
% \makeatother

%%%%%%%%%%%%%%%%%%%%%%%%%%%%%%%%%%%%%%%%%%%%%%%%%%%%%%%%%%%%%%%%%%%%%%
%%%%%%%%%%%%%%%%%%%%%%%%%%%%%%%%%%%%%%%%%%%%%%%%%%%%%%%%%%%%%%%%%%%%%%
%%
%% Some special settings that controls how text is typeset
%%
% \raggedbottom		% pages don't have to line up nicely on the last line
% \sloppy		% be a bit more relaxed in inter-word spacing
% \clubpenalty=10000	% try harder to avoid orphans
% \widowpenalty=10000	% try harder to avoid widows
% \tolerance=1000

%% And include some of our own useful macros
% This file provides examples of some useful macros for typesetting
% dissertations.  None of the macros defined here are necessary beyond
% for the template documentation, so feel free to change, remove, and add
% your own definitions.
%
% We recommend that you define macros to separate the semantics
% of the things you write from how they are presented.  For example,
% you'll see definitions below for a macro \file{}: by using
% \file{} consistently in the text, we can change how filenames
% are typeset simply by changing the definition of \file{} in
% this file.
% 
%% The following is a directive for TeXShop to indicate the main file
%%!TEX root = diss.tex

\newcommand{\NA}{\textsc{n/a}}	% for "not applicable"
\newcommand{\eg}{e.g.,\ }	% proper form of examples (\eg a, b, c)
\newcommand{\ie}{i.e.,\ }	% proper form for that is (\ie a, b, c)
\newcommand{\etal}{\emph{et al}}

% Some useful macros for typesetting terms.
\newcommand{\file}[1]{\texttt{#1}}
\newcommand{\class}[1]{\texttt{#1}}
\newcommand{\latexpackage}[1]{\href{http://www.ctan.org/macros/latex/contrib/#1}{\texttt{#1}}}
\newcommand{\latexmiscpackage}[1]{\href{http://www.ctan.org/macros/latex/contrib/misc/#1.sty}{\texttt{#1}}}
\newcommand{\env}[1]{\texttt{#1}}
\newcommand{\BibTeX}{Bib\TeX}

% Define a command \doi{} to typeset a digital object identifier (DOI).
% Note: if the following definition raise an error, then you likely
% have an ancient version of url.sty.  Either find a more recent version
% (3.1 or later work fine) and simply copy it into this directory,  or
% comment out the following two lines and uncomment the third.
\DeclareUrlCommand\DOI{}
\newcommand{\doi}[1]{\href{http://dx.doi.org/#1}{\DOI{doi:#1}}}
%\newcommand{\doi}[1]{\href{http://dx.doi.org/#1}{doi:#1}}

% Useful macro to reference an online document with a hyperlink
% as well with the URL explicitly listed in a footnote
% #1: the URL
% #2: the anchoring text
\newcommand{\webref}[2]{\href{#1}{#2}\footnote{\url{#1}}}

% epigraph is a nice environment for typesetting quotations
\makeatletter
\newenvironment{epigraph}{%
	\begin{flushright}
	\begin{minipage}{\columnwidth-0.75in}
	\begin{flushright}
	\@ifundefined{singlespacing}{}{\singlespacing}%
    }{
	\end{flushright}
	\end{minipage}
	\end{flushright}}
\makeatother

% \FIXME{} is a useful macro for noting things needing to be changed.
% The following definition will also output a warning to the console
\newcommand{\FIXME}[1]{\typeout{**FIXME** #1}\textbf{[FIXME: #1]}}

% END


% My Math Operators

% Dimensions
\DeclareMathOperator{\minkdim}{\dim_{\mathbf{M}}}
\DeclareMathOperator{\hausdim}{\dim_{\mathbf{H}}}
\DeclareMathOperator{\lowminkdim}{\underline{\dim}_{\mathbf{M}}}
\DeclareMathOperator{\upminkdim}{\overline{\dim}_{\mathbf{M}}}

\DeclareMathOperator{\lhdim}{\underline{\dim}_{\mathbf{M}}}
\DeclareMathOperator{\lmbdim}{\underline{\dim}_{\mathbf{MB}}}

% Indicator Functions
\DeclareMathOperator{\ind}{\mathbf{I}}

% Number Systems
\DeclareMathOperator{\RR}{\mathbf{R}}
\DeclareMathOperator{\ZZ}{\mathbf{Z}}
\DeclareMathOperator{\CC}{\mathbf{C}}
\DeclareMathOperator{\QQ}{\mathbf{Q}}

% Probability Operators
\DeclareMathOperator{\EE}{\mathbf{E}}
\DeclareMathOperator{\PP}{\mathbf{P}}

\DeclareMathOperator{\AAA}{\mathbf{A}}
\DeclareMathOperator{\Prob}{\mathbf{P}}
\DeclareMathOperator{\Expect}{\mathbf{E}}

\DeclareMathOperator{\B}{\mathcal{B}}
\DeclareMathOperator{\C}{\mathcal{C}}

\DeclareMathOperator{\Config}{\mathcal{C}}
\DeclareMathOperator{\diam}{\text{diam}}
\DeclareMathOperator{\divides}{\mid}










%%%%%%%%%%%%%%%%%%%%%%%%%%%%%%%%%%%%%%%%%%%%%%%%%%%%%%%%%%%%%%%%%%%%%%
%%%%%%%%%%%%%%%%%%%%%%%%%%%%%%%%%%%%%%%%%%%%%%%%%%%%%%%%%%%%%%%%%%%%%%
%%
%% Document meta-data: be sure to also change the \hypersetup information
%%

\title{Cartesian Products Avoiding Patterns}
%\subtitle{If you want a subtitle}

\author{Jacob Denson}
\previousdegree{BSc. Computing Science, University of Alberta, 2017}

% What is this dissertation for?
\degreetitle{Master of Science}

\institution{The University of British Columbia}
\campus{Vancouver}

\faculty{The Faculty of Science}
\department{Mathematics}
\submissionmonth{April}
\submissionyear{2019}

% details of your examining committee
\examiningcommittee{Malabika Pramanik, Mathematics}{Supervisor}
\examiningcommittee{Joshua Zahl, Mathematics}{Supervisor}

%% hyperref package provides support for embedding meta-data in .PDF
%% files
\hypersetup{
  pdftitle={Change this title!  (DRAFT: \today)},
  pdfauthor={Johnny Canuck},
  pdfkeywords={Your keywords here}
}

%% LaTeX's \includeonly commands causes any uses of \include{} to only
%% include files that are in the list.  This is helpful to produce
%% subsets of your thesis (e.g., for committee members who want to see
%% the dissertation chapter by chapter).  It also saves time by 
%% avoiding reprocessing the entire file.
%\includeonly{intro,conclusions}
%\includeonly{discussion}










\begin{document}

% Thesis Guidelines available at:
% 	http://www.grad.ubc.ca/current-students/dissertation-thesis-preparation/order-components

% 1. Title page (mandatory)
\maketitle

% 2. Committee page (mandatory): lists supervisory committee and if applicable, the examining committee
\makecommitteepage

% 3. Abstract (mandatory - maximum 350 words)
%% The following is a directive for TeXShop to indicate the main file
%%!TEX root = diss.tex

\chapter{Abstract}

% MAXIMUM 350 WORDS!

The pattern avoidance problem seeks to construct a set $X \subset \RR^d$ with large fractal dimension that avoids a prescribed pattern, such as three term arithmetic progressions, or more general patterns such as avoiding points $x_1, \dots, x_n \in \RR^d$ such that $f(x_1, \dots, x_n) = 0$ for a given function $f$. Previous work on the subject has considered patterns described by polynomials, or functions $f$ satisfying certain regularity conditions. We provide an exposition of some results in this setting, as well as considering new strategies to avoid `rough patterns. There are several problems that fit into the framework of rough pattern avoidance. As a first application, if $Y\subset[0,1]$ is a set with Minkowski dimension $s$, we construct a set $X\subset[0,1]$ with Hausdorff dimension $1-s$ so that $X+X$ is disjoint from $Y$. As a second application, given a set $Y$ with dimension close to one, we can construct a set $X\subset Y$ of dimension $1/2$ that avoids isosceles triangles.
\cleardoublepage

% 4. Lay Summary (Effective May 2017, mandatory - maximum 150 words)
%% The following is a directive for TeXShop to indicate the main file
%%!TEX root = diss.tex

%% https://www.grad.ubc.ca/current-students/dissertation-thesis-preparation/preliminary-pages
%% 
%% LAY SUMMARY Effective May 2017, all theses and dissertations must
%% include a lay summary.  The lay or public summary explains the key
%% goals and contributions of the research/scholarly work in terms that
%% can be understood by the general public. It must not exceed 150
%% words in length.

%% The lay or public summary explains the key goals and contributions of
%% the research\slash{}scholarly work in terms that can be understood by the
%% general public. It must not exceed 150 words in length.

\chapter{Lay Summary}

%Imagine a patch of carpet under a microscope. Zooming in, we see the carpet is really a collection of twines tied together. On closer inspection, those twines break off into smaller tufts of fabric. The carpet is rough at all scales. Shapes like circles or polygons do not have complexity at arbitrarily small scales. To model the roughness of a carpet, you'd need a fractal: a shape with complex structure at all small scales. Such models are often useful in small-scale physics or computer graphics.

%It is easy to construct polyhedra with geometric properties at a single scale. But it is non-trivial to construct fractals with properties at many scales. For instance, how do we construct a fractal which intersects any line in at most two points? In this thesis, we begin with an exposition on previous constructions in the literature, and then provide new construction techniques utilizing randomness.

In geometry, we are often interested in whether shapes with certain properties exist. For instance, given three points, can one find a circle connecting these three points? Most questions of this type involving classical shapes have been answered. But in the 20th century, mathematicians began discussing a new class of geometric shapes known as fractals: shapes with complex structure at all scales. Popular examples include the Koch snowflake, or Sierpinski triangle. Such shapes often occur in small scale physics and computer graphics.

Many open questions remain as to whether one can construct fractals with various properties. For instance, can one construct a large fractal so that one cannot form an isosceles triangle from three points lying on the fractal. This thesis focuses on fractal construction problems. We begin with an exposition on previous constructions in the literature, and then provide new construction techniques utilizing randomness.
\cleardoublepage

% 5. Preface
%% The following is a directive for TeXShop to indicate the main file
%%!TEX root = diss.tex

\chapter{Preface}

This thesis gives an exposition by the author, of the pattern avoidance problem and the geometric measure theory required to understand the pattern avoidance problem in the non-discrete setting. In Chapter \ref{ch:RoughSets}, the author presents details of joint work with his supervisors Dr. Joshua Zahl and Dr. Malabika Pramanik. The results of this section have been submitted for publication.
\cleardoublepage

% 6. Table of contents (mandatory - list all items in the preliminary pages
% starting with the abstract, followed by chapter headings and
% subheadings, bibliographies and appendices)
\tableofcontents
\cleardoublepage	% required by tocloft package

% 7. List of tables (mandatory if thesis has tables)
%\listoftables
%\cleardoublepage	% required by tocloft package

% 8. List of figures (mandatory if thesis has figures)
\listoffigures
\cleardoublepage	% required by tocloft package

% 9. List of illustrations (mandatory if thesis has illustrations)
% 10. Lists of symbols, abbreviations or other (optional)

% 11. Glossary (optional)
%% The following is a directive for TeXShop to indicate the main file
%%!TEX root = diss.tex

\chapter{Glossary}

This glossary uses the handy \latexpackage{acroynym} package to automatically
maintain the glossary.  It uses the package's \texttt{printonlyused}
option to include only those acronyms explicitly referenced in the
\LaTeX\ source.

% use \acrodef to define an acronym, but no listing
\acrodef{UI}{user interface}
\acrodef{UBC}{University of British Columbia}

% The acronym environment will typeset only those acronyms that were
% *actually used* in the course of the document
\begin{acronym}[ANOVA]
\acro{ANOVA}[ANOVA]{Analysis of Variance\acroextra{, a set of
  statistical techniques to identify sources of variability between groups}}
\acro{API}{application programming interface}
\acro{CTAN}{\acroextra{The }Common \TeX\ Archive Network}
\acro{DOI}{Document Object Identifier\acroextra{ (see
    \url{http://doi.org})}}
\acro{GPS}[GPS]{Graduate and Postdoctoral Studies}
\acro{PDF}{Portable Document Format}
\acro{RCS}[RCS]{Revision control system\acroextra{, a software
    tool for tracking changes to a set of files}}
\acro{TLX}[TLX]{Task Load Index\acroextra{, an instrument for gauging
  the subjective mental workload experienced by a human in performing
  a task}}
\acro{UML}{Unified Modelling Language\acroextra{, a visual language
    for modelling the structure of software artefacts}}
\acro{URL}{Unique Resource Locator\acroextra{, used to describe a
    means for obtaining some resource on the world wide web}}
\acro{W3C}[W3C]{\acroextra{the }World Wide Web Consortium\acroextra{,
    the standards body for web technologies}}
\acro{XML}{Extensible Markup Language}
\end{acronym}

% You can also use \newacro{}{} to only define acronyms
% but without explictly creating a glossary
% 
% \newacro{ANOVA}[ANOVA]{Analysis of Variance\acroextra{, a set of
%   statistical techniques to identify sources of variability between groups.}}
% \newacro{API}[API]{application programming interface}
% \newacro{GOMS}[GOMS]{Goals, Operators, Methods, and Selection\acroextra{,
%   a framework for usability analysis.}}
% \newacro{TLX}[TLX]{Task Load Index\acroextra{, an instrument for gauging
%   the subjective mental workload experienced by a human in performing
%   a task.}}
% \newacro{UI}[UI]{user interface}
% \newacro{UML}[UML]{Unified Modelling Language}
% \newacro{W3C}[W3C]{World Wide Web Consortium}
% \newacro{XML}[XML]{Extensible Markup Language}
	% always input, since other macros may rely on it

\textspacing		% begin one-half or double spacing

% 12. Acknowledgements (optional)
%% The following is a directive for TeXShop to indicate the main file
%%!TEX root = diss.tex

\chapter{Acknowledgments}

Thank those people who helped you. 

Don't forget your parents or loved ones.

You may wish to acknowledge your funding sources.


% 13. Dedication (optional)

% Body of Thesis (not all sections may apply)
\mainmatter

\acresetall	% reset all acronyms used so far

% 1. Introduction
\include{intro}

% 2. Main body
%% The following is a directive for TeXShop to indicate the main file
%%!TEX root = diss.tex

\chapter{Background}
\label{ch:Background}

\section{Configuration Avoidance}

We consider an ambient set $\AAA$. It's \emph{$n$-point configuration space} is
%
\[ \Config^n(\AAA) = \{ (x_1, \dots, x_n) \in X^n: x_i \neq x_j\ \text{if $i \neq j$} \}. \]
%
An \emph{$n$ point configuration}, or \emph{pattern}, is a subset of $\Config^n(\AAA)$. More generally, we define the general \emph{configuration space} of $\AAA$ as $\Config(\AAA) = \bigcup_{n = 1}^\infty \Config^n(\AAA)$, and a \emph{pattern}, or \emph{configuration}, on $\AAA$ is a subset of $\Config(\AAA)$.

Our main focus in this thesis is the \emph{pattern avoidance problem}. For a fixed configuration $\C$ on $\AAA$, we say a set $X \subset \AAA$ \emph{avoids} $\C$ if $\Config(X)$ is disjoint from $\C$. The pattern avoidance problem asks to find sets $X$ of maximal size avoiding a fixed configuration $\C$. The set $\C$ often describes the presence of algebraic or geometric structure, and so we are trying to find large sets which do not possess any instance of such structure.

\begin{example}[Isoceles Triangle Configuration]
	Let
	%
	\[ \C = \left\{ (x_1, x_2, x_3) \in \Config^3(\RR^2) : |x_1-x_2| = |x_1-x_3| \right\}. \]
	%
	Then $\C$ is a 3-point configuration, and a set $X \subset \RR^2$ avoids $\C$ if and only if it does not contain all three vertices of an isoceles triangle. %Notice that $|x_1 - x_2| = |x_1 - x_3|$ holds if and only if $|x_1 - x_2|^2 = |x_1 - x_3|^2$, which is an algebraic equation in the coordinates of $x_1,x_2$, and $x_3$. Thus $\C$ is an algebraic hypersurface of degree two in $\RR^6$.
\end{example}

\begin{example}[Linear Independence Configuration]
	Let $V$ be a vector space over a field $K$. We set
	%
	\[ \C = \bigcup_{n = 1}^\infty \{ (x_1, \dots, x_n) \in \C^n(V): \text{for any}\ a_1, \dots, a_n \in K,\ a_1x_1 + \dots + a_nx_n \neq 0 \}. \]
	%
	A subset $X \subset V$ avoids $\C$ if and only if $X$ is a linearly independent subset of $X$. Note that if $V$ is infinite dimensional, then $\C$ cannot be replaced by a $n$ point configuration for any $n$; arbitrarily large tuples must be considered. We will be interested in the case where $K = \QQ$, and $V = \RR$, where we will be looking for analytically large, linearly independant subsets of $V$.
\end{example}

%\begin{example}[General Position Configuration]
%	Suppose we wish to find a subset $X$ of $\RR^d$ such that for each positive integer $k \leq d$, and for each collection of $k+1$ distinct points $x_1, \dots, x_{k+1} \in X$, the points do not lie in a $k-1$ dimensional hyperplane. For each $k \leq d$, set
	%
%	\[ \C^{k+1} = \{ (x_0, x_1, \dots, x_k) \in \Config^{k+1}(\RR^d): x_1-x_0, \dots, x_k - x_0\ \text{are linearly dependant} \}. \]
	%
%	If we define $\C = \bigcup_{k = 2}^d \C^k$, then a set $X$ avoids $\C$ precisely when all finite collection of distinct points in $X$ lie in general position. Notice that
	%
%	\[ \C^{k+1} = \bigcup \left\{ \text{span}(y_1, \dots, y_k) \times \{ y \} : y = (y_1, \dots, y_k) \in \Config^k(\RR^d) \right\} \cap \Config^{k+1}(\RR^d). \]
	%
%	so each $\C^{k+1}$ is essentially a union of $k$ dimensional hyperplanes.
%\end{example}

Even though our problem formulation assumes configurations are formed by distinct sets of points, one can still formulate avoidance problems involving repeated points by a simple trick.

\begin{example}[Sum Set Configuration]
	Let $G$ be an abelian group, and fix $Y \subset G$. Set
	%
	\[ \C^1 = \{ g \in \Config^1(G): g + g \in Y \} \quad \text{and} \quad \C^2 = \{ (g_1,g_2) \in \Config^2(G): g_1 + g_2 \in Y \}. \]
	%
	Then set $\C = \C^1 \cup \C^2$. A set $X \subset G$ avoids $\C$ if and only if $(X + X) \cap Y = \emptyset$.
\end{example}

Depending on the structure of the ambient space $\AAA$ and the configuration $\C$, there are various ways of measuring the size of sets $X \subset \AAA$ for the purpose of the pattern avoidance problem:
%
\begin{itemize}
	\item If $\AAA$ is finite, the goal is to find a set $X$ with large cardinality.
	\item If $\{ \AAA_n \}$ is an increasing family of finite sets with $\AAA = \lim \AAA_n$, the goal is to find a set $X$ such that $X \cap \AAA_n$ has large cardinality asymptotically in $n$.
	\item If $\AAA = \RR^d$, but $\C$ is a discrete configuration, then a satisfactory goal is to find a set $X$ with large Lebesgue measure avoiding $\C$.
\end{itemize}
%
In this thesis, inspired by results in these three settings, we establish methods for avoiding non-discrete configurations $\C$ in $\RR^d$. Here, Lebesgue measure completely fails to measure the size of pattern avoiding solutions, as the next theorem shows, under the often true assumption that $\C$ is \emph{translation invariant}, i.e. that if $(a_1, \dots, a_n) \in \C$ and $b \in \RR^d$, $(a_1 + b, \dots, a_n + b) \in \C$.

\begin{theorem}
	Let $\C$ be a $n$-point configuration on $\RR^d$. Suppose
	%
	\begin{enumerate}
		\item \label{translationinvariance} $\C$ is translation invariant.
		\item \label{nonDiscreteConfig} For any $\varepsilon > 0$, there is $(a_1, \dots, a_n) \in \C$ with $\diam \{ a_1, \dots, a_n \} \leq \varepsilon$.
	\end{enumerate}
	%
	Then no set with positive Lebesgue measure avoids $\C$.
\end{theorem}
\begin{proof}
	Let $X \subset \RR^d$ have positive Lebesgue measure. The Lebesgue density theorem shows that there exists a point $x \in X$ such that
	%
	\begin{equation} \label{densityApplication} \lim_{l(Q) \to 0} \frac{|X \cap Q|}{|Q|} = 1, \end{equation}
	%
	where $Q$ ranges over all cubes in $\RR^d$ with $x \in Q$, and $l(Q)$ denotes the sidelength of $Q$. Fix $\varepsilon > 0$, to be specified later, and choose $r$ small enough that $|X \cap Q| \geq (1 - \varepsilon) |Q|$ for any cube $Q$ with $x \in Q$ and $l(Q) \leq r$. Now let $Q_0$ denote the cube centered at $x$ with $l(Q_0) \leq r$. Applying Property \ref{nonDiscreteConfig}, we find $C = (a_1, \dots, a_n) \in \C$ such that
	%
	\begin{equation} \label{equation690346024} \diam \{ a_1, \dots, a_n \} \leq l(Q_0)/2. \end{equation}
	%
	For each $p \in Q_0$, let $C(p) = (a_1(p), \dots, a_n(p))$, where $a_i(p) = p + (a_i - a_1)$. Property \ref{translationinvariance} implies $C(p) \in \C$ for each $p \in \RR^d$. A union bound shows
	%
	\begin{equation} \label{equation548} \left| \{ p \in Q_0 : C(p) \not \in \C(X) \} \right| \leq \sum_{i = 1}^d \left| \{ p \in Q_0 : a_i(p) \not \in X \} \right|.
	\end{equation}
	%
	We have $a_i(p) \not \in X$ precisely when $p + (a_i - a_1) \not \in X$, so
	%
	\begin{equation} \label{equation1243462}
		|\{ p \in Q_0 : a_i(p) \not \in X \}| = |(Q_0 + (a_i - a_1)) \cap X^c|.
	\end{equation}
	%
	Note $Q_0 + (a_i - a_1)$ is a cube with the same sidelength as $Q_0$. Equation \eqref{equation690346024} implies $|a_i - a_1| \leq l(Q_0)/2$, so $x \in Q_0 + (a_i - a_1)$. Thus \eqref{densityApplication} shows
	%
	\begin{equation} \label{equation543} |Q_0 + (a_i - a_1)) \cap X^c| \leq \varepsilon |Q_0|. \end{equation}
	%
	Combining \eqref{equation548}, \eqref{equation1243462}, and \eqref{equation543}, we find
	%
	\[ \left| \{ p \in Q_0 : C(p) \not \in \C(X) \} \right| \leq \varepsilon d |Q_0|. \]
	%
	Provided $\varepsilon d < 1$, this means there is $p \in Q_0$ with $C(p) \in \C(X)$.
\end{proof}

Since no set of positive Lebesgue measure can avoid non-discrete configurations, we cannot use the Lebesgue measure to quantify the size of pattern avoiding sets. Fortunately, there is a quantity which can distinguish between the size of sets of measure zero. This is the \emph{fractional dimension} of a set.

There are many variants of fractional dimension. Here we choose to use the Minkowski dimension, the Hausdorff dimension, and the Fourier dimension. They assign the same dimension to any smooth manifold\footnote{For Fourier dimension, the smooth manifold also needs to have nonvanishing curvature.}, but vary over more singular sets. One major difference is that Minkowski dimension measures relative density at a single scale, whereas Hausdorff dimension measures relative density at countably many scales. The Fourier dimension is a refinement of the Hausdorff dimension which gives greater control on the set in the frequency domain, and therefore gives additional structural information on sets.









\section{Minkowski Dimension}

We begin by discussing the Minkowski dimension, which is the simplest to define. Given $l > 0$, and a bounded set $E \subset \RR^d$, we let $N(l,E)$ denote the \emph{covering number} of $E$, the minimum number of sidelength $l$ cubes required to cover $E$. We define the \emph{lower} and \emph{upper} Minkowski dimension as
%
\[ \lowminkdim(E) = \liminf_{l \to 0} \frac{\log(N(l,E))}{\log(1/l)} \quad\text{and}\quad \upminkdim(E) = \limsup_{l \to 0} \frac{\log(N(l,E))}{\log(1/l)}. \]
%
If $\upminkdim(E) = \lowminkdim(E)$, then we refer to this common quantity as the \emph{Minkowski dimension} of $E$, denoted $\minkdim(E)$. Thus $\lowminkdim(E) < s$ if there exists a sequence of lengths $\{ l_k \}$ converging to zero with $N(l_k,E) \leq (1/l_k)^s$, and $\upminkdim(E) < s$ if $N(l,E) \leq (1/l)^s$ for \emph{all} sufficiently small lengths $l$.

\section{Hausdorff Dimension}

For $E \subset \RR^d$ and $\delta > 0$, we define the \emph{Hausdorff content}
%
\[ H_\delta^s(E) = \inf \left\{ \sum_{k = 1}^\infty l(Q_k)^s : E \subset \bigcup_{k = 1}^\infty Q_k, l(Q_k) \leq \delta \right\}. \]
%
The \emph{$s$-dimensional Hausdorff measure} of $E$ is
%
\[ H^s(E) = \lim_{\delta \to 0} H_\delta^s(E). \]
%
It is easy to see $H^s$ is an exterior measure on $\RR^d$, and $H^s(E \cup F) = H^s(E) + H^s(F)$ if the Hausdorff distance $d(E,F)$ between $E$ and $F$ is positive. So $H^s$ is actually a metric exterior measure, and the Caratheodory extension theorem shows all Borel sets are measurable with respect to $H^s$.

\begin{lemma} \label{HausdorffBoundary}
	Consider $t < s$, and $E \subset \RR^d$.
	%
	\begin{enumerate}
		\item[(i)] If $H^t(E) < \infty$, then $H^s(E) = 0$.
		\item[(ii)] If $H^s(E) \neq 0$, then $H^t(E) = \infty$.
	\end{enumerate}
\end{lemma}
\begin{proof}
	Suppose that $H^t(E) = A < \infty$. Then for any $\delta > 0$, there is a cover of $E$ by a collection of intervals $\{ Q_k \}$, such that $l(Q_k) \leq \delta$ for each $k$, and
	%
	\[ \sum l(Q_k)^t \leq A < \infty. \]
	%
	But then
	%
	\[ H^s_\delta(E) \leq \sum l(Q_k)^s \leq \sum l(Q_k)^{s-t} l(Q_k)^t \leq \delta^{s-t} A. \]
	%
	As $\delta \to 0$, we conclude $H^s(E) = 0$, proving \emph{(i)}. And \emph{(ii)} is just the contrapositive of (i), and therefore immediately follows.
\end{proof}

\begin{corollary} \label{corollaryhausdorffzero}
	If $s > d$, $H^s = 0$.
\end{corollary}
\begin{proof}
	The measure $H^d$ is just the Lebesgue measure on $\RR^d$, so
	%
	\[ H^d[-N,N]^d = (2N)^d. \]
	%
	If $s > d$, Lemma \ref{HausdorffBoundary} shows $H^s[-N,N]^d = 0$. By countable additivity, taking $N \to \infty$ shows $H^s(\RR^d) = 0$. Thus $H^s(E) = 0$ for all $E$ if $s > d$.
\end{proof}

Given any Borel set $E$, Corollary \ref{corollaryhausdorffzero}, combined with Lemma \ref{HausdorffBoundary}, implies there is a unique value $s_0 \in [0,d]$ such that $H^s(E) = 0$ for $s > s_0$, and $H^s(E) = \infty$ for $0 \leq s < s_0$. We refer to $s_0$ as the \emph{Hausdorff dimension} of $E$, denoted $\hausdim(E)$.

\begin{theorem}
	For any bounded set $E$, $\hausdim(E) \leq \lowminkdim(E) \leq \upminkdim(E)$.
\end{theorem}
\begin{proof}
	Given $l > 0$, we have a simple bound $H^s_l(E) \leq N(l,E) \cdot l^s$. If $\lowminkdim(E) < s$, then there exists a sequence $\{ l_k \}$ with $l_k \to 0$, and $N(l_k,E) \leq (1/l_k)^s$. Thus we conclude that
	%
	\[ H^s(E) = \lim_{k \to \infty} H^s_{l_k}(E) \leq \lim_{k \to \infty} N(l_k,E) \cdot l_k^s \leq 1, \]
	%
	This $\hausdim(E) \leq s$. Taking infima over all $s$, we find $\hausdim(E) \leq \lowminkdim(E)$.
\end{proof}

\begin{remark}
	If $\hausdim(E) < d$, then $|E| = H^d(E) = 0$. Thus any set with fractional dimension less than $d$ must have measure zero, and so the dimension is a way of distinguishing between sets of measure zero, which is precisely what we need to study the configuration avoidance problem for non-discrete configurations.
\end{remark}

The fact that Hausdorff dimension is defined with respect to multiple scales makes it more stable under analytical operations. In particular,
%
\[ \hausdim \left\{ \bigcup E_k \right\} = \sup \left\{ \hausdim(E_k). \right\} \]
%
This need not be true for the Minkowski dimension; a single point has Minkowski dimension zero, but the rational numbers, which are a countable union of points, have Minkowski dimension one. An easy way to make Minkowski dimension countably stable is to define the \emph{modified Minkowski dimensions}
%
\begin{align*}
	\lmbdim(E) &= \inf \left\{ s : E \subset \bigcup_{i = 1}^\infty E_i, \lowminkdim(E_i) \leq s \right\}\\
	&\text{and}\\
	\umbdim(E) &= \inf \left\{ s : E \subset \bigcup_{i = 1}^\infty E_i, \upminkdim(E_i) \leq s \right\}.
\end{align*}
%
This notion of dimension, in a disguised form, appears in Chapter BLAH.







\section{Dyadic Scales}

It is now useful to introduce the dyadic notation we utilize throughout this thesis. At the cost of losing topological perspective about $\RR^d$, applying dyadic techniques often allows us to elegantly discretize certain problems in Euclidean space. We introduce fractional dimension through a dyadic framework, and all our constructions will be done dyadically. This section introduces some notation we will use throughout the remainder of the thesis. We fix a sequence of positive integers $\{ N_k : k \geq 1 \}$, which will change over each argument in the thesis, but will remain constant throughout each argument.
%
\begin{itemize}
	\item For each $k \geq 0$, we define 
	%
	\[ \DQ_k^d = \left\{ \prod_{k = 1}^d \left[ \frac{m_k}{N_1 \dots N_k}, \frac{m_k + 1}{N_1 \dots N_k} \right] : m \in \ZZ^d \right\}. \]
	%
	These are the \emph{dyadic cubes of generation $k$}. We let $\DQ^d = \bigcup_{k \geq 0} \DQ_k^d$. Note that any two cubes $\DQ^d$ are either nested within one another, or their interiors do not intersect.

	\item We set
	%
	\[ l_k = \frac{1}{N_1 \dots N_k}. \]
	%
	Then $l_k$ is the sidelength of the cubes in $\DQ_k^d$.

	\item Given $Q \in \DQ_{k+1}^d$, we let $Q^* \in \DQ_k^d$ denote the \emph{parent cube} of $Q$, i.e. the unique dyadic cube of generation $k$ such that $Q \subset Q^*$.

	\item We say a set $E \subset \RR^d$ is \emph{$\DQ_k^d$ discretized} if it is a union of cubes in $\DQ_k^d$. In this case, we let
	%
	\[ \DQ_k(E) = \{ Q \in \DQ_k^d: Q \subset E \} \]
	%
	denote the family of cubes whose union is $E$.
\end{itemize}
%
It is most common to set $N_k = 2$ for all $k$, which gives the standard families of dyadic cubes. But in our methods, it is necessary to allow more general sequences $\{ N_k \}$, which are not necessarily bounded.

%The most common class of dyadic cubes in analysis is obtained by setting $N_k = 2$ for each $k$. We reserve a special notation for this class of dyadic cubes; the class of all such cubes is denoted by $\DD^d$, and the generation $k$ cubes by $\DD^d_k$. The fact that the sequence $\{ N_k \}$ is constant makes these cubes more easy to analyze. But in our methods, it is necessary for the sequence $\{ N_k \}$ to become unbounded as $k \to \infty$. This is why we have to introduce the more general family of dyadic cubes introduced above.

Sometimes, we need to rely on certain `intermediary' cubes that lie between the scales $\DQ_k^d$ and $\DQ_{k+1}^d$. In this case, we consider a supplementary sequence $\{ M_k : k \geq 1 \}$ with $M_k \divides N_k$ for each $k$.
%
\begin{itemize}
	\item For $k \geq 1$, we define
	%
	\[ \DR_k^d = \left\{ \prod_{k = 1}^d \left[ \frac{m_k}{N_1 \dots N_{k-1} M_k} , \frac{m_k + 1}{N_1 \dots N_{k-1} M_k} \right] : m \in \mathbf{Z}^d \right\}. \]

	\item We set
	%
	\[ r_k = \frac{1}{N_1 \dots N_{k-1} M_k} \]
	%
	to be the sidelength of a cube in $\DR_k^d$.

	\item For $E \subset \RR^d$, the notions of being $\DR_k^d$ discretized, and the collection of cubes $\DR_k^d(E)$, are defined as should be expected.
\end{itemize}
%
Thus the cubes in $\DR_k^d$ are coarser than those in $\DQ_k^d$, but finer than those in $\DQ_{k-1}^d$.

%For the purposes of discretization, and to simplify notation, it is very useful to identify those cubes in $\DQ_k^d$ which are subsets of $[0,1]^d$ with the set
%
%\[ \Sigma_k^d = [N_1]^d \times \dots \times [N_k]^d. \]
%
%Given $j \in \Sigma_m^d$, we let $Q_j \in \DQ_m^d$ denote the cube with left-hand corner $a = \sum_{k = 1}^m j_kl_k$. Thus subcubes of a cube corresponding to an index $j \in \Sigma_k^d$ correspond to indices obtained by appending additional integers onto $j$. In the case of intermediary scales, we abuse notation, writing $[N_k] = [K_k] \times [M_k]$, we have
%
%\[ \Sigma_k^d = \big([K_1] \times [M_1] \big)^d \times \dots \times \big([K_k] \times [M_k] \big)^d, \]
%
%where $n \in [N_k]$ is equal to $(k,m) \in [K_k] \times [M_k]$, where $k M_k + m = n$.

%One very useful property of the cubes in $\DQ^d$ is that cubes are either nested within one another, or \emph{almost disjoint} from one another, in the sense that only their boundaries intersect. Thus we can think of $\DQ^d$ as a forest under the partial ordering of inclusion, with the roots corresponding to the elements of $\DQ_0^d$. Each cube $Q \in \DQ^d_k$ has $N_{k+1}^d$ children. For each cube $Q \in \DQ_{k+1}^d$, we will let $Q^* \in \DQ_k^d$ denote it's parent, i.e. the unique cube in $\DQ_k^d$ with $Q \subset Q^*$. Similarily, given an index $I = (I_0, \dots, I_{k+1}) \in \Sigma_{k+1}$, we let $I^* = (I_0, \dots, I_k) \in \Sigma_k$, so that $Q_I \subset Q_{I^*}$.

%We often construct configuration avoiding sets as limits of dyadic discretizations. It is most convenient to describe this construction in terms of a sequence of sets $\{ S_k \}$ with $S_k \subset \Sigma_k^d$ for each $k$, and $(S_{k+1})^* \subset S_k$ for each $k$. We refer to such a sequence as a \emph{constructing sequence}. We can define $E_k = \bigcup \{ Q_j : j \in S_k \}$. Then $E_{k+1} \subset E_k$ for each $k$, and we can form a set $\bigcap E_k$, which is the final, non discretized limit of the sequence of discretizations.

%We can also give the spaces $\Sigma_k^d$ and $\Sigma^d$ a metric space structure, by defining, for $I \neq J$, $d(I,J) = l_k$, where $k$ is the smallest index such that $I_k \neq J_k$. Aside from the fact that some points in $\RR^d$ are duplicated in $\Sigma^d$, the main difference between the two spaces is that the balls in $\Sigma^d$ are discretized to the scales $\{ l_k \}$. From the point of view of geometric measure theory, the first point is neglible, but we shall find that discretization does matter. The degree to which the geometry of $\Sigma^d$ models the geometry of $\RR^d$ from the perspective of geometric measure theory is a key topic in this thesis. We shall find that the rate at which the lengths $l_k$ tend to zero will be a key factor in this relationship.

\section{Frostman Measures}

%\begin{example}
%	Let $s = 0$. Then $H_\delta^0(E)$ is the number of $\delta$ balls it takes to cover $E$, which tends to $\infty$ as $\delta \to 0$ unless $E$ is finite, and in the finite case, $H_\delta^0(E) \to \# E$. Thus $H^0$ is just the counting measure.
%\end{example}

%\begin{example}
%	Let $s = d$. If $E$ has Lebesgue measure zero, then for any $\varepsilon > 0$, there exists a sequence of balls $\{ B(x_k,r_k) \}$ covering $E$ with
	%
%	\[ \sum_{k = 1}^\infty r_k^d < \varepsilon^d. \]
	%
%	Then we know $r_k < \varepsilon$, so $H^s_\varepsilon(E) < \varepsilon^d$. Letting $\varepsilon \to 0$, we conclude $H^d(E) = 0$. Thus $H^d$ is absolutely continuous with respect to the Lebesgue measure. The measure $H^d$ is translation invariant, so $H^d$ is actually a constant multiple of the Lebesgue measure.
%\end{example}

It is often easy to upper bound Hausdorff dimension, but non-trivial to \emph{lower bound} the Hausdorff dimension of a given set. A key technique to finding a lower bound is \emph{Frostman's lemma}, which says that a set has large Hausdorff dimension if and only if it supports a probability measure which obeys a certain decay law on small sets. We say a Borel measure $\mu$ is a \emph{Frostman measure} of dimension $s$ if it is non-zero, compactly supported, and for any cube $Q$, $\mu(Q) \lesssim l(Q)^s$. The proof of Frostman's lemma will utilize a technique often useful, known as the \emph{mass distribution principle}.

\begin{lemma}[Mass Distribution Principle] \label{massdistributionprinciplelem}
	Let $\mu: \DQ^d \to [0,\infty)$ be a function such that for any $k$, and for any $Q_0 \in \DQ^d_k$,
	%
	\begin{equation} \label{equation73234091} \sum \left\{ \mu(Q) : Q \in \DQ^d_{k+1}, Q^* = Q_0 \right\} = \mu(Q_0). \end{equation}
	%
	Then $\mu$ extends uniquely to a regular Borel measure on $\RR^d$.
\end{lemma}
\begin{proof}
	We begin by defining a sequence of regular Borel measures $\{ \mu_k \}$ by setting, for each $f \in C_c(\RR^d)$,
	%
	\[ \int f d\mu_k = \sum_Q \mu(Q) \int_Q f, \]
	%
	where $Q$ ranges over all cubes in $\DQ_k^d$. Similarily, define a family of operators $\{ E_k \}$ on regular Borel measures by the formula
	%
	\[ \int f(x) dE_k(\nu) = \sum_Q \nu(Q) \int_Q f, \]
	%
	where $Q$ ranges over all cubes in $\DQ_k^d$. Equation \eqref{equation73234091} then says that $E_j(\mu_k) = \mu_j$ for each $j \leq k$.

	If $\nu_i \to \nu$ weakly, then $\nu_i(Q) \to \nu(Q)$ for each fixed $Q \in \DQ_k^d$. Thus if $f \in C_c(\RR^d)$, then the support of $f$ intersects only finitely many cubes in $\DQ_k^d$, and this implies
	%
	\[ \int f(x) dE_k(\nu_i) = \sum \nu_i(Q) \int_Q f\; dx \to \sum \nu(Q) \int_Q f\; dx = \int f dE_k(\nu). \]
	%
	Thus the operators $\{ E_k \}$ are continuous with respect to the weak topology.

	Now fix an interval $Q \in \DQ_0^d$. The measures $\{ \mu_k|_Q \}$, restricted to $Q$, all have total mass $\mu(Q)$. Thus the Banach-Alaoglu theorem implies that is a subsequence $\mu_{k_i}|_Q$ converging weakly on $Q$ to some measure $\mu_Q$. By continuity,
	%
	\[ E_j(\mu_Q) = \lim_{i \to \infty} E_j(\mu_{k_i}|_Q) = \lim_{i \to \infty} E_j(\mu_{k_i})|_Q = \mu_j|_Q. \]
	%
	This means precisely that $\mu_Q$ is the extension of $\mu$ to a Borel measure on $Q$. If we patch together the measures $\mu_Q$ over all choices of $Q \in \DQ_0^d$, i.e. setting
	%
	\[ \mu(E) = \sum_{n \in \mathbf{Z}^d} \mu_Q \left(E \cap [n_1, n_1 + 1) \times \dots \times [n_d, n_d + 1) \right), \]
	%
	then this extends the function $\mu$ to a regular Borel measure. The uniqueness of the extension is guaranteed, because the intervals in $\DQ^d$ generate the entire Borel sigma algebra.
\end{proof}

%\begin{lemma}
%	let $\mu^+$ be a function from $\B$ to $[0,\infty)$ such that for any $I \in \B(1/M^k,\RR^d)$,
	%
%	\[ \sum \left\{ \mu^+(J) :J \in \B(1/M^{k+1},I) \right\} \leq \mu^+(I) \]
	%
%	Assume there exists $c > 0$ such that for all $k$,
	%
%	\[ \sum \left\{ \mu^+(I) : I \in \B(1/M^k,I) \right\} \geq c \]
	%
%	and
	%
%	\[ \sum \left\{ \mu^+(I) : I \in \B(1,I) \right\} < \infty \]	
	%
%	Then there exists a non-zero Borel measure $\mu$ such that $\mu(I) \leq \mu^+(I)$ for $I \in \B$.
%\end{lemma}
%\begin{proof}
%	As in the last lemma, define the operators $E_k$ and the measures $\mu_k$. By weak compactness, a subsequence of these measures converge weakly to some measure $\mu$, and $E_k(\mu) = \lim E_k(\mu_{j_k}) \leq \mu_k$. The measure $\mu$ is nonzero, since $\| \mu_{j_k} \| \geq c$ for each $k$, and so $\| \mu \| \geq c$.
%\end{proof}

\begin{lemma}[Frostman's Lemma]
	If $E$ is Borel, $H^s(E) > 0$ if and only if there exists an $s$ dimensional Frostman measure supported on $E$.
\end{lemma}
\begin{proof}
	Suppose that $\mu$ is $s$ dimensional and supported on $E$. If $H^s(F) = 0$, then for each $\varepsilon > 0$ there is a sequence of cubes $\{ Q_k \}$ with $\sum_{k = 1}^\infty l(Q_k)^s \leq \varepsilon$. But then
	%
	\[ \mu(F) \leq \sum_{k = 1}^\infty \mu(Q_k) \lesssim \sum_{k = 1}^\infty l(Q_k)^s \leq \varepsilon. \]
	%
	Taking $\varepsilon \to 0$, we conclude $\mu(F) = 0$. Thus $\mu$ is absolutely continuous with respect to $H^s$. But since $\mu(E) > 0$, this means that $H^s(E) > 0$.

	Conversely, suppose $H^s(E) > 0$. We work with the classical family of dyadic cubes, i.e. with the sequence $\{ N_k = 2 : k \geq 1 \}$.  By translating, we may assume that $H^s(E \cap [0,1]^d) > 0$, and so without loss of generality we may assume $E \subset [0,1]^d$. Fix $m$, and for each $Q \in \DQ_k^d$, define $\mu^+(Q) = H^s_{1/2^m}(E \cap Q)$. Then $\mu^+(Q) \leq 1/2^{ks}$, and $\mu^+$ is subadditive. We use it to recursively define a Frostman measure $\mu$, such that $\mu(Q) \leq \mu^+(Q)$ for each $Q \in \DQ_k^d$. We initially define $\mu$ by setting $\mu([0,1]^d) = \mu^+([0,1]^d)$. Given $Q \in \DQ_k^d$, we enumerate it's children as $Q_1, \dots, Q_M \in \DQ_{k+1}^d$. We then consider any values $A_1, \dots, A_M \geq 0$ such that
	%
  	\begin{equation} \label{equation3424209034}
  		A_1 + \dots + A_M = \mu(Q),
  	\end{equation}
  	%
  	and for each $k$,
  	%
  	\begin{equation} \label{equation12039123012}
  		A_k \leq \mu^+(Q_k).
  	\end{equation}
	%
	This is feasible to do because $\mu^+(Q_1) + \dots + \mu^+(Q_M) \geq \mu^+(Q)$. We then define $\mu(Q_k) = A_k$ for each $k$. Equation \eqref{equation3424209034} implies the recursive constraint is satisfied, so $\mu$ is a well defined function. Equation \eqref{equation12039123012} implies \eqref{equation73234091} of Lemma \ref{massdistributionprinciplelem}, and so the mass distribution principle implies $\mu$ extends to a Borel measure which satisfies $\mu(Q) \leq \mu^+(Q) \leq 1/2^{ks}$ for each $Q \in \DQ_k^d$. Given any cube $Q$, we find $k$ with $1/2^{k-1} \leq l(Q) \leq 1/2^k$. Then $Q$ is covered by $O_d(1)$ dyadic cubes in $\DQ_k^d$, and so $\mu(Q) \lesssim l(Q)^s$ for any cube $Q$. And this means that we have shown directly that $\mu$ is a Frostman measure of dimension $s$.
\end{proof}

Frostman's lemma implies that to study the Hausdorff dimension of the set, it suffices to understand the class of measures which can be supported on that set. The Fourier dimension of a set also studies this perspective, by a slightly refinement of the measure bound required on the Frostman dimension of a measure.



\section{Fourier Dimension}

The applicability of Fourier analysis to the analysis of dimension begins by converting the measure bound required on the Frostman dimension onto a condition on the Fourier transform of the measure. For a Borel measure $\mu$, we define the \emph{$s$ energy} of $\mu$ as
%
\[ I_s(\mu) = \int \int \frac{d\mu(x) d\mu(y)}{|x - y|^s}. \]
%
A simple rearrangement shows that for each $y$,
%
\[ \int \int \frac{d\mu(x) d\mu(y)}{|x - y|^s} = s \int_0^\infty \frac{\mu(B(x,r))}{r^{s+1}}\; dr, \]
%
where $B(x,r)$ is the open ball of radius $r$ about the point $x$. This enables us to relate the Frostman bound with energy integrals.

\begin{theorem}
	For any set $E$,
	%
	\[ \hausdim(E) = \sup \{ s : \text{there is $\mu$ supported on $E$ with $I_s(\mu) < \infty$} \}. \]
\end{theorem}
\begin{proof}
	For each $x \in \RR^d$ and $r > 0$, let $B(x,r)$ denote the open ball of radius $r$ centered at $x$. We note $\mu$ is an $s$ dimensional Frostman measure if and only if $\mu(B(x,r)) \lesssim r^s$, since every dyadic cube with sidelength $r$ is contained in $O_d(1)$ balls of radius $r$, and every ball of radius $r$ is contained in $O_d(1)$ dyadic cubes with sidelength $r$. Thus if $\mu$ is a Frostman measure with dimension less than $s$, then $I_s(\mu) < \infty$. Conversely, if $I_s(\mu) < \infty$, then for $\mu$ almost every $y$,
	%
	\[ \int \frac{d\mu(x)}{|x - y|^s} < \infty \]
	%
	In particular, there is $M < \infty$, and $E_0$ with $\mu(E_0) > 0$ such that for any $y \in E_0$,
	%
	\[ \int \frac{d\mu(x)}{|x - y|^s} \leq M. \]
	%
	If we let $\nu(E) = \mu(E \cap E_0)$, then for any $y \in \RR^d$, if $\nu(B(y,r)) > 0$, there is $y_0 \in E \cap B(y,r)$, so $B(y,r) \subset B(y_0,2r)$, and this implies
	%
	\[ \nu(B(y,r)) \leq \nu(B(y_0,2r)) \leq \int_{B(y_0,2r)} d\nu(x) \leq 2^s r^s \int_{B(y_0,2r)} \frac{d\nu(x)}{|x - y|^s} \leq (2^s M) r^s. \]
	%
	Thus $\nu$ is a Frostman measure of dimension $s$ supported on $E$.
\end{proof}

\begin{lemma}
	There exists a constant $C(d,s)$ such that
	%
	\[ I_s(\mu) = C(d,s) \int k_{d-s}(\xi) |\widehat{\mu}(\xi)|^2\; d\xi. \]
\end{lemma}
\begin{proof}
	We can convert the energy integral into a condition on the Fourier transform. If we define the \emph{Riesz kernels} $k_s(x) = 1/|x|^s$, for $0 < s < d$, then
	%
	\[ I_s(\mu) = \int (k_s * \mu) d\mu. \]
	%
	Naively applying the multiplication formula for the Fourier transform, we find
	%
	\begin{align*}
		\int (k_s * \mu)(x) d\mu(x) &= \int \widehat{k_s * \mu}(\xi) \overline{\widehat{\mu}(\xi)}\; dx\\
		&= \int \widehat{k_s}(\xi) |\widehat{\mu}(\xi)|^2\; dx\\
		&= C(d,s) \int k_{d-s}(\xi) |\widehat{\mu}(\xi)|^2\; dx.
	\end{align*}
	%
	where we have used the fact that the Fourier transform of $k_s$ is equal to $C(d,s) k_{d-s}$ for some constant $C(d,s)$. This is not a rigorous argument, because the kernels $k_s$ and the measure $\mu$ do not lie in $L^1(\RR^d)$. But one can interpret this idea in a distributional sense, and therefore obtain the formula by a technical argument involving approximations to the identity, which we leave out of this thesis. A formal proof can be found in BLAH.
\end{proof}

Thus the application of energy integrals translates the problem of finding a Frostman measure bound to frequency space. In particular, if $\mu$ is a Frostman measure of dimension $s$, then
%
\[ \int \frac{|\widehat{\mu}(\xi)|^2}{|\xi|^{d-s}}\; d\xi < \infty. \]
%
A weak type bound implies that we should have
%
\begin{equation} \label{equation89041094242} |\widehat{\mu}(\xi)|^2 \lesssim |\xi|^{s/2} \end{equation}
%
for \emph{most} values $\xi$. Thus we obtain a stronger condition if we require that \eqref{equation89041094242} holds for \emph{all} values $\xi$. In particular, we say a measure $\mu$ is a measure with \emph{Fourier dimension} $s$ if $\eqref{equation89041094242}$ holds for \emph{all} values $\xi$. If this is true, then $I_t(\mu) < \infty$ for all $t < s$. Thus if we define the \emph{Fourier dimension} of a set $E$ as
%
\[ \fordim(E) = \sup \left\{ s : \begin{array}{c} \text{there is $\mu$ supported on $E$ with}\\ \text{$|\widehat{\mu}(\xi)| \lesssim |\xi|^{s/2}$ for all $\xi \in \RR^d$} \end{array} \right\}, \]
%
then $\fordim(E) \leq \hausdim(E)$. We say a set $E$ is \emph{Salem} if $\fordim(E) = \hausdim(E)$.

Most classical, self-similar fractals (for instance, the classical Cantor set) have Fourier dimension zero. Nonetheless, most \emph{random} $s$ dimensional subsets of $\RR^d$ are almost surely Salem. As of now, except in very particular cases, the only way to find sets with large Fourier dimension is to input randomness into their construction. Using randomness, in Chapter BLAH we are able to find Salem sets with large Fourier dimension.







\section{Dyadic Fractional Dimension}

We wish to establish results about fractional dimension `dyadically'. We begin with Minkowski dimension. For each $m$, let $N_{\DQ}(m,E)$ denote the minimal number of cubes in $\DQ_m^d$ required to cover $E$. This is often easy to calculate, up to a multiplicative constant, by greedily selecting cubes which intersect $E$.

\begin{lemma} \label{comparableCovers}
	For any set $E$,
	%
	\[ N_{\DQ}(m,E) \sim_d \# \{ Q \in \DQ_m^d : Q \cap E \neq \emptyset \} \sim_d N(l_m,E). \]
\end{lemma}
\begin{proof}
	Let $\mathcal{E} = \{ Q \in \DQ_m^d : Q \cap E \neq \emptyset \}$. Then $\mathcal{E}$ is certainly a cover of $E$ by dyadic cubes, which shows $N_{\DQ}(m,E) \leq \#(E)$. Conversely, let $\{ Q_k \}$ be a minimal cover of $E$. Then $Q_k \cap E \neq \emptyset$ for each $m$, so $\{ Q_k \} \subset \mathcal{E}$. But if $Q \in \mathcal{E}$, then it intersects a cube in $\{ Q_k \}$, and since each cube $Q_k$ intersects at most $2^d$ other cubes in $\DQ_k^d$, we conclude that $\#(\mathcal{E}) \leq (2^d + 1) N_{\DQ}(k,E)$. The bound $N(l_m,E) \leq N_{\DQ}(m,E)$ is obvious, whereas for each $Q$ with $l(Q) = l_m$, $Q$ is covered by $2^d + 1$ cubes in $\DQ_m^d$, so $N_{\DQ}(m,E) \leq (2^d + 1) N(l_m,E)$.
\end{proof}

Thus it is natural to ask whether it is true that for any set $E$,
%
\begin{equation} \label{definingSequence}
	\begin{aligned}
		\lowminkdim(E) &= \liminf_{k \to \infty} \frac{\log[N_{\DQ}(k,E)]}{\log[1/l_k]}\\
		\text{and}\\
		\upminkdim(E) &= \limsup_{k \to \infty} \frac{\log[N_{\DQ}(k,E)]}{\log[1/l_k]}.
	\end{aligned}
\end{equation}
%
The answer depends on the choice of $\{ N_k \}$.

%Subsets of $\Sigma^d$ can also be assigned a Minkowski dimension. We define
%
%\[ \lowminkdim(E) = \liminf_{k \to \infty} \frac{\log(\#(\sigma_k^d(E)))}{\log(1/l_k)}\quad\text{and}\quad\upminkdim(E) = \limsup_{k \to \infty} \frac{\log(\#(\sigma_k^d(E)))}{\log(1/l_k)}. \]
%
%This makes sense, because $\Sigma^d$ only really has `balls' of radius $\{ l_k \}$, for each $k$, and \emph{any} cover of $E$ by balls of radius $l_k$ contains $\sigma_k^d(E)$. In order 
%we have $N(E,l_k) = \# (\Sigma_k^d(E))$, since \emph{any} cover of $E$ by balls of radius $l_k$

\begin{lemma} \label{definingsequenceminkowski}
	If, for all $\varepsilon > 0$, $N_{k+1} \lesssim_\varepsilon (N_1 \dots N_k)^\varepsilon$, then \eqref{definingSequence} holds.
\end{lemma}
\begin{proof}
	Fix a length $l$, and find $k$ with $l_{k+1} \leq l \leq l_k$. Applying Lemma \ref{comparableCovers} shows
	%
	\[ N(l,E) \leq N(l_{k+1},E) \lesssim_d N_{\DQ}(k+1,E) \]
	%
	and
	%
	\[ N(l,E) \geq N(l_k,E) \gtrsim_d N_{\DQ}(k,E). \]
	%
	Thus
	%
	\[ \frac{\log[N(l,E)]}{\log[1/l]} \leq \left[ \frac{\log(1/l_{k+1})}{\log(1/l_k)} \right] \frac{\log[N_{\DQ}(k+1,E)]}{\log[1/l_{k+1}]} + O_d(1/k) \]
	%
	and
	%
	\[ \frac{\log[N(l,E)]}{\log[1/l]} \geq \left[ \frac{\log(1/l_k)}{\log(1/l_{k+1})} \right] \frac{\log[N_{\DQ}(k,E)]}{\log[1/l_k]} + O_d(1/k). \]
	%
	Thus, provided that
	%
	\begin{equation} \label{equivalenceofscales}
		\frac{\log(1/l_{k+1})}{\log(1/l_k)} \to 1,
	\end{equation}
	%
	the conclusion of the theorem is true. But \eqref{equivalenceofscales} is equivalent to the condition that
	%
	\[ \frac{\log(N_{k+1})}{\log(N_1) + \dots + \log(N_k)} \to 0, \]
	%
	and this is equivalent to the assumption of the theorem.
\end{proof}

Any constant branching factor satisfies the hypothesis of Lemma \ref{definingsequenceminkowski} for the Minkowski dimension. In particular, we can construct sets over classical dyadic cubes without any problems occuring. But more importantly for our work, we can let the sequence $\{ N_k \}$ increase rapidly.

\begin{lemma} \label{rapidBranching}
	If $N_k = 2^{\lfloor 2^{k \psi(k)} \rfloor}$, where $\psi(k)$ is any decreasing sequence of positive numbers tending to zero, but for which $k \psi(k) \to \infty$, then $N_{k+1} \lesssim_\varepsilon (N_1 \dots N_k)^\varepsilon$ for any $\varepsilon > 0$.
\end{lemma}
\begin{proof}
	We note that $\log(N_k) = 2^{k \psi(k)} + O(1)$. Thus
	%
	\begin{align*}
		\frac{\log(N_{k+1})}{\log(N_1) + \dots + \log(N_k)} &= \frac{2^{(k+1) \psi(k+1)} + O(1)}{2^{\psi(1)} + 2^{2 \psi(2)} + \dots + 2^{k \psi(k)} + O(k)}\\
		&\lesssim \frac{2^{(k+1) \psi(k+1)}}{2^{\psi(k)} + 2^{2 \psi(k)} + \dots 2^{k \psi(k-1)}}\\
		&\lesssim \frac{2^{(k+1) \psi(k+1)}}{2^{(k+1) \psi(k)}} ( 2^{\psi(k)} - 1 ) = 2^{\psi(k)} - 1 \to 0.
	\end{align*}
	%
	This is equivalent to the fact that $N_{k+1} \lesssim_\varepsilon (N_1 \dots N_k)^\varepsilon$ for any $\varepsilon > 0$.
\end{proof}
%
We refer to any sequence $\{ l_k \}$ constructed by $\{ N_k \}$ satisfying the conditions of Lemma \ref{rapidBranching} as a \emph{subhyperdyadic} sequence. If a sequence is generated by a sequence $\{ N_k = 2^{\lfloor 2^{ck} \rfloor} \}$, for some fixed $c > 0$, the lengths are referred to as \emph{hyperdyadic}. The next (counter) example shows that hyperdyadic sequences are essentially the `boundary' for sequences that can be used to measure the Minkowski dimension.

\begin{example}
	We consider a multi-scale dyadic construction. Fix $0 \leq c < 1$, and define $N_k = 2^{\lfloor 2^{ck} \rfloor}$, and $M_k = 2^{\lfloor c 2^{ck} \rfloor}$. Then $M_k \divides N_k$ for each $k$. We recursively define a nested family of sets $\{ E_k \}$, with each $E_k$ a $\DQ_k^d$ discretized set, and set $E = \bigcap E_k$. We define $E_0 = [0,1]$. Then, given $E_k$, we divide each sidelength $l_k$ dyadic interval in $E_k$ into $M_{k+1}$ intervals, and then keep the first sidelength $l_k/M_{k+1}$ interval from this set, which is formed from $N_{k+1}/M_{k+1}$ sidelength $l_{k+1}$ dyadic intervals. Thus $\#(\DQ_0(E_0)) = 1$, and
	%
	\[ \#(\DQ_{k+1}(E_{k+1})) = (N_{k+1}/M_{k+1}) \#(\DQ_k(E_k)) \]
	%
	so
	%
	\begin{equation} \label{equation12623} \#(\DQ_k(E_k)) = \frac{N_1 \dots N_k}{M_1 \dots M_k} \end{equation}
	%
	Noting that $\log(N_i) = 2^{ci} + O(1)$, and $\log(M_i) = c2^{ci} + O(1)$, we conclude that
	%
	\begin{align*}
		\frac{\log \#(\DQ_k(E_k))}{\log(1/l_k)} &= \frac{(1-c)(2^c + \dots + 2^{ck}) + O(k)}{(2^c + \dots + 2^{ck}) + O(k)} = 1-c + O\left(\frac{k}{2^{ck}} \right) \to 1-c.
	\end{align*}
	%
	On the other hand, if $r_{k+1} = l_k/M_{k+1}$, then for each $k$,
	%
	\[ \#(\DR_{k+1}^d(E_k)) = \#(\DQ_k(E_k)) = \frac{N_1 \dots N_k}{M_1 \dots M_k}, \]
	%
	and so
	%
	\begin{align*}
		\frac{\log \#(\DR_{k+1}^d(E_k))}{\log(1/r_{k+1})} &= \frac{(1-c)(2^c + \dots + 2^{ck}) + O(k)}{(2^c + \dots + 2^{ck}) + c2^{c(k+1)} + O(k)} \\
		&= \frac{\left( \frac{(1-c) 2^c}{2^c - 1} \right) 2^{ck} + O(k)}{ \left( \frac{2^c}{2^c - 1} + c2^c \right) 2^{ck} + O(k)}\\
		&= \frac{1-c}{1 - c + c2^c} + O \left( \frac{k}{2^{ck}} \right) \to \frac{1 - c}{1 - c + c2^c} < 1 - c.
	\end{align*}
	%
%
%	Fix $0 \leq c < 1$. Let $\{ M_k \}$ be a sequence of integers such that $M_k \divides N_k$ for each $k$. We recursively define a sequence of sets $\{ E_k \}$, with $E_k$ a union of length $l_k$ intervals
%
%	Construct a subset of $\RR$ as follows. Let $N_k = K_kM_k$, where $N_k$, $K_k$, and $M_k$ are parameters to be specified later. Define $E_0 = [0,1]$. Given $E_k$, define $E_{k+1}$ by dividing each sidelength $l_k$ dyadic interval in $E_k$ into $K_{k+1}$ intervals, and then keeping only the first interval. Then $\#(\DQ_{k+1}^d(E_{k+1})) = M_k \cdot \#(\DQ_k^d(E_k))$, and since $\#(\DQ_0^d(E_0)) = 1$, $\#(\DQ_k^d(E_k)) = M_1 \dots M_k$. Thus
%	\begin{align*}
%		\frac{\log \left[ N(l_k,E) \right]}{\log(1/l_k)} &\sim \frac{\log \left[ \#(\DQ_k^d(E_k)) \right]}{\log(1/l_k)}\\
%		&= \frac{\log(M_1) + \dots + \log(M_k)}{\log(N_1) + \dots + \log(N_k)}\\
%		&= 1 - \frac{\log(K_1) + \dots + \log(K_k)}{\log(N_1) + \dots + \log(N_k)}.
%	\end{align*}
	%
%	On the other hand, if $r_k = l_k/K_{k+1}$, $N(r_k,E) \sim_d \#(S_k) = M_1 \dots M_k$, so
	%
%	\begin{align*}
%		\frac{\log \left[ N(r_k,E) \right]}{\log(1/r_k)} &\sim \frac{\log \left[ \#(\DQ_k^d(E_k)) \right]}{\log(1/r_k)}\\
%		&= \frac{\log(M_1) + \dots + \log(M_k)}{\log(N_1) + \dots + \log(N_k) + \log(K_{k+1})}\\
%		&= 1 - \frac{\log(K_1) + \dots + \log(K_{k+1})}{\log(N_1) + \dots + \log(N_k) + \log(K_{k+1})}.
%	\end{align*}
	%
%	Set $N_k = 2^{\lfloor 2^{ck} \rfloor}$, and $K_k = 2^{\lfloor c 2^{ck} \rfloor}$. Then
	%
%	\[ \log(K_1) + \dots + \log(K_k) = O(k) + c \sum_{i = 1}^k 2^{ck} = O(k) + c \frac{2^{c(k+1)} - 2^c}{2^c - 1} \]
	%
%	and
	%
%	\[ \log(N_1) + \dots + \log(N_k) = O(k) + \sum_{i = 1}^k 2^{ck} = O(k) + \frac{2^{c(k+1)} - 2^c}{2^c - 1}. \]
	%
%	Thus
	%
%	\[ \frac{\log \left[ N(l_k,E) \right]}{\log(1/l_k)} \to 1 - c \quad \text{and} \quad \frac{\log \left[ N(r_k,E) \right]}{\log(1/l_k)} \to \frac{1 - c}{1 - c + c2^c}. \]
	In particular,
	%
	\[ \lowminkdim(E) \neq \liminf_{k \to \infty} \frac{\log \left[ N(l_k,E) \right]}{\log(1/l_k)} = \liminf_{k \to \infty} \frac{\log[N_{\DQ}(k,E)]}{\log(1/l_k)}, \]
	%
	so measurements at hyperdyadic scales fail to establish general results about the Minkowski dimension.

%	\[ \lim_{k \to \infty} \frac{\log(K_1) + \dots + \log(K_k)}{2^{c(k+1)} - 2^c} \neq \lim_{k \to \infty} \frac{\log(K_1) + \dots + \log(K_{k+1})}{2^{c(k+1)} - 2^c} \frac{1}{1 + \log(K_{k+1}) (2^c - 1)/(2^{c(k+1)} - 2^c)} \]

%	First, assume $N_k/N_{k-1} \in \mathbf{Z}$ for each $k$, and for convenience, set $N_0 = 1$. Define $S_0 = \{ 0 \}$, and then given $S_k$, recursively define
	%
%	\[ S_{k+1} = \{ (j,1), \dots, (j,N_{k+1}/N_k) : j \in S_k \} \]
	%
%	We then set $E = \pi(\lim S_k)$. If we let $E_k = \bigcup \{ Q_j : j \in S_k \}$, then $E_{k+1}$ can be constructed by dividing each sidelength $l_k$ dyadic interval in $E_k$ into $N_{k+1}$ intervals, and selecting the initial $N_{k+1}/N_k$ intervals, which have total length $1/(N_1 \dots N_k^2)$. Then $\#(S_{k+1}) = (N_{k+1}/N_k) \#(S_k)$, and since $\#(S_0) = 1$, $\#(S_k) = N_k$. Thus
	%
%	\[ \frac{\log \left[ N(l_k,E) \right]}{\log(1/l_k)} \sim_d \frac{\log \left[ \# (S_k) \right]}{\log(1/l_k)} = \frac{\log(N_k)}{\log(N_1) + \dots + \log(N_k)}. \]
	%
%	On the other hand, if $r_k = 1/(N_1 \dots N_k^2) = l_k/N_k$, then $N(l,E) = \#(S_k) = N_k$, so
	%
%	\[ \frac{\log \left[ N(r_k,E) \right]}{\log(1/r_k)} \sim_d \frac{\log \left[ \#(S_k) \right]}{\log(1/r_k)} = \frac{\log(N_k)}{\log(N_1) + \dots + 2\log(N_k)}. \]
%	In any case to which Lemma \ref{definingsequenceminkowski} applies, both of these estimates converge to zero as $k \to \infty$, so that $E$ has Minkowski dimension zero. On the other hand, if $N_k = 2^{\lfloor \psi(k) 2^k \rfloor}$ where $\psi(k)$ is an increasing sequence tending to $\infty$, then
	%
%	\begin{align*}
%		\frac{2^{(k-1) \psi(k-1)}}{2^{\psi(1)} + 2^{2 \psi(2)} + \dots + 2^{(k-1) \psi(k-1)}} &\geq \frac{\left(2^{(k-1) \psi(k-1)} \right) \left( 2^{\psi(k-1)} - 1 \right)}{2^{k \psi(k-1)} - 2^{\psi(k-1)}} \sim 1.
%	\end{align*}
	%
%	Thus
	%
%	\[ \lim_{k \to \infty} \frac{\log \left[ N(l_k,E) \right]}{\log(1/l_k)} = 1. \]
	%
%	On the other hand,
	%
%	\[ \frac{\log \left[ N(r_k, E) \right]}{\log(1/r_k)} = \frac{\log(N_k)}{\log(N_1) + \dots + 2\log(N_k)} \leq 1/2. \]
	%
%	Thus we cannot possibly have
	%
%	\[ \lowminkdim(E) = \liminf_{k \to \infty} \frac{\log \left( N(l_k,E) \right)}{\log(1/l_k)}, \]
	%
	%
%	A simple calculation shows this result even fails if $N_k = 2^{\lfloor 2^{ck} \rfloor}$, where $c > 1$.
\end{example}

%In particular, we can define the Minkowski dimensions of $E \subset \Sigma$ as
%
%\[ \lowminkdim(E) = \liminf_{k \to \infty} \frac{\log(\#(\Sigma_k^d(E)))}{\log(1/l_k)}\quad\text{and}\quad\upminkdim(E) = \limsup_{k \to \infty} \frac{\log(\#(\Sigma_k^d(E)))}{\log(1/l_k)}. \]
%
%Similarily, the Hausdorff measures $H^s$ are obtained by setting
%
% TODO: Fix this
%\[ H^s(E) = \left\{ \sum_m l_{k_m} : Q_m \in \Sigma_{k_m}^d\ \text{for each $k$}, \text{For any} \right\} \]
%
%and define the Hausdorff dimension correspondingly. A natural question is whether $\dim(\pi(E)) = \dim(E)$ for the various fractal dimensions we consider in this thesis. This is addressed in the next section.

We now move on to calculating Hausdorff dimension dyadically. The natural quantity to consider is the measure defined for any $E$ as $H^s_{\DQ}(E) = \lim H^s_{\DQ,m}(E)$, where
%
\[ H^s_{\DQ,m}(E) = \inf \left\{ \sum_k l(Q_k)^s : E \subset \bigcup_k^\infty Q_k,\ Q_k \in \bigcup_{i \geq m} \DQ_i^d\ \text{for each $k$} \right\}. \]
%
A similar argument to the standard Hausdorff measures shows there is a unique $s_0$ such that $H^s_{\DQ}(E) = \infty$ for $s < s_0$, and $H^s_{\DQ}(E) = 0$ for $s > s_0$. It is obvious that $H^s_{\DQ}(E) \geq H^s(E)$ for any set $E$, so we certainly have $s_0 \geq \hausdim(E)$. The next lemma guarantees that $s_0 = \hausdim(E)$, under the same conditions on the sequence $\{ N_k \}$ as found in Lemma \ref{definingsequenceminkowski}.

\begin{lemma} \label{lemma51464}
	If, for any $\varepsilon > 0$, $N_{k+1} \lesssim_\varepsilon (N_1 \dots N_k)^\varepsilon$, then for any $\varepsilon > 0$,
	%
	\[ H^s_{\DQ}(E) \lesssim_\varepsilon H^{s-\varepsilon}(E). \]
\end{lemma}
\begin{proof}
	Fix $\varepsilon > 0$ and $m$. Let $E \subset \bigcup Q_k$, where $l(Q_k) \leq l_m$ for each $k$. Then for each $k$, we can find $i_k$ such that $l_{i_k+1} \leq l(Q_k) \leq l_{i_k}$. Then $Q_k$ is covered by $O_d(1)$ elements of $\DQ_{i_k}^d$, and
	%
	\begin{equation} \label{equation824} H^s_{\DQ,m}(E) \lesssim_d \sum l_{i_k}^s \leq \sum (l_{i_k}/l_{i_k+1})^s l(Q_k)^s \leq \sum \left( l_{i_k}/l_{i_{k+1}} \right)^s l_{i_k}^\varepsilon l(Q_k)^{s - \varepsilon} \end{equation}
	%
	By assumption,
	%
	\begin{equation} \label{equation992352}
		l_{i_k+1} = \frac{1}{N_1 \dots N_{i_k} N_{i_k+1}} \gtrsim_{s,\varepsilon} (N_1 \dots N_{i_k})^{1+ \varepsilon/s} = l_{i_k}^{1 + \varepsilon/s}.
	\end{equation}
	%
	Putting \eqref{equation824} and \eqref{equation992352} together, we conclude that $H^s_{\DQ,m}(E) \lesssim_{d,s,\varepsilon} \sum l(Q_k)^{s-\varepsilon}$. Since $\{ Q_k \}$ was an arbitrary cover of $E$, we conclude $H^s_{\DQ,m}(E) \lesssim H^{s-\varepsilon}(E)$, and since $m$ was arbitrary, that $H^s_{\DQ}(E) \lesssim H^{s-\varepsilon}(E)$.
\end{proof}

Finally, we consider computing whether we can establish that a measure is a Frostman measure dyadically. First, we recognize the utility of this approach from the perspective of a dyadic construction. Suppose we have a sequence $\{ \mathcal{E}_k \}$, where $\mathcal{E}_k \subset \DQ_k^d$, and $\mathcal{E}_{k+1}^* = \mathcal{E}_k$. Then the sets $\{ E_k = \bigcup \mathcal{E}_k \}$ are nested, and we can set $E = \bigcap E_k$ as a `limit' of the discretizations $E_k$. We can associate with this construction a finite measure $\mu$ supported on $E$. It is defined by setting $\mu([0,1]^d) = 1$, and for each $Q \in \mathcal{E}_{k+1}$, setting
%
\[ \mu(Q) = \frac{\mu(Q^*)}{\# \{ Q_0 \in \mathcal{E}_{k+1} : Q_0^* = Q^* \}}. \]
%
The mass distribution principle extends $\mu$ to a Borel measure, and we refer to it as the \emph{canonical} measure associated with this construction. For this measure, it is often easy to show that $\mu(Q) \lesssim l(Q)^s$ if $Q \in \DQ^d$. The conditions of Lemma \ref{definingsequenceminkowski} are then sufficient to infer that $\mu$ is a Frostman measure of dimension $s - \varepsilon$ for all $\varepsilon > 0$.

\begin{theorem} \label{easyCoverTheorem}
	If $N_{k+1} \lesssim_\varepsilon (N_1 \dots N_k)^\varepsilon$ for all $\varepsilon > 0$, and if $\mu$ is a Borel measure such that $\mu(Q) \lesssim l(Q)^s$ for each $Q \in \DQ_k^d$, then $\mu$ is a Frostman measure of dimension $s - \varepsilon$ for each $\varepsilon > 0$.
\end{theorem}
\begin{proof}
	Given a cube $Q$, find $k$ such that $l_{k+1} \leq l(Q) \leq l_k$. Then $Q$ is covered by $O_d(1)$ cubes in $\DQ^d_k$, which shows
	%
	\[ \mu(Q) \lesssim_d l_k^s = [(l_k/l)^s l^\varepsilon ] l^{s - \varepsilon} \leq [l_k^{s + \varepsilon} / l_{k+1}^s] l^{s - \varepsilon} = \left[ \frac{N_{k+1}^s}{(N_1 \dots N_k)^\varepsilon} \right] l^{s-\varepsilon} \lesssim_\varepsilon l^{s-\varepsilon}. \qedhere \]
\end{proof}	

\begin{remark}
	The dyadic construction showing that Minkowski dimension cannot be measured only at hyperdyadic scales also shows that a bound on a measure $\mu$ on hyperdyadic cubes does not imply the correct bound at all scales. It is easy to show from \eqref{equation12623} that the canonical measure $\mu$ for this example satisfies $\mu(Q) \lesssim l(Q)^{1-c}$ when $Q$ is hyperdyadic, yet we know that for the set constructed in that example,
	%
	\[ \hausdim(E) \leq \lowminkdim(E) < 1 - c. \]
	%
	Frostman's lemma implies we cannot possibly have $\mu(Q) \lesssim_\varepsilon l(Q)^{1-c-\varepsilon}$ for all $\varepsilon > 0$ and \emph{all} cubes $Q$.
\end{remark}






%Our final method for interpolating requires extra knowledge of the dissection process, but enables us to choose the $l_k$ arbitrarily rapidly. The idea behind this is that there is an additional sequence of lengths $r_k$ with $l_k \leq r_k \leq l_{k-1}$. The difference between $r_k$ and $l_{k-1}$ is allowed to be arbitrary, but the decay rate between $l_k$ and $r_k$ is of polynomial-type, which enables us to use the covering methods of the previous section. In addition, we rely on a `uniform mass bound' between $r_k$ and $l_k$ to cover the remaining classes of intervals. Because we can take $r_k$ arbitrarily large relative to $l_k$, this renders any constants that occur in the construction to become immediately negligible. For two quantities $A$ and $B$, we will let $A \precsim_k B$ stand for an inequality with a hidden constant depending only on parameters with index smaller than $k$, i.e. $A \leq C(l_1, \dots, l_k, r_1,\dots,r_k) B$ for some constant $C(l_1, \dots, l_k, r_1, \dots, r_k)$ depending only on parameters with indices up to $k$.

\section{Beyond Hyperdyadics}

If we are to use a faster increasing sequence of branching factors than the last section guarantees, we therefore must exploit some extra property of our construction, which is not always present in general sets. Here, we rely on a \emph{uniform mass distribution} between scales. Given the uniformity assumption, the lengths can decrease as fast as desired. We utilize a multi-scale set of dyadic cubes.

\begin{lemma} \label{uniformMassFrostman}
	Let $\mu$ be a measure supported on a set $E$. Suppose that
    %
    \begin{enumerate}
    	\item \label{discreteBound} For any $Q \in \DQ_k^d$, $\mu(Q) \lesssim l_k^s$.
    	\item \label{controlledScale} For each $Q \in \DR_k^d$, $\# \{ Q \in \DQ_k^d : E \cap Q \neq \emptyset \} = O(1)$.
    	\item \label{uniformDist} For any $Q \in \DR_{k+1}^d$ with parent cube $Q^* \in \DQ_k^d$, $\mu(Q) \lesssim (r_{k+1}/l_k)^d \mu(Q^*)$.
    \end{enumerate}
	%
	Then $\mu$ is a Frostman measure of dimension $s$.
\end{lemma}
\begin{proof}
	We establish the general bound $\mu(Q) \lesssim l(Q)^s$ for all cubes $Q$ in two different cases:
	%
	\begin{itemize}
		\item Suppose there is $k$ with $r_{k+1} \leq l \leq l_k$. Then we can cover $Q$ by at most $O_d((1/r_{k+1})^d)$ cubes in $\DR_k^d$. By Properties \ref{discreteBound} and \ref{uniformDist}, each of these cubes has measure at most $O( (r_{k+1}/l_k)^d l_k^s)$, so we obtain that
    %
    \[ \mu(Q) \lesssim (l/r_{k+1})^d (r_{k+1}/l_k)^d l_k^s = l^d / l_k^{d-s} \lesssim l^s. \]

    \item Suppose there exists $k$ with $l_k \leq l \leq r_k$. Then we can cover $Q$ by $O_d(1)$ cubes in $\DR_k^d$, and for each such cube, Property \ref{uniformDist} shows there are $O(1)$ cubes in $\DQ_k^d$ which have mass. Thus
    %
    \[ \mu(Q) \lesssim l_k^s \leq l^s. \]
	\end{itemize}
	%
	This addresses all cases, so $\mu$ is a Frostman measure of dimension $s$.
\end{proof}

TODO: ADDRESS OTHER FRACTIONAL DIMENSION CASES?

%\begin{remark}
%    The condition $\mu_\beta(J) \lesssim_{N-1} (r_N/l_N) \mu_\beta(I)$ essentially means that the probability mass on a length $l_N$ interval $I$ is uniformly distributed over the length $r_N$ intervals it contains. This is what enables us to remove the discussion of the growth of the sequence $\beta$ over time from discussion.
%\end{remark}

%Since the construction is obtained as a limit of intervals, it is often possible to construct such a $\mu$ by the {\it mass distribution principle}. That is, we let $\mu$ denote the weak limit of the probability masses $\mu_n$, where $\mu_0$ is a uniform distribution over $\mu_0$, and $\mu_{n+1}$ is obtained from $\mu_n$ by distributing the mass $\mu_n(I)$ of each length $l_n$ interval $I$ contained in $X_n$ over the portion of $I$ that remains in $X_{n+1}$. The cumulative distribution functions of the $\mu_n$ uniformly converge, hence the $\mu_n$ converge weakly to some $\mu$, which satisfy $\mu(I) = \mu_n(I)$ for each interval $I$ as above. Because of this discreteness, it is most easy to establish a bound $\mu(I) \lesssim l_n^\alpha$ when $I \subset X_n$ is a length $l_n$ interval. Since any interval $I$ of length $l_n$ is contained within at least two such intervals (or is contained in other length $l_n$ intervals that $\mu$ assigns no mass to), we have the general bound $\mu(I) \lesssim l_n^\alpha$ for all intervals $I$ of length $l_n$. Hausdorff dimension is a local property of a set\footnote{If we define $\dim_{\mathbf{H}}(x) = \lim_{r \downarrow 0} \dim_{\mathbf{H}}(B_r(x) \cap X)$ then $\dim_{\mathbf{H}}(X) = \sup_{x \in X} \dim_{\mathbf{H}}(x)$.}, so it is natural to expect that we can obtain a general bound $\mu(I) \lesssim_\alpha|I|^\alpha$ given that one has established precisely the same estimate, but restricted to intervals $I$ with $|I| = l_N$. This section concerns itself with ways that we can establish this general bound, and thus prove that $\dim_{\mathbf{H}}(X) \geq \alpha$.

%\section{BLAH}

%The collection $\DQ^d$ forms a \emph{tree} under the partial ordering induced by inclusion, with a branching factor of $H$. We let $\DB^d$ denote the set of all branches of the tree $\DQ^d$. For each branch $\mathfrak{b} = \{ Q_k : k \geq 0 \}$, the set $\bigcap_{k \geq 0} Q_k$ contains a unique point, which induces a function $\pi: \DB^d \to \RR^d$. This function is obviously surjective, but unfortunately not injective. Nonetheless, for each $x \in \RR^d$, $\pi^{-1}(x)$ contains $O_d(1)$ points. This means that from the point of view of geometric measure theory, the spaces $\DB^d$ and $\RR^d$ are isomorphic\footnote{Formally, we can define a topology on $\DB^d$ as a subset of $\prod_{k \geq 0} \DQ_k^d$, and $\pi$ is continuous with respect to this topology. We can also define the Hausdorff measures $H^s$ on $\DB^d$ as limits of $H^s_m$, where
%
%\[ H^s_m(E) = \inf \left\{ \sum_{i = 1}^\infty l(Q_i)^s : Q_i \in \DQ_{k_i}^d, k_i \geq m, E \subset \bigcup \{ \mathfrak{b} \in \DB^d : \mathfrak{b}_k = Q \} \right\} \]
%
%Then for each value $s > 0$ the map $\pi$ is an isomorphism of $(\DB^d, H^s)$ and $(\RR^d, H^s)$.}. In particular, given any configuration $\C$ on $\RR^d$, we can define a configuration $\pi^{-1}(\C)$ on $\DB^d$ as
%
%\[ \pi^{-1}(\C) = \{ (\mathfrak{b_1}, \dots, \mathfrak{b}_n) : (\pi(\mathfrak{b_1}), \dots, \pi(\mathfrak{b_n})) \in \C \}. \]
%
%If we can find an $s$ dimensional set $E \subset \DB^d$ avoiding $\pi^{-1}(\C)$, then $\pi(E)$ avoids $\C$ and is $s$ dimensinoal. Conversely, if $E \subset \RR^d$ avoids $\C$ and is $s$ dimensional, then $\pi^{-1}(E)$ avoids $\pi^{-1}(\C)$ and is $s$ dimensional.

%The Dyadic model is useful, but if the lengths decrease too fast, the model fails to reflect the geometry of $\RR^n$ at all scales.

\section{Extras: Hyperdyadic Covers}

Nonetheless, it will be useful for us to know that we can `decompose' a set with a prescribed Hausdorff dimension hyperdyadically. We say a sequence of sets $\{ E_k \}$ is a \emph{strong cover} of a set $E$ if $E \subset \limsup E_k$, or equivalently, if every $x \in E$ lies in infinitely many of the sets $E_k$. In this section, we inclusively treat hyperdyadic cubes, i.e. we assume $l_k = 2^{-\lfloor (1 + \varepsilon)^k \rfloor}$ for some fixed $0 < \varepsilon \leq 1$.

%Fix two parameters $\delta > 0$ and $\varepsilon > 0$. Given two numbers $A = A_{\delta \varepsilon}$ and $B = B_{\delta \varepsilon}$, we say $A \lessapprox B$ if there exists constants $C_\varepsilon$, and $C$ such that $A \leq C_\varepsilon \delta^{-C\varepsilon} B$. We say $A \approx B$ if $A \lessapprox B$ and $B \lessapprox A$. We say a set $E$ is \emph{$\delta$ discretized} if it is a union of dyadic cubes with sidelength $\approx \delta$. We say a set $E$ is a \emph{$(\delta,\alpha)$ set} if it is $\delta$ discretized, and for any dyadic cube $I$ with $\delta \leq l(I) \leq 1$, $|E \cap I| \lessapprox \delta^{d-\alpha} l(I)^\alpha$. Thus $E$ is \emph{roughly} a $\delta$ thickening of an $\alpha$ dimensional set. A set $E$ is \emph{strongly covered} by a family of sets $\{ U_i \}$ if $E \subset \limsup_{i \to \infty} U_i$. We consider a fixed hyperdyadic sequence $l_k = 2^{- \lfloor (1 + \varepsilon)^k \rfloor}$.

%\begin{lemma}
%	If $C$ is sufficiently large, and for each $\varepsilon$, there is a $(\delta, \alpha - C \varepsilon)$ set $X_\delta$ for each hyperdyadic $\delta$ such that the $X_\delta$ strongly cover $X$, then $\dim(X) \leq \alpha$.
%\end{lemma}
%\begin{proof}
%	Let $X_\delta = \bigcup I_i$, where $\{ I_i \}$ are disjoint dyadic cubes such that $l(I_i) \approx \delta$, and with $|X_\delta| \lessapprox \delta^{d-\alpha + C\varepsilon}$. Then
	%
%	\[ |X_\delta(\delta/2)| \leq \sum (l(I_i) + \delta/2)^d \lessapprox \sum l(I_i)^d = |X_\delta| \lessapprox \delta^{d - \alpha + C\varepsilon}. \]
	%
%	A volumetric argument then guarantees that $N(X_\delta,\delta) \lessapprox \delta^{-\alpha + C\varepsilon}$, and so
	%
%	\[ H^\alpha_\infty(X_\delta) \leq N(X_\delta,\delta) \delta^\alpha \lessapprox \delta^{C\varepsilon}. \]
	%
%	Thus there is $C_\varepsilon$ and $C_0$ such that $H^\alpha_\delta(X_\delta) \leq C_\varepsilon \delta^{(C - C_0) \varepsilon}$. Since $C_0$ does not depend on $C$, if we set $C > C_0$, then
	%
%	\[ \sum_{i = 1}^\infty H^\alpha_\infty(X_\delta) < \infty, \]
	%
%	and so $H^\alpha_\infty(X) = 0$.
%\end{proof}

\begin{theorem}
	Suppose $E \subset [0,1]^d$ is a set with $\dim(E) \leq s$. Fix $\varepsilon > 0$, and write $l_k = 2^{-\lfloor (1 + \varepsilon)^k \rfloor}$. Then there exists a strong cover of $E$ by sets $\{ E_k \}$, where $E_k$ is a union of $O((1 + \varepsilon)^{2k} l_k^{-s})$ cubes in $\DQ_k^d$.
\end{theorem}
\begin{proof}
	For each hyperdyadic number $l_k$, we can find a collection of cubes $\{ Q_{k,i} \}$ covering $E$ with $l(Q_{k,i}) \leq l_k$ for all $i$, and
	%
	\begin{equation} \label{HausdorffBound5}
		\sum_{i = 1}^\infty l(Q_{k,i})^{s + C\varepsilon} \lesssim 1.
	\end{equation}
	%
	For each $k$ and $i$, find $j_{k,i}$ such that $l_{j_{k,i} + 1} \leq l(Q_{k,i}) \leq l_{j_{k,i}}$. Note $l_{j_{k,i} + 1} \lesssim l_{j_{k,i}}^{1 + \varepsilon}$, so
	%
	\begin{align*}
		\sum_{i = 1}^\infty l_{j_{k,i}}^{s + (C + s)\varepsilon} &= \sum_{i = 1}^\infty l(Q_{k,i})^{s + (C + s) \varepsilon} (l_{j_{k,i}} / l(Q_{k,i}))^{s + (C + s) \varepsilon}\\
		&\lesssim \sum_{i = 1}^\infty l(Q_{k,i})^{s + (C + s)\varepsilon} l_{j_{k,i}}^{-s \varepsilon} \lesssim \sum_{i = 1}^\infty l(Q_{k,i})^{s + C \varepsilon} \lesssim 1.
	\end{align*}
	%
	Thus, replacing $C$ with $C + s$, and replacing $Q_{k,i}$ with the $O_d(1)$ cubes in $\DQ_{j_{k,i}}(Q_{k,i})$, we may assume without loss of generality that all cubes in the decomposition corresponding to \eqref{HausdorffBound5} are hyperdyadic.

	For $k_2 \geq k_1$, we let
	%
	\[ Y_{k_1,k_2} = \bigcup \{ Q_{k_1,i} : l(Q_{k_1,i}) = l_{k_2}. \} \]
	%
	Note that $Y_{k_1,k_2}$ is the union of $O((1/l_{k_2})^{s + C\varepsilon})$ cubes in $\DQ_{k_2}$. We let $Z_{k_1,k_2}$ be the collection of hyperdyadic cubes covering $Y_{k_1,k_2}$ which minimize
	%
	\[ \sum \left\{ l(Q)^s : Q \in Z_{k_1,k_2} \right\} \]
	%
	and such that $l(Q) \geq l_{k_2}$ for each $Q$. Then clearly
	%
	\[ \sum_{Q \in Z_{k_1,k_2}} l(Q)^s \lesssim l_{k_2}^{- C\varepsilon}. \]
	%
	In particular, this means $l(Q) \lesssim l_{k_2}^{-C\varepsilon/s}$ for each $Q \in Z_{k_1,k_2}$. Moreover, for each hyperdyadic $Q_0$ with $l(Q_0) \geq l_{k_2}$,
	%
	\[ \sum_{Q \subset Q_0} l(Q)^s \leq l(Q_0)^s. \]
	%
	Now we define $E_k = \bigcup \left\{ \bigcup_{k_1,k_2} Z_{k_1,k_2} \cap \DQ_k \right\}$. Since $E_k$ only contains cubes from $Z_{k_1,k_2}$ where $k_1 \leq k_2$, and $l_k \lesssim l_{k_2}^{-\varepsilon/\alpha}$ TODO: THERE IS SOMETHING WRONG HERE THIS DOESN'T IMPLY $\log(1/l_k)^2$?. But this means that there are only $O(\log(1/l_k)^2) = O((1 + \varepsilon)^{2k})$ such choices of $(k_1,k_2)$. But this means that
	%
	\[ |E_k| = \sum l(Q)^d = l_k^{d - s} \sum l(Q)^s \lesssim (1 + \varepsilon)^{2k} l_k^{d-s}, \]
	%
	which implies that $E_k$ is the union of at most $O((1 + \varepsilon)^{2k} l_k^{-s})$ cubes in $\DQ_k$.
\end{proof}
%% The following is a directive for TeXShop to indicate the main file
%%!TEX root = diss.tex

\chapter{Related Work}
\label{ch:RelatedWork}

\section{Keleti: A Translate Avoiding Set}

Keleti's two page paper constructs a full dimensional subset $X$ of $[0,1]$ such that $X$ intersects $t + X$ in at most one place for each nonzero real number $t$. If this is true, we say that $X$ \emph{avoids translates}. Malabika has adapted this technique to construct high dimensional subsets avoiding nontrivial solutions to differentiable functions. In this section, and in the sequel, we shall find it is most convenient to avoid certain configurations by expressing them in terms of an equation, whose properties we can then exploit. One feature of translation avoidance is that the problem is specified in terms of a linear equation.

\begin{lemma}
    Let $X$ be a set. Then $X$ avoids translates if and only if there do not exists values $x_1 < x_2 \leq x_3 < x_4$ in $X$ with $x_2 - x_1 = x_4 - x_3$.
\end{lemma}
\begin{proof}

    Suppose $(t + X) \cap X$ contains two points $a < b$. Without loss of generality, we may assume that $t > 0$. If $a \leq b - t$, then the equation
    %
    \[ a - (a - t) = t = b - (b - t) \]
    %
    satisfies the constraints, since $a - t < a \leq b - t < b$ are all elements of $X$. We also have
    %
    \[ (b - t) - (a - t) = b - a, \]
    %
    which satisfies the constraints if $a - t < b - t \leq a < b$. This covers all possible cases. Conversely, if there are $x_1 < x_2 \leq x_3 < x_4$ in $X$ with
    %
    \[ x_2 - x_1 = t = x_4 - x_3, \]
    %
    then $X + t$ contains $x_2 = x_1 + (x_2 - x_1)$ and $x_4 = x_3 + (x_4 - x_3)$.
\end{proof}

%\footnote{We always assume $L_n/L_{n+1}$ is an integer so that intervals in $\mathcal{B}(L_n)$ are either almost disjoint from intervals in $\mathcal{B}(L_{n+1})$ or contained completely within such an interval}

The basic, but fundamental idea to Keleti's technique is to introduce memory into Cantor set constructions. Keleti constructs a nested family of discrete sets $X_0 \supset X_1 \supset \dots$ converging to $X$, with each $X_k$ a union of disjoint intervals in $\mathcal{B}(l_k, \RR^d)$, for a decreasing sequence of lengths $\{ l_k \}$ converging to zero, to be chosen later, but with $10 l_{k+1} \mid l_k$. We initialize $X_0 = [0,1]$, and $l_0 = 1$. Furthermore, we consider a queue of intervals, initially just containining $[0,1]$. To construct the sequence $\{ X_k \}$, Keleti iteratively performs the following procedure:
%
\begin{algorithm}
    \begin{algorithmic}%[1]
        \caption{Construction of the Sets $X_N$}
        \State{Set $k = 0$}
        \MRepeat
            \State{Take off an interval $I$ from the front of the queue}

            \MForAll{\ $J \in \B(l_k, \RR^d)$ contained in $X_k$:}
                \State{Order the intervals in $\B(l_{k+1}, \RR^d)$ contained in $J$ as $J_0, J_1, \dots, J_N$}

                \State{{\bf If} $J \subset I$, add all intervals $J_i$ to $X_{k+1}$ with $i \equiv 0$ modulo 10}
                \State{{\bf Else} add all $J_i$ with $i \equiv 5$ modulo 10}
            \EndForAll
            \State{Add all intervals in $\mathcal{B}(l_{k+1})$ to the end of the queue}
            \State{Increase $k$ by 1}
        \EndRepeat   
    \end{algorithmic}
\end{algorithm}

Each iteration of the algorithm produces a new set $X_k$, and so leaving the algorithm to repeat infinitely produces a sequence $\{ X_k \}$ converging to a set $X$, which is translate avoiding.

\begin{lemma}
    The set $X$ is translate avoiding.
\end{lemma}
\begin{proof}
    If $X$ is not translate avoiding, there is $x_1 < x_2 \leq x_3 < x_4$ with $x_2 - x_1 = x_4 - x_3$. Since $l_k \to 0$, there is a suitably large integer $N$ such that $x_1$ is contained in an interval $I \in \mathcal{B}(l_N,\RR^d)$ not containing $x_2,x_3$, or $x_4$. At stage $N$ of the algorithm, the interval $I$ is added to the end of the queue, and at a much later stage $M$, the interval $I$ is retrieved. Find the startpoints $x_1^\circ, x_2^\circ$, $x_3^\circ, x_4^\circ \in l_M \mathbf{Z}$ to the intervals in $\mathcal{B}(l_M,\RR^d)$ containing $x_1$, $x_2$, $x_3$, and $x_4$. Then we can find $n$ and $m$ such that $x_4^\circ - x_3^\circ = (10n)l_M$, and $x_2^\circ - x_1^\circ = (10m + 5)l_M$. In particular, this means that $|(x_4^\circ - x_3^\circ) - (x_2^\circ - x_1^\circ)| \geq 5L_M$. But
    %
    \begin{align*}
        |(x_4^\circ - x_3^\circ) - (x_2^\circ - x_1^\circ)| &= |[(x_4^\circ - x_3^\circ) - (x_2^\circ - x_1^\circ)] - [(x_4 - x_3) - (x_2 - x_1)]|\\
        &\leq |x_1^\circ - x_1| + \dots + |x_4^\circ - x_4| \leq 4 L_M
    \end{align*}
    %
    which gives a contradiction.
\end{proof}

The algorithm shows that
%
\[ \B(l_k,X_k) = (l_{k-1}/10l_k) \B(l_{k-1},X_{k-1}). \]
%
Thus closing the recursive definition shows
%
\[ \B(l_k,X_k) = \frac{1}{10^k l_k}. \]
%
In particular, this means $|X_k| = 1/10^k$, so $X$ has measure zero irrespective of our parameters. Nonetheless, the canonical measure $\mu$ on $X$ defined with respect to the decomposition $\{ X_k \}$ satisfies $\mu(I) = 10^k l_k$ for all $I \in \B(l_k,X)$. If $10^k l_k^\varepsilon \lesssim_\varepsilon 1$ for all $\varepsilon$, then we can establish the bounds $\mu(I) \lesssim_\varepsilon l_k^{1-\varepsilon}$ for all $\varepsilon$. In particular this is true if we set $l_k = 1/10^{\lfloor k \log k \rfloor}$. And because this sequence is not too fast increasing, we can apply Theorem \ref{easyCoverTheorem} to show $\mu$ is a Frostman measure of dimension $1-\varepsilon$ for each $\varepsilon > 0$, so $X$ has full Hausdorff dimension.







\section{Fraser/Pramanik: Extending Keleti Translation to Smooth Configurations}

Inspired by Keleti's result, Pramanik and Fraser obtained a generalization of the queue method which allows one to find sets avoiding solutions to {\it any} smooth function satisfying suitably mild regularity conditions. To do this, rather than making a linear shift in one of the intervals we avoid as in Keleti's approach, one must use the smoothness properties of the function to find large segments of an interval avoiding solutions to another interval.

\begin{theorem}
    Let $f: \mathbf{R}^{d+1} \to \mathbf{R}$ be a $C^1$ function, and consider sets $T_0, \dots, T_d \subset [0,1]$, with each $T_n$ a union of intervals in $\B(1/M,[0,1])$, and such that $\partial_0 f$ is non-vanishing on $T_0 \times \dots \times T_d$. Then there exists a rational number $C$, and arbitrarily large integers $N \in M \mathbf{Z}$ for which there are $S_k \subset T_k$, each a union of cubes in $\B(C/N^d,T_k)$ with
    %
    \begin{enumerate}
        \item $f(x) \neq 0$ for $x \in S_0 \times \dots \times S_d$.

        \item If $k \neq 0$, $S_k$ contains an interval in $\B(C/N^d,I)$ for each $I \in \B(1/N,T_k)$.

        \item For at least a fraction $1 - 1/M$ of the cubes $I \in \B(1/N,T_0)$, $|S_0 \cap I| \geq C/N$.
    \end{enumerate}
\end{theorem}
\begin{proof}
    We begin by dividing the sets $T_1, \dots, T_d$ into length $1/N$ intervals, and let $S_n$ be defined by including a length $C_0/N^d$ segment, for some constant $C_0$ to be chosen later. Then once we fix $C_0$, the $S_n$ will satisfy property (ii) of the theorem. We define
    %
    \[ \mathbf{A} = \{ a \in \mathbf{R}^{d-1} : a_n\ \text{is a startpoint of a length $1/N$ interval in}\ T_n \} \]
    %
    Then $|\mathbf{A}| \leq N^d$, since each interval $T_n$ is contained in $[0,1]$, and therefore can only contain at most $N$ almost disjoint intervals of length $1/N$. Hence if we define the set of `bad points' in $T_0$ as
    %
    \[ \mathbf{B} = \{ x \in T_0: \text{there is}\ a \in \mathbf{A}\ \text{such that}\ f(x,a) = 0 \} \]
    %
    Then $|\mathbf{B}| \leq MN^d$. This is because for each fixed $a$, the function $x \mapsto f(x,a)$ is either strictly increasing or decreasing over each interval in the decomposition of $T_0$, or which there are at most $M$ because $T_0 \subset [0,1]$. If we split $T_0$ into length $1/N$ intervals, and choose a subcollection of such intervals $I$ such that $|I \cap \mathbf{B}| \leq M^3N^{d-1}$, then we throw away at most $MN^d/M^3N^{d-1} = N/M^2$ intervals, and so we keep $(N/M)(1 - 1/M)$ intervals, which is $1 - 1/M$ of the total number of intervals in the decomposition of $T_0$. The lemma we prove after this theorem implies that there exists a constant $C_1$ such that if $x \in S_n$, and $f(y,x) = 0$, then $d(y,\mathbf{B}) \leq C_0C_1/N^d$. If we split each interval $I$ with $|I \cap \mathbf{B}| \leq M^3N^{d-1}$ into $4M^3N^{d-1}$ length $1/4M^3N^d$ intervals, and we choose $C_0$ such that $C_0C_1 < 1/4M^3$, then the set $S_0$ obtained by discarding each interval that contains or is adjacent to an interval containing an element of $\mathbf{B}$ satisfies $d(S_0,\mathbf{B}) > C_0C_1/N^d$, and therefore there does not exist any $x_n \in S_n$ and $y \in S_0$ such that $f(y,x) = 0$. $S_0$ satisfies property (iii) of the theorem since for the interval $I$ we are considering, we keep at least $M^3N^{d-1}$ length $1/4M^3N^d$ intervals, which in total has length at least $1/4N$.
\end{proof}

\begin{remark}
    The length $1/N$ portion of each interval guaranteed by (iii) is unneccesary to the Hausdorff dimension bound, since the slightly better bounds obtained on scales where an interval is dissected as a $1/N$ are decimated when we eventually divide the further subintervals into $1/N^{d-1}$ intervals. The importance of (iii) is that it implies that the set we will construct has full {\it Minkowski dimension}. The reason for this is that Minkowski dimension lacks the ability to look at varying dissection depths at once, and since, at any particular depth, there exists a length $1/N$ dissection, the process appears to Minkowski to be full dimensional, even though at later scales this $1/N$ dissection is dissected into $1/N^{d-1}$ intervals.
\end{remark}

\begin{lemma}
    Given the $f$, $T_0, \dots, T_d$, there exists a constant $C_1$ depending on these quantities, such that for any $C_0$, and $x \in S_1 \times \dots \times S_{d-1}$, if $f(y,x) = 0$, then $d(y, \mathbf{B}) \leq C_0C_1/N^{d-1}$.
\end{lemma}
\begin{proof}
    Since $T_0 \times \dots \times T_d$ breaks into finitely many cubes with sidelengths $1/M$, it suffices to prove the theorem for a particular cube $J$ in this decomposition, where we assume the zeroset of $f$ intersects $J$. If $J = I \times J'$, where $I$ is an interval, we let $U$ be the set of all $x \in J'$ for which there is $y$ in the interior of $I$ such that $f(y,x) = 0$. Then $U$ is open. The implicit function theorem implies that there exists a $C^1$ function $g: U \to I$ such that $f(x,y) = 0$ if and only if $y = g(x)$. Then the function $h(x) = f(x,g(x))$ vanishes uniformly, so
    %
    \[ 0 = \partial_n h(x) = (\partial_n f) (g(x),x) + (\partial_0 f) (g(x),x) \partial_n g(x) \]
    %
    Hence for $x \in U$,
    %
    \[ |(\nabla g)(x)| = \frac{|(\nabla f)(x)|}{|(\partial_d f)(x,g(x))|} \leq \frac{B}{A} \]
    %
    If $N$ is chosen large enough, then for every $x \in U \cap (S_1 \times \dots \times S_d)$ there is $a \in \mathbf{A} \cap U$ in the same connected component of $U$ as $x$ with $|x - a| \lesssim C_0/N^{d-1}$, and this means that
    %
    \[ |g(x) - g(a)| \leq \| \nabla g \|_\infty |x - a| \lesssim \frac{BC_0}{A N^{d-1}} \]
    %
    and $g(a) \in \mathbf{B}$, completing the proof.
\end{proof}

How do we use this lemma to construct a set avoiding solutions to $f$? We form an infinite queue which will eventually filter out all the possible zeroes of the equation. Divide the interval $[0,1]$ into $d$ intervals, and consider all orderings of $d - 1$ subsets of these intervals, and add them to the queue. Now on each iteration $N$ of the algorithm, we have a set $X_N \subset [0,1]$. We take a particular sequence of intervals $T_1, \dots, T_d$ from the queue, and then use the lemma above to dissect the $X_N \cap T_n$, which are unions of intervals, into sets avoiding solutions to the equation, and describe the remaining points as $X_{N+1}$. We then add all possible orderings of $d$ intervals created into the end of the queue, and rinse and repeat. The set $X = \lim X_n$ then avoids all solutions to the equation with distinct inputs.

What remains is to bound the Hausdorff dimension of $X$ by constructing a probability measure supported on $X$ with suitable decay. To construct our probability measure, we begin with a uniform measure on the interval, and then, whenever our interval is refined, we uniformly distribute the volume on that particular interval uniformly over the new refinement. Let $\mu$ denote the weak limit of this sequence of probability distributions. At each step $n$ of the process, we let $1/M_n$ denote the size of the intervals at the beginning of the $n$'th subdivision, $1/N_n$ denote the size of the split intervals in the lemma, and $C_n$ the $n$'th constant. We have the relation $1/M_{n+1} = C_n/N_n^{d-1}$. If $K$ is a length $1/M_{N+1}$ interval, $J$ a length $1/N_N$ interval, and $I$ a length $1/M_N$ interval with $K \subset J \subset I$ and all recieving some mass in $\mu$. To calculate a bound on their mass, we consider the decompositions considered in the algorithm:
%
\begin{itemize}
        \item If $J$ is subdivided in the non-specialized manner, then every length $1/N_N$ interval recieves the same mass, which is allocated to a single length $1/M_{N+1}$ interval it contains. Thus $\mu(K) = \mu(J) \leq (M_N/N_N) \mu(I)$.
        \item In the second case, at least a fraction $1 - 1/M_N$ of the length $1/N_N$ intervals are assigned mass, so $\mu(J) = (M_N/N_N)(1 - 1/M_N)^{-1} \leq (2M_N/N_N) \mu(I)$, and more than $C_N/N_N$ of each length $1/N_N$ interval is maintained, so
        %
        \[ \mu(K) = \frac{N_N}{C_NM_{N+1}} \mu(J) \leq \frac{2M_N}{C_NM_{N+1}} \mu(I) \]
\end{itemize}
%
Thus in both cases, we have $\mu(J) \lesssim (N_N/M_N) \mu(I)$, $\mu(K) \lesssim_N |K|$, and $N_N = M_{N+1}^{1/(d-1)}/C_N \lesssim_N M_{N+1}^{1/(d-1)}$. From this, we conclude using the results of the appendix that there exists a family of rapidly decaying parameters which gives a $1/(d-1)$ dimensional set.

\begin{remark}
    The set $X$ constructed is precisely a $1/(d-1)$ dimensional set. Recall that $X = \lim X_n$, where $X_n$ is a union of a certain number of length $1/M_n$ intervals $I_1, \dots, I_N$. For each $n$, the interval $I_i$ is inevitably subdivided at a stage $J_i$ into length $C_{J_i} N_{J_i}^{1-d}$ intervals for each length $1/N_{J_i}$ interval that $I_i$ contains. Thus
    %
    \[ H_{1/M_n}^\alpha(X) \leq \sum_{i = 1}^N \frac{N_{m_i}}{M_n} (C_{m_i} N_{m_i}^{1-d})^\alpha = \frac{1}{M_n} \sum_{i = 1}^N C_{m_i}^\alpha N_{m_i}^{1 - \alpha(d-1)} \]
    %
    We may assume that $C_{m_i} \leq 1$, so if $\alpha > 1/(d - 1)$, using the fact that $N \leq M_n$, since $X_n$ is contained in $[0,1]$, we obtain
    %
    \[ H_{1/M_n}^\alpha(X) \leq \frac{1}{M_n} \sum_{i = 1}^N N_{m_i}^{1 - \alpha(d-1)} \leq N_{\max(m_i)}^{1 - \alpha(d-1)} \leq 1 \]
    %
    Thus, taking $n \to \infty$, we conclude $H^\alpha(X) \leq 1 < \infty$, so as $\alpha \downarrow 1/(d - 1)$, we conclude that $X$ has Hausdorff dimension bounded above by $1/(d-1)$.
\end{remark}

%Thus, in both cases, we have $\mu(J) \lesssim (N_N/M_N) \mu(I)$, which means we can apply the second method of appendix to calculate Hausdorff dimension with rapidly growing constants, where $l_N = 1/M_N$ and $r_N = 1/N_N$. We have $\mu(K) \lesssim_N $ and $N_N = M_{N+1}^{1/(d-1)}/C_N$ and


%
%Thus, in both cases, we have $\mu(J) \lesssim_N 1/M_{N+1}$. If $J \subset I$ is any length $1/N_N$ interval considered in the algorithm, then either $\mu(J) = (M_N/N_N) \mu(I)$, as in the first case of the subdivision, or we can apply the second case of the subdivision, giving $\mu(J) = (M_N/N_N)(1 - 1/M_N)^{-1} \mu(I) \leq (2M_N/N_N) \mu(I)$. This means we can apply the second method in the appendix. The fact that 

%by induction, if $I$ is a length $1/M_N$ interval considered in the process, then
%
%\begin{align*}
%    \mu(I) \leq \prod_{n < N} \frac{M_n}{(C_n M_{n+1})^{\frac{1}{d-1}}} = \left( \prod_{n < N} \frac{M_{n+1}^{1-\frac{1}{d-1} }}{C_n^{\frac{1}{d-1}}} \right) \frac{1}{M_N} = \frac{A_N}{M_N}
%\end{align*}
%
%If $J \subset I$ is any length $1/N_N$ interval considered in the algorithm, then either $\mu(J) = (M_N/N_N) \mu(I)$, as in the first case, or in the second case, $\mu(J) = (M_N/N_N(1 - 1/M_N)) \leq 2M_N/N_N \mu(I)$, so in general $\mu(J) \leq 2A_N/N_N$. This means we can apply the second method in the appendix for bounding Hausdorff dimension, with $l_N = 1/M_N$ and $r_N = 1/N_N$. To obtain 

%Now if $1/N_N \leq |I| \leq 1/M_N$, then $I$ can be covered by $|I|N_N$ intervals of length $1/N_N$, and so
%
%\begin{align*}
%    \mu(I) &\leq 2|I|N_N \frac{A_N}{N_N} = 2A_N|I| = \frac{2A_{N-1} M_N^{1 - \frac{1}{d-1}}}{C_{N-1}^{\frac{1}{d-1}}} |I| \lesssim_\varepsilon |I|M_N^{1 - \frac{1}{d-1} - \varepsilon} \leq |I|^{\frac{1}{d-1} - \varepsilon}
%\end{align*}
%
%Provided that we can choose $M_N$ such that $A_N/C_N \lesssim_\varepsilon M_{N+1}^\varepsilon$ for all $\varepsilon$ (this is why it is incredibly important that the values in the lemma are independent of $N$ in the proof above). On the other hand, if $1/M_{N+1} \leq |I| \leq 1/N_N$, then $I$ can be covered by a single length $1/N_N$ interval, hence
%
%\[ \mu(I) \leq \frac{2A_N}{N_N} = \frac{2A_N}{N_N} = \frac{2A_N}{(C_NM_{N+1})^{\frac{1}{d-1}}} \lesssim_\varepsilon \frac{1}{M_{N+1}^{\frac{1}{d-1} - \varepsilon}} \leq |I|^{\frac{1}{d-1} - \varepsilon} \]
%
%Thus we obtain the theorem if $M_{N+1} = \exp(A_N/C_N)$, for instance.

\section{A Set Avoiding All Functions With A Common Derivative}

In the latter part's of their paper, Pramanik and Fraser apply an iterative technique to construct, for each $\alpha$ with $\sum \alpha_n = 0$ and $K > 0$, a set $E$ of positive Hausdorff dimension avoiding solutions to any function $f: \mathbf{R}^d \to \mathbf{R}$ satisfying wth $(\partial_n f)(0) = \alpha_n$,
%
\[ \left| f(x) - \sum \alpha_n x_n \right| \leq K \sum_{n \neq 1} (x_n - x_1)^2 \]
%
The set of such $f$ is an uncountable family, which makes this situation interesting. The technique to create such a set relies on another iterative procedure.

\begin{lemma}
    Let $I \subsetneq [1,d]$ be a strict subset of indices, and $\delta_0 > 0$. Then there exists $\varepsilon > 0$ such that for any $\lambda > 0$ and two disjoint intervals $J_1$ and $J_2$, with $J_1$ occuring before $J_2$, and if we set
    %
    \[ [a_n,b_n] = \begin{cases} J_1 & n \in I \\ J_2 & n \not \in I \end{cases} \]
    %
    then for $\delta < \delta_0$, either for all $x_n \in [a_n,a_n+\varepsilon \lambda]$ or for all $x_n \in [b_n - \varepsilon \lambda, b_n]$,
    %
    \[ \left| \sum \alpha_n x_n \right| \geq \delta \lambda \]
\end{lemma}
\begin{proof}
    If $C^* = \sum |\alpha_n|$, then for $|x_n - a_n| \leq \varepsilon \lambda$,
    %
    \[ |\sum \alpha_n (x_n - a_n)| \leq C^* \varepsilon \lambda \]
    %
    Thus if $|\sum \alpha_n a_n| > (\delta + \varepsilon C^*)\lambda$, then $|\sum \alpha_n x_n| \geq \delta \lambda$. If this does not occur
\end{proof}

\endinput
	%% The following is a directive for TeXShop to indicate the main file
%%!TEX root = diss.tex

\chapter{Avoiding Rough Sets}
\label{ch:RoughSets}

In the previous chapter, we saw that many authors have considered the pattern avoidance problem for configurations $\C$ which take the form of many general classes of smooth shapes; in Math\'{e}'s work, $\C$ can take the form of an algebraic variety of low degree, and in Pramanik and Fraser's work, $\C$ can take the form of a smooth manifold. In this chapter, we consider the pattern avoidance problem for an even more general class of `rough' patterns, that are the countable union of sets with controlled lower Minkowski dimension.
%
\begin{theorem}\label{mainTheorem}
	Let $s \geq d$, and suppose $\C \subset \C^n(\RR^d)$ is the countable union of precompact sets, each with lower Minkowski dimension at most $s$. Then there exists a set $X \subset [0,1]^d$ with Hausdorff dimension at least $(nd - s)/(n-1)$ avoiding $\C$.
\end{theorem}

\begin{remarks}
	\
	\begin{enumerate}
		\item[1.] When $s < d$, avoiding the configuration $\C$ is trivial. If we define $\pi: \C^n(\RR^d) \to \RR^d$ by $\pi(x_1, \dots, x_n) = x_1$, then the set $X = [0,1]^d - \pi(\C)$ is full dimension and avoids $\C$. Note that obtaining a full dimensional set in the case $s = d$, however, is still interesting.

		\item[2.] Theorem \ref{mainTheorem} is trivial when $s = dn$, since we can set $X = \emptyset$. We will therefore assume that $s < dn$ in our proof of the theorem.

		\item[3.] Let $f: \RR^{dn} \to \RR^m$ be a $C^1$ map such that $f$ has full rank at any point $(x_1, \dots, x_n) \in \C^n(\RR^d)$ with $f(x_1, \dots, x_n) = 0$. If we set
		%
		\[ \C = \{ x \in \C^n(\RR^d) : f(x) = 0 \}, \]
		%
		Then $\C$ is a $C^1$ submanifold of $\C^n(\RR^d)$ of dimension $nd - m$. A submanifold of Euclidean space is $\sigma$ compact, so we can write $\C = \bigcup K_i$, where each $K_i$ is a compact set. Applying Theorem \ref{ManifoldDimensionThm} shows $\lowminkdim(K_i) \leq nd - m$ for each $i$, so we can apply Theorem \ref{mainTheorem} with $s = nd - m$ to yield a set in $\RR^d$ with Hausdorff dimension at least
		%
		\[ \frac{nd - s}{n-1} = \frac{m}{n-1}. \]
		%
		This recovers Theorem \ref{pramanikandfrasertheorem}, making Theorem \ref{mainTheorem} a generalization of Pramanik and Fraser's result.

		\item[4.] Since Theorem \ref{mainTheorem} does not require any regularity assumptions on the set $\C$, it can be applied in contexts that cannot be addressed using previous methods in the literature. Two such applications, new to the best of our knowledge, have been recorded in Chapter \ref{ch:Applications}; see Theorems \ref{sumset-application} and \ref{C1IsoscelesThm} there.
	\end{enumerate}
\end{remarks}

Like with the results considered in the last chapter, we construct the set $X$ in Theorem \ref{mainTheorem} by repeatedly applying a discrete avoidance result at the scales corresponding to cubes $\DQ^d$, constructing a Frostman measure with equal mass at intermediary scales, and applying Lemma \ref{uniformMassFrostman}. However, our method has several innovations that simplify the analysis of the resulting set $X = \bigcap X_k$ than from previous results. In particular, through a probabilistic selection process we are able to use a simplified queuing technique then that used in \cite{KeletiDimOneSet} and \cite{MalabikaRob}, that required storage of data from each step of the iterated construction to be retrieved at a much later stage of the construction process.

%The details of a single step of this construction are described in Section \ref{discretesection}. In Section \ref{discretizationsection}, we explain how the branching factors $\{ N_k \}$ must be chosen, complete the construction of $X$, and prove $X$ avoids the configuration $\C$. In Section \ref{dimensionsection} we analyze the size of $X$ and show it has the Hausdorff dimension guaranteed by the conclusion of Theorem \ref{mainTheorem}.






\section{Avoidance at Discrete Scales}\label{discretesection}

In this section we describe a method for avoiding a discretized version of $\C$ at a single scale. We apply this technique in Section \ref{discretizationsection} at many scales to construct a set $X$ avoiding $\C$ at all scales.
%This single scale avoidance technique is the core building block of our construction, and the efficiency with which we can avoid $\C$ at a single scale has direct consequences on the Hausdorff dimension of the set $X$ obtained in Theorem \ref{mainTheorem}.
In the discrete setting, $\C$ is replaced by a union of cubes in $\DQ^{dn}_{k+1}$ denoted by $B$. We say a cube $Q = Q_1 \times \dots \times Q_n \in \DQ^{dn}_{k+1}$ is \emph{strongly non-diagonal} if the $n$ cubes $Q_1, \dots, Q_n$ are distinct. Given a $\DQ_k$ discretized set $T \subset \RR^d$, our goal is to construct a $\DQ_{k+1}$ discretized set $S \subset T$, such that $\DQ_{k+1}^{dn}(F^n)$ does not contain any strongly non-diagonal cubes of $\DQ_{k+1}^{dn}(B)$.

%To ensure the set $X$ obtained in Theorem \ref{mainTheorem} has large Hausdorff dimension regardless of the rapid decay of scales used in the construction of $X$, it is crucial that $F$ is uniformly distributed among $\DR_{k+1}^d(T)$ so that we can apply Lemma \ref{uniformMassFrostman}. This is what Lemma \ref{discretelemma} achieves, given that $N_{k+1}$ is suitably large in comparsion to $M_{k+1}$, i.e. so that \eqref{rBound} holds.

\begin{lemma} \label{discretelemma}
	Fix $k$, $s \in [1,dn)$, and $\varepsilon \in [0,(dn-s)/2)$. Let $T \subset \RR^d$ be a nonempty, $\DQ_k$ discretized set, and let $B \subset \RR^{dn}$ be a nonempty $\DQ_{k+1}$ discretized set such that
	%
	\[ \#(\DQ_{k+1}(B)) \leq N_{k+1}^{s + \varepsilon}. \]
	%
	Then there exists a constant $C(s,d,n) > 0$%\footnote{The proof will show that we can choose $C(s,d,n) \sim_d 4^{\frac{1}{dn - s}}$, which explodes as $s \to dn$.}
	, depending only on $s$, $d$, and $n$, such that, provided
	% C(s,d,n) \geq 4d, 2^{\frac{n+1}{dn - t}}
	\begin{equation} \label{rBound}
		N_{k+1} \geq C(s,d,n) \cdot M_{k+1}^{\frac{d(n-1)}{dn - s - \varepsilon}},
	\end{equation}
	%
	then there is a $\DQ_{k+1}$ discretized set $S \subset T$ satisfying the following three properties:
	%
	\begin{enumerate}
		\item\label{avoidanceItem} For any collection of $n$ distinct cubes $Q_1, \dots, Q_n \in \DQ_{k+1}(S)$,
		%
		\[ Q_1 \times \dots \times Q_n \not \in \DQ_{k+1}(B). \]

		\item\label{nonConcentrationItem} For each $Q \in \DQ_k(T)$, there exists $\mathcal{R}_Q \subset \DR_{k+1}(Q)$ such that
		%
		\[ \#(\mathcal{R}_Q) \geq \frac{\#(\DR_{k+1}(Q))}{2}, \]
		%
		and if $R \in \DR_{k+1}(Q)$,
		%
		\[ \#(\DQ_{k+1}(R \cap S)) = \begin{cases} 1 & : R \in \mathcal{R}_Q \\ 0 & : R \not \in \mathcal{R}_Q. \end{cases} \]
	\end{enumerate}
\end{lemma}

\begin{proof}
	For each $R \in \DR_{k+1}(T)$, pick $Q_R$ uniformly at random from $\DQ_{k+1}(R)$; these choices are independent as $R$ ranges over $\DR_{k+1}(T)$. Define
	%
	\[ A = \bigcup \left\{ Q_R \setcolon R \in \DR_{k+1}(T) \right\}, \]
	%
	and
	%
	\[ \mathcal{K}(A) = \{ K \in \DQ_{k+1}(B) \cap \DQ_{k+1}(A^n) \setcolon \text{$K$ strongly non-diagonal} \}. \]
	%
	The sets $A$ and $\mathcal{K}(A)$ are random, in the sense that they depend on the random variables $\{ Q_R \}$. Define
	%
	\begin{equation} \label{defnOfF}
		S(A) = \bigcup \Big[ \DQ_{k+1}(A) - \{ \pi(K) \setcolon K \in \mathcal{K}(A) \} \Big],
	\end{equation}
	%
	where $\pi \colon \RR^{dn} \to \RR^d$ is the projection map $(x_1, \dots, x_n) \mapsto x_1$, for $x_i \in \RR^d$.

	Given any strongly non-diagonal cube $K = K_1 \times \cdots \times K_n \in \DQ_{k+1}(B)$, either $K \not \in \DQ_{k+1}(A^n)$, or $K \in \DQ_{k+1}(A^n)$. If the former occurs then $K \not \in \DQ_{k+1}(S(A))$ since $S(A) \subset A$, so $\DQ_{k+1}(S(A)^n) \subset \DQ_{k+1}(A^n)$. If the latter occurs then $K \in \mathcal{K}(A)$, and since $\pi(K) = K_1$, $K_1 \not \in \DQ_{k+1}(S(A))$. In either case, $K \not \in \DQ_{k+1}(S(A)^n)$, so $S(A)$ satisfies Property \ref{avoidanceItem}. By construction, we know that for each $R \in \DR_{k+1}(T)$,
	%
	\begin{equation} \label{equation8423246093490} \#(\DQ_{k+1}(S(A) \cap R)) \leq \#(\DQ_{k+1}(A \cap R)) = 1. \end{equation}
	%
	To conclude that $S(A)$ satisfies Property \ref{nonConcentrationItem}, it therefore suffices to show that
	%
	\[ \# (\mathcal{K}(A)) \leq (1/2) \cdot M_{k+1}^d. \]
	%
	We now show this is true with non-zero probability.

	For each cube $Q \in \DQ_{k+1}(T)$, there is a unique `parent' cube $R \in \DR_{k+1}(T)$ such that $Q \subset R$. Note that \eqref{rBound} implies, if $C(s,d,n) \geq 4d$, that $N_{k+1} \geq 4d \cdot M_{k+1}$. Since $Q_R$ is chosen uniformly from $\DQ_{k+1}(R)$, for any $Q \in \DQ_{k+1}(R)$,
	%
	\[ \prob(Q \subset A) = \prob(Q_R = Q) = \left( \DQ_{k+1}(R) \right)^{-1} = (M_{k+1}/N_{k+1})^d. \]
	%
	Suppose $K_1, \dots, K_n \in \DQ_{k+1}(T)$ are distinct cubes. If the cubes have distinct parents in $\DR_{k+1}^d$, we can apply the independence of the random cubes $\{ Q_R \}$ to conclude that
	%
	\[ \prob(K_1, \dots, K_n \subset A) = (M_{k+1}/N_{k+1})^{dn}. \]
	%
	If the cubes $K_1, \dots, K_n$ do not have distinct parents, \eqref{equation8423246093490} shows
	%
	\[ \prob(K_1, \dots, K_n \subset A) = 0. \]
	%
	In either case, we conclude that
	%
	\begin{equation}\label{jointprob}
	\prob(K_1, \dots, K_n \subset A) \leq (M_{k+1}/N_{k+1})^{dn}.
	\end{equation}
	%
	Let $K = K_1 \times \dots \times K_n$ be a strongly non-diagonal cube in $\DQ_{k+1}(B)$. We deduce from \eqref{jointprob} that
	%
	\begin{equation}\label{probaKSubsetUn}
		\prob(K \subset A^n) = \prob(K_1, \dots, K_n \subset A) \leq (M_{k+1}/N_{k+1})^{dn}.
	\end{equation}
	%
	If
	%
	\[ C(s,d,n) \geq 4^{\frac{1}{dn - s}} \geq 2^{\frac{1}{dn - s - \varepsilon}}, \]
	%
	by \eqref{probaKSubsetUn}, linearity of expectation, and \eqref{rBound}, we conclude
	%
	\begin{align*}
		\expect(\#(\mathcal{K}(A))) &= \sum_{K \in \DQ_{k+1}(B)} \prob(K \subset A^n)\\
		&\leq \#(\DQ_{k+1}(B)) \cdot (M_{k+1}/N_{k+1})^{dn}\\
		&\leq M_{k+1}^{dn} / N_{k+1}^{dn - s - \varepsilon}\\
		&\leq (1/2) \cdot M_{k+1}^d.
	\end{align*}
	%
	In particular, there exists at least one (non-random) set $A_0$ such that
	%
	\begin{equation}\label{KU0Small}
		\#(\mathcal{K}(A_0)) \leq \expect(\# (\mathcal{K}(A))) \leq (1/2) \cdot M_{k+1}^d.
	\end{equation}
	%
	In other words, $S(A_0) \subset A_0$ is obtained by removing at most $(1/2) \cdot M_{k+1}^d$ cubes in $\B^d_s$ from $A_0$. For each $Q \in \B_l^d(T)$, we know that $\# \DQ_{k+1}(Q \cap A_0) = M_{k+1}^d$. Combining this with \eqref{KU0Small}, we arrive at the estimate 
	%
	\begin{align*}
		\# \DQ_{k+1}(Q \cap S(A_0)) &= \DQ_{k+1}(Q \cap A_0) - \# \{ \pi(K) \setcolon K \in \mathcal{K}(A_0), \pi(K) \in A_0 \}\\
		&\geq \DQ_{k+1}(Q \cap A_0) - \#(\mathcal{K}(A_0))\\
		&\geq M_{k+1}^d - (1/2) \cdot M_{k+1}^d \geq (1/2) \cdot M_{k+1}^d
	\end{align*}  
	%
	Thus, $S(A_0)$ satisfies Property \ref{nonConcentrationItem}. Setting $S = S(A_0)$ completes the proof.
\end{proof}

\begin{remarks}
	\
	\begin{enumerate}
		\item[1.] While Lemma \ref{discretelemma} uses probabilistic arguments, the proof of the lemma is still constructive. In particular, one can find a suitable $S$ constructively by checking every possible choice of $A$ (there are finitely many) to find one particular choice $A_0$ which satisfies \eqref{KU0Small}, and then defining $S$ by \eqref{defnOfF}. Thus the set we obtain in Theorem \ref{mainTheorem} exists by purely constructive means.
		
		\item[2.] As with the proofs in Chapter \ref{ch:RelatedWork}, the fact that we can choose
		%
		\[ N_{k+1} \sim M_{k+1}^{(1 - \varepsilon)/t} \]
		%
		where $t = (dn - s)/d(n-1)$, implies we should be able to iteratively apply Lemma \ref{discretelemma} to obtain a set with Hausdorff dimension $(dn - s)/(n-1)$.
	\end{enumerate} 
\end{remarks}









\section{Fractal Discretization}\label{discretizationsection}

In this section we construct the set $X$ from Theorem \ref{mainTheorem} by applying Lemma \ref{discretelemma} at many scales. Let us start by fixing a strong cover of $\C$, as well as the branching factors $\{ N_k \}$, that we will work with in the sequel.

\begin{lemma}\label{coveringLemma}
	Let $\C \subset \C^n(\RR^d)$ be a countable union of bounded sets with Minkowski dimension at most $s$, and let $\epsilon_k \searrow 0$ with $\epsilon_k < (dn - s)/2$ for all $k$. Then there exists a choice of branching factors $\{ N_k : k \geq 1 \}$, and a sequence of sets $\{ B_k : k \geq 1 \}$, with $B_k \subset \RR^{dn}$ for all $k$, such that
	%
	\begin{enumerate}
		\item\label{StrongCoverProperty} \emph{Strong Cover}: The interiors $\{ B_k^\circ \}$ of the sets $\{ B_k \}$ form a strong cover of $\C$.

		\item\label{DiscretenessProperty} \emph{Discreteness}: For all $k \geq 1$, $B_k$ is a $\DQ_k$ discretized subset of $\RR^{dn}$.

		\item\label{SparsityProperty} \emph{Sparsity}: For all $k \geq 1$, $\DQ_k(B_k) \leq N_k^{s+\epsilon_k}$.

		\item \label{RapidDecayProperty} \emph{Rapid Decay}: For all $\varepsilon > 0$ and $k \geq 1$, $N_k$ is a power of two such that
		%
		\[ N_1 \dots N_{k-1} \lesssim_\varepsilon N_k^\varepsilon, \]
		%
		and for all $k \geq 1$,
		%
		\[ N_k \geq C(s,d,n). \]
	\end{enumerate}
\end{lemma}
\begin{proof}
	We can write $\C = \bigcup_{i = 1}^\infty Y_i$, with $\lowminkdim(Y_i) \leq s$ for each $i$. Let $\{ i_k \}$ be a sequence of integers that repeats each integer infinitely often. The branching factors $\{ N_k \}$ and sets $\{ B_k \}$ are defined inductively. Suppose that the lengths $N_1, \ldots, N_{k-1}$ have been chosen. Since $\lowminkdim(Y_{i_k}) < s + \varepsilon_k$, there are arbitrarily small lengths $l \leq l_{k-1}$ such that if $N_k = l_{k-1}/l$,
	% N_k \geq (N_1 \dots N_{k-1})^{2s/\varepsilon_k}
	\begin{equation} \label{coveringOfBdnlZk}
		\# (\DQ_k(Y_{i_k}(l))) \leq (1/l)^{s + (\varepsilon_k/2)}.
	\end{equation}
	% l^{\varepsilon_k/2} \leq l_{k-1}^{s + \varepsilon_k}
	In particular, we can choose $l$ small enough that $l \leq l_{k-1}^{2s/\varepsilon + 2}$, which together with \eqref{coveringOfBdnlZk} implies
	%
	\begin{equation} \label{equation789891491}
		\# (\DQ_k(Y_{i_k})(l)) \leq N_k^{s + \varepsilon_k}.
	\end{equation}
	%
	We can certainly also choose $l$ small enough that
	%
	\[ N_k \geq \max(C(s,d,n),(N_1 \dots N_{k-1})^{1/\varepsilon_k}). \]
	%
	We know that $l_{k-1}$ is a power of two, so we can ensure $N_k$ is a power of two. We then set $l_k = l$. With this choice, Property \ref{RapidDecayProperty} is satisfied. And if $B_k = \bigcup \DQ_k(Y_{i_k}(l_k))$, then this choice of $B_k$ clearly satisfies Property \ref{DiscretenessProperty}. And Property \ref{SparsityProperty} is precisely Equation \eqref{equation789891491}.

	It remains to verify that the sets $\{ B_k^\circ \}$ strongly cover $\C$. Fix a point $z \in \C$. Then there exists an index $i$ such that $z \in Y_i$, and there is a subsequence $k_1, k_2, \dots$ such that $i_{k_j} = i$ for each $j$. But then $z \in Y_i \subset B_{i_{k_j}}^\circ$, so $z$ is contained in each of the sets $B_{i_{k_j}}^\circ$, and thus $z \in \limsup B_i^\circ$. Thus Property \ref{StrongCoverProperty} is proved.
\end{proof}

To construct $X$, we consider a nested, decreasing family of sets $\{ X_k \}$, where each $X_k$ is an $l_k$ discretized subset of $\RR^d$. We then set $X = \bigcap X_k$. The goal is to choose $X_k$ such that $X_k^n$ does not contain any {\it strongly non diagonal} cubes in $B_k$.

\begin{lemma} \label{stronglydiagonal}
	For each $k$, let $B_k \subset \RR^{dn}$ be a $\DQ_k$ discretized set, such that the interiors $\{ B_k^\circ \}$ strongly cover $\C$. For each index $k$, let $X_k$ be a $\DQ_k$ discretized set such that $\DQ_k(X_k^n) \cap \DQ_k(B_k)$ contains no strongly non diagonal cubes. If $X = \bigcap X_k$, then $X$ avoids $B$.
\end{lemma}
\begin{proof}
	Let $x = (x_1, \dots, x_n) \in \C$ be a point such that $x_1, \dots, x_n$ are distinct. Define
	%
	\[ \Delta = \{ (y_1, \dots, y_n) \in \RR^{dn} \setcolon \text{there exists $i \neq j$ such that $y_i = y_j$} \}. \]
	%
	Then $d(\Delta,x) > 0$, where $d$ is the Hausdorff distance between $\Delta$ and $x$. Since $\{ B_k \}$ strongly covers $\C$, there is a subsequence $\{ k_m \}$ such that $x \in B_{k_m}^\circ$ for every index $m$. Since $l_k$ converges to 0 and thus $l_{k_m}$ converges to $0$, if $m$ is sufficiently large then $\sqrt{dn} \cdot l_{k_m} < d(\Delta,x)$. Note that $\sqrt{dn} \cdot l_{k_m}$ is the diameter of a cube in $\DQ_{k_m}$. For such a choice of $m$, any cube $Q \in \DQ_{k_m}^d$ which contains $x$ is strongly non-diagonal. Thus $x \in B_{k_m}^\circ$. Since $X_{k_m}$ and $B_{k_m}$ share no cube which contains $x$, this implies $x \not \in X_{k_m}$. In particular, this means $x \not \in X^n$.
\end{proof}

All that remains is to apply the discrete lemma to choose the sequence $\{ X_k \}$. Given $\C$, apply Lemma \ref{coveringLemma} to choose a sequence $\{ N_k \}$ and a sequence $\{ B_k \}$. We recursively define the sequence $\{ X_k \}$. Set $X_0 = [0,1]^d$. Then, Property \ref{RapidDecayProperty} of Lemma \ref{coveringLemma} implies
%
\[ \#(\DQ_{k+1}(B_{k+1})) \leq N_{k+1}^{s + \varepsilon_k} \]
%
Let $M_{k+1}$ be the largest power of two smaller than
%
\begin{equation} \label{equation9249815935} \left( \frac{N_{k+1}}{C(s,d,n)} \right)^{\frac{dn - s -\varepsilon_k}{d(n-1)}} \end{equation}
%
Then $1 \leq M_{k+1} \leq N_{k+1}$, and \eqref{rBound} is satisfied. Set $B = B_{k+1}$, and $T = X_k$. Then we can apply Lemma \ref{discretelemma} to find a set $S$ satisfying all Properties of that Lemma, and set $X_{k+1} = S$.

Property \ref{avoidanceItem} of Lemma \ref{discretelemma} together with Lemma \ref{stronglydiagonal} implies that the set $X = \bigcap X_k$ avoids $\C$. Property \ref{nonConcentrationItem} of Lemma \ref{discretelemma} shows that the sequence $\{ X_k \}$ satisfies Property \ref{SingleSelection} of Theorem \ref{TheConstructionTheorem}. Property \ref{RapidDecayProperty} of Lemma \ref{coveringLemma} implies Property \ref{RapidDecrease} of Theorem \ref{TheConstructionTheorem} is satisfied. And by definition, i.e. the choice of $\{ M_k \}$ as given by \eqref{equation9249815935},
%
\begin{equation} \label{equation1403463468957983} N_k \leq M_k^{\frac{d(n-1)}{dn - s - \varepsilon_k}} \lesssim_\varepsilon M_k^{\frac{1 + \varepsilon}{t}} \end{equation}
%
where $t = (nd - s)/d(n-1)$. \eqref{equation1403463468957983} is a form of Property \ref{ChangeofScales} of Theorem \ref{TheConstructionTheorem}. Thus all assumptions of Theorem \ref{TheConstructionTheorem} are satisfied, and so we find $X$ has Hausdorff dimension $(nd - s)/(n-1)$, completing the proof of Theorem \ref{mainTheorem}.










\chapter{Applications} \label{ch:Applications}

As discussed in the introduction, Theorem \ref{mainTheorem} generalizes Theorems 1.1 and 1.2 from \cite{MalabikaRob}. In this chapter, we present two applications of Theorem \ref{mainTheorem} in settings where previous methods do not yield any results.

\section{Sum-sets avoiding specified sets}

\begin{theorem} \label{sumset-application} 
	Let $Y \subset \RR^d$ be a countable union of sets with lower Minkowski dimension at most $t$. Then there exists a set $X \subset \RR^d$ with Hausdorff dimension at least $d - t$ such that $X + X$ is disjoint from $Y$.
\end{theorem}
\begin{proof}
	Define $\C = \C_1 \cup \C_2$, where
	%
	\[ \C_1 = \{ (x,y) \setcolon x + y \in Y \} \quad \text{and} \quad \C_2 = \{ (x,y) \setcolon y \in Y/2 \}. \]
	%
	Since $Y$ is a countable union of sets with lower Minkowski dimension at most $t$, $\C$ is a countable union of sets with lower Minkowski dimension at most $d + t$. Applying Theorem \ref{mainTheorem} with $n = 2$ and $s = d + t$ produces a set $X \subset \RR^d$ with Hausdorff dimension $2d  - (d + t) = d - t$ such that $(x,y) \not \in \C$ for all $x,y \in X$ with $x \neq y$. We claim that $X+ X$ is disjoint from $Y$. To see this, first suppose $x, y \in X$, $x \neq y$. Since $X$ avoids $Z_1$, we conclude that $x + y \not \in Y$. Suppose now that $x = y \in X$. Since $X$ avoids $Z_2$, we deduce that $X \cap (Y/2) = \emptyset$, and thus for any $x \in X$, $x + x = 2x \not \in Y$. This completes the proof.
\end{proof}


\section{Subsets of Lipschitz curves avoiding isosceles triangles}

In \cite{MalabikaRob}, Fraser and the second author prove that there exists a set $S \subset [0,1]$ with dimension $\log_3 2$ such that for any simple $C^2$ curve $\gamma \colon [0,1] \to \RR^n$ with bounded non-vanishing curvature, $\gamma(S)$ does not contain the vertices of an isosceles triangle. Our method enables us to obtain a result that works for Lipschitz curves with small Lipschitz constants. The dimensional bound that we provide is slightly worse than \cite{MalabikaRob} ($1/2$ instead of $\log_3 2$), and the set we obtain only works for a single Lipschitz curve, not for many curves simultaneously.

\begin{theorem} \label{C1IsoscelesThm}
	Let $f \colon [0,1] \to \RR^{n-1}$ be Lipschitz with \[ \| f \|_{\text{Lip}}  := \sup \bigl\{|f(x) - f(y)|/|x-y| : x, y \in [0,1], x \ne y   \bigr\} < 1. \]  Then there is a set $X \subset [0,1]$ of Hausdorff dimension $1/2$ so that the set
	%
	\[ \{(t,f(t)) \setcolon t\in X\} \]
	%
	does not contain the vertices of an isosceles triangle.
\end{theorem}

\begin{corollary} \label{C1IsoscelesCor}
	Let $f\colon [0,1] \to \RR^{n-1}$ be $C^1$.  Then there is a set $X \subset [0,1]$ of Hausdorff dimension $1/2$ so that the set
	%
	\[ \{(t,f(t)) \setcolon t\in X\} \]
	%
	does not contain the vertices of an isosceles triangle.
\end{corollary} 
\begin{proof}[Proof of Corollary \ref{C1IsoscelesCor}]
	The graph of any $C^1$ function can be locally expressed, after possibly a translation and rotation, as the graph of a Lipschitz function with small Lipschitz constant. In particular, there exists an interval $I\subset[0,1]$ of positive length so that the graph of $f$ restricted to $I$, after being suitably translated and rotated, is the graph of a Lipschitz function $g\colon [0,1] \to \RR^{n-1}$ with Lipschitz constant at most $1/2$. Since isosceles triangles remain invariant under these transformations, the corollary is a consequence of Theorem \ref{C1IsoscelesThm}.  
\end{proof} 

\begin{proof}[Proof of Theorem \ref{C1IsoscelesThm}]
	Set
	%
	\begin{equation} \label{def-Z}
		\C = \left\{ (x_1,x_2,x_3) \in \C^3[0,1] \setcolon \; \begin{array}{c}
			\text{The points $p_j = (x_j,f(x_j))$ form}\\
			\text{the vertices of an isosceles triangle}
		\end{array} \right\}.
	\end{equation} 
	%
	In the next lemma, we show $\C$ has lower Minkowski dimension at most two. By Theorem \ref{mainTheorem}, there is a set $X \subset [0,1]$ of Hausdorff dimension $1/2$ so that $X$ avoids $\C$. This is precisely the statement that for each $x_1,x_2,x_3\in X$, the points $(x_1,f(x_1)),\ (x_2,f(x_2))$, and $(x_3,f(x_3))$ do not form the vertices of an isosceles triangle. 
\end{proof}

\begin{lemma}
	Let $f\colon [0,1] \to \RR^{n-1}$ be Lipschitz with $\| f \|_{\text{Lip}} < 1$. Then the set $\C$ given by \eqref{def-Z} satisfies $\upminkdim(\C) \leq 2$.
\end{lemma}
\begin{proof}
	First, notice that three points $p_1,p_2,p_3 \in \RR^n$ form an isosceles triangle, with $p_3$ as the apex, if and only if $p_3 \in H_{p_1,p_2}$, where
	%
	\begin{equation} \label{def-H}  H_{p_1,p_2} = \left\{ x \in \RR^n \setcolon \left( x - \frac{p_1 + p_2}{2} \right) \cdot (p_2 - p_1) = 0 \right\}. \end{equation} 
	%
	To prove $\C$ has Minkowski has dimension at most two, it suffices to show the set
	%
	\[ W = \left\{ x \in [0,1]^3 \setcolon p_3 = (x_3,f(x_3)) \in H_{p_1, p_2} \right\} \]
	%
	has upper Minkowski dimension at most 2. This is because $\C$ is covered by three copies of $W$, obtained by permuting coordinates. We work with the family of dyadic cubes $\DD^n$. To bound the upper Minkowski dimension of $W$, we prove the estimate
	%
	\begin{equation}\label{boundOnWCoveringNumber}
		\# \bigl(\mathcal \DD_k(W(1/2^k)) \bigr) \leq C k 4^k \quad \text{ for all } k \geq 1,  
	\end{equation}  
	%
	where $C$ is a constant independent of $k$. Then for any $\varepsilon > 0$, \eqref{boundOnWCoveringNumber} implies that for suitably large $k$,
	%
	\[ \# \bigl(\mathcal \DD_k(W(1/2^k)) \bigr) \leq 2^{(2 + \varepsilon)k}. \]
	%
	Since $\varepsilon$ was arbitrary, this shows $\upminkdim(W) \leq 2$.

	To establish \eqref{boundOnWCoveringNumber}, we write
	%
	\begin{equation}\label{deltaCoveringWSum}
		\# \bigl(\mathcal \DD_k(W(1/2^k)) \bigr) = \sum_{m = 0}^{2^k}\ \sum_{\substack{I_1, I_2 \in \DD_k[0,1]\\d(I_1,I_2) = m/2^k}} \# \left( \DD_k(W(1/2^k) \cap (I_1 \times I_2 \times [0,1]) \right).
	\end{equation}
	Our next task is to bound each of the summands in \eqref{deltaCoveringWSum}.
	 Let $I_1, I_2 \in \DD_k[0,1]$, and let $m = 2^k \cdot d(I_1,I_2)$. Let $x_1$ be the midpoint of $I_1$, and $x_2$ the midpoint of $I_2$. Let $(y_1,y_2,y_3) \in W \cap (I_1 \times I_2 \times [0,1])$. Then it follows from \eqref{def-H} that 
	%
	\[ \left( y_3 - \frac{y_1 + y_2}{2} \right) \cdot (y_2 - y_1) + \left( f(y_3) - \frac{f(y_2) + f(y_1)}{2} \right) \cdot (f(y_2) - f(y_1)) = 0. \]
	%
	We know $|x_1 - y_1|, |x_2 - y_2| \leq 1/2^{k+1}$, so
	%
	\begin{align} \label{xyDiff}
		&\left| \left( y_3 - \frac{y_1 + y_2}{2} \right) (y_2 - y_1) - \left( y_3 - \frac{x_1 + x_2}{2} \right) (x_2 - x_1) \right| \nonumber\\
		&\ \ \ \ \ \leq \frac{|y_1 - x_1| + |y_2 - x_2|}{2} |y_2 - y_1| + \Big( |y_1 - x_1| + |y_2 - x_2| \Big) \left| y_3 - \frac{x_1 + x_2}{2} \right|\\
		&\ \ \ \ \ \leq (1/2^{k+1}) \cdot 1 + (1/2^k) \cdot 1 \leq 3/2^{k+1}. \nonumber
	\end{align}
	%
	Conversely, $|f(x_1) - f(y_1)|, |f(x_2) - f(y_2)| \leq 1/2^{k+1}$ because $\| f \|_{\text{Lip}} \leq 1$, and a similar calculation yields
	%
	\begin{align} \label{fnDiff}
	\begin{split}
		&\Big| \left( f(y_3) - \frac{f(y_1) + f(y_2)}{2} \right) \cdot (f(y_2) - f(y_1))\\
		&\ \ \ \ \ - \left( f(y_3) - \frac{f(x_1) + f(x_2)}{2} \right) \cdot (f(x_2) - f(x_1)) \Big|\leq 3/2^{k+1}.
	\end{split}
	\end{align}
	%
	Putting \eqref{xyDiff} and \eqref{fnDiff} together, we conclude that
	%
	\begin{align} \label{hyperplanethick}
	\begin{split}
		&\Big| \left( y_3 - \frac{x_1 + x_2}{2} \right) (x_2 - x_1)\\
		&\ \ \ \ \ + \left( f(y_3) - \frac{f(x_2) + f(x_1)}{2} \right) \cdot (f(x_2) - f(x_1)) \Big| \leq 3/2^k.
	\end{split}
	\end{align}
	%
	Since $|(x_2-x_1,f(x_2)-f(x_1))| \geq |x_2-x_1| \geq m/2^k$, we can interpret \eqref{hyperplanethick} as saying the point $(y_3, f(y_3))$ is contained in a $3/k$ thickening of the hyperplane $H_{(x_1,f(x_1)), (x_2,f(x_2))}$. Given another value $y' \in W \cap (I_1 \cap I_2 \cap [0,1])$, it satisfies a variant of the inequality \eqref{hyperplanethick}, and we can subtract the difference between the two inequalities to conclude
	%
	\begin{equation} \label{diffinequality}
		\left| \left( y_3 - y_3' \right) (x_2 - x_1) + (f(y_3) - f(y_3')) \cdot (f(x_2) - f(x_1)) \right| \leq 6/2^k.
	\end{equation}
	%
	The triangle difference inequality applied with \eqref{diffinequality} implies
	%
	\begin{align} \label{yylowbound}
	\begin{split}
		(f(y_3) - f(y_3')) \cdot (f(x_2) - f(x_1)) &\geq |y_3 - y_3'||x_2-x_1| - 6/2^k\\ &= \frac{(m+1) \cdot |y_3 - y_3'| - 6}{2^k}.
	\end{split}
	\end{align}
	%
	Conversely,
	%
	\begin{align} \label{yyupbound}
	\begin{split}
		(f(y_3) - f(y_3')) \cdot (f(x_2) - f(x_1)) &\leq \| f \|_{\text{Lip}}^2 \cdot |y_3 - y_3'| |x_2 - x_1| \\ &=  \| f \|_{\text{Lip}}^2 \cdot (m+1)/2^k \cdot |y_3 - y_3'|.
	\end{split}
	\end{align}
	%
	Combining \eqref{yylowbound} and \eqref{yyupbound} and rearranging, we see that
	%
	\begin{equation}\label{y3minusY3Prime}
		|y_3 - y_3'| \leq \frac{6}{(m+1)(1 - \| f \|_{\text{Lip}}^2)}  \lesssim\frac{1}{m+1},
	\end{equation} 
	%
	where the implicit constant depends only on $\| f \|_{\text{Lip}}$.  We conclude that
	%
	\begin{equation}\label{coveringNumberBoundLargeK}
		\# \DQ_k(W(1/2^k) \cap (I_1 \times I_2 \times [0,1])) \lesssim \frac{2^k}{m+1},
	\end{equation} 
	%
	which holds uniformly over any value of $m$.

	We are now ready to bound the sum from \eqref{deltaCoveringWSum}. Note that for each value of $m$, there are at most $2^{k+1}$ pairs $(I_1,I_2)$ with $d(I_1,I_2) = m/2^k$. Indeed, there are $2^k$ choices for $I_1$ and then at most two choices for $I_2$. Equation  \eqref{coveringNumberBoundLargeK} shows 
	%
	\begin{align*}
		\# \DQ_k(W(1/2^k)) &= \sum_{m = 0}^{2^k} \sum_{\substack{I_1, I_2 \in \DQ_k[0,1]\\d(I_1,I_2) = m/2^k}} \# \DQ_k(W(1/2^k) \cap (I_1 \times I_2 \times [0,1]))\\
		&\lesssim 4^k \sum_{m = 0}^{2^k} \frac{1}{m+1} \lesssim k/4^k.
	\end{align*}
	%
	In the above inequalities, the implicit constants depend on $\| f \|_{\text{Lip}},$ but they are independent of $k$. This establishes \eqref{boundOnWCoveringNumber} and completes the proof.
\end{proof}

\endinput
%% The following is a directive for TeXShop to indicate the main file
%%!TEX root = diss.tex

\chapter{Extensions to Low Rank Configurations}
\label{ch:LowRank}



\section{Boosting the Dimension of Pattern Avoiding Sets by Low Rank Coordinate Changes}

We now consider finding subsets of $[0,1]$ avoiding solutions to the equation $y = f(Tx)$, where $T$ is a rank $k$ linear transformation with integer coefficients with respect to standard coordinates, and $f$ is real-valued and Lipschitz continuous. Fix a constant $A$ bounding the operator norm of $T$, in the sense that $|Tx| \leq A|x|$ for all $x \in \mathbf{R}^n$, and a constant $B$ such that $|f(x+y) - f(x)| \leq B|y|$ for all $x$ and $y$ for which the equation makes sense (if $f$ is $C^1$, this is equivalent to a bound $\| \nabla f \|_\infty \leq B$). Consider sets $J_0, J_1, \dots, J_n, \subset [0,1]$, which are unions of intervals of length $1/M$, with startpoints lying on integer multiples of $1/M$. The next theorem works as a `building block lemma' used in our algorithm for constructing a set avoiding solutions to the equation with Hausdorff dimension $k$ and full Minkowski dimension.

\begin{theorem}
    For infinitely many integers $N$, there exists $S_i \subset J_i$ avoiding solutions to $y = f(Tx)$ with $y \in S_0$ and $x_n \in S_n$, such that
    %
    \begin{itemize}
        \item For $n \neq 0$, if we decompose each $J_i$ into length $1/N$ consecutive intervals, $S_i$ contains an initial portion $\Omega(1/N^k)$ of each length $1/N$ interval. This part of the decomposition gives the Hausdorff dimension $1/k$ bound for the set we will construct.

        \item If we decompose $J_i$ into length $1/N$ intervals, and then subdivide these intervals into length $\Omega(1/N^k)$ intervals, then $S_0$ contains a subcollection of these $1/N^k$ intervals which contains a total length $\Omega(1/N)$ of a fraction $1 - 1/M$ of the length $1/N$ intervals. This property gives that our resultant set will have full Minkowski dimension.
    \end{itemize}
    %
    The implicit constants in these bounds depend only on $A$, $B$, $n$, and $k$.
\end{theorem}
\begin{proof}
Split each interval of $J_a$ into length $1/N$ intervals, and then set
%
\[ \mathbf{A} = \{ x : x_a\ \text{is a startpoint of a $1/N$ interval in $J_a$} \} \]
%
Since the startpoints of the intervals are integer multiples of $1/N$, $T(\mathbf{A})$ is contained with a rank $k$ sublattice of $(\mathbf{Z}/N)^m$. The operator norm also guarantees $T(\mathbf{A})$ is contained within the ball $B_A$ of radius $A$ in $\mathbf{R}^m$. Because of the lattice structure of the image, $| x - y | \gtrsim_n 1/N$ for each distinct pair $x,y \in T(\mathbf{A})$. For any $R$, we can cover $\Sigma \cap B_A$ by $O_{n,k}((A/R)^k)$ balls of radius $R$. If $R \gtrsim_n 1/N$, then each ball can contain only a single element of $T(\mathbf{A})$, so we conclude that $|T(\mathbf{A})| \lesssim_{n,k} (AN)^k$. If we define the set of `bad points' to be
%
\[ \mathbf{B} = \{ y \in [0,1] : \text{there is $x \in \mathbf{A}$ such that $y = f(T(x))$} \} \]
%
Then
%
\[ |\mathbf{B}| = |f(T(\mathbf{A}))| \leq |T(\mathbf{A})| = O_{A,n,k}(N^k) \]
%
For simplicity, we now introduce an integer constant $C_0 = C_0(A,n,k,M)$ such that $|\mathbf{B}| \leq (C_0/M^2) N^k$. We now split each length $1/M$ interval in $J_0$ into length $1/N$ intervals, and filter out those intervals containing more than $C_0N^{k-1}$ elements of $\mathbf{B}$. Because of the cardinality bound we have on $\mathbf{B}$ there can be at most $N/M^2$ such intervals, so we discard at most a fraction $1/M$ of any particular length $1/M$ interval in $J_0$. If we now dissect the remaining intervals into $4C_0N^{k-1}$ intervals of length $1/4C_0N^k$, and discard any intervals containing an element of $\mathbf{B}$, or adjacent to such an interval, then the remaining such intervals $I$ satisfy $d(I,\mathbf{B}) \geq 1/4C_0N^k \gtrsim_{A,n,M,k}(1/N^k)$, and because of our bound on the number of elements of $\mathbf{B}$ in these intervals, there are at least $C_0N^{k-1}$ intervals remaining, with total length exceeding $C_0N^{k-1}/4C_0N^k = \Omega(1/N)$. If $f$ is $C^1$ with $\| \nabla f \|_\infty \leq B$, or more generally, if $f$ is Lipschitz continuous of magnitude $B$, then
%
\[ | f(Tx) - f(Tx')| \leq AB |x - x'| \]
%
and so we may choose $S_i \subset J_i$ by thickening each startpoint $x \in J_i$ to a length $O(1/N^k)$ interval while still avoiding solutions to the equation $y = f(T(x))$.
\end{proof}

\begin{remark}
    If $T$ is a rank $k$ linear transformation with rational coefficients, then there is some number $a$ such that $aT$ has integer coefficients, and then the equation $y = f(Tx)$ is the same as the equation $y = f_0((aT)(x))$, where $f_0(x) = f(x)/a$. Since $f_0$ is also Lipschitz continuous, we conclude that we still get the dimension $1/k$ bound if $T$ has rational rather than integral coefficients. More generally, this trick shows the result applies unperturbed if all coefficients of $T$ are integer multiples of some fixed real number. More generally, by varying the lengths of our length $1/N$ decomposition by a constant amount, we can further generalize this to the case where each column of $T$ are integers multiples of some fixed real number.
\end{remark}

\begin{remark}
    To form $\mathbf{A}$, we take startpoints lying at equal spaced $1/N$ points. However, by instead taking startpoints at varying points in the length $1/N$ intervals, we might be able to make points cluster more than in the original algorithm. Maybe the probabalistic method would be able to guarantee the existence of a choice of startpoints whose images are tightly clustered together.
\end{remark}

\begin{remark}
    Since the condition $y = f(Tx)$ automatically assumes a kind of `non-vanishing derivative' condition on our solutions, we do not need to assume the regularity of $f$, and so the theorem extends naturally to a more general class of functions than Rob's result, i.e. the Lipschitz continuous functions.
\end{remark}

Using essentially the same approach as the last argument shows that we can avoid solutions to $y = f(Tx)$, where $y$ and $x$ are now vectors in some $\mathbf{R}^m$, and $T$ has rank $k$. If we consider unions of $1/M$ cubes $J_0, \dots, J_n$. If we fix startpoints of each $x_k$ forming lattice spaced apart by $\Omega(1/N)$, and consider the space $\mathbf{A}$ of products, then there are $O(N^k)$ points in $T(\mathbf{A})$, and so there are $O(N^k)$ elements in $\mathbf{B}$. We now split each $1/M$ cube in $J_0$ into length $1/N$ cubes, and discard those cubes which contain more than $O(N^{k-m})$ bad points, then we discard at most $1 - 1/M$ of all such cubes. We can dissect the remaining length $1/N$ cubes into $O(N^{k-m})$ length $\Omega(1/N^{k/m})$ cubes, and as in the previous argument, the cubes not containing elements of $\mathbf{B}$ nor adjacent to an element have total volume $\Omega(1/N^m)$, which we keep. The startpoints in the other intervals $T_i$ may then be thickened to a length $\Omega(1/N^{m/k})$ portion while still avoiding solutions. This gives a set with full Minkowski dimension and Hausdorff dimension $m/k$ avoiding solutions to $y = f(Tx)$. (I don't yet understand Minkowski dimension enough to understand this, but the techniques of the appendix make proving the Hausdorff dimension $1/k$ bound easy)

\section{Extension to Well Approximable Numbers}

If the coefficients of the linear transformation $T$ in the equation $y = f(Tx)$ are non-rational, then the images of startpoints under the action of $T$ do not form a lattice, and so points may not overlap so easily when avoiding solutions to the equations $y = f(Tx)$. However, if $T$ is `very close' to a family of rational coefficient linear transformations, then we can show the images of the startpoints are `very close' to a lattice, which will still enable us to find points avoiding solutions by replacing the direct combinatorial approach in the argument for integer matrices with a covering argument.

Suppose that $T$ is a real-coefficient linear transformation with the property that for each coefficient $x$ there are infinitely many rational numbers $p/q$ with $|x - p/q| \leq 1/q^\alpha$, for some fixed $\alpha$. For infinitely many $K$, we can therefore find a linear transformation $S$ with coefficients in $\mathbf{Z}/K$ with each coefficient of $T$ differing from the corresponding coefficient in $S$ by at most $1/K^\alpha$. Then for each $x$, we find
%
\[ \| (T - S)(x) \|_\infty \leq (n/K^\alpha) \| x \|_\infty \]
%
If we now consider $T_0, \dots, T_n$, splitting $T_1, \dots, T_n$ into length $1/N$ intervals, and considering $\mathbf{A}$ as in the last section, then $S(\mathbf{A})$ lie in a $k$ dimensional sublattice of $(\mathbf{Z}/KN)^m$, hence containing at most $(2A)^k (KN)^k = O_{T,n}((KN)^k)$ points. By our error term calculation of $T-S$, the elements of $T(\mathbf{A})$ are contained in cubes centered at these lattice points with side-lengths $2n/K^\alpha$, or balls centered at these points with radius $n^{3/2}/K^\alpha$. If $\| \nabla f \| \leq B$, then the images of the radius $n^{3/2}/K^\alpha$ balls under the action of $f$ are contained in length $Bn^{3/2}/K^\alpha$ intervals. Thus the total length of the image of all these balls under $f$ is $(Bn^{3/2}/K^\alpha)(2A)^k(KN)^k = (2A)^k Bn^{3/2} K^{k-\alpha} N^k$. If $k < \alpha$, then we can take $K$ arbitrarily large, so that there exists intervals with $\text{dist}(I,\mathbf{B}) = \Omega_{A,k,M}(1/N^k)$. But I believe that, after adding the explicit constants in, we cannot let $k = \alpha$.

\begin{remark}
    One problem is that, if $T$ has rank $k$, we might not be able to choose $S$ to be rank $k$ as well. Is this a problem? If $T$ has full rank, then the set of all such matrices is open so if $T$ and $S$ are close enough, $S$ also has rank $k$, but this need not be true if $T$ does not have full rank.
\end{remark}

\begin{example}
    If $T$ has rank 1, then Dirichlet's theorem says that every irrational number $x$ can be approximated by infinitely many $p/q$ with $|x - p/q| < 1/q^2$, so every real-valued rank 1 linear transformation can be avoided with a dimension one bound.
\end{example}

\section{Equidistribution and Real Valued Matrices}

If $T$ is a non-invertible matrix containing irrational coefficients, then the values $Tx$, for $x \in \mathbf{Z}^n$, do not form a lattice, and therefore we cannot use the direct combinatorial arguments of the past section to obtain the decomposition lemma. However, without loss of generality, we can write $T(x) = S(x_1) + U(x_2)$, where $x = (x_1,x_2)$, $x_1 \in \mathbf{R}^k$, $x_2 \in \mathbf{R}^{n-k}$, and $S$ has full rank $k$. Then $S$ is an embedding of $\mathbf{R}^k$ into $\mathbf{R}^n$, so $\Gamma = S((\mathbf{Z}/N)^k)$ forms a lattice with points spaced apart by a distance on the order of $\Omega(1/N)$. Since $T$ has rank $k$, the image of $U$ is contained within the image of $S$. We let $\mathbf{T}$ denote the torus obtained by quotienting the $k$ dimensional subplane forming the image of $T$ by $\Gamma$. Then the image of $S$ in $\mathbf{T}$ is contained within an $\alpha$ dimensional subtorus of $\mathbf{T}$. Note that $\alpha = 0$ precisely when $T$ still has rank $k$ over the rational numbers, so that in a suitable basis $T$ is an integer valued matrix. If $\alpha = 1$, then by an appropriate scaling in the values $x_2$ we can still make the values of $S$ lie at lattice points, which should give a Hausdorff dimension one set. When $\alpha = 2$, we run into problems.

If $\pi: \mathbf{R}^k \to T$ is the homomorphism obtained by composing the quotient map onto the torus with the linear map $T$, then $\pi(x_1,0) = 0$ for all $x_1 \in (\mathbf{Z}/N)^k$. On the other hand, Ratner's theorem implies that for each $x_2 \in \mathbf{R}^{n-k}$, there is some $M$ such that the sequence $\pi(0, nMx_2) = Mn \pi(0, x_2)$ is equidistributed on a subtorus of $T$. Equidistribution may be useful in extending the rational matrix result to all real matrices with some dimension loss, since a matrix is rational if and only if $n M \pi(0, x_2)$ is equidistributed on a zero dimensional lattice -- it may ensure that points are closely clustered to lattice points.

Lets consider the simplest case, where $n = k+1$, so $x_2 \in \mathbf{R}$. Consider our setup, with intervals $J_0, \dots, J_n$, and an equation $y = f(Tx)$, where $f$ is Lipschitz with Lipschitz norm bounded by $B$. Now if $T = S + U$, then $S(\mathbf{Z}^k)$ forms a rank $k$ lattice, and $U(\mathbf{Z})$ equidistributes over an $\alpha$ dimensional subtorus of the torus generated over the lattice. In particular, since the set $\mathbf{Z} \cap NJ_n$ contains $\Omega(N |J_n|) = \Omega(N)$ consecutive points, for any $\varepsilon$ and suitably large $N$, $\mathbf{Z} \cap NJ_n$ contains $\Omega(Nr^\alpha)$ points $x$ such that $S(x)$ is within a distance $r$ from a lattice point, for any $r$. Dividing by $N$ tells us that we have $O(Nr^\alpha)$ points $x$ in $\mathbf{Z}/N \cap J_n$ such that $U(x)$ is at a distance $r/N$ from a lattice point in $S((\mathbf{Z}/N)^k)$. If $r = 1/N^\beta$, then  we have $O(N^{1-\alpha\beta})$ points at a distance $1/N^{\beta + 1}$ from a lattice point. There are $O(N^k)$ points in the lattice, and so provided that $k < 1 + \beta$, we can find a large subset avoiding the images of these startpoints, and we should be able to thicken the startpoints to lengt h $\Omega(1/N^k)$ intervals, hence we should expect the set we construct to have Hausdorff dimension $1/k(k-1)$ if this process is repeated to construct our solution avoiding set.

\section{Applications of Low Rank Coordinate Changes}

\begin{example}
Our initial exploration of low rank coordinate changes was inspired by trying to find solutions to the equation
%
\[ y - x = (u - w)^2 \]
%
Our algorithm gives a Hausdorff dimension $1/2$ set avoiding solutions to this equation. This equals Math\'{e}'s result. But this dimension for us now depends on the shifts involved in the equation, not on the exponent, so we can actually avoid solutions to the equation
%
\[ y - x = (u - w)^n \]
%
for any $n$, in a set of Hausdorff dimension $1/2$. More generally, if $X$ is a set, then given a smooth function $f$ of $n$ variables, we can find a set $X$ of Hausdorff dimension $1/n$ such that there is no $x \in X$, and $y_1, \dots, y_n \in X - X$ such that $x = f(y_1, \dots, y_n)$. This is better than the $1/2n$ bound that is obtained by Malabika and Fraser's result.
\end{example}

\begin{example}
For any fixed $m$, we can find a set $X \subset \mathbf{R}^n$ of full Hausdorff dimension which contains no solutions to
%
\[ a_1x_1 + \dots + a_nx_n = 0 \]
%
for {\it any} rational numbers $a_n$ which are not all zero. Since Malabika/Fraser's technique's solutions are bounded by the number of variables, they cannot let $n \to \infty$ to obtain a linearly independant set over the rational numbers. But since the Hausdorff dimension of our sets now only depends on the rank of $T$, rather than the total number of variables in $T$, we can let $n \to \infty$ to obtain full sets linearly independant over the rationals. More generally, for any Lipschitz continuous function $f: \mathbf{R} \to \mathbf{R}$, we can find a full Hausdorff dimensional set such that there are no solutions
%
\[ f(a_1x_1 + \dots + a_nx_n, y) \]
%
for any $n$, and for any rational numbers $a_n$ that are not all zero.
\end{example}

\begin{example}
The easiest applications of the low rank coordinate change method are probably involving configuration problems involving pairwise distances between $m$ points in $\mathbf{R}^n$, where $m \ll n$, since this can best take advantage of our rank condition. Perhaps one way to encompass this is to avoid $m$ vertex polyhedra in $n$ dimensional space, where $m \ll n$. In order to distinguish this problem from something that can be solved from Math\'{e}'s approach, we can probably find a high dimensional set avoiding $m$ vertex polyhedra on a parameterized $n$ dimensional manifold, where $m \ll n$. There is a result in projective geometry which says that every projectively invariant property of $m$ points in $\mathbf{RP}^d$ is expressible as a function in the ${m \choose d}$ bracket polynomials with respect to these $m$ points. In particular, our result says that we can avoid a countable collection of such invariants in a dimension $1/{m \choose d}$ set. This is a better choice of coordinates than Euclidean coordinates if ${m \choose d} \leq m$. Update: I don't think this is ever the case.
\end{example}

\begin{remark}
    Because of how we construct our set $X$, we can find a dimension $1/k$ set avoiding solutions to $y = f(Tx)$ for {\it all} rank $k$ rational matrices $T$, without losing any Hausdorff dimension. Maybe this will help us avoid solutions to more general problems?
\end{remark}

\begin{example}
Given a smooth curve $\Gamma$ in $\mathbf{R}^n$, can we find a subset $E$ with high Hausdorff dimension avoiding isoceles triangles. That is, if the curve is parameterized by $\gamma: [0,1] \to \mathbf{R}^n$, can we find $E \subset [0,1]$ such that for any $t_1, t_2, t_3$, $\gamma(t_1)$, $\gamma(t_2)$, and $\gamma(t_3)$ do not form the vertices of an isoceles triangle. This is, in a sense, a non-linear generalization of sets avoiding arithmetic progressions, since if $\Gamma$ is a line, an isoceles triangle is given by arithmetic progressions. Assuming our curve is simple, we must avoid zeroes of the function
%
\[ |\gamma(t_1) - \gamma(t_2)|^2 = |\gamma(t_2) - \gamma(t_3)|^2 \]
%
If we take a sufficiently small segment of this curve, and we assume the curve has non-zero curvature on this curve, we can assume that $t_1 < t_2 < t_3$ in our dissection method.

If the coordinates of $\gamma$ are given by polynomials with maximum degree $d$, then the equation
%
\[ |\gamma(t_1) - \gamma(t_2)|^2 - |\gamma(t_2) - \gamma(t_3)|^2 \]
%
is a polynomial of degree $2d$, and so Math\'{e}'s result gives a set of dimension $1/2d$ avoiding isoceles triangles. In the case where $\Gamma$ is a line, then the function $f(t_1,t_2,t_3) = \gamma(t_1) + \gamma(t_3) - 2\gamma(t_2)$ avoids arithmetic progresions, and Math\'{e}'s result gives a dimension one set avoiding such progressions. Rob and Malabika's algorithm easily gives a set with dimension $1/2$ for any curve $\Gamma$. Our algorithm doesn't seem to be able to do much better here.
\end{example}

\begin{example}
    What is the largest dimension of a set in Euclidean space such that for any value $\lambda$, there is at most one pair of points $x,y$ in the set such that $|x - y| = \lambda$.
\end{example}

\begin{example}
    What is the largest dimension of a set which avoids certain angles, i.e. for which a triplet $x,y,z$ avoids certain planar configurations.
\end{example}

\begin{example}
    A set of points $x_0, \dots, x_d \in \mathbf{R}^d$ lie in a hyperplane if and only if the determinant formed by the vectors $x_n - x_0$, for $n \in \{ 1, \dots, d \}$, is zero. This is a degree $d$ polynomial, hence Math\'{e}'s result gives a dimension one set with no set of $d+1$ points lying in a hyperplane. On the other hand, a theorem of Mattila shows that every analytic set $E$ with dimension exceeding one contains $d + 1$ points in a hyperplane. Can we generalize this to a more general example avoiding points on a rotational, translation invariant family of manifolds using our results?
\end{example}

\begin{example}
    Given a set $F$ not containing the origin, what is the largest Hausdorff dimension of a set $E$ such that for any for any distinct rational $a_1, \dots, a_N$, the sum $a_1E + \dots + a_NE$ does not contain any elements of $F$. Thus the vector space over the rationals generated by $E$ does not contain any elements of $F$. We can also take the non-linear values $f(a_1E + \dots + a_NE)$ avoiding elements of $F$. $F$ must have non-empty interior for the problem to be interesting. Then can we find a smooth function $f$ with non-nanishing derivative which vanishes over $F$, or a family of smooth functions with non-vanishing derivative around $F$.
\end{example}

\section{Idea: Generalizing This Problem to low rank smooth functions}

Suppose we are able to find dimension $1/k$ sets avoiding configurations $y = g(f(x))$, where $f$ is a smooth function from $\mathbf{R}^n \to \mathbf{R}^m$ of rank $k$. Then given any function $g(f(x))$, where $f$ has rank $k$, if $g(f(x)) = 0$, then the implicit function theorem guarantees that there is a cover $U_\alpha$ and functions $h_\alpha: U^k \to \mathbf{R}^{n-k}$ such that for each if $g(f(x)) = 0$, for $x \in U_\alpha$, then there is a subset of $k$ indices $I$ such that $x_{I^c} = h_\alpha(x_I)$.

Then given any function $f(x)$ with rank $k$, we can use the implicit function theorem to find sets $U_\alpha$, indices $n_\alpha$, and functions $g_\alpha$ such that if $f(x) = 0$, for $x \in U_\alpha$, then $x_{n_\alpha} = g_\alpha(x_1, \dots, \widehat{x_{n_\alpha}}, \dots, x_n)$. Thus we need only avoid this type of configuration to avoid configurations of a general low rank function. If we don't believe that we are able to get dimension $1/(k-1)$ sets for rank $k$ configurations, then we shouldn't be able to find sets of Hausdorff dimension $1/k$ avoiding configurations of the form $y = f(x)$, where $f$ has rank $k$. 

\begin{remark}
    If the functions $g_\alpha$ are only partially defined, this makes the problem easier than if the functions were globally defined, because the constraint condition is now smaller than the original constraint.
\end{remark}

\begin{remark}
    This would solve our problem of avoiding $y - x = (u - v)^2$, since if $f(x,y,u,v) = y - x - (u - v)^2$, then
    %
    \[ \nabla f \]
\end{remark}

\section{Idea: Algebraic Number Fields}

If $\mathbf{Q}(\omega)$ is a quadratic extension of the rational numbers, then the ring of integers in this field form a lattice. Perhaps we can use this to generalize our approach to avoiding configurations $y = f(Tx)$, where all coefficients of the matrix $T$ lie in some common quadratic extension of the rational numbers.

\section{A Scheme for Avoiding Configurations}

Math\'{e}'s result can be reconfigured in terms of a building block strategy for implementation in our algorithm.

\begin{theorem}
    Let $f$ be a polynomial of degree $m$, and consider unions of length $1/M$ intervals $T_0, \dots, T_d \subset [0,1]$, with rational start-points. If $\partial_0 f$ is non-vanishing on $T_0 \times \dots \times T_d$, then there exists arbitrarily large integers $N$ and a constant $C$ not depending on $N$ and sets $S_n \subset T_n$ such that
    %
    \begin{itemize}
        \item $f(x) \neq 0$ for $x \in S_0 \times \dots \times S_d$.
        \item If $T_0, \dots, T_d$ are split into length $1/N$ intervals, then $S_n$ contains a length $C/N^d$ region of each interval.
    \end{itemize}
\end{theorem}
\begin{proof}
    Without loss of generality (by subdividing the initial intervals), let $M$ be the greatest common divisor of all of the startpoints of the intervals in $T_n$. Divide each interval $T_n$ into length $1/N$ intervals, and let $\mathbf{A} \subset (\mathbf{Z}/N)^d$ be the cartesian product of all startpoints of these length $1/N$ intervals. Since $f$ has degree $m$, $f(\mathbf{A}) \subset \mathbf{Z}/N^m$. If $A_0 \leq |\partial_0 f| \leq A_1$ on $T_0 \times \dots \times T_d$, then for any $a \in \mathbf{A}$, and $\delta_0$, there exists $\delta_1$ between $0$ and $\delta_0$ for which
    %
    \[ |f(a + \delta_0 e_0)| - f(a)| = \delta_0 |(\partial_0 f)(a + \delta_1)| \]
    %
    If $K$ is fixed such that $A_1 \leq (K-1)A_0$, so that we can choose
    %
    \[ \frac{1/K}{A_0N^m} \leq \delta_0 \leq \frac{\left( 1 - 1/K \right)}{A_1N^m} \]
    %
    Then
    %
    \[ \frac{1/K}{N^m} \leq |f(a + \delta_0 e_0) - f(a)| \leq \frac{1 - 1/K}{N^m} \]
    %
    Thus $d(f(\mathbf{A} + \delta_0 e_0), \mathbf{Z}/N^m) \geq 1/KN^m$. Thus if we thicken the coordinates of $\mathbf{A} + \delta_0$ to intervals of length $O(1/N^m)$, then we obtain sets $S_0, \dots, S_n$ avoiding solutions.
\end{proof}

TODO: CAN WE USE THE COMBINATORIAL NULLSTELLENSATZ TO COME UP WITH AN ALTERNATE BUILDING BLOCK LEMMA FOR ARBITRARY FIELDS?

\section{Square Free Sets}

We now look at avoiding solutions to the equation $x - y = (u - v)^2$. We consider two sets $I$ and $J$. Suppose that we can select a subset $\mathbf{S}$ from $\mathbf{Z} \cap N^2I$ such that if $x,y \in \mathbf{S}$ are distinct, $x - y$ is not a perfect square, and $|\mathbf{S}| \gtrsim |\mathbf{Z} \cap N^2I|^\alpha$. Then for any distinct $u,v \in \mathbf{Z} \cap NJ$, $(u - v)^2 \not \in \mathbf{S} - \mathbf{S}$. But this means that if we thicken the points in $\mathbf{S}/N^2$ to length $O(1/N^2)$ intervals, and the points in $\mathbf{Z}/N \cap J$ into length $O(1/N)$ intervals, then the resultant set will avoid solutions to $x - y = (u - v)^2$. This should give a dimension $\alpha$ set.

\endinput
%% The following is a directive for TeXShop to indicate the main file
%%!TEX root = diss.tex

\chapter{Fourier Dimension and Fractal Avoidance}
\label{ch:RoughSets}


In the last few chapters, we have discovered that the presence of many configurations is not guaranteed by large Hausdorff dimension. We now turn to an analysis of a different dimensional quantity which in many cases gives much more structure than Hausdorff dimension. 




\section{Schmerkin's Method}

We now try and adapt Schmerkin's Method to our situation, introducing translations into the avoiding set. Let $K \subset \mathbf{Z}_N^n$. Our goal is to find $X \subset \mathbf{Z}_N$ such that if $x_1, \dots, x_n$ are distinct, then $(x_1, \dots, x_n) \not \in K$, such that if $f: \mathbf{Z}_N \to \mathbf{R}$ is the probability distribution which uniformly randomly chooses a point on $X$, such that the Fourier transform of $f$ has small $L^\infty$ norm.

We consider an intermediary scale $M$, with $M \divides N$. Then we consider the set
%
\[ K' = \{ x \in \mathbf{Z}_N^n : \text{There is $y \in K$ with $x - y \in M \mathbf{Z}_N^n$} \}. \]
%
We construct $X$ such that if $x_1, \dots, x_n \in X$ are distinct, then $(x_1, \dots, x_n) \not \in Y$. In particular, this means that if we define, for each $x \in X$, a uniformly random selected integer $n_x \in M \mathbf{Z}_N$, then we can define $x' = x + n_x$, and $X' = \{ x' : x \in X \}$. Now if $I_x: \mathbf{Z}_N \to \mathbf{R}$ denotes the delta function at $x'$, then
%
\[ \widehat{I_x}(k) = (1/N) e(-kx'/N) = (1/N) e(-kx/N) e(-n_x). \]
%
And so
%
\begin{align*}
	\mathbf{E}(\widehat{I_x}(k)) &= (M/N) e(-kx/N) \sum_{m = 1}^{N/M} e(-(kM/N)m).
\end{align*}
%
If $kM/N \not \in \mathbf{Z}$, $\mathbf{E}(\widehat{I_x}(k)) = 0$. If $kM/N \in \mathbf{Z}$, then $\mathbf{E}(\widehat{I_x}(k)) = e(-kx/N)$. And
%
\[ |\widehat{I_x}(k)| \leq 1. \]
%
Since $I_x$ is independant, we can apply Hoeffding's inequality to conclude that
%
\[ \mathbf{P} \left( \widehat{f}(k) \geq t \right) \leq \exp \left( \frac{-t^2}{4|X|} \right) \]
%
so $\widehat{f}(k) \lesssim |X|^{1/2}$ with high probability. This should give the Fourier dimension bound we want? Except we do lose some dimension by introducing the parameter $M$. If we pick $N$ to be very large, but $M$ very small, then this shouldn't introduce many problems? What about if we pick $M$ randomly to prevent problems on the linear frequencies?






\section{Tail Bounds on Cube Selection}

The aim of this section is to obtain tail bounds on the total number of intersections between the random set we select in the discrete selection schema, with the goal of obtaining a Fourier dimension bound on the set we construct. Consider the random process generating our interval selection scheme. We have three fixed lengths $l$, $r$, and $s$. We start with a set $E$, which is a non-empty union of cubes in $\B^d_l$, and a set $G$, which is a union of cubes in $\B^{dn}_s$. For each cube $J \in \B^d_r(E)$, we select a single random element of $\B^d_s(J)$, denoted $J_I$. We let $U = \bigcup J_I$. The aim of this section is to obtain tail bounds on the size of the random set
%
\[ \mathcal{K}(U) = \{ K \in \B^{dn}_s(G): K \in U^n, K\ \text{strongly non-diagonal} \}. \]
%
Here, this is obtained by applying the statistical decoupling techniques of Bourgain and Tzafriri.

To understand the distribution of this random variable, it will be convenient to simplify notation. We write $\mathbf{I}(I_J = I_0)$ as $X_{J,I_0}$. We then set
%
\[ Z = \# \mathcal{K}(U) = \sum_{J_1, \dots, J_n} \sum_K \left( \prod_{i = 1}^n X_{J_i,K_i} \right), \]
%
where $J_1, \dots, J_n$ ranges over all distinct choices of $n$ elements from $\B^d_l(E)$, and $K = K_1 \times \dots \times K_n$ ranges over all elements of $\B^{dn}_l(G) \cap \B^{dn}_s(J_1 \times \dots \times J_n)$. The random cubes $\{ I_J \}$ form an independant family, which should make the distribution easier to bound. But $Z$ is a multiplicative linear combination of this independant family. Random variables of this form as known as a \emph{chaos}.

The idea behind statistical decoupling is to obtain tail bounds on the chaos $Z$ by studying the tail bounds on the modified random variable
%
\[ W = \sum_{J_1, \dots, J_n} \sum_K \left( \prod_{i = 1}^n X_{J_i,K_i}^i \right), \]
%
where $X^1, \dots, X^n$ are independant and identically distributed copies of $X$.

\begin{lemma} \label{decouplinglemma}
	Let $F: [0,\infty) \to [0,\infty)$ be a monotone, convex function. Then
	%
	\[ \EE(F(Z)) \leq \EE(F(n^n W)). \]
\end{lemma}
\begin{proof}
	Consider a partition $\mathcal{J}_1, \dots, \mathcal{J}_n$ selected at random from all possible partitions of $\B^d_s(E)$ into $n$ classes. Then, for any distinct set of intervals $J_1, \dots, J_n \in \B^d_s(E)$,
	%
	\begin{equation} \label{partitionprob}
		\PP(J_1 \in \mathcal{J}_1, \dots, J_n \in \mathcal{J}_n) = \PP(J_1 \in \mathcal{J}_1) \dots \PP(J_n \in \mathcal{J}_n) = 1/n^n.
	\end{equation}
	%
	Thus if we consider
	%
	\[ Z' = \sum_{J_1 \in \mathcal{J}_1, \dots, J_n \in \mathcal{J}_n} \sum_K \left( \prod_{i = 1}^n X_{J_i,K_i} \right), \]
	%
	where $J_i$ ranges over elements of $\mathcal{J}_i$ for each $i \in [n]$. If we let $\EE_{\mathcal{J}}$ denote conditioning with respect to all random variables except those depending on $\mathcal{J}_1, \dots, \mathcal{J}_n$, then \eqref{partitionprob} implies
	%
	\begin{align*}
		\EE_{\mathcal{J}}(Z') &= \sum_{J_1, \dots, J_n} \sum_K \PP(J_1 \in \mathcal{J}_1, \dots, J_n \in \mathcal{J}_n) \left( \prod_{i = 1}^n X_{J_i,K_i} \right) = Z/n^n.
	\end{align*}
	%
	Thus Jensen's inequality implies that
	%
	\[ F(Z) = F(n^n \EE_{\mathcal{J}}(Z')) \leq \EE_{\mathcal{J}} F(n^n Z'). \]
	%
	In particular, this means we can select a single, non-random partition $\mathcal{J}_1, \dots, \mathcal{J}_n$ of $\B^d_s(E)$ such that $F(Z) \leq F(n^n Z')$. Because $\mathcal{J}_1, \dots, \mathcal{J}_n$ are disjoint sets, $Z'$ is identically distributed to
	%
	\[ W' = \sum_{J_1 \in \mathcal{J}_1, \dots, J_n \in \mathcal{J}_n} \sum_K \left( \prod_{i = 1}^n X_{J_i,K_i}^i \right). \]
	%
	But $W' \leq W$, and so by monotonicity,
	%
	\[ \EE(F(Z)) \leq \EE(F(n^n W')) \leq \EE(F(n^n W)). \qedhere \]
\end{proof}

\begin{lemma}
	MOMENT BOUND THEOREM.
\end{lemma}
\begin{proof}
	We now bound the moments of $W$, which by Lemma \ref{decouplinglemma} bounds the moments of $Z$. We write
	%
	\[ W^m = \sum_{J_{j,i}} \sum_{K_1, \dots, K_m} \left( \prod_{i = 1}^n \prod_{j = 1}^m X_{J_{j,i}, K_{j,i}}^i \right). \]
	%
	Thus
	%
	\[ \EE(W^m) = \sum_{J_{j,i}} \sum_{K_1, \dots, K_m} \prod_{i = 1}^n \EE \left( \prod_{j = 1}^m X_{J_{j,i}, K_{j,i}}^i \right) \]
	%
	The inner expectation is zero except if $\pi(K_{j,i}) = J_{j,i}$ for all $j$ and $i$, and also, if $\pi(K_{j_0,i}) = \pi(K_{j_1,i})$, then $K_{j_0,i} = K_{j_1,i}$. And then the expectation is $\varepsilon^{\# \{ K_{1,i}, \dots, K_{m,i} \}}$.
\end{proof}

\section{Singular Value Decomposition of Tensors}

An \emph{$n$ tensor} with respect to $\RR^{k_1}, \dots, \RR^{k_n}$, is a multi-linear function
%
\[ A: \RR^{k_1} \times \dots \times \RR^{k_n} \to \RR. \]
%
The family of all $n$ tensors forms a vector space, denoted $\RR^{k_1} \otimes \dots \otimes \RR^{k_n}$, or $\bigotimes \RR^{k_i}$. Given a family of vectors $v_i$, with $v_i \in \RR^{k_i}$, we can consider a tensor $v_1 \otimes \dots \otimes v_n$, such that
%
\[ (v_1 \otimes \dots \otimes v_n)(w_1, \dots, w_n) = (v_1 \cdot w_1) \dots (v_n \cdot w_n). \]
%
It is easily verified that a basis for $\RR^{k_1} \otimes \dots \otimes \RR^{k_n}$ is given by
%
\[ \{ e_{j_1} \otimes \dots \otimes e_{j_n} : j_i \in [k_i] \}. \]
%
Thus a tensor $A$ can be identified with a family of coefficients
%
\[ \{ A_{j_1 \dots j_n} : 1 \leq j_i \leq k_i \} \]
%
such that
%
\[ A(v_1,\dots,v_n) = \sum A_{j_1 \dots j_n} \cdot e_{j_1} \otimes \dots \otimes e_{j_n}. \]
%
From this perspective, a tensor is a multi-dimensional array given by $n$ indices. This basis induces a natural inner product for which the basis is orthonormal. The norm with respect to this inner product is called the \emph{Frobenius norm}, denoted $\| A \|_F$.

Recall that if $A$ is a rank $r$, $m \times n$ matrix, then the \emph{singular value decomposition} says that there are orthonormal families of vectors $\{ u_1, \dots, u_r \}$ and $\{ v_1, \dots, v_r \}$ in $\RR^m$ and $\RR^n$ respectively, and values $s_1, \dots, s_r > 0$ such that $A = \sum s_i (u_i^T v_i)$. We may view $A$ as a 2 tensor in $\RR^m \otimes \RR^n$ by setting $A(v,w) = v^T A w$. Then the singular value decomposition says that $A = \sum s_i (u_i \otimes v_i)$. Thus, viewing $A$ as a bilinear function, for any vectors $x \in \RR^m$ and $y$ in $\RR^n$, $A(x,y) = \sum s_i (u_i \cdot x) (v_i \cdot y)$. We want to try and generalize this result to $n$ tensors in $\RR^{k_1} \otimes \dots \otimes \RR^{k_n}$.

One can reduce certain quantities over tensors to quantities defined with respect to linear maps. For instance, for each index $i_0 \in [n]$, we can define a linear map
%
\[ A_i: \bigotimes_{i' \neq i} \RR^{k_{i'}} \to \RR^{k_i} \]
%
by letting $A_{i_0}(v_1, \dots, \widehat{v_i}, \dots, v_n)$ to be the unique vector $w \in \RR^{k_i}$ such that for any vector $v_i \in \RR^{k_i}$,
%
\[ v_i \cdot w = A(v_1, \dots, v_n). \]
%
The matrix $A_i$ is known as the \emph{$i$'th unfolding of $A$}. We define the $i$ rank of $A$ to be $\text{rank}_i(A) = \text{rank}(A_i)$. This generalizes the column and row rank of matrices. Unlike in the case of matrices, these quantities can differ for different indices $i$. One can also obtain the $i$ rank

The natural definition of the rank of a tensor in the study of the higher order singular value decomposition is defined in analogy with the fact that a rank $r$ matrix can be written as the sum of $r$ rank one matrices. We say a tensor is rank one if it can be written in the form $v_1 \otimes \dots \otimes v_n$ for some vectors $v_i \in \RR^{k_i}$. The \emph{rank} of a tensor $A$, denoted $\text{rank}(A)$, is the minimal number of rank one tensors which can be written in a linear combination to form $A$. We have $\text{rank}(A_i) \leq \text{rank}(A)$ for each $i$, but there may be a gap in these ranks.

To obtain an SVD decomposition, fix an index $i$. Then we can perform a singular value decomposition on $A_i$, obtaining orthonormal basis

\section{Decoupling Concentration Bounds}

We require some methods in non-asymptotic probability theory in order to understand the concentration structure of the random quantities we study. Most novel of these is relying on a probabilistic decoupling bound of Bourgain.

In this section, we fix a single underlying probability space, assumed large enough for us to produce any additional random quantities used in this section. Given a scalar valued random variable $X$, we define its \emph{Orlicz norm} as
%
\[ \| X \|_{\psi_2} = \inf \left\{ \lambda \geq 0 : \EE \left(\exp \left( |X|^2/\lambda^2 \right) \right) \leq 2 \right\}. \]
%
The family of random variables with finite Orlicz norm forms a Banach space, known as the space of \emph{subgaussian random variables}.

\begin{theorem}
	Let $X$ be a sub-gaussian random variable.
	%
	\begin{itemize}
		\item For all $t \geq 0$,
		%
		\[ \PP \left( |X| \geq t \right) \leq 2 \exp \left( \frac{-t^2}{\| X \|_{\psi_2}^2} \right). \]

		\item Conversely, if there is a constant $K > 0$ such that for all $t \geq 0$,
		%
		\[ \PP \left( |X| \geq t \right) \leq 2 \exp \left( \frac{-t^2}{K^2} \right), \]
		%
		then $\| X \|_{\psi_2} \leq 2K$.

		\item For any bounded random variable $X$, $\| X \|_{\psi_2} \lesssim \| X \|_\infty$.

		\item $\| X - \EE X \|_{\psi_2} \lesssim \| X \|$.

		\item If $X_1, \dots, X_n$ are independent, then
		%
		\[ \| X_1 + \dots + X_n \|_{\psi_2} \lesssim \left( \| X_1 \|_{\psi_2}^2 + \dots + \| X_n \|_{\psi_2}^2 \right)^{1/2}. \] 
	\end{itemize}
\end{theorem}
\begin{proof} See \cite{Vershynin}. \end{proof}

Given a random vector $X$, taking values in $\RR^n$, we say it is \emph{uniformly subgaussian} if there is a constant $A$ such that for any $x \in \RR^n$, $\| X \cdot x \|_{\psi_2} \leq A \cdot |x|$. We define
%
\[ \| X \|_{\psi_2} = \sup_{|x| = 1} \| X \cdot x \|_{\psi_2}, \]
%
so that $\| X \cdot x \|_{\psi_2} \leq \| X \|_{\psi_2} \cdot |x|$ for any random vector $X$.

The classic Hanson-Wright inequality gives a concentration bound for the magnitude of a quadratic form $X^T A X$ in terms of the matrix norms of $A$, assuming $X$ is a random vector with independant, mean zero, subgaussian coordinates.

\begin{theorem}[Hanson-Wright]
	Let $X = (X_1, \dots, X_k)$ be a random vector with independant, mean zero, subgaussian coordinates, with $\| X_i \|_{\psi_2} \leq K$ for each $i$, and let $A$ be a diagonal-free $k \times k$ matrix. Then there is a universal constant $c$ such that for every $t \geq 0$,
	%
	\[ \PP \left( |X^TAX| \geq t \right) \leq 2 \exp \left( -c \cdot \min \left( \frac{t^2}{K^4 \| A \|_F^2}, \frac{t}{K^2 \| A \|} \right) \right). \]
	%
	Here $\| A \|$ is the operator norm of $A$, and $\| A \|_F$ it's Frobenius norm.
\end{theorem}

In our work, we require a modified version of the Hanson Wright inequality where we replace the independant random variables $X_1, \dots, X_k$ with independant random vectors taking values in $\RR^d$, and where $A$ is replaced with a $n$ tensor on $\RR^{k \times d}$. To make things more tractable, we assume that $A_I = 0$ for each multi-index $I = ((i_1,j_1), \dots, (i_n,j_n))$ if there is $r \neq r'$ with $i_r = i_{r'}$. If this is satisfied, we say $A$ is a \emph{non-diagonal tensor}.

\begin{theorem}
	Let $X = (X_1, \dots, X_k)$, where $X_1, \dots, X_k$ are independant, mean zero random vectors taking values in $\RR^d$, with $\| X_i \|_{\psi_2} \leq K$ for each $i$. Let $A$ be a non-diagonal $n$-tensor in $\RR^{k \times d}$. Then for $t \geq 0$,
	%
	\[ \PP \left( |A(X,\dots,X)| \geq t \right) \leq 2 \cdot \exp \left( - c(n) \cdot \min \left( \frac{t^2}{K^4 \| A \|_F^2}, \dots, \frac{t^{2/n}}{K^{4/n} \| A \|_F^{2/n}} \right) \right). \]
\end{theorem}

The main technique involved is a statistical decoupling result due to J. Bourgain and L. Tzafriri. We follow the proof of the Hanson-Wright inequality from \cite{Vershynin}, altering various calculations to incorporate the vector-valued coordinates $X_1, \dots, X_n$, and incorporating more robust calculations of the moments of higher order Gaussian chaos.

\begin{lemma}[Decoupling]
	If $F: \RR \to \RR$ is convex, and $X = (X_1, \dots, X_k)$, where $X_1, \dots, X_k$ are independant, mean zero random vectors taking values in $\RR^d$. If $A$ is a non-diagonal $n$-tensor in $\RR^{d \times k}$, then
	%
	\[ \EE(F(A(X,\dots,X))) \leq \EE(F(n^n \cdot A(X^1,X^2,\dots,X^n))), \]
	%
	where $X^1, \dots, X^n$ are independant copies of $X$.
\end{lemma}
\begin{proof}
	Given a set $I \subset [k]$, let $\pi_I: \RR^{k \times d} \to \RR^{k \times d}$ be defined by
	%
	\[ \pi_{I}(X)_{ij} = \begin{cases} X_{ij} &: \text{if}\ i \in I, \\ 0 &: \text{if}\ i \not \in I \end{cases}. \]
	%
	Thus given $X = (X_1, \dots, X_k)$, the vector $\pi_I(X)$ is obtained by zeroeing out all vectors $X_i$ with $i \not \in I$. Consider a random partition of $[k]$ into $n$ classes $I_1, \dots, I_n$, chosen uniformly at random. Thus given any $x \in \RR^{d \times k}$, we have
	%
	\[ \EE(A(\pi_{I_1}(x),\dots,\pi_{I_n}(x))) = \sum_{I=((i_1,j_1),\dots,(i_n,j_n))} A_I \cdot x_{i_1 j_1} \dots x_{i_n j_n} \PP(i_1 \in I_1, \dots, i_n \in I_n). \]
	%
	If $A_I \neq 0$, the indices $\{ i_r \}$ are distinct from one another. Since $(I_1,\dots,I_n)$ is selected uniformly at random, this means the positions of the indices $i_r$ are independent of one another. Thus
	%
	\[ \PP(i_1 \in I_1, \dots, i_n \in I_n) = \PP(i_1 \in I_1) \dots \PP(i_n \in I_n) = 1/n^n. \]
	%
	Thus we have shown
	%
	\[ A(x,\dots,x) = n^n \cdot \EE(A(\pi_{I_1}(x), \dots, \pi_{I_n}(x))). \]
	%
	In particular, substituting $X$ for $x$, we have
	%
	\[ A(X,\dots,X) = n^n \cdot \EE_I(A(\pi_{I_1}(X), \dots, \pi_{I_n}(X))) \]
	%
	Applying Jensen's inequality, we find
	%
	\begin{align*}
		\EE_X(F(A(X,\dots,X))) &= \EE_X(F(n^n \cdot \EE_I(A(\pi_{I_1}(X), \dots, \pi_{I_n}(X)))))\\
		&\leq \EE_X \EE_I F(n^n \cdot A(\pi_{I_1}(X), \dots, \pi_{I_n}(X)))\\
		&= \EE_I \EE_X F(n^n \cdot A(\pi_{I_1}(X), \dots, \pi_{I_n}(X))).
	\end{align*}
	%
	Thus we may select a \emph{deterministic} partition $I$ for which
	%
	\[ \EE(F(A(X,\dots,X))) \leq \EE(F(n^n \cdot A(\pi_{I_1}(X),\dots,\pi_{I_n}(X)))). \]
	%
	Since the classes $I_1, \dots, I_n$ are disjoint from one another, $(\pi_{I_1}(X), \dots, \pi_{I_n}(X))$ is identically distributed to $(\pi_{I_1}(X^1), \dots, \pi_{I_n}(X^n))$, so
	%
	\[ \EE(F(n^n \cdot A(\pi_{I_1}(X),\dots,\pi_{I_n}(X)))) = \EE(F(n^n \cdot A(\pi_{I_1}(X^1),\dots,\pi_{I_n}(X^n)))). \]
	%
	Let $Z = A(X^1,\dots,X^n) - A(\pi_{I_1}(X^1),\dots,\pi_{I_n}(X^n))$. If $\EE'$ denotes conditioning with respect to $\pi_{I_1}(X^1), \dots, \pi_{I_n}(X^n)$, then this expectation fixes $A(\pi_{I_1}(X_1),\dots,\pi_{I_n}(X_n))$, but $\EE'(Z) = 0$. Thus we can apply Jensen's inequality again to conclude
	%
	\begin{align*}
		\EE(F(n^n \cdot A(\pi_{I_1}(X^1),\dots,\pi_{I_n}(X^n)))) &= \EE(F(n^n \cdot \EE'(A(\pi_{I_1}(X^1), \dots, \pi_{I_n}(X^n)) + Z)))\\
		&= \EE(F(n^n \cdot \EE'(A(X^1,\dots,X^n))))\\
		&\leq \EE(\EE'(F(n^n A(X^1,\dots,X^n))))\\
		&= \EE(F(n^n A(X^1,\dots,X^n))). \qedhere
	\end{align*}
\end{proof}

\begin{remark}
	Note this theorem also implies that for each $\lambda \in \RR$,
	%
	\[ \EE(F((\lambda/n^n) \cdot A(X,\dots,X))) \leq \EE(F(\lambda \cdot A(X^1,\dots,X^n))). \]
\end{remark}

\begin{lemma}[Comparison]
	Let $X^1,\dots,X^n$ be independant random vectors taking values in $\RR^{k \times d}$
\end{lemma}

To make things easy for ourselves, we assume that $A$ vanishes along it's diagonal. Reinterpreting the quantities in the Hanson-Wright inequality in this setting gives the result. For each $i$ and $j$, we define $A_{ij}$ be an $d \times d$ matrix, then we set $X^T A X = \sum X_i^T A_{ij} X_j$,
%
\[ \| A \|_F = \sqrt{\sum \| A_{ij} \|_F^2}, \quad\text{and}\quad \| A \| = \sup \left\{ |\sum x_i^T A_{ij} y_j| : \sum |x_i|^2 = \sum |y_j|^2 = 1 \right\}. \]
%
These are precisely the definitions of the operator norm and Frobenius norm if we interpret $A$ as a $dn \times dn$ matrix by expanding out the entries of each $A_{ij}$, and $X$ as a random vector taking values in $\RR^{dn}$ by expanding out the entries of each $X_i$.

\endinput
%% The following is a directive for TeXShop to indicate the main file
%%!TEX root = diss.tex

\chapter{Constructing Squarefree Sets}
\label{ch:Squarefree}

\section{Ideas For New Work}

A continuous formulation of the squarefree difference problem is not so clear to formulate, because every positive real number has a square root. Instead, we consider a problem which introduces a similar structure to avoid in the continuous domain rather than the discrete. Unfortunately, there is no direct continuous anology to the squarefree subset problem on the interval $[0,1]$, because there is no canonical subset of $[0,1]$ which can be identified as `perfect squares', unlike in $\mathbf{Z}$. If we only restrict ourselves to perfect squares of a countable set, like perfect squares of rational numbers, a result of Keleti gives us a set of full Hausdorff dimension avoiding this set. Thus, instead, we say a set $X \subset [0,1]$ is (continuously) {\bf squarefree} if there are no nontrivial solutions to the equation $x - y = (u - v)^2$, in the sense that there are no $x,y,u,v \in X$ satisfying the equation for $x \neq y, u \neq v$. In this section we consider some blue sky ideas that might give us what we need.

How do we adopt Rusza's power series method to this continuous formulation of the problem? We want to scale up the problem exponentially in a way we can vary to give a better control of the exponentials. Note that for a fixed $m$, every elements $x \in [0,1]$ has an essentially unique $m$-ary expansion
%
\[ x = \sum_{n = 1}^\infty \frac{x_n}{m^n} \]
%
and the pullback to the Haar measure on $\mathbf{F}_m^\infty$ is measure preserving (with respect to the natural Haar measure on $\mathbf{F}_m^\infty$), so perhaps there is a way to reformulate the problem natural as finding nice subsets of $\mathbf{F}_m^\infty$ avoiding squares. In terms of this expansion, the equation $x - y = (u - v)^2$ can be rewritten as
%
\[ \sum_{n = 1}^\infty \frac{x_n - y_n}{m^n} = \left( \sum_{k = 1}^\infty \frac{u_n - v_n}{m^n} \right)^2 = \sum_{n = 1}^\infty \left( \sum_{k = 1}^{n-1} (u_k - v_k)(u_{n-k} - v_{n-k}) \right) \frac{1}{m^n} \]
%
One problem with this expansion is that the sums of the differences of each element do not remain in $\{ 0, \dots, m-1 \}$, so the sum on the right cannot be considered an equivalent formal expansion to the expansion on the left. Perhaps $\mathbf{F}_m^\infty$ might be a simpler domain to explore the properties of squarefree subsets, in relation to Ruzsa's discrete strategy. What if we now consider the problem of finding the largest subset $X$ of $\mathbf{F}_m^\infty$ such that there do not exist $x,y,u,v \in \mathbf{F}_m^\infty$ such that if $x,y,u,v \in X$, $x \neq y$, $u \neq v$, then for any $n$
%
\[ x_n - y_n \neq \sum_{k = 1}^{n-1} (u_k - v_k)(u_{n-k} - v_{n-k}) \]

What if we consider the problem modulo $m$, so that the convolution is considered modulo $m$, and we want to avoid such differences modulo $m$. So in particular, we do not find any solutions to the equation
%
\begin{align*}
    x_2 - y_2 &= (u_1 - v_1)^2\\
    x_3 - y_3 &= 2 (u_1 - v_1)(u_2 - v_2)\\
    x_4 - y_4 &= (u_1 - v_1)(u_3 - v_3) + (u_2 - v_2)^2\\
    &\ \ \vdots
\end{align*}
%
which are considered modulo $m$. The topology of the $p$-adic numbers induces a power series relationship which `goes up' and might be useful to our analysis, if the measure theory of the $p$-adic numbers agrees with the measure theory of normal numbers in some way, or as an alternate domain to analyze the squarefree problem as with $\mathbf{F}_m^\infty$.

The problem with the squarefree subset problem is that we are trying to optimize over two quantities. We want to choose a set $X$ such that the number of distinct differences $x - y$ as small as possible, while keeping the set as large as possible. This double optimization is distinctly different from the problem of finding squarefree difference subsets of the integers. Perhaps a more natural analogy is to fix a set $V$, and to find the largest subset $X$ of $[0,1]$ such that $x - y = (u - v)^2$, where $x \neq y \in X$, and $u \neq v \in V$. Then we are just avoiding subsets of $[0,1]$ which avoid a particular set of differences, and I imagine this subset has a large theory. But now we can solve the general subset problem by finding large subsets $X$ such that $(X - X)^2 \subset V$ and $X$ containing no differences in $V$. Does Rusza's method utilize the fact that the problem is a single optimization? Can we adapt Rusza's method work to give better results about finding subsets $X$ of the integers such that $X - X$ is disjoint from $(X - X)^2$?

\section{Squarefree Sets Using Modulus Techniques}

We now try to adapt Ruzsa's idea of applying congruences modulo $m$ to avoid squarefree differences on the integers to finding high dimensional subsets of $[0,1]$ which satisfy a continuous analogy of the integer constraint. One problem with the squarefree problem is that solutions are non-scalable, in the sense that if $X \subset [N]$ is squarefree, $\alpha X$ may not be squarefree. This makes sense, since avoiding solutions to $\alpha (x - y) = \alpha^2 (u - v)^2$ is clearly not equivalent to the equation $x - y = (u - v)^2$. As an example, $X = \{ 0, 1/2 \}$ is squarefree, but $2X = \{ 0, 1 \}$ isn't. On the other hand, if $X$ avoids squarefree differences modulo $N$, it {\it is} scalable by a number congruent to 1 modulo $N$. More generally, if $\alpha$ is a rational number of the form $p/q$, then $\alpha X$ will avoid nontrivial solutions to $q (x - y) = p (u - v)^2$, and if $p$ and $q$ are both congruent to 1 modulo $N$, then $X$ is squarefree, so modulo arithmetic enables us to scale down. Since the set of rational numbers with numerator and denominator congruent to 1 is dense in $\mathbf{R}$, {\it essentially} all scales of $X$ are continuously squarefree. Since $X$ is discrete, it has Hausdorff dimension zero, but we can `fatten' the scales of $X$ to obtain a high dimension continuously squarefree set. To initially simplify the situation, we now choose to avoid nontrivial solutions to $y - x = (z - x)^2$, removing a single degree of freedom from the domain of the equation.

So we now fix a subset $X$ of $\{ 0, \dots, m-1 \}$ avoiding squares modulo $m$. We now ask how large can we make $\varepsilon$ such that nontrivial solutions to $x - y = (x - z)^2$ in the set
%
\[ E = \bigcup_{x \in X} [\alpha x, \alpha x + \varepsilon) \]
%
occur in a common interval, if $\alpha$ is just short of $1/m^n$. This will allow us to recursively place a scaled, `fattened' version of $X$ in every interval, and then consider a limiting process to obtain a high dimensional continuously squarefree set. If we have a nontrivial solution triple, we can write it as $\alpha x + \delta_1, \alpha y + \delta_2$, and $\alpha z + \delta_3$, with $\delta_1, \delta_2, \delta_3 < \varepsilon$. Expanding the solution leads to
%
\[ \alpha (x - y) + (\delta_1 - \delta_2) = \alpha^2 (x - z)^2 + 2\alpha(x - z)(\delta_1 - \delta_3) + (\delta_1 - \delta_3)^2 \]
%
If $x$, $y$, and $z$ are all distinct, then, as we have discussed, we cannot have $\alpha (x - y) = \alpha^2 (x - z)^2$. if $\alpha$ is chosen close enough to $1/m^n$, then we obtain an approximate inequality
%
\[ |\alpha (x - y) - \alpha^2 (x - z)^2| \geq \alpha^2 \]
%
(we require $\alpha$ to be close enough to $1/n$ for some $n$ to guarantee this). Thus we can guarantee at least two of $x$, $y$, and $z$ are equal to one another if
%
\[ |2\alpha(x - z)(\delta_1 - \delta_3) + (\delta_1 - \delta_3)^2 - (\delta_1 - \delta_2)| < \frac{1}{m^{2n}} \]
%
We calculate that
%
\[ 2\alpha(x - z)(\delta_1 - \delta_3) + (\delta_1 - \delta_3)^2 - (\delta_1 - \delta_2) < 2\alpha(m-1)\varepsilon + \varepsilon^2 + \varepsilon \]
\[ (\delta_1 - \delta_2) - 2\alpha(x - z)(\delta_1 - \delta_3) - (\delta_1 - \delta_3)^2 \leq \varepsilon + 2\alpha(m-1)\varepsilon \]
%
So it suffices to choose $\varepsilon$ such that
%
\[ \varepsilon^2 + [2\alpha(m-1) + 1]\varepsilon \leq \alpha^2 \]
%
This is equivalent to picking
%
\[ \varepsilon \leq \sqrt{\left( \frac{2 \alpha(m - 1) + 1}{2} \right)^2 + \alpha^2} - \frac{2\alpha(m-1) + 1}{2} \approx \frac{\alpha^2}{2\alpha(m-1) + 1} \]

We split the remaining discussion of the bound we must place on $\varepsilon$ into the three cases where two of $x$, $y$, and $z$ are equal, but one is distinct, to determine how small $\varepsilon$ must be to prevent this from happening. Now
%
\begin{itemize}
    \item If $y = z$, but $x$ is distinct, then because we know $\alpha(x - y) = \alpha^2(x - y)^2$ has no solution in $X$, we obtain that (provided $\alpha$ is close enough to $1/m^n$),
    %
    \[ |\alpha(x - y) - \alpha^2(x-y)^2| \geq \alpha^2 \]
    %
    and the same inequality that worked for the case where the three equations are distinct now applies for this case.

    \item If $x = y$, but $z$ is distinct, we are left with the equation
    %
    \[ \delta_1 - \delta_2 = \alpha^2(x - z)^2 + 2\alpha(x - z)(\delta_1 - \delta_3) + (\delta_1 - \delta_3)^2 \]
    %
    Now $\alpha^2(x - z)^2 \geq \alpha^2$, and
    %
    \[ \delta_1 - \delta_2 - 2\alpha(x-z)(\delta_1 - \delta_3) - (\delta_1 - \delta_3)^2 < \varepsilon + 2\alpha(m-1)\varepsilon \]
    %
    so we need the additional constraint $\varepsilon + 2\alpha(m-1)\varepsilon \leq \alpha^2$, which is equivalent to saying
    %
    \[ \varepsilon \leq \frac{\alpha^2}{1 + 2\alpha(m-1)} \]

    \item If $x = z$, but $y$ is distinct, we are left with the equation
    %
    \[ \alpha(x - y) + (\delta_1 - \delta_2) = (\delta_1 - \delta_3)^2 \]
    %
    Now $|\alpha(x-y)| \geq \alpha$, and
    %
    \[ (\delta_1 - \delta_3)^2 - (\delta_1 - \delta_2) < \varepsilon^2 + \varepsilon \]
    \[ (\delta_1 - \delta_2) - (\delta_1 - \delta_3)^2 < \varepsilon \]
    %
    so to avoid this case, we need $\varepsilon^2 + \varepsilon \leq \alpha$, or
    %
    \[ \varepsilon \leq \frac{\sqrt{1 + 4\alpha} - 1}{2} \approx \alpha \]
\end{itemize}
%
Provided $\varepsilon$ is chosen as above, all solutions in $E$ must occur in a common interval. Thus, if we now replace the intervals with a recursive fattened scaling of $X$, all solutions must occur in smaller and smaller intervals. If we choose the size of these scalings to go to zero, these solutions are required to lie in a common interval of length zero, and thus the three values must be equal to one another. Rigorously, we set $\varepsilon \approx 1/m^2$, and $\alpha \approx 1/m$, we can define a recursive construction by setting
%
\[ E_1 = \bigcup_{x \in X} [\alpha x, \alpha x + \varepsilon_1) \]
%
and if we then set $X_n$ to be the set of startpoints of the intervals in $E_n$, then
%
\[ E_{n+1} = \bigcup_{x \in X_n} (x + \alpha^2 E_n) \]
%
Then $\bigcap E_n$ is a continuously squarefree subset. But what is it's dimension?

\section{Idea; Delaying Swaps}

By delaying the removing in the pattern removal queue, we may assume in our dissection methods that we are working with sets with certain properties, i.e. we can swap an interval with a dimension one set avoiding translates.

\section{Squarefree Subsets Using Interval Dissection Methods}

The main idea of Keleti's proof was that, for a function $f$, given a method that takes a sequence of disjoint unions of sets $J_1, \dots, J_N$, each a union of almost disjoint closed intervals of the same length, and gives large subsets $J_n' \subset J_n$, each a union of almost disjoint intervals of a much smaller length, such that $f(x_1, \dots, x_n) \neq 0$ for $x_n \in J_n'$. Then one can find high dimensional subsets $K$ of the real line such that $f(x_1, \dots, x_n) \neq 0$ for a sequence of distinct $x_1, \dots, x_n \in K$. The larger the subsets $J_n'$ are compared to $J_n$, the higher the Hausdorff dimension of $K$. We now try and apply this method to construct large subsets avoiding solutions to the equation $f(x,y,z) = (x - y) - (x - z)^2$. In this case, since solutions to the equation above satisfy $y = x - (x-z)^2$, given $J_1, J_2, J_3$, finding $J_1', J_2', J_3'$ as in the method above is the same as choosing $J_1'$ and $J_3'$ such that the image of $J_1' \times J_3'$ under the map $g(x,z) = x - (x-z)^2$ is small in $J_2$. We begin by discretizing the problem, splitting $J_1$ and $J_3$ into unions of smaller intervals, and then choosing large subsets of these intervals, and finding large intervals of $J_2$ avoiding the images of the startpoints to these intervals.

So suppose that $J_1,J_2$, and $J_3$ are unions of intervals of length $1/M$, for which we may find subsets $A,B \subset [M]$ of the integers such that
%
\[ J_1 = \bigcup_{a \in A} \left[\frac{a}{M}, \frac{a + 1}{M} \right]\ \ \ \ \ J_3 = \bigcup_{b \in B} \left[ \frac{b}{M} , \frac{b + 1}{M} \right] \]
%
If we split $J_1$ and $J_3$ into intervals of length $1/NM$, for some $N \gg M$ to be specified later (though we will assume it is a perfect square), then
%
\[ J_1 = \bigcup_{\substack{a \in A\\0 \leq k < N}} \left[ \frac{Na + k}{NM}, \frac{Na + k}{NM} + \frac{1}{NM} \right]\ \ \ \ \ J_3 = \bigcup_{\substack{b \in A\\0 \leq l < N}} \left[ \frac{Nb + l}{NM}, \frac{Nb + l}{NM} + \frac{1}{NM} \right] \]
%
We now calculate $g$ over the startpoints of these intervals, writing
%
\begin{align*}
    g \left( \frac{Na + k}{NM}, \frac{Nb + l}{NM} \right) &= \frac{Na + k}{NM} - \left( \frac{N(a - b) + (k-l)}{NM} \right)^2\\
    &= \frac{a}{M} - \frac{(a-b)^2}{M^2} + \frac{k}{NM} - \frac{2(a-b)(k-l)}{NM^2} + \frac{(k-l)^2}{(NM)^2}
\end{align*}
%
which splits the terms into their various scales. If we write $m = k - l$, then $m$ can range on the integers in $(-N,N)$, and so, ignoring the first scale of the equation, we are motivated to consider the distribution of the set of points of the form
%
\[ \frac{k}{NM} - \frac{2(a-b)m}{NM^2} + \frac{m^2}{(NM)^2} \]
%
where $k$ is an integer in $[0,N)$, and $m$ an integer in $(-N,N)$. To do this, fix $\varepsilon > 0$. Suppose that we find some value $\alpha \in [0,1]$ such that $S$ intersects
%
\[ \left[ \alpha , \alpha + \frac{1}{N^{1 + \varepsilon}} \right] \]
%
Then there is $k$ and $m$ such that
%
\[ 0 \leq \frac{kNM - 2N(a-b)m + m^2}{(NM)^2} - \alpha \leq \frac{1}{N^{1 + \varepsilon}} \]
%
Write $m = q \sqrt{N} + r$ (remember that we chose $N$ so it's square root is an integer), with $0 \leq r < \sqrt{N}$. Then $m^2 = qN + 2qr \sqrt{N} + r^2$, and if $2qr = Q\sqrt{N} + R$, where $0 \leq R < \sqrt{N}$, then we find
%
\[ -\frac{R}{M^2 N^{3/2}} - \frac{r^2}{(NM)^2} \leq \frac{kM - 2(a-b)m + q + Q}{NM^2} - \alpha \leq \frac{1}{N^{1 + \varepsilon}} - \frac{R}{M^2 N^{3/2}} - \frac{r^2}{(NM)^2} \]
%
Thus
%
\[ d(\alpha, \mathbf{Z}/NM^2) \leq \max \left( \frac{1}{N^{1+\varepsilon}} - \frac{R}{\sqrt{N}} - \frac{r^2}{N}, \frac{R}{M^2 N^{3/2}} + \frac{r^2}{(NM)^2} \right) \]
%
If we now restrict our attention to the set $S$ consisting of the expressions we are studying where $R \leq (\delta_0/2) \sqrt{N}$, $r \leq \sqrt{\delta_0 N/2}$, then if the interval corresponding to $\alpha$ intersects $S$, then
%
\[ d(\alpha, \mathbf{Z}/NM^2) \leq \max \left( \frac{1}{N^{1+\varepsilon}} , \frac{\delta_0}{NM^2} \right) \]
%
If $N^\varepsilon \geq M^2/\delta_0$, then we can force $d(\alpha, \mathbf{Z}/NM^2) \leq \delta_0/NM^2$ for all $\alpha$ intersecting $S$. Thus, if we split $J_2$ into intervals starting at points of the form
%
\[ \frac{k + 1/2}{NM^2} \]
%
each of length $1/N^{1+\varepsilon}$, then provided $\delta_0 < 1/2$, we conclude that these intervals do not contain any points in $S$, since
%
\[ d \left( \frac{k + 1/2}{NM^2}, \mathbf{Z}/NM^2 \right) = \frac{1}{2NM^2} > \frac{\delta_0}{NM^2} \]
%
So we're well on our way to using Pramanik and Fraser's recursive result, since this argument shows that, provided points in $J_1$ and $J_3$ are chosen carefully, we can keep $O_M(1/N^{1 + \varepsilon})$ of each interval in $J_2$, which should lead to a dimension bound arbitrarily close to one.

\section{Finding Many Startpoints of Small Modulus}

To ensure a high dimension corresponding to the recursive construction, it now suffices to show $J_1$ and $J_3$ contain many startpoints corresponding to points in $S$, so that the refinements can be chosen to obtain $O_M(1/N)$ of each of the original intervals. Define $T$ to be the set of all integers $m \in (-N,N)$ with $m = q \sqrt{N} + r$ and $r \leq \sqrt{\delta_0 N/2}$ and $2qr = Q\sqrt{N} + R$ with $R \leq (\delta_0/2) \sqrt{N}$. Because of the uniqueness of the division decomposition, we find $T$ is in one to one correspondence with the set $T'$ of all pairs of integers $(q,r)$, with $q \in (-\sqrt{N},\sqrt{N})$ and $r \in [0,\sqrt{N})$, with $r \leq \sqrt{\delta_0 N/2}$, $2qr = Q \sqrt{N} + R$, and $R \leq (\delta_0/2) \sqrt{N}$. Thus we require some more refined techniques to better upper bound the size of this set.

Let's simplify notation, generalizing the situation. Given a fixed $\varepsilon$, We want to find a large number of integers $n \in (-N,N)$ with a decomposition $n = qr$, where $r \leq \varepsilon \sqrt{N}$, and $q \leq \sqrt{N}$. The following result reduces our problem to understanding the distribution of the smooth integers.

\begin{lemma}
    Fix constants $A,B$, and let $n \leq AN$ be an integer. If all prime factors of $n$ are $\leq BN^{1-\delta}$, then $n$ can be decomposed as $qr$ with $r \leq \varepsilon \sqrt{N}$ and $q \leq \sqrt{N}$.
\end{lemma}
\begin{proof}
    Order the prime factors of $n$ in increasing order as $p_1 \leq p_2 \leq \dots \leq p_K$. Let $r = p_1 \dots p_m$ denote the largest product of the first prime factors such that $r \leq \varepsilon \sqrt{N}$. If $r = n$, we can set $q = 1$, and we're finished. Otherwise, we know $r p_{m+1} > \varepsilon \sqrt{N}$, hence
    %
    \[ r > \frac{\varepsilon \sqrt{N}}{p_{m+1}} \geq \frac{\varepsilon \sqrt{N}}{B N^{1-\delta}} = \frac{\varepsilon}{B} N^{\delta - 1/2} \]
    %
    And if we set $q = n/r$, the inequality above implies
    %
    \[ q < \frac{nB}{\varepsilon} N^{1/2 - \delta} \leq \frac{AB}{\varepsilon} N^{3/2-\delta} \]
    %
    But now we run into a problem, because the only way we can set $q < \sqrt{N}$ while keeping $A$, $B$, and $\varepsilon$ fixed constants is to set $\delta = 1$, and $AB/\varepsilon \leq 1$.
\end{proof}

\begin{remark}
    Should we expect this method to work? Unless there's a particular reason why values of $(q,r)$ should accumulate near $Q = 0$, we should expect to lose all but $N^{-1/2}$ of the $N$ values we started with, so how can we expect to get $\Omega(N)$ values in our analysis. On the other hand, if a number $n$ is suitably smooth, in a linear amount of cases we should be able to divide up primes into two numbers $q$ and $r$ such that $r$ is small and $q$ fits into a suitable value of $Q$, so maybe this method will still work.
\end{remark}

Regardless of whether the lemma above actually holds through, we describe an asymptotic formula for perfect numbers which might come in handy. If $\Psi(N,M)$ denotes the number of integers $n \leq N$ with no prime factor exceeding $M$, then Karl Dickman showed
%
\[ \Psi(N,N^{1/u}) = N \rho(u) + O \left( \frac{u N}{\log N} \right) \]
%
This is essentially linear for a fixed $u$, which could show the set of $(q,r)$ is $\Omega_\varepsilon(N)$, which is what we want. Additional information can be obtained from Hildebrand and Tenenbaum's survey paper ``Integers Without Large Prime Factors''.

\section{A Better Approach}

Remember that we can write a general value in our set as
%
\[ x = \frac{-2(a-b)m}{NM^2} + \frac{q}{NM^2} + \frac{Q}{NM^2} + \frac{R}{N^{3/2} M^2} + \frac{r^2}{N^2M^2} \]
%
with the hope of guaranteeing the existence of many points, rather than forcing $R$ to be small, we now force $R$ to be close to some scaled value of $\sqrt{N}$, 
%
\[ |R - n \varepsilon \sqrt{N}| = \delta \sqrt{N} \leq \varepsilon \sqrt{N} \]
%
Then
%
\[ x = \frac{-2(a-b)m + q + Q + n\varepsilon + \delta}{NM^2} + \frac{r^2}{N^2M^2} \]
%
So
%
\[ d \left( x, \mathbf{Z}/NM^2 + n\varepsilon / NM^2 \right) \leq \frac{\varepsilon}{NM^2} + \frac{1}{4NM^2} = \frac{\varepsilon + 1/4}{NM^2} \]
%
By the pidgeonhole principle, since $R < \sqrt{N}$, there are $1/\varepsilon$ choices for $n$, whereas there are


The choice has the benefit of automatically possessing a lot of points by the pidgeonhole principle,


\endinput
%% The following is a directive for TeXShop to indicate the main file
%%!TEX root = diss.tex

\chapter{Future Work}
\label{ch:Conclusions}

To conclude this thesis, we sketch some ideas developing the theory of `rough sets avoiding patterns', which we introduced in Chapter \ref{ch:RoughSets}. Section 6.1 attempts to exploit additional geometric information about certain rough configurations to find sets with large Hausdorff dimension avoiding patterns, and Section 6.2 finds configuration avoiding sets supported a measure with large Fourier decay.

\section{Low Rank Avoidance}

One way we can extend the results of Chapter \ref{ch:RoughSets} is to utilize addition geometric structure of the rough configuration $\C$ to obtain larger avoiding sets. Recall that in Chapter \ref{ch:RoughSets}, we studied the avoidance problem for configurations with low Minkowski dimension. This condition means precisely that these configurations are efficiently covered by cubes at all scales. The idea of this section is to study configurations which are efficiently covered by other families of geometric objects at all scales. Here, we study the simple setting where our set is efficiently covered by families of thickened hyperplanes or thickened lines. We note that a set $E$ is efficiently covered by a family of thickened parallel hyperplanes at all scales if and only, for a linear transformation $M$ with that hyperplane as a kernel, $M(E)$ has low Minkowski dimension.

\begin{theorem} \label{theorem9063909014901}
    Let $\C \subset \C(\RR)$ be the countable union of sets $\{ \C_i \}$ such that
    %
    \begin{itemize}
        \item For each $i$, there exists $n_i$ such that $\C_i$ is a pre-compact subset of $\C^{n_i}(\RR)$.

        \item There exists an integer $m_i > 0$ and $s_i \in [0,m_i)$, together with a full-rank rational-coefficient linear transformation $M_i: \RR^{n_i} \to \RR^{m_i}$ such that $M_i(\C_i)$ has lower Minkowski dimension at most $s_i$.
    \end{itemize}
    %
    Then there exists a set $X \subset [0,1]$ avoiding $\C$ with Hausdorff dimension at least
    %
    \[ \inf_i \left( \frac{m_i - s_i}{m_i} \right). \]
\end{theorem}

\begin{remarks}
    \
    \begin{enumerate}
        \item[1.] A useful feature of this method is that the resulting set does not depend on the number of points in a configuration. This is a feature only shared by Math\'{e}'s result, Theorem \ref{mathemainresult} in Section 3.3. We exploit this feature later on in this section to find large subsets avoiding a countable family of equations with arbitrarily many variables.

        \item[2.] It might be expected, based on the result of Theorem \ref{mainTheorem}, that one should be able to obtain a set $X \subset [0,1]$ avoiding $\C$ with Hausdorff dimension
        %
        \[ \inf_i \left( \frac{m_i - s_i}{m_i - 1} \right), \]
        %
        whenever $s_i \geq 1$ for all $i$. We plan to pursue whether this conjecture is true in further research.

        \item[3.] Compared to Theorem \ref{mainTheorem}, this result only applies in the one-dimensional configuration avoidance setting. We also plan to find higher dimensional analogues to this theorem, when $d > 1$, in the near future.
    \end{enumerate}
\end{remarks}

For purpose of brevity, here we only describe a solution to the discretized version of the problem. This can be fleshed out into a full proof of Theorem \ref{theorem9063909014901} by techniques analogous to those given in Chapters \ref{ch:RelatedWork} and \ref{ch:RoughSets}. Thus we discuss a single linear transformation $M: \RR^{dn} \to \RR^m$, and try to avoid a discretized version of a low dimensional set.

Before we describe the discretized result, let us simplify the problem slightly. Since our transformation $M$ has full rank, we may find indices
%
\[ i_1, \dots, i_m \in \{ 1, \dots, n \} \]
%
such that the transformation $M$ is invertible when restricted to the span of $\{ e_{i_1}, \dots, e_{i_m} \}$. By an affine change of coordinates in the range of $M$, which preserves the Minkowski dimension of any set, we may assume without loss of generality that $M(e_{i_j}) = e_j$ for each $1 \leq j \leq m$.

\begin{theorem} \label{theorem059891891829}
    Fix $s \in [0,m)$ and $\varepsilon \in [0, (m-s)/2)$. Let $T_1, \dots, T_n \subset [0,1]$ be disjoint, $\DQ_k$ discretized sets, and let $B \subset \RR^m$ be a $\DQ_{k+1}$ discretized set such that
    %
    \begin{equation} \label{equation6091904232093}
        \#(\DQ_{k+1}(B)) \leq N_{k+1}^{s + \varepsilon}.
    \end{equation}
    %
    Then there exists a constant $C(n,m,M) > 0$, and an integer constant $A(M) > 0$, such that if $A(M) \divides N_{k+1}$, and 
    %
    \begin{equation} \label{equation19024u1298352389}
        N_{k+1} > C(n,m,M) \cdot M_{k+1}^{\frac{m}{m - (s + \varepsilon)}}.
    \end{equation}
    %
    then there exists $\DQ_{k+1}$ discretized sets $S_1 \subset T_1$, \dots, $S_n \subset T_n$ such that
    %
    \begin{enumerate}
        \item For any collection of $n$ distinct cubes $Q_i \in \DQ_{k+1}(S_i)$,
        %
        \[ Q_1 \times \dots \times Q_n \not \in \DQ_{k+1}(B). \]

        \item For each $i$, and for each $Q \in \DQ_k(T_i)$, there exists $\DR_Q \subset \DR_{k+1}(Q)$ such that
        %
        \[ \#(\DR_Q) \geq \frac{\#(\DR_{k+1}(Q))}{A(M)}, \]
        %
        and if $R \in \DR_{k+1}(Q)$,
        %
        \[ \#(\DQ_{k+1}(R \cap S_i)) = \begin{cases} 1 &: R \in \DR_Q, \\ 0 &: R \not \in \DR_Q. \end{cases} \]
    \end{enumerate}
\end{theorem}
\begin{proof}
    For each $i \not \in \{ i_1, \dots, i_m \}$, there are rational numbers $a_{ij} = p_{ij}/q_{ij} \in \mathbf{Q}$ such that $M(e_i) = \sum a_{ij} e_j$. Set $A(M) = \prod_{ij} q_{ij}$. For each interval $R \in \DR_{k+1}(T_i)$, we let
    %
    \[ a(R) \in \{ 0, \dots, N_1 \dots N_k M_{k+1} - 1 \} \]
    %
    be the unique integer such that
    %
    \[ R = \left[ \frac{a(R)}{N_1 \dots N_k M_{k+1}}, \frac{a(R) + 1}{N_1 \dots N_k M_{k+1}} \right]. \]
    %
    Let $X \in \{ 0, \dots, N_{k+1}/M_{k+1} - 1 \}^m$. For each $1 \leq j \leq m$, define
    %
    \[ S_{i_j}(X) = \bigcup_{R \in \DR_{k+1}(T_{i_j})} \left[ \frac{a(R)}{N_1 \dots N_k M_{k+1}} + \frac{X_j}{N_1 \dots N_{k+1}}, \frac{a(R)}{N_1 \dots N_k M_{k+1}} + \frac{X_j + 1}{N_1 \dots N_{k+1}} \right]. \]
    %
    For $i \not \in \{ i_1, \dots, i_m \}$, define
    %
    \[ S_i(X) = \bigcup_{\substack{R \in \DR_{k+1}(T_i)\\ \prod q_{ij} \divides a(R)}} \left[ \frac{a(R)}{N_1 \dots N_k M_{k+1}}, \frac{a(R)}{N_1 \dots N_k M_{k+1}} + \frac{1}{N_1 \dots N_{k+1}} \right] \]
    %
    For each $i$, we let $\mathcal{S}_i(X)$ denote the set of startpoints to intervals in $S_i$. Then
    %
    \[ \mathcal{S}_{i_j}(X) \subset \frac{\ZZ}{N_1 \dots N_k M_{k+1}} + \frac{X_j}{N_1 \dots N_{k+1}} \]
    %
    and for $i \not \in \{ i_1, \dots, i_m \}$,
    %
    \[ \mathcal{S}_i(X) \subset \frac{\prod q_{ij} \ZZ}{N_1 \dots N_k M_{k+1}}. \]
    %
    It therefore follows that if
    %
    \[ \mathcal{A}(X) = M(\mathcal{S}_1(X) \times \dots \times \mathcal{S}_n(X)), \]
    %
    then
    %
    \[ \mathcal{A}(X) \subset \frac{\ZZ^m}{N_1 \dots N_k M_{k+1}} + \frac{X}{N_1 \dots N_{k+1}}. \]
    %
    In particular, if $X \neq X'$, $\mathcal{A}(X)$ and $\mathcal{A}(X')$ are disjoint. Equation \eqref{equation19024u1298352389} implies there is a constant $C(n,m,M)$, such that
    %
    \begin{equation} \label{equation69129319031209}
    \begin{split}
        \#& \left\{ n \in \mathbf{Z}^m : d \left( \frac{n}{N_1 \dots N_{k+1}}, B \right) \leq \frac{2}{\sqrt{d} \cdot \| M \|} \frac{1}{N_1 \dots N_{k+1}} \right\}\\
        &\ \ \ \ \ \ \ \ \ \ \ \ \ \ \ \ \ \ \ \ \ \ \ \ \ \ \ \ \ \ \ \ \ \ \leq C(n,m,M)^{m - (s + \varepsilon)} \cdot N_{k+1}^{s + \varepsilon}.
    \end{split}
    \end{equation}
    %
    Applying the pigeonhole principle, \eqref{equation6091904232093}, and \eqref{equation69129319031209}, there exists some value $X_0$ such that
    %
    \begin{align*}
        \# \left\{ n \in \mathbf{A}(X_0) : d(n,B) \leq \frac{2}{\sqrt{d} \cdot \| M \|} \frac{1}{N_1 \dots N_{k+1}} \right\} &\leq \frac{C(n,m,M)^{m - (s + \varepsilon)} \cdot N_{k+1}^{s + \varepsilon}}{(N_{k+1}/M_{k+1})^m}\\
        &\leq \frac{C(n,m,M)^{m - (s + \varepsilon)} \cdot M_{k+1}^m}{N_{k+1}^{m - (s + \varepsilon)}} < 1.
    \end{align*}
    %
    In particular, this set is actually empty. But this means that the set
    %
    \[ M(S_1(X_0) \times \dots \times S_n(X_0)) \]
    %
    is disjoint from $B$. Taking $S_i = S_i(X_0)$ for each $i$ completes the proof.
\end{proof}

Before we move on, consider one application of Theorem \ref{theorem9063909014901}, which gives an extension of Theorem \ref{sumset-application} to arbitrarily large sums.

\begin{theorem}
    Let $Y \subset \RR$ be a countable union of pre-compact sets with lower Minkowski dimension at most $t$. Then there exists a set $X \subset \RR$ with Hausdorff dimension at least $1 - t$ such that for any integer $n > 0$, for any $a_1, \dots, a_n \in \QQ$, and for any $x_1, \dots, x_n \in X$,
    %
    \[ (a_1X + \dots + a_n X) \cap Y \subset (0). \]
\end{theorem}
\begin{proof}
    Let $Y = \bigcup_{i = 1}^\infty Y_i$, where each $Y_i$ has lower Minkowski dimension at most $t$. For each $n$, $i$, and $a = (a_1, \dots, a_n) \in \QQ^n$ with $a \neq 0$, let
    %
    \[ \C_{n,a,i} = \{ (x_1, \dots, x_n) \in \C^n : a_1x_1 + \dots + a_nx_n \in Y_i \}, \]
    %
    and let $\C = \bigcup \C_{n,a,i}$. Let $T_{n,a}(x_1,\dots,x_n)$ be the linear map given by
    %
    \[ T_{n,a}(x_1,\dots,x_n) = a_1x_1 + \dots + a_nx_n. \]
    %
    Then $T_{n,a}$ is nonzero, and $T_{n,a}(\C_{n,i,a})$ how lower Minkowski dimension at most $t$. Applying Theorem \ref{theorem9063909014901}, we obtain a set $X \subset [0,1]$ avoiding $\C$ with Hausdorff dimension at least $1 - t$.

    We prove $X$ satisfies the conclusions of this theorem by induction on $n$. Consider the case $n = 1$, and fix $a \in \QQ$. If $a \neq 0$, then because $X$ avoids $\C_{n,a,i}$ for each $i$, if $x \in X$, $ax \not \in Y$, so $aX \cap Y = \emptyset$. If $a = 0$, then $aX = 0$, so $(aX) \cap Y \subset (0)$.

    In general, consider $a = (a_1, \dots, a_{n+1}) \in \QQ^{n+1}$. If $a \neq 0$, then because $X$ avoids $\C_{n,a,i}$ for each $i$, we know if $x_1, \dots, x_{n+1} \in X$ are distinct, then $a_1 x_1 + \dots + a_{n+1} x_{n+1} \not \in Y$. If the values $x_1, \dots, x_{n+1} \in X$ are not distinct, then by rearranging both the values $\{ x_i \}$ and $\{ a_i \}$, we may without loss of generality assume that $x_n = x_{n+1}$. Then
    %
    \begin{align*}
        a_1 x_1 + \dots + a_{n+1} x_{n+1} &= a_1 x_1 + \dots + a_{n-1} x_{n-1} + (a_n + a_{n+1}) x_n\\
        &\subset (a_1 X + \dots + a_{n-1} X + (a_n + a_{n+1}) X).
    \end{align*}
    %
    By induction,
    %
    \[ (a_1 X + \dots + a_{n-1} X + (a_n + a_{n+1}) X) \cap Y \subset (0), \]
    %
    so we conclude that either $a_1 x_1 + \dots + a_{n+1} x_{n+1} \not \in Y$, or $a_1x_1 + \dots + a_{n+1} x_{n+1} = 0$. The only remaining case we have not covered is if $a \in \QQ^{n+1}$ is equal to zero. But in this case,
    %
    \[ (a_1 X + \dots + a_n X) = (0 + \dots + 0) = 0, \]
    %
    and so it is trivial that $(a_1 X + \dots + a_n X) \cap Y \subset (0)$.
\end{proof}

\section{Fourier Dimension}

Recently, there has been much interest in determining whether sets with large Fourier dimension can avoid configurations. Results published recently in the literature include \cite{PramanikLaba} and \cite{Shmerkin}. In this Section, we attempt to modify the procedure of Theorem \ref{mainTheorem} to obtain a set with large Fourier dimension. We obtain such a result, though with a suboptimal dimension to what we expect from Theorem \ref{mainTheorem}, and only holds in the setting where $d = 1$. We are currently researching methods to resolve the deficiencies in this method.

\begin{theorem} \label{FourierTheorem}
    Suppose $\C$ is a configuration on $\RR$, formed from the countable union of pre-compact sets, each with lower Minkowski dimension at most $s$. Then there exists a set $X \subset [0,1]$ with Fourier dimension at least $(n - s)/n$ avoiding $\C$.
\end{theorem}

We begin with a lemma which uses the Poisson summation theorem to restrict the analysis of the Fourier decay of the probability measures we study to the analysis of frequencies in $\ZZ$.

\begin{lemma} \label{discretefouriermeasures}
    Fix $s \in [0,d]$. Suppose $\mu$ is a compactly supported finite Borel measure on $\RR^d$. Then there exists a constant $A \geq 1$, depending only on the dimension of $d$ and the radius of the support of $\mu$, such that
    %
    \[ \sup_{\xi \in \RR^d} |\xi|^{s/2} |\widehat{\mu}(\xi)| \leq 1 + A \left( \sup_{m \in \ZZ^d} |m|^{s/2} |\widehat{\mu}(m)| \right). \]
\end{lemma}
\begin{proof}
    Without loss of generality, we may assume that $\mu$ is supported on a compact subset of $[1/3,2/3)^d$, since every compactly supported measure is a finite sum of translates of measures of this form. Let
    %
    \[ C = \sup_{m \in \ZZ^d} |m|^{s/2} |\widehat{\mu}(m)|, \]
    %
    which we may assume, without loss of generality, to be finite. Consider the distribution $\Lambda = \sum_{m \in \mathbf{Z}^d} \delta_m$, where $\delta_m$ is the Dirac delta distribution at $m$. Then the Poisson summation formula says that the Fourier transform of $\Lambda$ is itself. If $\psi \in C_c(\RR^d)$ is a bump function supported on $[0,1)^d$, with $\psi(x) = 1$ for $x \in [1/3,2/3)^d$, then $\mu = \psi (\Lambda * \mu)$, so
    %
    \begin{equation} \label{mubounded}
    \begin{split}
        |\widehat{\mu}(\xi)| &= \left| \left[ \widehat{\psi} * (\Lambda \widehat{\mu}) \right](\xi) \right|\\
        &= \left| \sum_{m \in \mathbf{Z}^d} \widehat{\mu}(m)(\widehat{\psi} * \delta_m)(\xi) \right|\\
        &= \left| \sum_{m \in \mathbf{Z}^d} \widehat{\mu}(m) \widehat{\psi}(\xi - m) \right|.
%       &\lesssim \sum_{n \in \mathbf{Z}^d} |\widehat{\mu}(n)| \prod_{i = 1}^d \frac{1}{1 + |n_i - \xi_i|}
    \end{split}
    \end{equation}
    %
    Since $\psi$ is smooth, we know that for all $\eta \in \RR^d$, $|\widehat{\psi}(\eta)| \lesssim 1/|\eta|^{d+1}$. If we perform a dyadic decomposition, we find
    %
    \begin{equation}
        \label{calculation1}
    \begin{split}
        \sum_{1 \leq |m - \xi| \leq |\xi|/2} |\widehat{\mu}(m)| |\widehat{\psi}(\xi - m)| &\leq C \sum_{1 \leq |m - \xi| \leq |\xi|/2} |\xi|^{-s/2} |\widehat{\psi}(\xi - m)|\\
        &\lesssim C \sum_{k = 1}^{\log |\xi|} \sum_{\frac{|\xi|}{2^{k+1}} \leq |m - \xi| \leq \frac{|\xi|}{2^{k}}} |\xi|^{-s/2} \left( 2^k/|\xi| \right)^{d+1}\\
        &\lesssim C \sum_{k = 1}^{\log |\xi|} |\xi|^{-s/2} (2^k / |\xi| ) \lesssim C |\xi|^{-s/2}.
    \end{split}
    \end{equation}
    %
    There are $O_d(1)$ points $m \in \mathbf{Z}^d$ with $|m - \xi| \leq 1$, so if $|\xi| \geq 2$,
    %
    \begin{equation} \label{calculation2}
        \sum_{|m - \xi| \leq 1} |\widehat{\mu}(m)| |\widehat{\psi}(m - \xi)| \lesssim C |\xi|^{-s/2}.
    \end{equation}
    %
    We can also perform another dyadic decomposition, using the fact that for all $\eta \in \RR^d$, $|\widehat{\psi}(\eta)| \lesssim 1/|\eta|^{2d}$, to find that
    %
    \begin{equation} \label{calculation3}
    \begin{split}
        \sum_{|m - \xi| \geq |\xi|/2} |\widehat{\mu}(m)| |\widehat{\psi}(m - \xi)| &\lesssim \sum_{k = 0}^\infty \sum_{|\xi| 2^{k-1} \leq |m - \xi| \leq |\xi| 2^k} \frac{|\widehat{\mu}(m)|}{|\xi|^{2d} 2^{2dk}}\\
        &\lesssim C \sum_{k = 0}^\infty |\xi|^{-d} 2^{-dk} \lesssim C |\xi|^{-d}.
    \end{split}
    \end{equation}
    %
    Combining \eqref{calculation1}, \eqref{calculation2}, and \eqref{calculation3} with \eqref{mubounded}, we conclude that there exists a constant $A \geq 1$ depending only on the dimension $d$ such that if $|\xi| \geq 2$,
    %
    \begin{equation} \label{endequation53}
        |\widehat{\mu}(\xi)| \leq A \cdot C \cdot |\xi|^{-s/2}.
    \end{equation}
    %
    Since $\| \widehat{\mu} \|_{L^\infty(\RR^d)} \leq 1$, \eqref{endequation53} actually holds for all $\xi \in \RR^d$, provided $C \geq 1$.
\end{proof}

Our goal now is to carefully modify the discrete selection strategy and discretized probability measures we use to obtain have sharp control over the Fourier transform of these measures at each scale of our construction. A key strategy is to obtain high probability bounds controlling the Fourier transform of functions on the sets we choose using Hoeffding's inequality.

\begin{theorem}[Hoeffding's Inequality]
    Let $\{ X_i \}$ be an independent family of $N$ mean-zero random variables, and let $A > 0$ be a constant such that $\| X_i \|_\infty \leq A$ for each $i$. Then for each $t > 0$,
    %
    \[ \PP \left( \left| \frac{1}{N} \sum_{i = 1}^N X_i \right| \geq t \right) \leq 2 \exp \left( (N/A^2) \cdot (- t^2) \right). \]
\end{theorem}

As with Theorem \ref{mainTheorem}, we perform a multi-scale analysis, using the notations introduced in Section \ref{sec:Dyadics}. Lemma \ref{discretefouriermeasures} implies that we only need control over integer-valued frequencies. The discretized measures $\{ \nu_k \}$ we select are, for each $k$, a sum of point mass distributions at the points $\ZZ/N_1 \dots N_k$. Therefore, $\widehat{\nu_k}$ will be $N_1 \dots N_k$ periodic, in the sense that for any $m \in (N_1 \dots N_k) \ZZ$ and $\xi \in \RR$, $\widehat{\nu_k}(\xi + m) = \widehat{\nu_k}(\xi)$. Since we are only concerned with integer valued frequencies, it will therefore suffice to control the Fourier transform of $\nu_k$ on frequencies lying in $\{ 1, \dots, N_1 \dots N_k \}$.

In the discrete lemma below, we rely on a variant of the proof strategy of Theorem 2.1 of \cite{Shmerkin}, but modified so that we can allow the branching factors $\{ N_k \}$ to increase arbitrarily fast. For each $\DQ_k$ discretized set $E \subset [0,1]$, we define a probability measure
%
\[ \nu_E = \frac{1}{\#(\DQ_k(E))} \sum_{Q \in \DQ_k(E)} \delta(a(Q)), \]
%
where for each $x \in \RR$, $\delta(x)$ is the Dirac delta measure at $x$, and for each $Q \in \DQ_k$, $a(Q)$ is the startpoint of the interval $Q$. Also, for each $k$, we define a probability measure
%
\[ \eta_k = \frac{1}{N_k} \sum_{i_1, \dots, i_d = 0}^{N_k - 1} \delta \left( \frac{i}{N_1 \dots N_k} \right).  \]
%
The purpose of introducing $\eta_k$ is so that, given a measure $\mu$ which is a sum of point mass distributions in $\ZZ/N_1 \dots N_k$, the probability measure $\mu * \eta_{k+1}$ is a sum of point mass distributions in $\ZZ/N_1 \dots N_{k+1}$, uniformly distributed at the scale $1/N_1 \dots N_{k+1}$.

%Our goal now is now to carefully modify the discrete selection strategy and discretized probability measures we use, so that with high probability, the measures have the appropriate Fourier decay for the Fourier dimension bound we wish to obtain. Surprisingly, here we only need to perform a single scale analysis with the family of cubes $\DQ^d$, rather than a multi scale analysis involving the cubes $\DQ^d$ and $\DR^d$ as in Chapter \ref{ch:RoughSets}.

\begin{lemma} \label{discreteFourierBuildingBlock}
    Fix $s \in [1,dn)$, and $\varepsilon \in [0,(n-s)/4)$. Let $T \subset \RR$ be a nonempty, $\DQ_k$ discretized set, and let $B \subset \RR^n$ be a nonempty $\DQ_{k+1}$ discretized set such that
    %
    \begin{equation} \label{equation982589128942189}
    \begin{split}
        \#(\DQ_{k+1}(B)) \leq N_{k+1}^{s + \varepsilon}.
    \end{split}
    \end{equation}
    %  \leq N_{k+1}^d
    %
    Provided that
    %
    \begin{equation} \label{equation5523786128439}
        M_{k+1} \leq N_{k+1}^{\frac{n-s-2\varepsilon}{n}} \leq 2 M_{k+1},
    \end{equation}
    %
    %\begin{equation} \label{equation5523786128439}
    %    M_{k+1}^{\frac{n}{n - s - 2\varepsilon}} \leq N_{k+1} \leq 2 M_{k+1}^{\frac{n}{n - s - 2\varepsilon}},
    %\end{equation}
    %
    \begin{equation} \label{equation189248914891}
        \quad N_{k+1} \geq 3^{1/\varepsilon},
    \end{equation}
    %
    \begin{equation} \label{equation8941894189238912}
        N_{k+1} \geq \exp \left( \left( \frac{4n}{n-s} \right)^4 N_1 \dots N_k \right),
    \end{equation}
    %
    and
    %
    \begin{equation} \label{equation77871247817841278}
        N_{k+1} \geq (1/\varepsilon)^{1/\varepsilon},
    \end{equation}
    %
    there exists a universal constant $A(n,s)$ and a $\DQ_{k+1}$ discretized set $S \subset T$, satisfying the following properties:
    %
    \begin{enumerate}
        \item[(A)] For any collection of $n$ distinct cubes $Q_1, \dots, Q_n \in \DQ_{k+1}(S)$,
        %
        \[ Q_1 \times \dots \times Q_n \not \in \DQ_{k+1}(B). \]

        \item[(B)] For any $m \in \ZZ$,
        %
        \[ |\widehat{\nu_S}(m) - \widehat{\eta_{k+1}}(m) \widehat{\nu_T}(m)| \leq A(n,s) \cdot (N_1 \dots N_{k+1})^{-\frac{n - s}{2n} + 2\varepsilon}. \]
    \end{enumerate}
\end{lemma}
\begin{proof}
    For each $R \in \DR_{k+1}(T)$, let $Q_R$ be randomly selected from $\DQ_{k+1}(R)$, such that the collection $\{ Q_R \}$ forms an independent family of random variables. Then, set $S = \bigcup \{ Q_R: R \in \DR_{k+1}(T) \}$. We then have
    %
    \begin{equation} \label{equation6900921094190290}
        \#(\DQ_{k+1}(S)) = \#(\DR_{k+1}(T)) = M_{k+1} \cdot \DQ_k(T).
    \end{equation}
    %
    Without loss of generality, removing cubes from $B$ if necessary, we may assume that for every cube $Q_1 \times \dots \times Q_n \in \DQ_{k+1}(B)$, the values $Q_1, \dots, Q_n$ occur in distinct intervals in $\DR_{k+1}(T)$. In particular, given any such cube, just as in Lemma \ref{discretelemma}, we have
    %
    \begin{equation} \label{equation12043910293120909}
        \mathbf{P}(Q_1 \times \dots Q_n \in \DQ_{k+1}(S^n)) = (M_{k+1}/N_{k+1})^n.
    \end{equation}
    %
    Thus \eqref{equation982589128942189}, \eqref{equation5523786128439}, and \eqref{equation12043910293120909} imply
    %
    \begin{equation} \label{equation999992482}
        \mathbf{E} \left[ \#(\DQ_{k+1}(B) \cap \DQ_{k+1}(S^n)) \right] \leq M_{k+1}^n/N_{k+1}^{n - (s + \varepsilon)} \leq 1/N_{k+1}^\varepsilon.
    \end{equation}
    %
    Markov's inequality, together with \eqref{equation189248914891} and \eqref{equation999992482}, imply
    %
    \begin{equation} \label{fourierdim2}
    \begin{split}
        \mathbf{P}(\DQ_{k+1}(B) \cap \DQ_{k+1}(S^n) \neq \emptyset) &= \mathbf{P}(\# (\DQ_{k+1}(B) \cap \DQ_{k+1}(S^n)) \geq 1)\\
        &\leq 1/N_{k+1}^\varepsilon \leq 1/3.
    \end{split}
    \end{equation}
    %
    Thus $\DQ_{k+1}(S^n)$ is disjoint from $\DQ_{k+1}(B)$ with high probability.

    Now we analyze the Fourier transform of the measures $\nu_S$. For each cube $R \in \DR_{k+1}(T)$, and for each $m \in \ZZ$, let
    %
    \[ A_R(m) = e^{\frac{-2 \pi i m \cdot a(Q_R)}{N_1 \dots N_{k+1}}} - \frac{1}{N_{k+1}} \sum_{l = 0}^{N_{k+1} - 1} e^{\frac{-2 \pi i m \cdot [N_{k+1} a(Q) + l]}{N_1 \dots N_{k+1}}}. \]
    %
    Then $\EE[A_R(m)] = 0$, $|A_R(m)| \leq 2$ for each $m$, and
    %
    \[ \widehat{\nu_S}(m) - \widehat{\eta_{k+1}}(m) \widehat{\nu_T}(m) = \frac{1}{\#(\DR_{k+1}(T))} \sum_{R \in \DR_{k+1}(T)} A_R(m). \]
    %
    Fix a particular value of $m$. Since the random variables $\{ A_R(m) : R \in \DR_{k+1}(T) \}$ are bounded and independent from one another, we can apply Hoeffding's inequality with \eqref{equation6900921094190290} to conclude that for each $t > 0$,
    %
    \begin{equation} \label{equation5551902402919120}
    \begin{split}
        \PP \left( |\widehat{\nu_S}(m) - \widehat{\eta_{k+1}}(m) \widehat{\nu_T}(m)| \geq t \right) &\leq 2 \exp \left( \frac{- \#(\DR_{k+1}(T)) t^2}{4} \right)\\
        &= 2 \exp \left( \frac{- \#(\DQ_k(T)) M_{k+1} t^2}{4} \right).
    \end{split}
    \end{equation}
    %
    The function $\widehat{\nu_S} - \widehat{\eta_{k+1}} \widehat{\nu_T}$ is $N_1 \dots N_{k+1}$ periodic. Thus, to uniformly bound this function, we need only bound the function over $N_1 \dots N_{k+1}$ values of $m$. Applying a union bound with \eqref{equation5551902402919120}, we conclude
    %
    \begin{equation} \label{equation6662410242191209}
        \PP \left( \| \widehat{\nu_S} - \widehat{\eta_{k+1}} \widehat{\nu_T} \|_{L^\infty(\ZZ)} \geq t \right) \leq 2 N_1 \dots N_{k+1} \exp \left( \frac{- \#(\DQ_k(T)) M_{k+1} t^2}{4} \right).
    \end{equation}
    %
    In particular, \eqref{equation5523786128439}, applied to \eqref{equation6662410242191209}, shows
    %
    \begin{align*}
        \PP & \left( \| \widehat{\nu_S} - \widehat{\eta_{k+1}} \widehat{\nu_T} \|_{L^\infty(\ZZ)} \geq (N_1 \dots N_k M_{k+1})^{-1/2} \log(M_{k+1}) \right)\\
        &\ \ \ \ \leq 2N_1 \dots N_{k+1} \exp \left( - \frac{\#(\DQ_k(T)) \log(M_{k+1})^2}{4 N_1 \dots N_k} \right)\\
        &\ \ \ \ = 2 N_1 \dots N_k \exp \left( \log(N_{k+1}) - \frac{\log(M_{k+1})^2}{4 N_1 \dots N_k} \right)\\
        %&\ \ \ \ \leq 2 N_1 \dots N_k \exp \left( \log(N_{k+1}) - \frac{\log \left( N_{k+1}^{\frac{n-s-2\varepsilon}{2n}}/2 \right)^2}{4 N_1 \dots N_k} \right)\\
        &\ \ \ \ \leq 2 N_1 \dots N_k \exp \left( \log(N_{k+1}) - \left[ \left( \frac{n - s}{4n} \right) \log(N_{k+1}) - \log(2) \right]^2 \frac{1}{N_1 \dots N_k} \right).
    \end{align*}
    % A = 2N_1 ... N_k
    % D = 1/N_1 ... N_k
    % B = (n-s/4n)
    % C = log(2)
    %
    % A e(X - D (BX + C)^2) <= 3
    % X - D(BX + C)^2 <= log(3/A)
    % X - (B^2D)X^2 - 2BCDX - C^2D <= log(3/A)
    % (B^2D) X^2 + (2BCD - 1)X + [log(3/A) - C^2D] >= 0
    % X >= (1/2B^2D - C/B) + sqrt((2BCD - 1)^2 - 4(B^2D)(log(3/A) - C^2D))/2B^2D
    % X >= (4n/n-s)^4[N_1 ... N_k]
    % N_{k+1} \geq \exp \left( (4n/n-s)^4 [N_1 ... N_k] \right)
    % 
    Thus \eqref{equation8941894189238912} implies
    %
    \begin{equation} \label{equation90120931902390190}
        \PP \left( \| \widehat{\nu_S} - \widehat{\eta_{k+1}} \widehat{\nu_T} \|_{L^\infty(\ZZ)} \geq (N_1 \dots N_k M_{k+1})^{-1/2} \log(M_{k+1}) \right) \leq 1/3.
    \end{equation}
    %
    Taking a union bound over \eqref{fourierdim2} and \eqref{equation90120931902390190}, we conclude that there is a non-zero probability that the set $S$ satisfies Property (A), and
    %
    \[ \| \widehat{\nu_S} - \widehat{\eta_{k+1}} \widehat{\nu_T} \|_{L^\infty(\ZZ)} \leq (N_1 \dots N_k M_{k+1})^{-1/2} \log(M_{k+1}). \]
    %
    Since \eqref{equation77871247817841278} holds,
    % \log(x) \leq x^\varepsilon
    % 1/\varepsilon^{1/\varepsilon} \leq x
    % N_{k+1} \geq 1/\varepsilon^{1/\varepsilon}
    %
    \begin{align*}
        (N_1 \dots N_k M_{k+1})^{-1/2} \log(M_{k+1}) &\lesssim_{n,s} \log(N_{k+1}) (N_1 \dots N_{k+1})^{- \frac{n-s-2\varepsilon}{2n}}\\
        &\leq (N_1 \dots N_{k+1})^{- \frac{n-s}{2n} + 2\varepsilon}.
    \end{align*}
    %
    Thus the set $S$ also satisfies Property (B) with an appropriately chosen constant $A(n,s)$.
\end{proof}

\begin{comment}

Let us describe the measures we construct. For each $k$, we let
%
\[ \psi_k = N_1 \dots N_k \cdot \mathbf{I}_{\left[ 0, \frac{1}{N_1 \dots N_k} \right]}. \]
%
Then for each $\xi \in \RR^d$,
%
\[ |\widehat{\psi_k}(\xi)| = \left| N_1 \dots N_k \cdot \frac{e^{- \frac{2 \pi i \xi}{N_1 \dots N_k}} - 1}{- 2\pi i \xi} \right| \lesssim \frac{N_1 \dots N_k}{1 + |\xi|}. \]
%
Thus $\widehat{\psi_k}$ has fast decay for $|\xi| \geq N_1 \dots N_k$. For any $\DQ_k$ discretized set $E$, we let $\mu_E$ by the absolutely continuous probability measure with density function
%
\[ \frac{d\mu_E}{dx} = \frac{1}{\#(\DQ_k(E))} \sum_{Q \in \DQ_k(E)} \mathbf{I}_Q. \]
%
Then $\mu_E$ is supported on $E$. Since $\mu_E$ can be viewed as a convolution of $\psi_k$ with a discrete probability measure at the left-hand edges of the intervals in $\DQ_k(E)$, for each $\xi \in \RR^d$,
%
\[ |\widehat{\mu_E}(\xi)| \leq |\widehat{\psi_k}(\xi)| \lesssim \frac{N_1 \dots N_k}{1 + |\xi|}. \]
%
Thus $\widehat{\mu_E}$ has large decay for frequencies of large magnitude, and it suffices to construct $E$ such that we can control the Fourier transform of $\mu_E$ on low magnitude frequencies. To prove this, we take $E$ to be a random set, and apply Hoeffding's inequality to obtain tail bounds on the magnitude of the Fourier transform.

\begin{theorem}[Hoeffding's Inequality]
    Let $\{ X_i \}$ be a family of $N$ independant, mean zero random variables, such that $|X_i| \leq B$ for all $i$. Then
    %
    \[ \PP \left( \sum X_i \geq t \right) \leq 2 \exp \left( \frac{-t^2}{2B^2 N} \right). \]
\end{theorem}

\begin{lemma} \label{discreteFourierBuildingBlock}
    Fix $s \in [1,n)$, $\varepsilon_1 \in [0,(n-s)/4)$, and $\varepsilon_2 \in (0,\infty)$. Let $T \subset \RR$ be a nonempty, $\DQ_k$ discretized set, and let $B \subset \RR^n$ be a nonempty $\DQ_{k+1}$ discretized set such that
    %
    \begin{equation} \label{equation982589128942189}
    \begin{split}
        \#(\DQ_{k+1}(B)) \leq N_{k+1}^{s + \varepsilon_1}.
    \end{split}
    \end{equation}
    %  \leq N_{k+1}^d
    %
    Then there exists a constant $A(d,n,s)$ such that, provided
    %
    \begin{equation} \label{equation5523786128439}
        M_{k+1}^{\frac{dn}{dn - s - 2\varepsilon}} \leq N_{k+1} \leq 2 M_{k+1}^{\frac{dn}{dn - s - 2\varepsilon}},
    \end{equation}
    %
    \begin{equation} \label{equation189248914891}
        \quad N_{k+1} \geq 3^{1/\varepsilon},
    \end{equation}
    %
    and
    %
    \begin{equation} \label{equation77871247817841278}
        N_{k+1} \geq 3^{1/d} \left( \left\lceil \frac{A(d,n,s)}{\varepsilon} \right\rceil! \right)^{3/d} (N_1 \dots N_k),
    \end{equation}
    %
    then there exists a $\DQ_{k+1}$ discretized set $S \subset T$, satisfying the following properties:
    %
    \begin{enumerate}
        \item[(A)] For any collection of $n$ distinct cubes $Q_1, \dots, Q_n \in \DQ_{k+1}(S)$,
        %
        \[ Q_1 \times \dots \times Q_n \not \in \DQ_{k+1}(B). \]

        \item[(B)] For any $m \in \ZZ$,
        %
        \[ |\widehat{\mu_S}(m) - \widehat{\mu_T}(m)| \leq N_{k+1}^{-(1 - \varepsilon_2) \frac{n - s}{2n}}. \]
    \end{enumerate}
\end{lemma}
\begin{proof}
    For each $R \in \DR_{k+1}(T)$, let $Q_R$ be randomly selected from $\DQ_{k+1}(R)$, independently from all other selections $Q_{R'}$. Then, set $S = \bigcup \{ Q_R: R \in \DR_{k+1}(T) \}$. We then have
    %
    \[ \#(\DQ_{k+1}(S)) = \#(\DR_{k+1}(T)) = M_{k+1}^d \DQ_k(T). \]
    %
    Without loss of generality, removing cubes from $B$ if necessary, we may assume that every cube $Q_1 \times \dots \times Q_n \in \DQ_{k+1}(B)$, the values $Q_1, \dots, Q_n$ are distinct. In particular, given any such cube, just as in Lemma \ref{discretelemma}, we have
    %
    \[ \mathbf{P}(Q_1 \times \dots Q_n \in \DQ_{k+1}(S^n)) = (M_{k+1}/N_{k+1})^n. \]
    %
    Thus \eqref{equation982589128942189} and \eqref{equation5523786128439} imply
    %
    \[ \mathbf{E}(\#(\DQ_{k+1}(B) \cap \DQ_{k+1}(S^n))) \leq M_{k+1}^n/N_{k+1}^{n - (s + \varepsilon_1)} \leq 1/N_{k+1}^{\varepsilon_1}. \]
    %
    Markov's inequality, together with \eqref{equation189248914891} implies
    %
    \begin{equation} \label{fourierdim2}
    \begin{split}
        \mathbf{P}(\DQ_{k+1}(B) \cap \DQ_{k+1}(S^n) \neq \emptyset) &= \mathbf{P}(\# (\DQ_{k+1}(B) \cap \DQ_{k+1}(S^n)) \geq 1)\\
        &\leq 1/N_{k+1}^\varepsilon \leq 1/3.
    \end{split}
    \end{equation}
    %
    Thus $\DQ_{k+1}(S^n)$ is disjoint from $\DQ_{k+1}(B)$ with high probability.

    Now we analyze the Fourier transform of the measure $\mu_S$. For each $R \in \DR_{k+1}(T)$, and $m \in \ZZ$, let
    %
    \[ A_R(m) = \int_R \left( \mu_S(x) - \mu_T(x) \right) e^{-2 \pi i m \cdot x}\; dx. \]
    %
    We note that for each $Q \in \DQ_{k+1}(R)$, and $x \in Q^\circ$,
    %
    \begin{align*}
        \EE(\mu_S(x) - \mu_T(x)) &= (N_1 \dots N_{k+1}) \frac{\PP(Q_R = 1)}{\#(\DQ_{k+1}(S))} - (N_1 \dots N_k) \frac{1}{\#(\DQ_k(T)}\\
        &= (N_1 \dots N_{k+1}) \frac{(M_{k+1}/N_{k+1})}{M_{k+1} \cdot \#(\DQ_k(T))} - (N_1 \dots N_k) \frac{1}{\#(\DQ_k(T))} = 0.
    \end{align*}
    %
    Thus
    %
    \[ \EE[A_R(m)] = \int_R \EE \left[ \mu_S(x) - \mu_T(x) \right] e^{-2 \pi i m \cdot x}\; dx = 0. \]
    %
    Notice that for each fixed $m$, the family $\{ A_R(m) \}$ are a family of $M_{k+1} \cdot \DQ_k(T)$ independant random variables, and
    %
    \[ |A_R(m)| \leq \int_R |\mu_S(x) - \mu_T(x)|\; dx = \frac{2}{M_{k+1}}. \]
    %
    We note that
    %
    \[ \widehat{\mu_S}(m) - \widehat{\mu_T}(m) = \sum_{R \in \DR_{k+1}(T)} A_R(m). \]
    %
    Applying Hoeffding's inequality, we conclude that for each $t > 0$,
    %
    \[ \PP \left( \left| \widehat{\mu_S}(m) - \widehat{\mu_T}(m) \right| \geq t \right) \leq 2 \exp \left( \frac{- M_{k+1}}{8 \cdot \#(\DQ_k(T))} \cdot t^2 \right). \]
    %
    In particular,
    %
    \begin{equation} \label{equation68994812893189}
        \PP \left( \left| \widehat{\mu_S}(m) - \widehat{\mu_T}(m) \right| \geq N_{k+1}^{-(1 - \varepsilon_2) \frac{n-s}{2n}} \right) \leq 2 \exp \left( \frac{-M_{k+1} N_{k+1}^{-(1 - \varepsilon_2) \frac{n-s}{n}}}{8 \cdot \#(\DQ_k(T))} \right).
    \end{equation}
    %
    Applying a union bound with \eqref{equation68994812893189}, we conclude that if
    %
    \[ I = \{ m \in \ZZ : |m| \leq (N_1 \dots N_k)^{A} \}, \]
    %
    then
    %
    \[ \PP \left( \left\| \widehat{\mu_S} - \widehat{\mu_T} \right\|_{L^\infty(I)} \geq N_{k+1}^{- \left(1 - \varepsilon_2 \right) \frac{n-s}{2n}} \right) \leq 2 (N_1 \dots N_k)^A \exp \left( \frac{-M_{k+1} N_{k+1}^{-(1 - \varepsilon_2) \frac{n-s}{n}}}{8 \cdot \#(\DQ_k(T))} \right). \]
    %
    But if $|m| \geq (N_1 \dots N_k)^A$, then
    %
    \[ \PP \left( \widehat{\mu_S}(m) \right) \]



    For each cube $R \in \DR_{k+1}(T)$, and for each $m$, let
    %
    \[ A_R(m) = e^{\frac{-2 \pi i m \cdot a(Q_R)}{N_1 \dots N_{k+1}}} - \frac{1}{N_{k+1}^d} \sum_{k_1, \dots, k_d = 0}^N e^{\frac{-2 \pi i m \cdot [N_{k+1} a(Q) + k]}{N_1 \dots N_{k+1}}}. \]
    %
    Then $\EE[A_R(m)] = 0$, $|A_R(m)| \leq 2$ for each $m$, and
    %
    \[ \widehat{\nu_S}(m) - \widehat{\eta_{k+1}}(m) \widehat{\nu_T}(m) = \frac{1}{\#(\DR_{k+1}(T))} \sum_{R \in \DR_{k+1}(T)} A_R(m). \]
    %
    Now fix a particular value of $m$. Since the random variables $A_R(m)$ are independant from one another as $R$ ranges over $\DR_{k+1}(T)$, we can apply Hoeffding's inequality to conclude that for each $t > 0$,
    %
    \[ \PP \left( |\widehat{\nu_S}(m) - \widehat{\eta_{k+1}}(m) \widehat{\nu_T}(m)| \geq t \right) \leq e^{-\#(\DR_{k+1}(T)) t^2/2} = e^{-\#(\DQ_k(T)) M_{k+1}^d t^2/2}. \]
    %
    In particular,
    %
    \[ \PP \left( |\widehat{\nu_S}(m) - \widehat{\eta_{k+1}}(m) \widehat{\nu_T}(m)| \geq M_{k+1}^{-d/2 - \varepsilon} \right) \leq \exp(- \#(\DQ_k(T)) M_{k+1}^\varepsilon / 2 ). \]
    %
    The function $\widehat{\nu_S} - \widehat{\eta_{k+1}}$ is $N_1 \dots N_{k+1}$ periodic. Thus, to uniformly bound $\widehat{\nu_S}(m) - \widehat{\eta_{k+1}}(m)$, we need only bound the function over $(N_1 \dots N_{k+1})^d$ values. Applying a union bound with \eqref{equation5523786128439}, we find that
    %
    \begin{equation} \label{equation81298398120412}
    \begin{split}
        \PP \left( \| \widehat{\nu_S} - \widehat{\eta_{k+1}} \widehat{\nu_T} \|_{L^\infty(\ZZ^d)} \geq M_{k+1}^{- d/2 - \varepsilon} \right) &\leq (N_1 \dots N_{k+1})^d \exp \left( - \#(\DQ_k(T)) M_{k+1}^\varepsilon / 2 \right)\\
        &\leq 2^{\lceil \frac{4d^2n}{\varepsilon(dn - s)} \rceil} \left\lceil \frac{4d^2n}{\varepsilon(dn-s)} \right\rceil! \frac{(N_1 \dots N_{k+1})^d}{M_{k+1}^{4d^2n/(dn - s)}}\\
        &\leq 2^{\lceil 8d^2n/(dn - s) \rceil} \left\lceil \frac{4d^2n}{\varepsilon(dn-s)} \right\rceil! \frac{(N_1 \dots N_k)^d}{N_{k+1}^d}\\
        &\leq \left( \left\lceil \frac{4d^2n}{\varepsilon(dn-s)} \right\rceil! \right)^3 \frac{(N_1 \dots N_k)^d}{N_{k+1}^d}.
    \end{split}
    \end{equation}
    %
    Note that
    %
    \begin{equation} \label{equation8998724714871}
        M_{k+1}^{-d/2 - \varepsilon} \lesssim N_{k+1}^{-(1 - 2\varepsilon/d) \frac{dn - s - 2\varepsilon}{2n}} \leq N_{k+1}^{-(1 - \varepsilon) \frac{dn - s - 2\varepsilon}{2n}}.
    \end{equation}
    %
    If we set $A(d,n,s) = (4d^2n/(dn - s))$, then \eqref{equation77871247817841278}, \eqref{equation81298398120412}, and \eqref{equation8998724714871} allow us to conclude that
    %
    \begin{equation} \label{equation1241751}
        \PP \left( \| \widehat{\nu_S} - \widehat{\eta_{k+1}} \widehat{\nu_T} \|_{L^\infty(\ZZ^d)} \geq N_{k+1}^{-(1 - \varepsilon)} \right) \leq 1/3.
    \end{equation}
    %
    Taking a union bound over \eqref{fourierdim2} and \eqref{equation1241751}, we find there is a non-zero probability that a set $S$ exists satisfying Property (A) and (B).
    \begin{comment}

    %
    Then for each $m \in \mathbf{Z}^d$,
    %
    \[ \widehat{\nu_S}(m) - \widehat{\eta_{k+1}}(m) \widehat{\nu_T}(m) = \sum_{Q \in \DQ_{k+1}(T)} A_Q e^{-\frac{2 \pi i m \cdot a(Q)}{N_1 \dots N_{k+1}}}. \]
    %
    We calculate that for each $Q \in \DQ_{k+1}(T)$,
    %
    \begin{align*}
        \EE[A_Q|\#(\DQ_k(S))] &= \frac{\PP (X_Q = 1 | \#(\DQ_k(S)))}{\#(\DQ_{k+1}(S))} - \frac{1}{N_{k+1}^d \#(\DQ_{k+1}(T))}\\
        &= \frac{\#(\DQ_{k+1}(S)) / N_{k+1}^d \#(\DQ_{k+1}(T))}{\#(\DQ_{k+1}(S))} - \frac{1}{N_{k+1}^d \#(\DQ_{k+1}(T))} = 0.
    \end{align*}
    %
    In particular, $\EE[A_Q] = 0$. Now for each $q \geq 1$,
    %
    \begin{align*}
        \EE[A_Q^q|\DQ_{k+1}(S)] &= \frac{\PP(X_Q = 1 | \DQ_{k+1}(S))}{\#(\DQ_{k+1}(S))^q} &= \frac{1}{\#(\DQ_{k+1}(T)) \cdot \#(\DQ_{k+1}(S))^{q-1}}.
    \end{align*}
    %
    Thus
    %
    \begin{align*}
        \EE[A_Q^q] &= \frac{1}{\#(\DQ_{k+1}(T))} \EE \left[ \frac{1}{\#(\DQ_{k+1}(S))^{q-1}} \right]\\
        &\leq s
    \end{align*}


    Now fix $m \in \{ -N_1 \dots N_{k+1}, N_1 \dots N_{k+1} \}^d$.



    We can then apply Hoeffding's inequality to conclude that for each $t > 0$,
    %
    \begin{align*}
        \PP \left( |\widehat{\nu_S}(m)| \geq \frac{t p \cdot \#(\DQ_{k+1}(T))^{1/2}}{\#(\DQ_{k+1}(S))} \right) &= \PP \left( \sum \left| \#(\DQ_{k+1}(S)) A_Q e^{-\frac{2 \pi i m \cdot a(Q)}{N_1 \dots N_{k+1}}} \right| \geq A \right) \\
        &\leq 2 \cdot \exp \left( - 2 t^2) \right)
    \end{align*}
    %
    If we now take a union bound over all $m \in \{ -N_1 \dots N_{k+1}, \dots, N_1 \dots N_{k+1} \}^d$, we can guarantee that
    %
    \begin{equation} \label{fourierdim3}
        \mathbf{P} \left( |\widehat{f}(m)| \leq \log(N_{k+1})/S\ \text{for all $m \in \{ -N, \dots, N\}^d$} \right) \geq 1 - 2^{d+1}/N^{c \log N - d}.
    \end{equation}
    %
    Since $\widehat{\nu_S}$ is $N_1 \dots N_{k+1}$ periodic, this means we can control all integer values of $\widehat{\nu_S}$ with high probability.

    Combining \eqref{fourierdim1}, \eqref{fourierdim2}, and \eqref{fourierdim3}, we conclude that there exists a constant $C$ such that with probability at least
    %
    \[ 1 - 2 \exp \left( \frac{-N^{d-s/n}}{A^{1/n} (\log N)^{1/n}} \right) - 1/\log N - \frac{2^{d+1}}{N^{c \log N - d}} \geq 1 - C / \log N, \]
    %
    the set $X$ avoids $K$, and for all $m \in \{ -N, \dots, N \}^d$,
    %
    \[ |\widehat{f}(m)| \leq \frac{C (\log N)^{1-1/n}}{N^{d-s/n}}. \qedhere \]



    Let
    %
    \[ p = \frac{1}{(N_{k+1}^{s + \varepsilon} \log(N_{k+1}))^{1/n}}, \]
    %
    and let $\{ X_Q \}$ be a family of independent and identically distributed $\{ 0, 1 \}$ valued Bernoulli random variables, for each $Q \in \DQ_{k+1}(T)$, such that $\PP(X_Q = 1) = p$. Then, define $S = \bigcup \{ Q : X_Q = 1 \}$. Then $\#(\DQ_{k+1}(S)) = \sum_Q X_Q$ is the sum of $\#(\DQ_k(T)) \cdot N_{k+1}^d$ independant and identically distributed random variables, and so Chernoff's inequality implies that
    %
    \[ \PP \left( \left| \#(\DQ_{k+1}(S)) - p \cdot \DQ_k(T) \cdot N_{k+1}^d \right| \leq \frac{p \cdot \# \DQ_k(T) \cdot N_{k+1}^d}{2} \right) \leq 10 e^{- p \DQ_k(T) N_{k+1}^d}. \]
    %
    Substituting in the value of $p$, we conclude
    %
    \begin{equation} \label{fourierdim1}
    \begin{split}
        \PP& \left( \left| \#(\DQ_{k+1}(S)) - \frac{\#(\DQ_k(T)) N_{k+1}^{\frac{dn - (s + \varepsilon)}{n}}}{\log(N_{k+1})^{1/n}} \right| \leq \frac{\#(\DQ_k(T)) N_{k+1}^{\frac{dn - (s + \varepsilon)}{n}}}{2 \log(N_{k+1})^{1/n}} \right)\\
        &\ \ \ \ \ \ \ \ \ \ \leq 10 \exp \left( \frac{- N_{k+1}^{\frac{dn - (s + \varepsilon)}{n}} \cdot \DQ_k(T)}{\log(N_{k+1})^{1/n}} \right)\\
        &\ \ \ \ \ \ \ \ \ \ \leq 10 \exp \left( \frac{-N_{k+1}^{\frac{dn - (s + \varepsilon)}{n}}}{\log(N_{k+1})^{1/n}} \right)
    \end{split}
    \end{equation}
    %
    Thus $S$ is the union of a large number of cubes, with high probability.

    Without loss of generality, removing cubes from $B$ if necessary, we may assume that every cube $Q_1 \times \dots \times Q_n \in \DQ_{k+1}(B)$, the values $Q_1, \dots, Q_n$ are distinct. Just as in Lemma

    In particular, given any such cube, we have
    %
    \[ \mathbf{P}(Q_1 \times \dots Q_n \subset S) = \mathbf{P}(X_{Q_1} = 1, \dots, X_{Q_n} = 1) = p^n. \]
    %
    Thus
    %
    \[ \mathbf{E}(\#(\DQ_{k+1}(B) \cap \DQ_{k+1}(S^n))) \leq N_{k+1}^{s+\varepsilon} p^n = \log(N_{k+1})^{-1}. \]
    %
    Markov's inequality implies
    %
    \begin{equation} \label{fourierdim2}
    \begin{split}
        \mathbf{P}(\DQ_{k+1}(B) \cap \DQ_{k+1}(S^n) \neq \emptyset) &= \mathbf{P}(\# (\DQ_{k+1}(B) \cap \DQ_{k+1}(S^n)) \geq 1)\\
        &\leq \log(N_{k+1})^{-1}.
    \end{split}
    \end{equation}
    %
    Thus $\DQ_{k+1}(S^n)$ is disjoint from $\DQ_{k+1}(B)$ with high probability.

    Now we analyze the Fourier transform of the measure $\nu_S$. For each cube $Q \in \DR_{k+1}(T)$, we can define
    %
    \[ A_Q = \nu_S(a(Q)) - (\eta_{k+1} * \nu_T)(a(Q)) = \begin{cases} \frac{1}{\#(\DQ_{k+1}(S))} - \frac{1}{N_{k+1}^d \#(\DQ_k(T))} &: X_Q = 1, \\ - \frac{1}{ N_{k+1}^d \#(\DQ_k(T))} &: X_Q = 0. \end{cases} \]
    %
    Then for each $m \in \mathbf{Z}^d$,
    %
    \[ \widehat{\nu_S}(m) - \widehat{\eta_{k+1}}(m) \widehat{\nu_T}(m) = \sum_{Q \in \DQ_{k+1}(T)} A_Q e^{-\frac{2 \pi i m \cdot a(Q)}{N_1 \dots N_{k+1}}}. \]
    %
    We calculate that for each $Q \in \DQ_{k+1}(T)$,
    %
    \begin{align*}
        \EE[A_Q|\#(\DQ_k(S))] &= \frac{\PP (X_Q = 1 | \#(\DQ_k(S)))}{\#(\DQ_{k+1}(S))} - \frac{1}{N_{k+1}^d \#(\DQ_{k+1}(T))}\\
        &= \frac{\#(\DQ_{k+1}(S)) / N_{k+1}^d \#(\DQ_{k+1}(T))}{\#(\DQ_{k+1}(S))} - \frac{1}{N_{k+1}^d \#(\DQ_{k+1}(T))} = 0.
    \end{align*}
    %
    In particular, $\EE[A_Q] = 0$. Now for each $q \geq 1$,
    %
    \begin{align*}
        \EE[A_Q^q|\DQ_{k+1}(S)] &= \frac{\PP(X_Q = 1 | \DQ_{k+1}(S))}{\#(\DQ_{k+1}(S))^q} &= \frac{1}{\#(\DQ_{k+1}(T)) \cdot \#(\DQ_{k+1}(S))^{q-1}}.
    \end{align*}
    %
    Thus
    %
    \begin{align*}
        \EE[A_Q^q] &= \frac{1}{\#(\DQ_{k+1}(T))} \EE \left[ \frac{1}{\#(\DQ_{k+1}(S))^{q-1}} \right]\\
        &\leq s
    \end{align*}


    Now fix $m \in \{ -N_1 \dots N_{k+1}, N_1 \dots N_{k+1} \}^d$.



    We can then apply Hoeffding's inequality to conclude that for each $t > 0$,
    %
    \begin{align*}
        \PP \left( |\widehat{\nu_S}(m)| \geq \frac{t p \cdot \#(\DQ_{k+1}(T))^{1/2}}{\#(\DQ_{k+1}(S))} \right) &= \PP \left( \sum \left| \#(\DQ_{k+1}(S)) A_Q e^{-\frac{2 \pi i m \cdot a(Q)}{N_1 \dots N_{k+1}}} \right| \geq A \right) \\
        &\leq 2 \cdot \exp \left( - 2 t^2) \right)
    \end{align*}
    %
    If we now take a union bound over all $m \in \{ -N_1 \dots N_{k+1}, \dots, N_1 \dots N_{k+1} \}^d$, we can guarantee that
    %
    \begin{equation} \label{fourierdim3}
        \mathbf{P} \left( |\widehat{f}(m)| \leq \log(N_{k+1})/S\ \text{for all $m \in \{ -N, \dots, N\}^d$} \right) \geq 1 - 2^{d+1}/N^{c \log N - d}.
    \end{equation}
    %
    Since $\widehat{\nu_S}$ is $N_1 \dots N_{k+1}$ periodic, this means we can control all integer values of $\widehat{\nu_S}$ with high probability.

    Combining \eqref{fourierdim1}, \eqref{fourierdim2}, and \eqref{fourierdim3}, we conclude that there exists a constant $C$ such that with probability at least
    %
    \[ 1 - 2 \exp \left( \frac{-N^{d-s/n}}{A^{1/n} (\log N)^{1/n}} \right) - 1/\log N - \frac{2^{d+1}}{N^{c \log N - d}} \geq 1 - C / \log N, \]
    %
    the set $X$ avoids $K$, and for all $m \in \{ -N, \dots, N \}^d$,
    %
    \[ |\widehat{f}(m)| \leq \frac{C (\log N)^{1-1/n}}{N^{d-s/n}}. \qedhere \]
%\end{comment}
\end{proof}
\end{comment}

The construction of the set $X$ follows essentially the construction of the configuration avoiding set in Chapter \ref{ch:RoughSets}. We choose a decreasing sequence of parameters $\{ \varepsilon_k \}$ such that $\varepsilon_k < (n-s)/4$ for each $k$, as well as parameters $\{ N_k \}$ such that
%
\[ N_k \geq 3^{1/\varepsilon_k}, \]
%
\[ N_k \geq \exp \left( \left( \frac{4n}{n-s} \right)^4 N_1 \dots N_{k-1} \right), \]
%
\[ N_k \geq (1/\varepsilon_k)^{1/\varepsilon_k}, \]
%
\begin{equation} \label{equation13895891489132}
    N_k \geq (N_1 \dots N_{k-1})^{2/\varepsilon_k}.
\end{equation}
%
The choice of $N_k$ is also chosen sufficiently large that we can find a $\DQ_k$ discretized set $B_k$ such that
%
\[ \#(\DQ_k(B_k)) \leq (N_1 \dots N_k)^{s + \varepsilon_k/2} \leq N_k^{s + \varepsilon_k}, \]
%
and such that the collection $\{ B_k \}$ forms a strong cover of the configuration $\C$. We then choose a sequence $\{ M_k \}$  such that for each $k$,
%
\[ M_k \leq N_k^{\frac{n-s-2\varepsilon_k}{n}} \leq 2 M_{k+1}. \]
%
Just as was done in Chapter \ref{ch:RoughSets}, this choice of parameters enables us to find a nested family of sets $\{ X_k \}$, obtained by setting $X_0 = [0,1]$, and letting $X_{k+1}$ be obtained from $X_k$ by applying Lemma \ref{discreteFourierBuildingBlock} with $\varepsilon = \varepsilon_{k+1}$, $T = X_k$, and $B = B_{k+1}$. We set $X = \bigcap X_k$. Since Property (A) of Lemma \eqref{discreteFourierBuildingBlock} is true at each step of the process, this is sufficient to guarantee that $X$ avoids the configuration $\C$. The remainder of this section is devoted to showing that Property (B) of Lemma \ref{discreteFourierBuildingBlock} is sufficient to obtain the Fourier dimension bound on $X$ guaranteed by Theorem \ref{FourierTheorem}.

Let $\nu_k = \nu_{X_k}$ for each $k$. Property (B) of Lemma \ref{discreteFourierBuildingBlock} implies that for each $k$,
%
\begin{equation} \label{equation77770123091293120}
    \left\| \widehat{\nu_{k+1}} - \widehat{\eta_{k+1}} \widehat{\nu_k} \right\|_{L^\infty(\ZZ)} \leq A(n,s) \cdot (N_1 \dots N_{k+1})^{- \frac{n-s}{2n} + 2\varepsilon_{k+1}}.
\end{equation}
%
We shall form a sequence of measures $\{ \mu_k \}$ by convolving the measures $\{ \nu_k \}$ with an appropriate family of mollifiers, which will be sufficient to obtain the required asymptotic bound.

\begin{lemma}
    There exists a sequence of probability measures $\{ \mu_k \}$, with $\mu_k$ supported on $X_k$ for each $k$, such that for each $\varepsilon > 0$,
    %
    \[ \sup_{k > 0} \sup_{m \in \ZZ} |m|^{\frac{n-s}{2n} - \varepsilon} |\widehat{\mu_k}(m)| < \infty. \]
\end{lemma}
\begin{proof}
    For each $k$, let
    %
    \[ \psi_k(x) = (N_1 \dots N_k) \cdot \mathbf{I}_{\left[ 0, \frac{1}{N_1 \dots N_k} \right]}. \]
    %
    Then it is easy to calculate that
    %
    \begin{equation} \label{equation901418294891481792}
        |\widehat{\psi_k}(m)| \lesssim \min \left( 1, \frac{N_1 \dots N_k}{|m|} \right).
    \end{equation}
    %
    Note that the measures $\mu_k = \nu_k * \psi_k$ are still supported on $X_k$, and
    %
    \[ \widehat{\mu_k}(\xi) = \widehat{\nu_k}(\xi) \widehat{\psi_k}(\xi). \]
    %
    Also note that $\psi_k = \psi_{k+1} * \eta_{k+1}$. If $\varepsilon > 0$, then we can apply \eqref{equation77770123091293120} with \eqref{equation901418294891481792} to conclude
    %
    \begin{equation} \label{equation6892489214781278}
    \begin{split}
        &|\widehat{\mu_{k+1}}(m) - \widehat{\mu_k}(m)|\\
        &\ \ \ \ = |\widehat{\psi_{k+1}}(m)| |\widehat{\nu_k}(m) - \widehat{\eta_{k+1}}(m) \widehat{\nu_k}(m)|\\
        &\ \ \ \ \lesssim \min \left( 1, \frac{N_1 \dots N_{k+1}}{|m|} \right) (N_1 \dots N_{k+1})^{-\frac{n-s}{2n} + 2\varepsilon_{k+1}}.\\
        &\ \ \ \ = \min \left( \frac{|m|^{\frac{n-s}{2n} - \varepsilon}}{(N_1 \dots N_{k+1})^{\frac{n-s}{2n} - 2\varepsilon_{k+1}}}, \frac{(N_1 \dots N_k)^{1 - \frac{n-s}{2n} + 2\varepsilon_{k+1}}}{|m|^{1 - \frac{n-s}{2n} + \varepsilon}} \right) |m|^{- \frac{n-s}{2n} + \varepsilon}.
    \end{split}
    \end{equation}
    %
    The minima is maximized when $|m| = N_1 \dots N_{k+1}$, which gives
    %
    \[ \min \left( \frac{|m|^{\frac{n-s}{2n} - \varepsilon}}{(N_1 \dots N_{k+1})^{\frac{n-s}{2n} - 2\varepsilon_{k+1}}}, \frac{(N_1 \dots N_k)^{1 - \frac{n-s}{2n} + 2\varepsilon_{k+1}}}{|m|^{1 - \frac{n-s}{2n} + \varepsilon}} \right) \leq (N_1 \dots N_{k+1})^{2\varepsilon_{k+1} - \varepsilon}. \]
    %
    Thus, for all $k$, for all $m \in \ZZ$, and for all $\varepsilon > 0$,
    %
    \begin{equation} \label{equation11020404120}
    \begin{split}
        |\widehat{\mu_{k+1}}(m) - \widehat{\mu_k}(m)| \lesssim \frac{(N_1 \dots N_{k+1})^{2\varepsilon_{k+1} - \varepsilon}}{|m|^{\frac{n-s}{2n} - \varepsilon}}.
    \end{split}
    \end{equation}
    %
    For each $k$, let
    %
    \[ A_k = \sup_{m \in \ZZ} |\widehat{\mu_k}(m)| |m|^{\frac{n-s}{2n} - \varepsilon}. \]
    %
    Then \eqref{equation11020404120} implies that
    %
    \[ A_{k+1} = A_k + O \left( (N_1 \dots N_{k+1})^{2\varepsilon_{k+1} - \varepsilon} \right). \]
    %
    Thus for all $k > 0$,
    %
    \[ A_k = O \left( \sum_{k = 1}^\infty (N_1 \dots N_k)^{2\varepsilon_{k+1} - \varepsilon} \right). \]
    %
    Provided the sum on the right hand side converges for each $\varepsilon > 0$, this gives a uniform bound of $A_k$ in $k$ for each $\varepsilon > 0$, completing the proof. But for suitably large $k$, depending on $\varepsilon$, it is eventually true that $\varepsilon_{k+1} \leq \varepsilon/8$, and so
    %
    \begin{align*}
        A_k &= O_\varepsilon(1) + \sum_{k = 1}^\infty (N_1 \dots N_k)^{-\varepsilon/4} = O_\varepsilon(1) + \sum_{k = 1}^\infty 2^{-k\varepsilon/4} = O_\varepsilon(1). \qedhere
    \end{align*}
\end{proof}

Just as for the sequence of measures in Theorem \ref{massdistributionprinciplelem}, the sequence $\{ \mu_k \}$ is a Cauchy sequence of probability measures, and therefore converges weakly to some measure $\mu$. Because for each $k$, $\mu_k$ is supported on $X_k$, $\mu$ is supported on $\bigcap X_k = X$. Furthermore, the Fourier transform of each $\mu_k$ converges pointwise to the Fourier transform of $\mu$. Thus we find that for each $\varepsilon > 0$,
%
\[ \sup_{m \in \ZZ} |m|^{\frac{n-s}{2n} - \varepsilon} |\widehat{\mu}(m)| \leq \sup_{k > 0} \sup_{m \in \ZZ} |m|^{\frac{n-s}{2n} - \varepsilon} |\widehat{\mu_k}(m)| < \infty. \]
%
Combined with Lemma \ref{discretefouriermeasures}, this implies $X$ has Fourier dimension $(nd - s)/n$.

\endinput

% 3. Notes
% 4. Footnotes

% 5. Bibliography
\begin{singlespace}
\raggedright
\bibliographystyle{abbrvnat}
\bibliography{biblio}
\end{singlespace}

%\appendix
% 6. Appendices (including copies of all required UBC Research
% Ethics Board's Certificates of Approval)
% \include{reb-coa}	% pdfpages is useful here
%\chapter{Supporting Materials}

This would be any supporting material not central to the dissertation.
For example:
\begin{itemize}
\item additional details of methodology and/or data;
\item diagrams of specialized equipment developed.;
\item copies of questionnaires and survey instruments.
\end{itemize}


\nocite{*}
\backmatter
% 7. Index
% See the makeindex package: the following page provides a quick overview
% <http://www.image.ufl.edu/help/latex/latex_indexes.shtml>

\end{document}
