%% The ubcdiss class provides several options:
%%   gpscopy (aka fogscopy)
%%       set parameters to exactly how GPS specifies
%%         * single-sided
%%         * page-numbering starts from title page
%%         * the lists of figures and tables have each entry prefixed
%%           with 'Figure' or 'Table'
%%       This can be tested by `\ifgpscopy ... \else ... \fi'
%%   10pt, 11pt, 12pt
%%       set default font size
%%   oneside, twoside
%%       whether to format for single-sided or double-sided printing
%%   balanced
%%       when double-sided, ensure page content is centred
%%       rather than slightly offset (the default)
%%   singlespacing, onehalfspacing, doublespacing
%%       set default inter-line text spacing; the ubcdiss class
%%       provides \textspacing to revert to this configured spacing
%%   draft
%%       disable more intensive processing, such as including
%%       graphics, etc.

% For submission to GPS
%\documentclass[gpscopy,onehalfspacing,12pt]{ubcdiss}

% For your own copies (looks nicer)
\documentclass[balanced,twoside,12pt]{ubcdiss}

%%
%% FONTS:
%% 
%% The defaults below configures Times Roman for the serif font,
%% Helvetica for the sans serif font, and Courier for the
%% typewriter-style font.

%% NFSS font specification (New Font Selection Scheme)
\usepackage{times,mathptmx,courier}
\usepackage[scaled=.92]{helvet}

\usepackage{amsmath, amsfonts, amssymb, accents, mdwlist}

\usepackage{amsthm}
\theoremstyle{plain}
\newtheorem{lemma}{Lemma}
\newtheorem*{example}{Example}
\newtheorem*{fact}{Fact}
\newtheorem*{corollary}{Corollary}
\newtheorem{theorem}{Theorem}
\newtheorem*{remark}{Remark}

\usepackage{algorithm}
\usepackage[noend]{algpseudocode}

\algblockdefx{MRepeat}{EndRepeat}{\textbf{Repeat}}{}
\algnotext{EndRepeat}

\algblockdefx{MForAll}{EndForAll}{\textbf{For all}}{}
\algnotext{EndForAll}

\usepackage{checkend} % better error messages on left-open environments
\usepackage{graphicx} % for incorporating external images
\usepackage{booktabs} % Provides commands for typesetting tables.
\usepackage{comment} % Comment environment.

\usepackage{listings} % Supports for source doe listings.
\lstset{basicstyle=\sffamily\scriptsize,showstringspaces=false,fontadjust}

\usepackage[printonlyused,nohyperlinks]{acronym} % Acronyms and Glossaries
% Typeset acronyms in small-caps, as recommended by Bringhurst.
\renewcommand{\acsfont}[1]{{\scshape \MakeTextLowercase{#1}}}

%% For an excellent set of examples, see Tufte's "Visual Display of
%% Quantitative Information" or "Envisioning Information".
\usepackage{color, xcolor}
\definecolor{greytext}{gray}{0.5}
\definecolor{crimsonred}{RGB}{132,22,23}
\definecolor{darkblue}{RGB}{72,61,139}

%% Provides citing commands such as \citeauthor{} to provide author names
%% \citet{} to produce author-and-reference citation,
%% \citep{} to produce parenthetical citations.
%% We use \citeeg{} to provide examples
\usepackage[numbers,sort&compress]{natbib}
\newcommand{\citeeg}[1]{\citep[e.g.,][]{#1}}

%% The titlesec package provides commands to vary how chapter and
%% section titles are typeset.  The following uses more compact
%% spacings above and below the title.  The titleformat that follow
%% ensure chapter/section titles are set in singlespace.
\usepackage[compact]{titlesec}
\titleformat*{\section}{\singlespacing\raggedright\bfseries\Large}
\titleformat*{\subsection}{\singlespacing\raggedright\bfseries\large}
\titleformat*{\subsubsection}{\singlespacing\raggedright\bfseries}
\titleformat*{\paragraph}{\singlespacing\raggedright\itshape}

%% The caption package provides support for varying how table and
%% figure captions are typeset.
\usepackage[format=hang,indention=-1cm,labelfont={bf},margin=1em]{caption}

%% url: for typesetting URLs and smart(er) hyphenation.
%% \url{http://...} 
\usepackage{url}
\urlstyle{sf}	% typeset urls in sans-serif

\usepackage{pdfpages}	% insert pages from other PDF files
\usepackage{longtable}	% provide tables spanning multiple pages
\usepackage{chngpage}	% for changing the page widths on demand
\usepackage{tabularx}   % enhanced tabular environment
\usepackage{subfig}     % include subfigures in figures.

\usepackage{enumitem}   % allows pausing and resuming enumerate environments.
\setlist[enumerate]{label*={\normalfont(\Alph*)},ref=(\Alph*)}

%% ragged2e: provides several new new commands \Centering, \RaggedLeft,
%% \RaggedRight and \justifying and new environments Center, FlushLeft,
%% FlushRight and justify, which set ragged text and are easily
%% configurable to allow hyphenation.
%\usepackage{ragged2e}

%% The ulem package provides a \sout{} for striking out text and
%% \xout for crossing out text.  The normalem and normalbf are
%% necessary as the package messes with the emphasis and bold fonts
%% otherwise.
%\usepackage[normalem,normalbf]{ulem}    % for \sout

%%%%%%%%%%%%%%%%%%%%%%%%%%%%%%%%%%%%%%%%%%%%%%%%%%%%%%%%%%%%%%%%%%%%%%
%% HYPERREF:
%% The hyperref package provides for embedding hyperlinks into your
%% document.  By default the table of contents, references, citations,
%% and footnotes are hyperlinked.
%%
%% Hyperref provides a very handy command for doing cross-references:
%% \autoref{}.  This is similar to \ref{} and \pageref{} except that
%% it automagically puts in the *type* of reference.  For example,
%% referencing a figure's label will put the text `Figure 3.4'.
%% And the text will be hyperlinked to the appropriate place in the
%% document.
%%
%% Generally hyperref should appear after most other packages

%% The following puts hyperlinks in very faint grey boxes.
%% The `pagebackref' causes the references in the bibliography to have
%% back-references to the citing page; `backref' puts the citing section
%% number.  See further below for other examples of using hyperref.
%% 2009/12/09: now use `linktocpage' (Jacek Kisynski): GPS now prefers
%%   that the ToC, LoF, LoT place the hyperlink on the page number,
%%   rather than the entry text.
\usepackage[bookmarks,bookmarksnumbered,%
%    allbordercolors={0.8 0.8 0.8},%
    pagebackref,linktocpage%
    ]{hyperref}
\hypersetup{
    colorlinks=true,
    citecolor=black,
    filecolor=black,
    linkcolor=black,
    urlcolor=black,
    linktoc=all
}
%% The following change how the the back-references text is typeset in a
%% bibliography when `backref' or `pagebackref' are used
%%
%% Change \nocitations if you'd like some text shown where there
%% are no citations found (e.g., pulled in with \nocite{xxx})
\newcommand{\nocitations}{\relax}
%%\newcommand{\nocitations}{No citations}
%%
%\renewcommand*{\backref}[1]{}% necessary for backref < 1.33
\renewcommand*{\backrefsep}{,~}%
\renewcommand*{\backreftwosep}{,~}% ', and~'
\renewcommand*{\backreflastsep}{,~}% ' and~'
\renewcommand*{\backrefalt}[4]{%
\textcolor{greytext}{\ifcase #1%
\nocitations%
\or
\(\rightarrow\) page #2%
\else
\(\rightarrow\) pages #2%
\fi}}


%% The following uses most defaults, which causes hyperlinks to be
%% surrounded by colourful boxes; the colours are only visible in
%% PDFs and don't show up when printed:
%\usepackage[bookmarks,bookmarksnumbered]{hyperref}

%% The following disables the colourful boxes around hyperlinks.
%\usepackage[bookmarks,bookmarksnumbered,pdfborder={0 0 0}]{hyperref}

%% The following disables all hyperlinking, but still enabled use of
%% \autoref{}
%\usepackage[draft]{hyperref}

%% The following commands causes chapter and section references to
%% uppercase the part name.
\renewcommand{\chapterautorefname}{Chapter}
\renewcommand{\sectionautorefname}{Section}
\renewcommand{\subsectionautorefname}{Section}
\renewcommand{\subsubsectionautorefname}{Section}

%% If you have long page numbers (e.g., roman numbers in the 
%% preliminary pages for page 28 = xxviii), you might need to
%% uncomment the following and tweak the \@pnumwidth length
%% (default: 1.55em).  See the tocloft documentation at
%% http://www.ctan.org/tex-archive/macros/latex/contrib/tocloft/
% \makeatletter
% \renewcommand{\@pnumwidth}{3em}
% \makeatother

%%%%%%%%%%%%%%%%%%%%%%%%%%%%%%%%%%%%%%%%%%%%%%%%%%%%%%%%%%%%%%%%%%%%%%
%%%%%%%%%%%%%%%%%%%%%%%%%%%%%%%%%%%%%%%%%%%%%%%%%%%%%%%%%%%%%%%%%%%%%%
%%
%% Some special settings that controls how text is typeset
%%
% \raggedbottom		% pages don't have to line up nicely on the last line
% \sloppy		% be a bit more relaxed in inter-word spacing
% \clubpenalty=10000	% try harder to avoid orphans
% \widowpenalty=10000	% try harder to avoid widows
% \tolerance=1000

%% And include some of our own useful macros
\input{macros}

% My Math Operators
\DeclareMathOperator{\minkdim}{\dim_{\mathbf{M}}}
\DeclareMathOperator{\hausdim}{\dim_{\mathbf{H}}}
\DeclareMathOperator{\lowminkdim}{\underline{\dim}_{\mathbf{M}}}
\DeclareMathOperator{\upminkdim}{\overline{\dim}_{\mathbf{M}}}

\DeclareMathOperator{\lhdim}{\underline{\dim}_{\mathbf{M}}}
\DeclareMathOperator{\lmbdim}{\underline{\dim}_{\mathbf{MB}}}

\DeclareMathOperator{\RR}{\mathbf{R}}
\DeclareMathOperator{\ZZ}{\mathbf{Z}}
\DeclareMathOperator{\CC}{\mathbf{C}}
\DeclareMathOperator{\QQ}{\mathbf{Q}}

\DeclareMathOperator{\AAA}{\mathbf{A}}
\DeclareMathOperator{\Prob}{\mathbf{P}}
\DeclareMathOperator{\Expect}{\mathbf{E}}

\DeclareMathOperator{\B}{\mathcal{B}}
\DeclareMathOperator{\C}{\mathcal{C}}

\DeclareMathOperator{\Config}{\mathcal{C}}
\DeclareMathOperator{\diam}{\text{diam}}
\DeclareMathOperator{\divides}{\mid}










%%%%%%%%%%%%%%%%%%%%%%%%%%%%%%%%%%%%%%%%%%%%%%%%%%%%%%%%%%%%%%%%%%%%%%
%%%%%%%%%%%%%%%%%%%%%%%%%%%%%%%%%%%%%%%%%%%%%%%%%%%%%%%%%%%%%%%%%%%%%%
%%
%% Document meta-data: be sure to also change the \hypersetup information
%%

\title{Cartesian Products Avoiding Patterns}
%\subtitle{If you want a subtitle}

\author{Jacob Denson}
\previousdegree{BSc. Computing Science, University of Alberta, 2017}

% What is this dissertation for?
\degreetitle{Master of Science}

\institution{The University of British Columbia}
\campus{Vancouver}

\faculty{The Faculty of Science}
\department{Mathematics}
\submissionmonth{April}
\submissionyear{2019}

% details of your examining committee
\examiningcommittee{Malabika Pramanik, Mathematics}{Supervisor}
\examiningcommittee{Joshua Zahl, Mathematics}{Supervisor}

%% hyperref package provides support for embedding meta-data in .PDF
%% files
\hypersetup{
  pdftitle={Change this title!  (DRAFT: \today)},
  pdfauthor={Johnny Canuck},
  pdfkeywords={Your keywords here}
}

%% LaTeX's \includeonly commands causes any uses of \include{} to only
%% include files that are in the list.  This is helpful to produce
%% subsets of your thesis (e.g., for committee members who want to see
%% the dissertation chapter by chapter).  It also saves time by 
%% avoiding reprocessing the entire file.
%\includeonly{intro,conclusions}
%\includeonly{discussion}










\begin{document}

% Thesis Guidelines available at:
% 	http://www.grad.ubc.ca/current-students/dissertation-thesis-preparation/order-components

% 1. Title page (mandatory)
\maketitle

% 2. Committee page (mandatory): lists supervisory committee and if applicable, the examining committee
\makecommitteepage

% 3. Abstract (mandatory - maximum 350 words)
%% The following is a directive for TeXShop to indicate the main file
%%!TEX root = diss.tex

\chapter{Abstract}

% MAXIMUM 350 WORDS!

The pattern avoidance problem seeks to construct a set $X \subset \RR^d$ with large dimension that avoids a prescribed pattern such as three term arithmetic progressions, or more general patterns such as avoiding points $x_1, \dots, x_n$ such that $f(x_1, \dots, x_n) = 0$ for a given function $f$. Previous work on the subject has considered patterns described by polynomials, or functions $f$ satisfying certain regularity conditions. We provide an exposition of this work, as well as considering new strategies to avoid `rough patterns. There are several problems that fit into the framework of rough pattern avoidance. As a first application, if $Y\subset[0,1]$ is a set with Minkowski dimension $\alpha$, we construct a set $X\subset[0,1]$ with Hausdorff dimension $1-\alpha$ so that $X+X$ is disjoint from $Y$. As a second application, given a set $Y$ with dimension close to one, we can construct a set $X\subset Y$ of dimension $1/2$ that avoids isosceles triangles.
\cleardoublepage

% 4. Lay Summary (Effective May 2017, mandatory - maximum 150 words)
%% The following is a directive for TeXShop to indicate the main file
%%!TEX root = diss.tex

%% https://www.grad.ubc.ca/current-students/dissertation-thesis-preparation/preliminary-pages
%% 
%% LAY SUMMARY Effective May 2017, all theses and dissertations must
%% include a lay summary.  The lay or public summary explains the key
%% goals and contributions of the research/scholarly work in terms that
%% can be understood by the general public. It must not exceed 150
%% words in length.

%% The lay or public summary explains the key goals and contributions of
%% the research\slash{}scholarly work in terms that can be understood by the
%% general public. It must not exceed 150 words in length.

\chapter{Lay Summary}

%Imagine a patch of carpet under a microscope. Zooming in, we see the carpet is really a collection of twines tied together. On closer inspection, those twines break off into smaller tufts of fabric. The carpet is rough at all scales. Shapes like circles or polygons do not have complexity at arbitrarily small scales. To model the roughness of a carpet, you'd need a fractal: a shape with complex structure at all small scales. Such models are often useful in small-scale physics or computer graphics.

%It is easy to construct polyhedra with geometric properties at a single scale. But it is non-trivial to construct fractals with properties at many scales. For instance, how do we construct a fractal which intersects any line in at most two points? In this thesis, we begin with an exposition on previous constructions in the literature, and then provide new construction techniques utilizing randomness.

In geometry, we are often interested in constructing shapes with certain properties. For instance, given three points, can one find a circle connecting these three points? Most questions of this type involving classical shapes like circles and polygons have been answered. But many open questions remain about more modern families of shapes. Here, we focus on \emph{fractals}; Examples of fractals include the Koch snowflake and Sierpinski triangle. This class of shapes often occurs in small scale physics and computer graphics.

In this thesis, we focus on constructing large fractals which avoid the existence of certain configurations. For example, can one construct a large fractal so that one cannot form an equilateral triangle from three points lying on the fractal? We begin with an exposition on constructions in the literature for avoiding configurations, and then provide new construction techniques utilizing randomness.
\cleardoublepage

% 5. Preface
%% The following is a directive for TeXShop to indicate the main file
%%!TEX root = diss.tex

\chapter{Preface}

This thesis gives an exposition by the author, of the pattern avoidance problem and the geometric measure theory required to understand the pattern avoidance problem in the non-discrete setting. In Chapter \ref{ch:RoughSets} and \ref{ch:Applications}, the author presents details of joint work with his supervisors Dr. Joshua Zahl and Dr. Malabika Pramanik. The results of these sections have been accepted for publication in the Springer series \emph{Harmonic Analysis and Applications}. As is the norm in mathematical research, all researchers are assumed to have contributed equally to these results. But to list concrete contributions, the author of this thesis reviewed the background literature detailed in the bibliography to this paper, and came up with the main problem statement behind Theorem \ref{mainTheorem}. In Chapter \ref{ch:Conclusions}, the author presents details on partially completed results emerging from discussions with Dr. Joshua Zahl and Dr. Malabika Pramanik, which he hopes can be refined and published in the near future.
\cleardoublepage

% 6. Table of contents (mandatory - list all items in the preliminary pages
% starting with the abstract, followed by chapter headings and
% subheadings, bibliographies and appendices)
\tableofcontents
\cleardoublepage	% required by tocloft package

% 7. List of tables (mandatory if thesis has tables)
%\listoftables
%\cleardoublepage	% required by tocloft package

% 8. List of figures (mandatory if thesis has figures)
\listoffigures
\cleardoublepage	% required by tocloft package

% 9. List of illustrations (mandatory if thesis has illustrations)
% 10. Lists of symbols, abbreviations or other (optional)

% 11. Glossary (optional)
\input{glossary}	% always input, since other macros may rely on it

\textspacing		% begin one-half or double spacing

% 12. Acknowledgements (optional)
%% The following is a directive for TeXShop to indicate the main file
%%!TEX root = diss.tex

\chapter{Acknowledgments}

I would like to thank my advisors, Dr. Malabika Pramanik and Dr. Josh Zahl for their key insights in the past two years of research. Without their tough scrutiny of my writing style over the past year in the writing of our recently submitted research paper, this thesis would be exponentially less legible.

I'd also like to thank the University of British Columbia and the NSERC research fund, for their financial support during the writing of the thesis.

Thank you to my `unofficial undergraduate research advisor', Dr. Zachary Friggstad. I can't state how much our long discussions in your office have changed the way I think about mathematics. Your influence is felt throughout the research detailed here.

I would also like to thank my family for their support. I would especially like to thank my grandfather, Ted Mcclung. Without your encouragement in my early years of undergraduate education, it is unlikely I would have found my passion for higher mathematics.

Last, but certainly not least, I'd like to thank the UBC mathematics department, and greater student community, for keeping me grounded during many stressful moments over the past two years of research. You know who you are.

% 13. Dedication (optional)

% Body of Thesis (not all sections may apply)
\mainmatter

\acresetall	% reset all acronyms used so far

% 1. Introduction
\include{intro}

% 2. Main body
%% The following is a directive for TeXShop to indicate the main file
%%!TEX root = diss.tex

\chapter{Background}
\label{ch:Background}

This thesis discusses methods to form large sets avoiding patterns. First, we give a precise definition by what we mean by a pattern, and what it means to avoid a pattern. We consider an ambient set $\AAA$, often discrete, or equipped with some topological or metric structure. It's {\it $n$-point configuration space} is the set of distinct tuples of $n$ points in $\AAA$, i.e.
%
\[ \Config^n(\AAA) = \{ (x_1, \dots, x_n) \in X^n: x_i \neq x_j\ \text{if $i \neq j$} \}. \]
%
The {\it configuration space} of $\AAA$ is $\Config(\AAA) = \bigcup_{n = 1}^\infty \Config^n(\AAA)$. A {\it pattern}, or {\it configuration}, on $\AAA$ is a subset of $\Config(X)$, and we say a subset $Y$ of $X$ avoids a configuration $\C$ if $\Config(Y)$ is disjoint from $\C$. An {\it $n$ point configuration} on $X$ will be a configuration $\C$ which is a subset of $\Config^n(X)$.

\begin{example}[Isoceles Triangle Configuration]
	Consider the problem of finding a set avoiding the vertices of an isoceles triangle in the plane. Set
	%
	\[ \C = \{ (x_1, x_2, x_3) \in \Config^3(\RR^2) : |x_1-x_2| = |x_1-x_3| \}. \]
	%
	Then $\C$ is a 3-point configuration, and a set $X \subset \RR^2$ avoids $\C$ if and only if it contains no vertices of an isoceles triangle. Notice that $|x_1 - x_2| = |x_1 - x_3|$ holds if and only if $|x_1 - x_2|^2 = |x_1 - x_3|^2$, which is an algebraic equation in the coordinates of $x_1,x_2$, and $x_3$. Thus $\C$ is an algebraic hypersurface in $\RR^6$.
\end{example}

\begin{example}[General Position Configuration]
	Suppose we wish to find a subset of $\RR^d$ such that every collection of $k+1$ points in the set, for $k < d$, lies in `general position', i.e. they do not lie in a $k$ dimensional hyperplane. Set
	%
	\[ \C^k = \{ (x_0, x_1, \dots, x_k) \in \Config^{k+1}(\RR^d): x_1-x_0, \dots, x_k - x_0\ \text{are linearly dependant} \}. \]
	%
	and then consider the configuration $\C = \bigcup_{k = 1}^{d-1} \C^k$. A set $X$ avoids $\C$ if and only if all of it's points lie in general position. Notice that
	%
	\[ \C^k = \bigcup \left\{ \text{span}(y_1, \dots, y_k) \times \{ y \} : y = (y_1, \dots, y_k) \in \Config^k(\RR^d) \right\} \cap \Config^{k+1}(\RR^d). \]
	%
	so each $\C^k$ is essentially a union of $k$ dimensional hyperplanes.
\end{example}

Even though our problem formulation assumes configurations are formed by distinct sets of points, one can still formulate avoidance problems involving repeated points in our framework, because an instance of a configuration involving $n$ points which may contain repetitions can be seen as an instance of a configuration involving fewer than $n$ distinct points.

\begin{example}[Sum Set Configuration]
	Let $G$ be an abelian group, and fix $Y \subset G$. Set
	%
	\[ \C^1 = \{ g \in \Config^1(G): 2g \in Y \} \quad \text{and} \quad \C^2 = \{ (g_1,g_2) \in \Config^2(G): g_1 + g_2 \in Y \}. \]
	%
	Then set $\C = \C^1 \cup \C^2$. A set $X \subset G$ avoids $\C$ if and only if $(X + X) \cap Y = \emptyset$.
\end{example}

Our main focus in this thesis is on the {\it pattern avoidance problem}: Given a configuration $\C$ on $\AAA$, how large can $X \subset \AAA$ be avoiding $\C$. If $\AAA$ is discrete, i.e. finite, or a discrete limit of finite sets $\AAA_n$, the goal is to find $X$ with large cardinality, or such that $X \cap \AAA_n$ has large cardinality asymptotically in $n$. If $\AAA = \RR^d$, but $\C$ is a sufficiently discrete configuration, then a satisfactory goal is to find $X$ with large Lebesgue measure avoiding $\C$. But in this thesis we establish methods for avoiding non-discrete configurations $\C$, i.e. those for which $X^m \cap \C^m$ are dense in $X^m$, but taking inspiration from methods in the discrete setting. The next section shows that Lebesgue measure completely fails to measure the degree of success for a solution to the pattern avoidance problems, but provides an alternate measurement which does succeed to establish this point.

\section{Fractal Dimension}

The Lebesgue measure is not the correct measurement for how large a pattern-avoiding set for non-discrete patterns. This is because for most of these patterns, every pattern-avoiding set has measure zero. One intuition as to why this is true is that a set with positive Lebesgue measure behaves in many respects like an open set, and an open set certainly intersects a dense set somewhere. The rigorous instance of this phenomenon we use is the Lebesgue density theorem. It's proof takes us too far afield into differentiation theory, so we merely state the result without proof. The intuitive idea of the result is that a set of positive Lebesgue measure locally contains a large percentage of it's surrounding points.

\begin{theorem}[Lebesgue Density Theorem]
	Let $E \subset \RR^d$ have positive Lebesgue measure. Then for almost every point $x \in E$,
	%
	\[ \lim_{r \to 0} \frac{|E \cap B_r(x)|}{|B_r(x)|} = 1. \]
\end{theorem}

Under mild non-discreteness conditions, which are certainly satisfied by the example configurations given in the last section, no set with positive Lebesgue measure can avoid a configuration.

\begin{theorem}
	Let $\C$ be an $n$-point configuration on $\RR^d$ such that
	%
	\begin{enumerate}
		\item \emph{Translation Invariance}: For any $b \in \RR^d$, $\C + b \subset \C$.
		\item \emph{Non-Discrete}: For any $\varepsilon > 0$, there is an instance of the configuration $(c_1, \dots, c_n) \in C$ such that $\diam \{ c_1, \dots, c_n \} \leq \varepsilon$.
	\end{enumerate}
	%
	Then no set with positive Lebesgue measure avoids $\C$.
\end{theorem}
\begin{proof}
	Let $X \subset \RR^d$ have positive Lebesgue measure. Applying the Lebesgue density theorem, we find a point $x_0 \in X$ such that
	%
	\[ \lim_{r \to 0} \frac{|X \cap B_r(x_0)|}{|B_r(x_0)|} = 1 \]
	%
	We fix $r_0$ such that $|X \cap B_{r_0}(x_0)| \geq (1 - \varepsilon)|B_{r_0}(x)|$. Let $C = (c_1, \dots, c_n) \in \C$ be an instance of the configuration such that $\diam \{ c_1, \dots, c_n \} \leq \varepsilon r_0$. For each $p \in B_{r_0}(x_0)$, let $C(p) = (c_1(p), \dots, c_n(p)) \in \C$, where $c_i(p) = p + (c_i - c_1)$. Then a union bound gives
	%
	\begin{align*}
		\left| \bigcup_{i = 1}^n \{ p \in B_{r_0/2}(x_0) : c_i(p) \not \in X \} \right| &\leq \sum_{i = 1}^n |B_{r_0/2}(x_0) \cap (X + (c_1 - c_i))^c|\\
		&= \sum_{i = 1}^n |B_{r_0/2}(x_0)| - |B_{r_0/2}(x_0 + c_i - c_1) \cap X|\\
		&\leq n \left( |B_{r_0/2}(x_0)| - |B_{r_0(1/2 - \varepsilon)}(x_0) \cap X| \right)\\
		&\leq n \left[ (r_0/2)^d - (1 - \varepsilon)(1/2 - \varepsilon)^d r_0^d \right] \\
		&\leq n \left( 1 - (1 - \varepsilon) (1 - 2d\varepsilon) \right) (r_0/2)^d\\
		&\leq \varepsilon n (2d + 1) (r_0/2)^d
	\end{align*}
	%
	If $\varepsilon < 1/n(2d + 1)$, we conclude that
	%
	\begin{align*}
		\left| \bigcap_{i = 1}^n \{ p \in B_{r_0/2}(x_0) : c_i(p) \in X \} \right| &= |B_{r_0/2}(x_0)| - \left| \bigcap_{i = 1}^n \{ p \in B_{r_0/2}(x_0) : c_i(p) \not \in X \} \right|\\
		&\geq (r_0/2)^d - \varepsilon n(2d + 1) (r_0/2)^d > 0
	\end{align*}
	%
	Thus there exists $p$ such that $C(p) \in \Config(X)$, and so $X$ does not avoid $\C$.
\end{proof}

Fortunately, we have a second order notion of size, which is able to distinguish between sets of measure zero, known as {\it fractional dimension}.  Intuitively, the fractional dimension provides a measure of the local density of a set, and so we view a set which is more dense as `larger', for the purposes of a pattern avoidance problem. There are two definitions of fractional dimension we use in this thesis: Minkowski dimension and Hausdorff dimension. The main difference between the two is that Minkowski dimension measures relative density at a single scale, whereas Hausdorff dimension measures relative density at various small scales.

We begin by discussing the Minkowski dimension, which is the easiest of the two dimension to define. If $E$ is a bounded set in $\RR^d$, then we can consider the {\it delta thickening} $E_\delta = \{ x: d(x,E) < \delta \}$. We define the {\it upper} and {\it lower} Minkowski dimension as
%
\[ \overline{\dim}_M(E) = \limsup_{\delta \to 0} \left( d - \frac{\log|E_\delta|}{\log \delta} \right)\ \ \ \ \ \underline{\dim}_M(E) = \liminf_{\delta \to 0} \left( d - \frac{\log|E_\delta|}{\log \delta} \right) \]
%
If $\overline{\dim}_M(E) = \underline{\dim}_M(E)$, then we refer to this common quantity as the {\it Minkowski dimension} of $E$, denote $\dim_M(E)$. This means that as $\delta \to 0$, if $\dim_M(E) = \alpha$, then $|E_\delta| = \delta^{n - \alpha + o(1)}$. In particular, every set with positive Lebesgue measure has Minkowski dimension $d$, so Minkowski dimension is really only interesting for sets with Lebesgue measure zero. We can extend the lower and upper Minkowski dimension to unbounded sets $E$ by considering the supremum and infinum of the dimensions of all bounded subsets of $E$.

%\begin{example}
%	If $E = B^k \times \{ 0 \}^{n-k}$, where $B^k$ is the $k$ dimensional unit ball, then
	%
%	\[ B^k \times \delta B^{n-k} \subset E_\delta \subset (1 + \delta)B^k \times \delta B^{n-k} \]
	%
%	which shows that
	%
%	\[ \delta^{n-k} \lesssim |E_\delta| \lesssim (1 + \delta)^k \delta^{n-k} \]
	%
%	Thus $\dim_M(E) = k$.
%\end{example}

\begin{example}
	Let
	%
	\[ C = \left\{ \sum_{i = 1}^\infty a_i/4^i : a_i \in \{ 0, 3 \} \right\} \]
	%
	If $1/4^{N+1} \leq \delta \leq 1/4^N$, then
	%
	\[ \left\{ \sum_{i = 1}^\infty a_i/4^i : a_1, \dots, a_{N+1} \in \{ 0, 3 \} \right\} \subset C_\delta \subset \left\{ \sum_{i = 1}^\infty a_i/4^i : a_1, \dots, a_N \in \{ 0, 3 \} \right\} \]
	%
	The former set has volume $2^{N+1}/4^{N+1} = 1/2^{N+1} \geq \delta^{1/2}$, whereas the latter set has volume $2^N/4^N = 1/2^N \leq (2\delta)^{1/2}$. Thus $\log |C_\delta| = \log \delta/ 2 + O(1)$, and so
	%
	\[ \minkdim(C) = \lim_{\delta \to 0} \left( 1 - \frac{\log |C_\delta|}{\log \delta} \right) = 1 - 1/2 = 1/2 \]
	%
	So $C$ has Minkowski dimension $1/2$.
\end{example}

There are a few things to notice about this calculation. First, we performed an upper and lower bound on powers of $1/4^k$, and then used this to obtain bounds at all scales. And indeed, if we fix $M$, and consider $1/M^{k+1} \leq \delta \leq 1/M^k$, then as $k \to \infty$, $\log \delta \sim \log(1/M^k)$, and combined with the fact that $C_{1/M^{k+1}} \subset C_\delta \subset C_{1/M^k}$, we find
%
\[ \frac{|C_{1/M_k}|}{\log(1/M_k)} \sim \frac{|C_{1/M^k}|}{\log(\delta)} \leq \frac{|C_\delta|}{\log(\delta)} \leq \frac{|C_{1/M^{k+1}}|}{\log(\delta)} \sim \frac{|C_{1/M^{k+1}}|}{\log(1/M_{k+1})} \]
%
Thus for any set $E$ and any integer $M$,
%
\[ \lowminkdim(E) = \liminf_{n \to \infty} \frac{|C_{1/M^k}|}{\log(1/M^k)}\quad\text{and}\quad \upminkdim(E) = \limsup_{n \to \infty} \frac{|C_{1/M^k}|}{\log(1/M^k)} \]
%
The second point is that we understood the dimension of $C$ via a `Cantor-type' decomposition. Indeed, the set $C$ can be understood as the limit of a family of sets which are simple unions of intervals. If we set
%
\[ C_k = \left\{ \sum_{i = 1}^\infty a_i/4^i : a_1, \dots, a_{N+1} \in \{ 0, 3 \} \right\}. \]
%
Then $\{ C_k \}$ is a nested family of sets, with $C = \lim C_k$. If for an index $k$, we set
%
\[ \B^d_l = \{ [a_1, a_1 + l] \times \cdots \times [a_d, a_d + l] : a_k \in l \cdot \ZZ \}. \]
%
then $C_k$ is a union of $2^k$ cubes in $\B^d_{4^k}$, and
%
\[ \frac{\log(2^k)}{\log(4^k)} = \log_4(2) = 1/2. \]
%
One can also calculate the Minkowski dimension by counting the number of cubes in $\B^d_l$ intersecting the set. For a set $E$, set $\B^d_l(E)$ to be the set of all cubes in $\B^d_l$ intersecting $E$, i.e. $\B^d_l(E) = \{ I \in \B^d_l: I \cap E \neq \emptyset \}$.

\begin{lemma}
	If $E$ is a bounded set in $\RR^d$ and $M$ is an integer, then
	%
	\[ \lowminkdim(E) = \liminf_{l \to 0} \frac{\# \B^d_l(E)}{\log(1/l)}\quad\text{and}\quad \upminkdim(E) = \limsup_{l \to 0} \frac{\# \B^d_l(E)}{\log(1/l)} \]
	%
	where the limit is taken over lengths $l = 1/M^k$.
\end{lemma}
\begin{proof}
	Let $l = 1/M^k$. For each cube $I \in \B^d_l$, the $l$ thickening $I_l$ is contained in $3^d$ cubes in $\B^d_l$. Conversely, if $I \in \B^d_l(E)$, then $I \subset E_{d^{1/2} l}$, so $I \subset E_{M^{k_0} l}$ for $M^{k_0} \geq d^{1/2}$. Thus
	%
	\[ |E_l| \leq 3^d \# \B^d_l(E) l^d\quad\text{and}\quad|E_{M^{k_0} l}| \geq \# \B^d_l(E) l^d \]
	%
	So as $l \to 0$,
	%
	\[ d - \frac{\log |E_l|}{\log l} = \frac{\log(|E_l| l^{-d})}{\log(1/l)} \leq \frac{\log(3^d \# \B^d_l(E))}{\log(1/l)} = \frac{\# \B^d_l(E)}{\log(1/l)} + o_d(1) \]
	%
	and
	%
	\[ d - \frac{\log|E_{M^{k_0} l} |}{\log(M^{k_0} l)} = \frac{\log(|E_{M^{k_0} l}| (d^{1/2} l)^{-d})}{\log(1/l)} \geq o_d(1) + \frac{\# \B^d_l(E)}{\log(1/l)} \]
	%
	Taking limits completes the proof.
\end{proof}

We will often use `Cantor-type' constructions to form pattern avoiding sets. And Lemma 1 will be crucial either for calculating the Minkowski dimension of these constructions. It is especially useful when trying to work at discrete scales, because we can view $\bigcup \B^d_l(E)$ as a `discretization' of a set $E$ at the scale $l$.

\begin{example}
	We will often consider sets whose dimension behaves differently at various scales. This often occurs when performing multi-scale constructions where the scales in the construction decay inverse superexponentially. A toy example of this phenomenon can be obtained by modifying the last example slightly so that the construction of the Cantor set behaves differently at various scales. We fix an increasing sequence of integers $\{ N_k \}$, with $N_0 = 0$, and consider
	%
	\[ C = \left\{ \sum_{i = 1}^\infty a_i/4^i : a_i \in \{ 0, 3 \}\ \text{if there is $k \geq 0$ such that}\ N_{2k} \leq i \leq N_{2k+1} \right\} \]
	%
	Then $C = \lim C_n$, where
	%
	\[ C_n = \left\{ \sum_{i = 1}^\infty a_i/4^i : a_i \in \{ 0, 3 \}\ \text{if there is $k \leq n$ such that}\ N_{2k} \leq i \leq N_{2k+1} \right\} \]
	%
	Notice that $C_n$ is the union of
	%
	\[ 2^{N_1 - N_0} 4^{N_2 - N_1} 2^{N_3 - N_2} \dots 4^{N_{2n} - N_{2n-1}} = 2^{-N_1+N_2-N_3+\dots - N_{2n-1} + 2N_{2n}} \geq 2^{2N_{2n} - N_{2n-1}} \]
	% -N_1 + N_2 - N_3 + 2N_4
	sidelength $l_n$ cubes, where $l_n = 1/4^{N_{2n}}$. Each of these cubes intersects $C$, so
	%
	\[ \frac{\log \# \B^1_{l_n}(C)}{\log(1/l_n)} \geq \frac{2N_{2n} - N_{2n-1}}{2N_{2n}} = 1 - \frac{N_{2n-1}}{2N_{2n}} \]
	%
	Provided that $N_{2n-1}/N_{2n} = o(1)$, which occurs if the values $N_k$ increase superexponentially, i.e. if $N_k = 2^{k^2}$, we conclude that $\upminkdim(C) = 1$. On the other hand, $C_n$ is also the union of
	%
	\[ 2^{N_1 - N_0} 4^{N_2 - N_1} 2^{N_3 - N_2} \dots 4^{N_{2n} - N_{2n-1}} 2^{N_{2n+1} - N_{2n}} = 2^{- N_1 + N_2 - \dots + N_{2n} + N_{2n+1}} \leq 2^{N_{2n} + N_{2n+1}}. \]
	% -N_1 + N_2 - N_3 + N_4 + N_5
	%
	sidelength $r_n$ cubes, where $r_n = 1/4^{N_{2n+1}}$. Thus
	%
	\[ \frac{\log \# \B^1_{l_n}(C)}{\log(1/l_n)} \leq \frac{N_{2n} + N_{2n+1}}{2N_{2n+1}} = 1/2 + \frac{N_{2n}}{2N_{2n+1}} \]
	%
	Again, if $N_{2n}/N_{2n+1} = o(1)$, then $\lowminkdim(C) \leq 1/2$. One can fairly easily check that the values $r_n$ are the `worst case' scales, so that $\lowminkdim(C) = 1/2$. Thus the set $C$ `looks' half dimension between $l_n$ and $r_n$, for each $n$, but `looks' full dimensional between the scales $r_n$ and $l_{n+1}$.
\end{example}

Hausdorff dimension is a more stable version of fractal dimension which is obtained by finding a canonical `$s$ dimensional measure' $H^s$ on $\RR^d$ for $s \in [0,\infty)$, and then setting the dimension of $E$ to be the supremum of the values $s$ such that $H^s(E) < \infty$. A naive way the Hausdorff measure to construct is to assign a mass $r^s$ to each radius $r$ ball in $\mathbf{R}^n$, and then define
%
\[ H^{s,\infty}(E) = \inf \left\{ \sum r_k^s : E \subset \bigcup B(x_k,r_k) \right\} \]
%
This is an outer measure, and so Caratheodory's extension theorem gives a $\sigma$ algebra of measurable sets. Unfortunately, most sets are not measurable with respect to this $\sigma$ algebra. For instance, take $s = 1/2$, and $E = (a,b)$. On one hand, $H^{s,\infty}(E) \leq [(b-a)/2]^{1/2}$. On the other hand, if $(a,b)$ is covered by balls $B(x_k,r_k)$, then $\sum 2r_k \geq b - a$, so applying the concavity of $x \mapsto x^{1/2}$, we conclude
%
\[ \sum r_k^{1/2} \geq \left( \sum r_k \right)^{1/2} \geq \left( \frac{b - a}{2} \right)^{1/2} \]
%
Thus $H^{s,\infty}(E) = [(b-a)/2]^{1/2}$. But now we see that the additivity property begins to breakdown, since $H^{1/2,\infty}[0,1] = 2^{-1/2}$, whereas $H^{1/2,\infty}[0,1/2] = H^{1/2,\infty}[1/2,1] = 1/2$, and so $H^{1/2,\infty}[0,1] < H^{1/2,\infty}[0,1/2] + H^{1/2,\infty}[1/2,1]$. The reason for this is that $[0,1]$ is most efficiently covered by one large ball, rather than covering $[0,1/2]$ and $[1/2,1]$ separately. This is fixed by limiting the Hausdorff measure to be the value of the most efficient cover by arbitrarily small balls.

For a subset $E$ of Euclidean space, we define
%
\[ H_\delta^s(E) = \inf \left\{ \sum_{n = 1}^\infty \text{diam}(B_n)^s : E \subset \bigcup_{n = 1}^\infty B_n, \text{diam}(B_n) \leq \delta \right\} \]
%
We then define $H^s(E) = \lim_{\delta \to 0} H_\delta^s(E)$. Then $H^s$ is an exterior measure, and $H^s(E \cup F) = H^s(E) + H^s(F)$ if $d(E,F) > 0$. Thus all Borel sets are measurable with respect to $H^s$, which is certainly more satisfactory than the last definition.

\begin{example}
	Let $s = 0$. Then $H_\delta^0(E) = N_\delta^{\text{Ext}}(E)$, which tends to $\infty$ as $\delta \to 0$ unless $E$ is finite, and then $H_\delta^0(E) \to \# E$. Thus $H^0$ is just the counting measure.
\end{example}

\begin{example}
	Let $s = n$. If $E$ has Lebesgue measure zero, then for any $\varepsilon > 0$, there exists countable many balls $B(x_k,r_k)$ covering $E$ with $\sum r_k^n < \varepsilon$. Then $r_k < \varepsilon^{1/n}$, so $H^n_{\varepsilon^{1/n}}(E) < \varepsilon$. Letting $\varepsilon \to 0$, we conclude $H^n(E) = 0$. Thus $H^n$ is absolutely continuous with respect to the Lebesgue measure. The measure $H^n$ is translation invariant, so $H^n$ is actually a constant multiple of the Lebesgue measure. We let the constant multiple be defined $1/\omega_n$. The value $\omega_n$ can be defined as the volume of a unit ball in $\mathbf{R}^n$, since $H^n(B) = 1$ if $B$ is a unit ball.
\end{example}

The same argument shows that if $V$ is an $m$ dimensional subspace of $\mathbf{R}^n$, then $H^m$, restricted to subsets of $V$, is a constant multiple of the $m$ dimensional Lebesgue measure on $V$. More generally, $H^m$ measures the $m$ dimensional surface area of smooth, $m$ dimensional submanifolds of $\mathbf{R}^n$.

\begin{theorem}
	Let $U$ be an open subset of $\mathbf{R}^d$, and let $\phi: U \to \mathbf{R}^n$ be a smooth immersion. Then for any compact set $E$,
	%
	\[ H^d(\phi(E)) \propto \frac{1}{\omega_d} \int_E J(x)\; dx \]
	%
	where $J(x)$ is the square root of the sums of squares of the $d \times d$ minors of $D\phi(x)$.
\end{theorem}
\begin{proof}
	We may cover $E$ by finitely many open sets $U_1, \dots, U_N$, together with coordinate charts $y_1, \dots, y_N$ such that $(y_k \circ \phi)(x) = (x,f_k(x))$ for some smooth $f_k$, and fix $J_k$ such that for any $x \in U_k$, $|J(x) - J_k| < \varepsilon$. TODO: PROVE REST OF THEOREM.
\end{proof}

\begin{lemma}
	If $t < s$ and $H^t(E) < \infty$, $H^s(E) = 0$, and if $H^s(E) = \infty$, $H^t(E) = \infty$.
\end{lemma}
\begin{proof}
	If, for any cover of $E$ by balls $B(x_k,r_k)$, $\sum r_k^t \leq A$, and $r_k \leq \delta$, then $\sum r_k^s \leq \sum r_k^{s-t} r_k^t \leq \delta^{s-t} A$. Thus $H^s_\delta(E) \leq \delta^{s-t} A $, and taking $\delta \to 0$, we conclude $H^s(E) = 0$. The latter point is just proved by taking contrapositives.
\end{proof}

Thus given any Borel set $E$, there is $s$ such that $H^{s_0}(E) = 0$ for $s_0 < s$, and $H^{s_1}(E) = \infty$ for $s_1 > s$. We refer to $s$ as the Hausdorff dimension of $E$, denoted $\dim_H(E)$.

\begin{example}
	Consider $S = \{ (x,\sin(1/x)) : 0 < x \leq 1 \}$. Then for each $\delta > 0$, the set $S \cap [\delta,1] \times \mathbf{R}$ is a smooth curve, and therefore has Hausdorff dimension $1$. Thus for any $\varepsilon > 0$, $H^{1 + \varepsilon}(S \cap [\delta,1] \times \mathbf{R}) = 0$. But then taking limits as $\delta \to 0$, we conclude $H^{1+\varepsilon}(S) = 0$. Since $H^1(S) > 0$, this shows $S$ has Hausdorff dimension 1. Compare this to the Minkowski dimension $3/2$ result we obtained previously.
\end{example}

An easy way to compare the approaches to fractal dimension given by Minkowski and Hausdorff dimension is that Minkowski dimension measures the efficiency of covers of a set at a fixed scale, whereas Hausdorff dimension measures the efficiency of covers of a set at various, small scales.



% TODO: Explain how cantor set makes a continuous problem into a sequence of discrete problems.

\section{Branching Processes}

\section{Rusza: Difference Sets Without Squares}

In this section, we describe the work of Ruzsa on the discrete squarefree difference problem, which provides inspiration for our speculated results for the squarefree subset problem in the continuous setting. If $X$ and $Y$ are subsets of integers, we shall let $X \pm Y = \{ x \pm y: x \in X, y \in Y, x \pm y > 0 \}$ denote the sums and difference of the set. The {\it differences} of a set $X$ are elements of $X - X$, and so the squarefree difference set problem asks to consider how large a subset of the integers can be, whose differences do not contain the square of any positive integer. We let $D(N)$ denote the maximum number of integers which can be selected from $[1,N]$ whose differences do not contain a square.

\begin{example}
    The set $X = \{ 1, 3, 6, 8 \}$ is squarefree, because $X - X = \{ 2, 3, 5, 7 \}$, and none of these elements are perfect squares. On the other hand, $\{ 1, 3, 5 \}$ is not a squarefree subset, because $5 - 1 = 4$ is a perfect square.
\end{example}

There are a few tricks to constructing large subsets of integers avoiding squares. If $p$ is prime, then $p \mathbf{Z} \cap [1,p^2)$ avoids squares, because the difference of two numbers must be divisible by $p$, but not by $p^2$. If $N = p^2$, this gives a set with $N^{1/2}$ elements. However, we can do just as well without using any properties of the set of squares except for their sparsity, by greedily applying a sieve. We start by writing out a large list of integers $1,2,3,4,\dots,N$. Then, while we still have numbers to pick, we greedily select the smallest number $x_*$ we haven't crossed out of the list, add it to our set $X$ of squarefree numbers, and then cross out all integers $y$ such that $y - x_*$ is a positive square. Thus we cross out $x_*$, $x_* + 1$, $x_* + 4$, and so on, all the way up to $x_* + m^2$, where $m$ is the largest integer with $x_* + m^2 \leq N$. This implies $m \leq \sqrt{N - x_*} \leq \sqrt{N-1}$, hence we cross out at most $\sqrt{N-1} + 1$ integers whenever with add a new element $x_*$ to $X$. When the algorithm terminates, all integers must be crossed out, and if the algorithm runs $n$ iterations, a union bound gives that we cross out at most $n[\sqrt{N-1} + 1]$ integers, hence $n[\sqrt{N-1} + 1] \geq N$. It follows that the set $X$ we end up with contains $\Omega(N^{1/2})$ elements. What's more, this algorithm generates an increasing family of squarefree subsets of the integers as $n$ increases, so we may take the union of these subsets over all $N$ to find an infinite squarefree subset $X$ with $|X \cap [1,N]| = \Omega(\sqrt{N})$ for all $N$.

In 1978, S\'{a}rk\"{o}zy proved an upper bound on the size of squarefree subsets of the integers, showing $D(N) = O(N (\log N)^{-1/3 + \varepsilon})$ for every $\varepsilon > 0$. In particular, this proves a conjecture of Lov\'{a}sz that every infinite squarefree subset has density zero, because if $X$ is any infinite squarefree subset, then $|X \cap [1,N]| = o(N)$. S\'{a}rk\"{o}zy even conjectured that $D(N) = O(N^{1/2 + \varepsilon})$ for all $\varepsilon > 0$. Thus the sieve technique is essentially optimal, an incredibly pessimistic point of view, since the Sieve method doesn't depend on any properties of the set of perfect squares. Ruzsa's results shows we should be more optimistic, taking advantage of the digit expansion of numbers to obtain infinite squarefree subsets $X$ with $|X \cap [1,N]| = \Omega(N^{0.73})$. The method reduces the problem to a finitary problem of maximizing squarefree subsets modulo a squarefree integer $m$.

\begin{theorem}
    If $m$ is a squarefree integer, then
    %
    \[ D(N) \geq \frac{n^{\gamma_m}}{m} = \Omega_m(n^{\gamma_m}) \]
    %
    where
    %
    \[ \gamma_m = \frac{1}{2} + \frac{\log_m |R^*|}{m} \]
    %
    and $R^*$ denotes the maximal subset of $[1,m]$ whose differences contain no squares modulo $m$. Setting $m = 65$ gives
    %
    \[ \gamma_m = \frac{1}{2} \left( 1 + \frac{\log 7}{\log 65} \right) = 0.733077 \dots \]
    %
    and therefore $D(N) = \Omega(n^{0.7})$. For $m = 2$, we find $D(N) \geq \sqrt{N}/2$, which is only slightly worse than the sieve result.
\end{theorem}

\begin{remark}
    Let us look at the analysis of the sieve method backwards. Rather than fixing $N$ and trying to find optimal solutions of $[1,N]$, let's fix a particular strategy (to start with, the sieve strategy), and think of varying $N$ and seeing how the size of the solution given by the strategy on $[1,N]$ increases over time. In our analysis, the size of a solution is directly related to the number of iterations the stategy can produce before it runs out of integers to add to a solution set. Because we apply a union bound in our analysis, the cost of each particular new iteration is the same as the cost of the other iterations. If the cost of each iteration was independant of $N$, we could increase the solution size by increasing $N$ by a fixed constant, leading to family of solutions which increases on the order of $N$. However, as we increase $N$, the cost of each iteration increases on the order of $\sqrt{N}$, leading to us only being able to perform $N/\sqrt{N} = \sqrt{N}$ iterations for a fixed $N$. Rusza's method applies the properties of the perfect squares to perform a similar method of expansion. At an exponential cost, Rusza's method increases the solution size exponentially. The advantage of exponentials is that, since Rusza's is based on a particular parameter, a squarefree integer $m$, we can vary $m$ to improve the iteration numbers more naturally.
\end{remark}

The idea of Rusza's construction is to break the problem into exponentially large intervals, upon which we can solve the problem modulo an integer. More enerally, Rusza constructs a set whose differences are free of $d$'th powers.

\begin{theorem}
    Let $R \subset [1,m]$ be a subset of integers such that no difference is a power of $d$ modulo $m$, where $m$ is a {\it squarefree integer}. Construct the set
    %
    \[ A = \left\{ \sum_{k = 0}^n r_k m^k : 0 \leq n < \infty, r_k \in \left. \begin{cases} R & d\ \text{divides}\ N\\ [1,m] & \text{otherwise} \end{cases} \right\} \right\} \]
    %
    Then $A$ is squarefree.
\end{theorem}
\begin{proof}
    Suppose that we can write $\sum (r_k - r_k') m^k = N^d$. Let $s$ to be the smallest index with $r_s \neq r_s'$. Then $(r_s - r_s') m^s + M m^{s+1} = N^d$ where $M$ is some positive integer. If $s = ds_0$, then $(N/m^{s_0})^d = (r_s - r_s') + M m$, and this contradicts the fact that $r_s - r_s'$ cannot be a $d$'th power modulo $m$. On the other hand, we know $m^s$ divides $N^d$, but $m^{s+1}$ does not. This is impossible if $s$ is not divisible by $d$, because primes in $N^d$ occur in multiples of $d$, and $m$ is squarefree.
\end{proof}

For any $n$, we find
%
\[ A \cap [1,m^n - 1] = \left\{ \sum_{k = 0}^{n-1} r_km^k : r_k \in [1,m], r_k \in R\ \text{when $d$ divides $k$} \right\} \]
%
which therefore has cardinality
%
\begin{align*}
    |R|^{1 + \lfloor \frac{n-1}{d} \rfloor} m^{n-1- \lfloor \frac{n-1}{d} \rfloor} = m^n \left( \frac{|R|}{m} \right)^{1 + \lfloor \frac{n-1}{d} \rfloor} \geq m^n \left( \frac{|R|}{m} \right)^{n/d} = m^{n \gamma(m,d)}
\end{align*}
%
where $\gamma(m,d) = 1 - 1/d + \log_m |R|/d$. Therefore, for $m^{n+1} - 1 \geq k \geq m^n - 1$
%
\[ A \cap [1,k] \geq A \cap [1,m^n] \geq m^{n \gamma(m,d)} = \frac{m^{(n+1) \gamma(m,d)}}{m} \geq \frac{k^{\gamma(m,d)}}{m} \]
%
This completes Rusza's construction. Thus we have proved a more general result than was required.

\begin{theorem}
    For every $d$ and squarefree integer $m$, we can construct a set $X$ whose differences contain no $d$th powers and
    %
    \[ |X \cap [1,n]| \geq \frac{n^{\gamma(d,m)}}{m} = \Omega(n^{\gamma(d,m)}) \]
    %
    where $\gamma(d,m) = 1 - 1/d + \log_m |R^*|/d$, and $R^*$ is the largest subset of $[1,m]$ containing no $d$'th powers modulo $m$.
\end{theorem}

For $m = 65$, the group $\mathbf{Z}_{65}^* \cong \mathbf{Z}_{5}^* \times \mathbf{Z}_{13}^*$ has a set of squarefree residues of the form $\{ (0,0), (0,2), (1,8), (2,1), (2,3), (3,9), (4,7) \}$, which gives the required value for $\gamma_{65}$. In 2016, Mikhail Gabdullin proved that if $m$ is squarefree, then in $\mathbf{Z}_m$, any set $R$ such that $R - R$ is squarefree has $|R| \leq me^{-c \log m / \log \log m}$, where $n$ denotes the number of odd prime divisors of $m$, so that
%
\[ \gamma(d,m) \leq 1 - 1/d + \frac{\log(me^{-c \log m / \log \log m})}{m} \]
%
Rusza believes that we cannot choose $m$ to construct squarefree subsets of the integers growing better than $\Omega(n^{3/4})$, and he claims to have proved this assuming $m$ is squarefree and consists only of primes congruent to 1 modulo 4. Looking at some sophisticated papers in number theory (Though I forgot to write down the particular references), it seems that using modern estimates this is quite easy to prove. Thus expanding on Rusza's result in the discrete case requires a new strategy, or perhaps Rusza's result is the best possible.

Let $D(N,d)$ denote the largest subset of $[1,N]$ containing no $d$th powers of positive integer. The last part of Rusza's paper is devoted to lower bounding the polynomial growth of $D(N,d)$ asymptotically.

\begin{theorem}
    If $p$ is the least prime congruent to one modulo $2d$, then
    %
    \[ \limsup_{N \to \infty} \frac{\log D(N,d)}{\log N} \geq 1 - \frac{1}{d} + \frac{\log_p d}{d} \]
\end{theorem}
\begin{proof}
    The set $X$ we constructed in the last theorem shows that for any $m$,
    %
    \[ \frac{\log D(N,d)}{\log n} \geq \gamma(d,m) - \frac{\log m}{\log n} = 1 - \frac{1}{d} + \frac{\log_m |R^*|}{d} - \frac{\log m}{\log n} \]
    %
    Hence
    %
    \[ \limsup_{N \to \infty} \frac{\log D(N,d)}{\log n} \geq 1 - \frac{1}{d} + \frac{\log_m |R^*|}{d} \]
    %
    The claim is then completed by the following lemma.
\end{proof}

\begin{lemma}
    If $p$ is a prime congruent to $1$ modulo $2d$, then we can construct a set $R \subset [1,p]$ whose differences do not contain a $d$th power modulo $p$ with $|R| \geq d$.
\end{lemma}
\begin{proof}
    Let $Q \subset [1,p]$ be the set of powers $1^k, 2^k, \dots, p^k$ modulo $p$. We have
    %
    \[ |Q| = \frac{p-1}{k} + 1 \]
    %
    This follows because the nonzero elements of $Q$ are the images of the group homomorphism $x \mapsto x^k$ from $\mathbf{Z}_p^*$ to itself. Since $\mathbf{Z}_p^*$ is cyclic, the equation $x^k = 1$ has the same number of solutions as the equation $kx = 0$ modulo $p-1$, and since $p \equiv 1$ modulo $2k$, there are exactly $k$ solutions to this equation. The sieve method yields a $k$th power modulo $p$ free subset of size greater than or equal to
    %
    \[ p/q = \frac{p}{1 + \frac{p-1}{k}} = \frac{pk}{p + k - 1} \to k \]
    %
    as $p \to \infty$, which is greater than $k-1$ for large enough $p$. This shows the theorem is essentially trivial for large primes. However, for smaller primes a more robust analysis is required. We shall construct a sequence $b_1, \dots, b_k \in \mathbf{Z}_p$ such that $b_i - b_j \not \in Q$ for any $i,j$ and $|B_j + Q| \leq 1 + j(q-1)$. Given $b_1, \dots, b_j$, let $b_{j+1}$ be any element of $(B_j + Q + Q) - (B_j + Q)$. Since $b_{j+1} \not \in B_j + Q$, $b_{j+1} - b_i \not \in Q$ for any $i$. Since $b_{j+1} \in B_j + Q + Q$, the sets $B_j + Q$ and $b_{j+1} + Q$ are not disjoint (we have used $Q = -Q$, which is implied when $p \equiv 1$ mod $2k$), and so
    %
    \begin{align*}
        |B_{j+1} + Q| &= |(B_j + Q) \cup (b_{j+1} + Q)|\\
        &\leq |B_j + Q| + |b_{j+1} + Q| - 1\\
        &\leq 1 + j(q-1) + q - 1\\
        &= 1 + (j+1)(q-1)
    \end{align*}
    %
    This procedure ends when $B_j + Q + Q = B_j + Q$, and this can only happen if $B_j + Q = \mathbf{Z}_p$, because we can obtain all integers by adding elements of $Q$ recursively, so $1 + j(q-1) \geq p$, and thus $j \geq k$.
\end{proof}

\begin{corollary}
    In the special case of avoiding squarefree numbers, we find 
    \[ \limsup \frac{\log D(N)}{\log N} \geq \frac{1}{2} + \frac{\log_5 2}{2} = 0.71533\dots \]
    %
    which is only slightly worse than the bound we obtain with $m = 65$.
\end{corollary}

Rusza's leaves the ultimate question of whether one can calculate
%
\[ \alpha = \lim_{N \to \infty} \log D(N) / \log N \]
%
or even whether it exists at all. The consequence of this would essentially solve the squarefree integers problem, since it gives the exact growth of $D(N)$ in terms of a monomial. Because of how conclusive this problem is, we should not expect to find a nontrivial way to calculate this constant.









\section{Keleti: A Translate Avoiding Set}

Keleti's two page paper constructs a full dimensional subset $X$ of $[0,1]$ such that $X$ intersects $t + X$ in at most one place for each nonzero real number $t$. Malabika has adapted this technique to construct high dimensional subsets avoiding nontrivial solutions to differentiable functions. In this section, and in the sequel, we shall find it is most convenient to avoid certain configurations by expressing them in terms of an equation, whose properties we can then exploit. One feature of translation avoidance is that the problem is specified in terms of a linear equation.

\begin{lemma}
    A set $X$ avoids translates if and only if there do not exists values $x_1 < x_2 \leq x_3 < x_4$ in $X$ with $x_2 - x_1 = x_4 - x_3$.
\end{lemma}
\begin{proof}

    Suppose $t + X \cap X$ contains two points $a < b$. Without loss of generality, we may assume that $t > 0$. If $a \leq b - t$, then the equation $a - (a - t) = t = b - (b - t)$ satisfies the constraints, since $a - t < a \leq b - t < b$ are all elements of $X$. We also have $(b - t) - (a - t) = b - a$ which satisfies the constraints if $a - t < b - t \leq a < b$. This covers all possible cases. Conversely, if there are $x_1 < x_2 \leq x_3 < x_4$ in $X$ with $x_2 - x_1 = t = x_4 - x_3$, then $X + t$ contains $x_2 = x_1 + (x_2 - x_1)$ and $x_4 = x_3 + (x_4 - x_3)$.
\end{proof}

%\footnote{We always assume $L_n/L_{n+1}$ is an integer so that intervals in $\mathcal{B}(L_n)$ are either almost disjoint from intervals in $\mathcal{B}(L_{n+1})$ or contained completely within such an interval}

The basic, but fundamental idea to Keleti's technique is to introduce memory into Cantor set constructions. Keleti constructs a nested family of discrete sets $X_0 \supset X_1 \supset \dots$ converging to $X$, with each $X_N$ a union of disjoint intervals in $\mathcal{B}(L_N)$, for a decreasing sequence of lengths $L_N$ converging to zero, to be chosen later, but with $10 L_{N+1} \mid L_N$. We initialize $X_0 = [0,1]$, and $L_0 = 1$. Furthermore, we consider a queue of intervals, initially just containining $[0,1]$. To construct $X_1, X_2, \dots$, Keleti iteratively performs the following procedure:
%
\begin{algorithm}
    \begin{algorithmic}%[1]
        \caption{Construction of the Sets $X_N$}
        \State{Set $N = 0$}
        \MRepeat
            \State{Take off an interval $I$ from the front of the queue}

            \MForAll{\ $J \in \mathcal{B}(L_N)$ contained in $X_N$:}
                \State{Order the intervals in $\mathcal{B}(L_{N+1})$ contained in $J$ as $J_0, J_1, \dots, J_M$}

                \State{{\bf If} $J \subset I$, add all intervals $J_i$ to $X_{N+1}$ with $i \equiv 0$ modulo 10}
                \State{{\bf Else} add all $J_i$ with $i \equiv 5$ modulo 10}
            \EndForAll
            \State{Add all intervals in $\mathcal{B}(L_{N+1})$ to the end of the queue}
            \State{Increase $N$ by 1}
        \EndRepeat   
    \end{algorithmic}
\end{algorithm}

After each iteration of the algorithm, we obtain a new set $X_{N+1}$, and so leaving the algorithm to repeat infinitely produces a sequence of sets $X_1, X_2, \dots$ converging to a set $X$. We claim that with the appropriate choice of parameters, $X$ is a translate avoiding set.

If $X$ is not translate avoiding, there is $x_1 < x_2 \leq x_3 < x_4$ with $x_2 - x_1 = x_4 - x_3$. Since $L_N \to 0$, there is $N$ such that $x_1$ is contained in an interval $I \in \mathcal{B}(L_N)$ that $x_2,x_3, x_4$ are not contained in. At stage $N$ of the algorithm, the interval $I$ is added to the end of the queue, and at a much later stage $M$, the interval $I$ is retrieved. Find the startpoints $x_1^\circ, x_2^\circ$, $x_3^\circ, x_4^\circ \in L_M \mathbf{Z}$ to the intervals in $\mathcal{B}(L_M)$ containing $x_1$, $x_2$, $x_3$, and $x_4$. Then we can find $n$ and $m$ such that $x_4^\circ - x_3^\circ = (10n)L_M$, and $x_2^\circ - x_1^\circ = (10m + 5)L_M$. In particular, this means that $|(x_4^\circ - x_3^\circ) - (x_2^\circ - x_1^\circ)| \geq 5L_M$. But
%
\begin{align*}
    |(x_4^\circ - x_3^\circ) - (x_2^\circ - x_1^\circ)| &= |[(x_4^\circ - x_3^\circ) - (x_2^\circ - x_1^\circ)] - [(x_4 - x_3) - (x_2 - x_1)]|\\
    &\leq |x_1^\circ - x_1| + \dots + |x_4^\circ - x_4| \leq 4 L_M
\end{align*}
%
which gives a contradiction.

The algorithm shows that $X_N$ contains $L_{N-1} / 10 L_N$ times the number of intervals that $X_{N-1}$ has, but they are at a length $L_N$ rather than $L_{N-1}$. This means that in total, $X_N$ contains $1/10^N L_N$ intervals, of length $L_N$. Since $L_N / 10^N L_N = o(1)$, this shows our set will have Lebesgue measure zero irrespective of our parameters. However, if $L_N$ decays suitably fast, then we might have $L_N^{1 - \varepsilon}/10^N L_N \gtrsim_\varepsilon 1$ for all $\varepsilon > 0$, which would imply that $X$ has positive $1 - \varepsilon$ dimensional Hausdorff measure for all $\varepsilon$, so $X$ still has Hausdorff dimension one. For this to be true, $L_N$ must decay superexponentially, i.e. the inequalities above are equivalent to $L_N \lesssim_B 1/B^N$ for all choices of $B$. Choosing an arbitrarily fast decaying sequence, such as $L_N = 1/N! \cdot 10^N$ or $L_N = 1/10^{10^N}$, suffices to obtain a Hausdorff dimension one set.

\begin{lemma}
    If $L_N$ decays superexponentially, $X$ has Hausdorff dimension one.
\end{lemma}
\begin{proof}
%Recall Frostman's lemma, which says that the $s$ dimensional Hausdorff measure $H_s(X)$ of a Euclidean set $X$ is positive if and only if there is a finite positive Borel measure $\mu$ supported on $X$ with $\mu(B_r(x)) \lesssim r^s$, for a universal constant depending only on $\mu$. If such a measure can be constructed on a set $X$, it therefore follows that $\dim_{\mathbf{H}}(X) \geq s$. Thus to prove $X$ has dimension one, it suffices to construct a probability measure $\mu$ on $X$ with $\mu(B_r(x)) \lesssim_s r^s$, for each $s < 1$. We can construct such a measure using what is often called the {\it mass distribution principle}; we construct a probability measure $\mu_n$ supported on $X_n$ in such a way that a weak limit $\mu = \lim \mu_n$ exists, in which case $\mu$ is supported on $X$. To do this, we let $\mu_1$ be the uniform probability measure on $[0,1]$. Then, to construct $\mu_{n+1}$ from $\mu_n$, we divide the mass of each interval $J$ in $X_n$ uniformly over the intervals in $X_{n+1}$ contained in $J$. The distribution functions of these measures converge uniformly, and therefore the $\mu_n$ converge weakly to a measure $\mu$ supported on $X$.

We use the mass distribution principle, as used in our note on calculating Hausdorff dimensions. It is easy to establish the bounds $\mu_N(I) \lesssim_\varepsilon L(I)^{1-\varepsilon}$ for $I \in \mathcal{B}(L_N)$, and since we can choose $L_N$ suitably slowly decreasing to use the epsilon of room technique, this gives the result. Alternatively, we can use the uniform distribution bounds with $L_N = R_N$, since if $J \in \mathcal{B}(L_{N+1})$, $I \in \mathcal{B}(L_N)$, $\mu(J) = 1/10^{N+1} L_{N+1}$, $\mu(I) = 1/10^N L_N$, and so $\mu(J) \lesssim (L_{N+1}/L_N) \mu(I)$. This gives the result if $L_N$ grows too fast.
\end{proof}

%\begin{remark}
%    Here's why we need the tighter bound $\mu(I) \lesssim_\varepsilon |I|^{1-\varepsilon}/(n!)^{\varepsilon/2}$ at the discrete scales to successively interpolate our bounds to all interval scales, rather than just the simpler bound $\mu(I) \lesssim_\varepsilon |I|^{1-\varepsilon}$. If $L_{N+1} \leq |I| \leq L_N$, and we cover $I$ by $|I|L_{N+1}^{-1}$ length $L_{N+1}$ intervals, then we obtain that
    %
%    \[ \mu(I) \lesssim_\varepsilon |I|L_{N+1}^{-1} L_{N+1}^{1-\varepsilon} = |I| L_{N+1}^{-\varepsilon} = \left( \frac{|I|}{L_{N+1}} \right)^\varepsilon |I|^{1-\varepsilon} \]
    %
%    Similarily, if we cover $I$ by a single length $L_N$ interval, then
    %
%    \[ \mu(I) \lesssim_\varepsilon L_N^{1-\varepsilon} = \left( \frac{L_N}{|I|} \right)^{1-\varepsilon} |I|^{1-\varepsilon} \]
    %
%    If we are to hope that these bounds give us a $\lesssim_\varepsilon |I|^{1-\varepsilon}$ bound for all $\varepsilon$, then we must have
    %
%    \[ \max_{L_{N+1} \leq |I| \leq L_N} \min \left( \left( \frac{|I|}{L_{N+1}} \right)^\varepsilon, \left( \frac{L_N}{|I|} \right)^{1-\varepsilon} \right) \lesssim_\varepsilon 1 \]
    %
%    The minimization is maximized when
    %
%    \[ \left( \frac{|I|}{L_{N+1}} \right)^\varepsilon = \left( \frac{L_N}{|I|} \right)^{1-\varepsilon} \]
    %
%    or when $|I| = L_N^{1-\varepsilon} L_{N+1}^\varepsilon$. Inputting this into the formula, we obtain that
    %
%    \[ \max_{L_{N+1} \leq |I| \leq L_N} \min \left( \left( \frac{|I|}{L_{N+1}} \right)^\varepsilon, \left( \frac{L_N}{|I|} \right)^{1-\varepsilon} \right) = \left( \frac{L_N}{L_{N+1}} \right)^{\varepsilon (1 - \varepsilon)} \]
    %
%    With the choice of parameters given, we have $L_N/L_{N+1} = 8(n+1)$, and we do not have $(8(n+1))^{\varepsilon(1-\varepsilon)} \lesssim_\varepsilon 1$. Thus, with the bounds we have used, there is no way to obtain a constant coefficient bound for all scales lying inbetween the discrete scales if we use the $\mu(I) \lesssim_\varepsilon |I|^{1-\varepsilon}$ bound for the discrete scales. However, the tighter bound $\mu(I) \lesssim_\varepsilon |I|^{1-\varepsilon}/(n!)^{\varepsilon/2}$ causes the $O(n)$ term for $L_N/L_{N+1}$ to be annihilated, which results in a constant term bound at the continuous range of scales.
%\end{remark}

\begin{remark}
    Keleti briefly remarks that by replacing the 10 in the algorithm with a slowly increasing set of numbers, one can obtain a Hausdorff dimension one set which is linearly independant over the rational numbers. To see why this works, the condition of linear independence would fail if $\smash{a_1 x_1 + \dots + a_M x_M = 0}$, where $\smash{x_1 < x_2 < \dots < x_M}$, and $\smash{a_1, \dots, a_M}$ are integers with no common factor. One can again reduce this by picking intervals with indices congruent to a certain large modulus.

%     Just as before, we find $x_n^\circ$ with $0 \leq x_n - x_n^\circ \leq L_N$. Provided that $2 (a_1 + \dots + a_M) L_N < M_M$, and the $x_n^\circ$ lie at integer multiples of $\varepsilon_N$, we conclude that $a_1 x_1^\circ + \dots + a_M x_M^\circ = 0$. If $K$ is an integer not dividing $a_1$, then for suitably large $N$ we assume that each $x_2^\circ, \dots, x_M^\circ$ lies at multiples of $K\Delta_N$. Shifting $x_1^\circ$ by a single multiple of $\Delta_N$ then breaks the equation from ever occuring in the first place. In order to guarantee this, we must first set $\varepsilon_n = A_n! L_n$ where $A_n$ is an increasing sequence with $A_n \to \infty$. We also guarantee that $x_2^\circ, \dots, x_M^\circ$ lies at multiples of $A_n! \Delta_n$. This can be guaranteed by induction if $A_{n+1}! \Delta_{n+1} \divides \Delta_n, \varepsilon_{n+1}$. Thus the parameters
    %
%    \[ \Delta_n = L_n\ \ \ \varepsilon_n = A_n! L_n\ \ \ L_{n+1} = \frac{L_n}{2N_{n+1}A_{n+1}!} \]
    %
%    give a linearly independant set. Assuming the $A_n$ grow incredibly slowly relative to the $N_n$, i.e.
    %
%    \[ N_n = n\ \ A_n = \log \log n + O(1)\ \ \ \ N_n = 2^n\ \ A_n = \log n + O(1)\ \ \ \ N_n = 2^{n^2}\ \ A_n = n \]
    %
%    then we obtain a set with Hausdorff dimension one.
%Assuming the $A_n$ grow incredibly slowly, we can still hope for this set to have Hausdorff dimension one. fI we construct the probability measure $\mu$ as before, we find that for any length $l_n$ interval $J$
    %
%    \[ \mu(J) \leq \frac{2}{n!} = \frac{2}{n!l_n^{1-\varepsilon}} l_n^{1-\varepsilon} = \left( \frac{2}{(n!)^\varepsilon} \left( \prod A_m! \right)^{1-\varepsilon} \right) l_n^{1-\varepsilon} \]
    %
%    We can choose the $A_m$ to grow slowly enough that for any $\varepsilon > 0$,
    %
%    \[ \left( \prod A_m! \right)^{1-\varepsilon} \lesssim_\varepsilon (n!)^{\varepsilon/2} \]
    %
%    Testing this inequality leads to the fact that $A_{n+1}! \leq (n+1)^{\varepsilon/2(1-\varepsilon)}$ must eventually hold for $n$ large enough, so taking $A_n \to \infty$ but growing slower than any polynomial in $n$ satisfies the inequality, i.e. if $A_n!$ is the largest factorial smaller than $\log n$. Thus
    %
%    \[ \mu(J) \lesssim_\varepsilon \frac{l_n^{1-\varepsilon}}{(n!)^{\varepsilon/2}} \]
    %
%    and the interpolation bound as in the previous problem then guarantee $\mu(I) \lesssim_\varepsilon |I|^{1-\varepsilon}$ for all $I$, since $l_n/l_{n+1} = O(nA_n!) = O((n!)^{1/2})$.
\end{remark}

%\begin{remark}
%    We attempted to obtain a squarefree subset of $[0,1]$ by combining Ruzsa's squarefree discrete strategy with Keleti's decomposition approach to find a high dimensional continuous squarefree set. However, using these techniques we were only able to obtain a dimension 1/2 set, which is only slightly better than a dimension 1/3 set which exists from the general results given by Math\'{e}'s result, or Pramanik and Fraser's result, and is much less than the dimension 1 set that Malabika expects.
%\end{remark}







\section{Fraser/Pramanik: Extending Keleti Translation to Smooth Configurations}

Inspired by Keleti's result, Pramanik and Fraser obtained a generalization of the queue method which allows one to find sets avoiding solutions to {\it any} smooth function satisfying suitably mild regularity conditions. To do this, rather than making a linear shift in one of the intervals we avoid as in Keleti's approach, one must use the smoothness properties of the function to find large segments of an interval avoiding solutions to another interval.

\begin{theorem}
    Suppose that $f: \mathbf{R}^{d+1} \to \mathbf{R}$ is a $C^1$ function, and there are sets $T_0, \dots, T_d \subset [0,1]$, with each $T_n$ a union of almost disjoint closed intervals of length $1/M$ such that $A \leq |\partial_0 f|$ and $|\nabla f| \leq B$ on $T_0 \times \dots \times T_d$. There there exists a rational constant $C$ and arbitrarily large integers $N \in M \mathbf{Z}$ for which there exist subsets $S_n \subset T_n$ such that
    %
    \begin{itemize}
        \item[(i)] $f(x) \neq 0$ for $x \in S_0 \times \dots \times S_d$.

        \item[(ii)] For $n \neq 0$, if we divide each interval $T_n$ into length $1/N$ intervals, then $S_n$ contains an interval of length $C/N^d$ of each of these intervals.

        \item[(iii)] If $T_0$ is split into length $1/N$ intervals, then for a fraction $1 - 1/M$ of such intervals, $S_0$ is a union of length $C/N^d$ intervals with total length $C/N$.
    \end{itemize}
\end{theorem}
\begin{proof}
    We begin by dividing the sets $T_1, \dots, T_d$ into length $1/N$ intervals, and let $S_n$ be defined by including a length $C_0/N^d$ segment, for some constant $C_0$ to be chosen later. Then once we fix $C_0$, the $S_n$ will satisfy property (ii) of the theorem. We define
    %
    \[ \mathbf{A} = \{ a \in \mathbf{R}^{d-1} : a_n\ \text{is a startpoint of a length $1/N$ interval in}\ T_n \} \]
    %
    Then $|\mathbf{A}| \leq N^d$, since each interval $T_n$ is contained in $[0,1]$, and therefore can only contain at most $N$ almost disjoint intervals of length $1/N$. Hence if we define the set of `bad points' in $T_0$ as
    %
    \[ \mathbf{B} = \{ x \in T_0: \text{there is}\ a \in \mathbf{A}\ \text{such that}\ f(x,a) = 0 \} \]
    %
    Then $|\mathbf{B}| \leq MN^d$. This is because for each fixed $a$, the function $x \mapsto f(x,a)$ is either strictly increasing or decreasing over each interval in the decomposition of $T_0$, or which there are at most $M$ because $T_0 \subset [0,1]$. If we split $T_0$ into length $1/N$ intervals, and choose a subcollection of such intervals $I$ such that $|I \cap \mathbf{B}| \leq M^3N^{d-1}$, then we throw away at most $MN^d/M^3N^{d-1} = N/M^2$ intervals, and so we keep $(N/M)(1 - 1/M)$ intervals, which is $1 - 1/M$ of the total number of intervals in the decomposition of $T_0$. The lemma we prove after this theorem implies that there exists a constant $C_1$ such that if $x \in S_n$, and $f(y,x) = 0$, then $d(y,\mathbf{B}) \leq C_0C_1/N^d$. If we split each interval $I$ with $|I \cap \mathbf{B}| \leq M^3N^{d-1}$ into $4M^3N^{d-1}$ length $1/4M^3N^d$ intervals, and we choose $C_0$ such that $C_0C_1 < 1/4M^3$, then the set $S_0$ obtained by discarding each interval that contains or is adjacent to an interval containing an element of $\mathbf{B}$ satisfies $d(S_0,\mathbf{B}) > C_0C_1/N^d$, and therefore there does not exist any $x_n \in S_n$ and $y \in S_0$ such that $f(y,x) = 0$. $S_0$ satisfies property (iii) of the theorem since for the interval $I$ we are considering, we keep at least $M^3N^{d-1}$ length $1/4M^3N^d$ intervals, which in total has length at least $1/4N$.
\end{proof}

\begin{remark}
    The length $1/N$ portion of each interval guaranteed by (iii) is unneccesary to the Hausdorff dimension bound, since the slightly better bounds obtained on scales where an interval is dissected as a $1/N$ are decimated when we eventually divide the further subintervals into $1/N^{d-1}$ intervals. The importance of (iii) is that it implies that the set we will construct has full {\it Minkowski dimension}. The reason for this is that Minkowski dimension lacks the ability to look at varying dissection depths at once, and since, at any particular depth, there exists a length $1/N$ dissection, the process appears to Minkowski to be full dimensional, even though at later scales this $1/N$ dissection is dissected into $1/N^{d-1}$ intervals.
\end{remark}

\begin{lemma}
    Given the $f$, $T_0, \dots, T_d$, there exists a constant $C_1$ depending on these quantities, such that for any $C_0$, and $x \in S_1 \times \dots \times S_{d-1}$, if $f(y,x) = 0$, then $d(y, \mathbf{B}) \leq C_0C_1/N^{d-1}$.
\end{lemma}
\begin{proof}
    Since $T_0 \times \dots \times T_d$ breaks into finitely many cubes with sidelengths $1/M$, it suffices to prove the theorem for a particular cube $J$ in this decomposition, where we assume the zeroset of $f$ intersects $J$. If $J = I \times J'$, where $I$ is an interval, we let $U$ be the set of all $x \in J'$ for which there is $y$ in the interior of $I$ such that $f(y,x) = 0$. Then $U$ is open. The implicit function theorem implies that there exists a $C^1$ function $g: U \to I$ such that $f(x,y) = 0$ if and only if $y = g(x)$. Then the function $h(x) = f(x,g(x))$ vanishes uniformly, so
    %
    \[ 0 = \partial_n h(x) = (\partial_n f) (g(x),x) + (\partial_0 f) (g(x),x) \partial_n g(x) \]
    %
    Hence for $x \in U$,
    %
    \[ |(\nabla g)(x)| = \frac{|(\nabla f)(x)|}{|(\partial_d f)(x,g(x))|} \leq \frac{B}{A} \]
    %
    If $N$ is chosen large enough, then for every $x \in U \cap (S_1 \times \dots \times S_d)$ there is $a \in \mathbf{A} \cap U$ in the same connected component of $U$ as $x$ with $|x - a| \lesssim C_0/N^{d-1}$, and this means that
    %
    \[ |g(x) - g(a)| \leq \| \nabla g \|_\infty |x - a| \lesssim \frac{BC_0}{A N^{d-1}} \]
    %
    and $g(a) \in \mathbf{B}$, completing the proof.
\end{proof}

How do we use this lemma to construct a set avoiding solutions to $f$? We form an infinite queue which will eventually filter out all the possible zeroes of the equation. Divide the interval $[0,1]$ into $d$ intervals, and consider all orderings of $d - 1$ subsets of these intervals, and add them to the queue. Now on each iteration $N$ of the algorithm, we have a set $X_N \subset [0,1]$. We take a particular sequence of intervals $T_1, \dots, T_d$ from the queue, and then use the lemma above to dissect the $X_N \cap T_n$, which are unions of intervals, into sets avoiding solutions to the equation, and describe the remaining points as $X_{N+1}$. We then add all possible orderings of $d$ intervals created into the end of the queue, and rinse and repeat. The set $X = \lim X_n$ then avoids all solutions to the equation with distinct inputs.

What remains is to bound the Hausdorff dimension of $X$ by constructing a probability measure supported on $X$ with suitable decay. To construct our probability measure, we begin with a uniform measure on the interval, and then, whenever our interval is refined, we uniformly distribute the volume on that particular interval uniformly over the new refinement. Let $\mu$ denote the weak limit of this sequence of probability distributions. At each step $n$ of the process, we let $1/M_n$ denote the size of the intervals at the beginning of the $n$'th subdivision, $1/N_n$ denote the size of the split intervals in the lemma, and $C_n$ the $n$'th constant. We have the relation $1/M_{n+1} = C_n/N_n^{d-1}$. If $K$ is a length $1/M_{N+1}$ interval, $J$ a length $1/N_N$ interval, and $I$ a length $1/M_N$ interval with $K \subset J \subset I$ and all recieving some mass in $\mu$. To calculate a bound on their mass, we consider the decompositions considered in the algorithm:
%
\begin{itemize}
        \item If $J$ is subdivided in the non-specialized manner, then every length $1/N_N$ interval recieves the same mass, which is allocated to a single length $1/M_{N+1}$ interval it contains. Thus $\mu(K) = \mu(J) \leq (M_N/N_N) \mu(I)$.
        \item In the second case, at least a fraction $1 - 1/M_N$ of the length $1/N_N$ intervals are assigned mass, so $\mu(J) = (M_N/N_N)(1 - 1/M_N)^{-1} \leq (2M_N/N_N) \mu(I)$, and more than $C_N/N_N$ of each length $1/N_N$ interval is maintained, so
        %
        \[ \mu(K) = \frac{N_N}{C_NM_{N+1}} \mu(J) \leq \frac{2M_N}{C_NM_{N+1}} \mu(I) \]
\end{itemize}
%
Thus in both cases, we have $\mu(J) \lesssim (N_N/M_N) \mu(I)$, $\mu(K) \lesssim_N |K|$, and $N_N = M_{N+1}^{1/(d-1)}/C_N \lesssim_N M_{N+1}^{1/(d-1)}$. From this, we conclude using the results of the appendix that there exists a family of rapidly decaying parameters which gives a $1/(d-1)$ dimensional set.

\begin{remark}
    The set $X$ constructed is precisely a $1/(d-1)$ dimensional set. Recall that $X = \lim X_n$, where $X_n$ is a union of a certain number of length $1/M_n$ intervals $I_1, \dots, I_N$. For each $n$, the interval $I_i$ is inevitably subdivided at a stage $J_i$ into length $C_{J_i} N_{J_i}^{1-d}$ intervals for each length $1/N_{J_i}$ interval that $I_i$ contains. Thus
    %
    \[ H_{1/M_n}^\alpha(X) \leq \sum_{i = 1}^N \frac{N_{m_i}}{M_n} (C_{m_i} N_{m_i}^{1-d})^\alpha = \frac{1}{M_n} \sum_{i = 1}^N C_{m_i}^\alpha N_{m_i}^{1 - \alpha(d-1)} \]
    %
    We may assume that $C_{m_i} \leq 1$, so if $\alpha > 1/(d - 1)$, using the fact that $N \leq M_n$, since $X_n$ is contained in $[0,1]$, we obtain
    %
    \[ H_{1/M_n}^\alpha(X) \leq \frac{1}{M_n} \sum_{i = 1}^N N_{m_i}^{1 - \alpha(d-1)} \leq N_{\max(m_i)}^{1 - \alpha(d-1)} \leq 1 \]
    %
    Thus, taking $n \to \infty$, we conclude $H^\alpha(X) \leq 1 < \infty$, so as $\alpha \downarrow 1/(d - 1)$, we conclude that $X$ has Hausdorff dimension bounded above by $1/(d-1)$.
\end{remark}

%Thus, in both cases, we have $\mu(J) \lesssim (N_N/M_N) \mu(I)$, which means we can apply the second method of appendix to calculate Hausdorff dimension with rapidly growing constants, where $l_N = 1/M_N$ and $r_N = 1/N_N$. We have $\mu(K) \lesssim_N $ and $N_N = M_{N+1}^{1/(d-1)}/C_N$ and


%
%Thus, in both cases, we have $\mu(J) \lesssim_N 1/M_{N+1}$. If $J \subset I$ is any length $1/N_N$ interval considered in the algorithm, then either $\mu(J) = (M_N/N_N) \mu(I)$, as in the first case of the subdivision, or we can apply the second case of the subdivision, giving $\mu(J) = (M_N/N_N)(1 - 1/M_N)^{-1} \mu(I) \leq (2M_N/N_N) \mu(I)$. This means we can apply the second method in the appendix. The fact that 

%by induction, if $I$ is a length $1/M_N$ interval considered in the process, then
%
%\begin{align*}
%    \mu(I) \leq \prod_{n < N} \frac{M_n}{(C_n M_{n+1})^{\frac{1}{d-1}}} = \left( \prod_{n < N} \frac{M_{n+1}^{1-\frac{1}{d-1} }}{C_n^{\frac{1}{d-1}}} \right) \frac{1}{M_N} = \frac{A_N}{M_N}
%\end{align*}
%
%If $J \subset I$ is any length $1/N_N$ interval considered in the algorithm, then either $\mu(J) = (M_N/N_N) \mu(I)$, as in the first case, or in the second case, $\mu(J) = (M_N/N_N(1 - 1/M_N)) \leq 2M_N/N_N \mu(I)$, so in general $\mu(J) \leq 2A_N/N_N$. This means we can apply the second method in the appendix for bounding Hausdorff dimension, with $l_N = 1/M_N$ and $r_N = 1/N_N$. To obtain 

%Now if $1/N_N \leq |I| \leq 1/M_N$, then $I$ can be covered by $|I|N_N$ intervals of length $1/N_N$, and so
%
%\begin{align*}
%    \mu(I) &\leq 2|I|N_N \frac{A_N}{N_N} = 2A_N|I| = \frac{2A_{N-1} M_N^{1 - \frac{1}{d-1}}}{C_{N-1}^{\frac{1}{d-1}}} |I| \lesssim_\varepsilon |I|M_N^{1 - \frac{1}{d-1} - \varepsilon} \leq |I|^{\frac{1}{d-1} - \varepsilon}
%\end{align*}
%
%Provided that we can choose $M_N$ such that $A_N/C_N \lesssim_\varepsilon M_{N+1}^\varepsilon$ for all $\varepsilon$ (this is why it is incredibly important that the values in the lemma are independent of $N$ in the proof above). On the other hand, if $1/M_{N+1} \leq |I| \leq 1/N_N$, then $I$ can be covered by a single length $1/N_N$ interval, hence
%
%\[ \mu(I) \leq \frac{2A_N}{N_N} = \frac{2A_N}{N_N} = \frac{2A_N}{(C_NM_{N+1})^{\frac{1}{d-1}}} \lesssim_\varepsilon \frac{1}{M_{N+1}^{\frac{1}{d-1} - \varepsilon}} \leq |I|^{\frac{1}{d-1} - \varepsilon} \]
%
%Thus we obtain the theorem if $M_{N+1} = \exp(A_N/C_N)$, for instance.

\section{A Set Avoiding All Functions With A Common Derivative}

In the latter part's of their paper, Pramanik and Fraser apply an iterative technique to construct, for each $\alpha$ with $\sum \alpha_n = 0$ and $K > 0$, a set $E$ of positive Hausdorff dimension avoiding solutions to any function $f: \mathbf{R}^d \to \mathbf{R}$ satisfying wth $(\partial_n f)(0) = \alpha_n$,
%
\[ \left| f(x) - \sum \alpha_n x_n \right| \leq K \sum_{n \neq 1} (x_n - x_1)^2 \]
%
The set of such $f$ is an uncountable family, which makes this situation interesting. The technique to create such a set relies on another iterative procedure.

\begin{lemma}
    Let $I \subsetneq [1,d]$ be a strict subset of indices, and $\delta_0 > 0$. Then there exists $\varepsilon > 0$ such that for any $\lambda > 0$ and two disjoint intervals $J_1$ and $J_2$, with $J_1$ occuring before $J_2$, and if we set
    %
    \[ [a_n,b_n] = \begin{cases} J_1 & n \in I \\ J_2 & n \not \in I \end{cases} \]
    %
    then for $\delta < \delta_0$, either for all $x_n \in [a_n,a_n+\varepsilon \lambda]$ or for all $x_n \in [b_n - \varepsilon \lambda, b_n]$,
    %
    \[ \left| \sum \alpha_n x_n \right| \geq \delta \lambda \]
\end{lemma}
\begin{proof}
    If $C^* = \sum |\alpha_n|$, then for $|x_n - a_n| \leq \varepsilon \lambda$,
    %
    \[ |\sum \alpha_n (x_n - a_n)| \leq C^* \varepsilon \lambda \]
    %
    Thus if $|\sum \alpha_n a_n| > (\delta + \varepsilon C^*)\lambda$, then $|\sum \alpha_n x_n| \geq \delta \lambda$. If this does not occur
\end{proof}









\section{Equidistribution Results}

The classical Weyl equidistribution theorem says that if $\alpha$ is an irrational number, then the decimal parts of the numbers $n \alpha$ are equidistributed in $\mathbf{T} = \mathbf{R}/\mathbf{Z}$, in the sense that for any continuous function $f: \mathbf{T} \to \mathbf{R}$,
%
\[ \frac{1}{N} \sum_{n \leq N} f(n \alpha) \to \int_{\mathbf{T}} f(x)\; dx \]
%
This is equivalent to prove that for any interval $[a,b] \in [0,1)$,
%
\[ \frac{\# \{ 1 \leq n \leq N : x_n \in [a,b] \}}{N} \to b - a \]
%
as $N \to \infty$. By approximating a continuous function $f$ by a Fourier series, to prove this is true for a particular sequence, it suffices to prove it for $f(x) = e(nx) = e^{2 \pi i n x}$, for each nonzero integer $n$. Certain techniques we are developing in the theory of cantor decompositions require a higher dimensional variant of such a result, so this section details some information which might help us in the future. We will encounter sequences that are not equidistributed over the entire space, so if $G$ is any closed subgroup of $\mathbf{T}$, we say a sequence $x_n$ is equidistributed over $G$ if $x_n \in G$ for all $n$, and for any continuous function $f: G \to \mathbf{R}$,
%
\[ \frac{1}{N} \sum_{n \leq N} f(n \alpha) \to \int_G f(x)\; dx \]
%
or alternatively, if for any closed set $K$ in $G$,
%
\[ \frac{\# \{ 1 \leq n \leq N: x_n \in K \}}{N} \to |K| \]
%
where $|K|$ is taken with respect to the Haar probability measure on $G$.

\begin{theorem}
    A sequence $x_1, x_2, \dots$ is equidistributed in $\mathbf{T}^d$ if and only if for every nonzero $\xi \in \mathbf{Z}^n$,
    %
    \[ \frac{1}{N} \sum_{n \leq N} e(\xi \cdot x_n) \to 0 \]
\end{theorem}
\begin{proof}
    We prove that the exponential sum condition implies the general result, noting that the other direction is clear. Clearly the exponential sum condition implies the result for all functions $f$ which are trigonometric polynomials. But then, by the Stone Weirstrass theorem and basic Abelian Harmonic analysis, the multivariate trigonometric polynomials are dense in $C(\mathbf{T}^d)$, and we may apply a standard limiting argument.
\end{proof}

\begin{example}
    If $x_n = n \alpha + \beta$, then
    %
    \begin{align*}
        \frac{1}{N} \sum_{n \leq N} e(\xi \cdot x_n) &= \frac{e(\xi \cdot \beta)}{N} \sum_{n \leq N} e(n \xi \cdot \alpha) = \begin{cases} \frac{1}{N} \frac{e(\xi \cdot \beta)(e((n+1) \xi \cdot \alpha) - 1)}{e(\xi \cdot \alpha) - 1} & : \xi \cdot \alpha \not \in \mathbf{Z} \\ e(\xi \cdot \beta) & : \xi \cdot \alpha \in \mathbf{Z} \end{cases}
    \end{align*}
    %
    Weyl's exponential sum theorem implies that $x_n$ is equidistributed on $\mathbf{T}^n$ precisely when $\xi \cdot \alpha \not \in \mathbf{Z}$ for all $\xi \in \mathbf{Z}^n$. What's more, it is simple to see from this that $x_n$ is still equidistributed for any subsequence whose indices form an arithmetic progression.
\end{example}

Thus in one dimension, an arithmetic sequence is either equidistributed over the entire torus, or over a discrete set of points forming a {\it discrete subgroup} of the torus. We can think of this discrete subgroup as a zero dimension torus, which leads us to suspect that in higher dimensions, an arithmetic sequence is always equidistributed, but not necessarily over the whole torus, but instead over a lower dimensional subtorus.

\begin{theorem}[Ratner]
    If $x_n = n \alpha + \beta$, then we can write $\alpha = \alpha_0 + \alpha_1$, where the sequence $n \alpha_0 + \beta$ is periodic in $\mathbf{T}^d$, and $n \alpha_1 + \beta$ is equidistributed over a subtorus of $\mathbf{T}^d$. In particular, if $\beta = 0$, then Ratner's theorem says that $x_n$ has an evenly spaced subsequence equidistributed over a subtorus of the space containing the origin.
\end{theorem}
\begin{proof}
    We induct on the dimension. For $d = 1$, the theorem is obvious, since either $\alpha$ is rational, and therefore $n \alpha + \beta$ is periodic, or $\alpha$ is irrational, and $n \alpha + \beta$ is equidistributed on $\mathbf{T}$. So now let us consider a sequence on the torus $\mathbf{T}^{d+1}$. If $\alpha$ is irrational, then $x_n$ is equidistributed, and the theorem is obvious. Otherwise, there exists $\xi \in \mathbf{Z}^n$ such that $\xi \cdot \alpha \in \mathbf{Z}$. We may write $\alpha = \alpha_0 + \alpha_1$, where $\alpha_0 \in \mathbf{Q}^d$, and $\xi \cdot \alpha_1 = 0$. The sequence $n\alpha_0 + \beta$ is periodic, whereas $n \alpha_1$ takes values in the subtorus $T$ of points $x$ with $\xi \cdot x = 0$. Since $\xi \neq 0$, $T$ is a $d$ dimensional compact subgroup of $\mathbf{T}^{d+1}$ isomorphic to $\mathbf{T}^d$. To see this, we assume for simplicity that $\xi_d \neq 0$. Then $\xi^\perp$ is generated by the basis of $d$ vectors $v_n = \xi_d e_n - \xi_n e_d$, for $1 \leq n \leq d$. Thus we get a homomorphism between $\mathbf{R}^d$ and $T$ given by the map $x \mapsto \sum x_n v_n$. If $\sum x_n v_n \in \mathbf{Z}$, so that $\sum x_n v_n = 0$ on $T$, then this means that $x_n \in \mathbf{Z}/\xi_d$ for each $n$, implying that the kernel of this homomorphism is discrete. Lattice theory implies that we can write the kernel as $\bigoplus \mathbf{Z} \langle w_n \rangle$, for $d$ generating vectors $w_1, \dots, w_d$. But then the map $f(x) = \sum x_n w_n$ gives an isomorphism between $\mathbf{T}^d$ and $T$. If we set $\beta = f^{-1}(\alpha_0)$, then $f^{-1}(n\alpha_0) = n \beta$, and so by induction, we can write $\beta = \beta_0 + \beta_1$, where $n\beta_0$ is periodic, and $n \beta_1$ is equidistributed over a subtorus of $\mathbf{T}^d$. But then $n f(\beta_0) = f(n \beta_0)$ is periodic on $T$, and $n f(\beta_1) = f(n \beta_1)$ is equidistributed over a subtorus of $T$. Since the sum of two periodic sequences is periodic, $n \alpha_0 + n f(\beta_0)$ is periodic, and $n f(\beta_1)$ is equidistributed over a subtorus. We have $\alpha_0 + f(\beta_0) + f(\beta_1) = \alpha_0 + f(\beta) = \alpha_0 + \alpha_1$, completing the proof.
\end{proof}

\begin{corollary}
    Any linear sequence in $\mathbf{T}^d$ is equidistributed in a finite union of cosets of a subtorus of $\mathbf{T}^d$.
\end{corollary}

In general, ergodic theory results do not give rates on how long it takes for a sequence to equidistribute over a set. This is not a problem in the constructions we perform, since the rates that our intervals shrink can be arbitrarily fast. However, it is important to note that the convergence rates are uniform across all intervals.

\begin{theorem}
    If $x_n$ is equidistributed over a torus $\mathbf{T}^d$, and
    %
    \[ A_N = \sup_I \left| \frac{\# \{ 1 \leq n \leq N : x_n \in I \}}{N} - |I| \right| \]
    %
    where $I$ ranges over all boxes in $\mathbf{T}^d$, then $A_N \to 0$ as $N \to \infty$.
\end{theorem}
\begin{proof}
    For notational simplicity, we let
    %
    \[ \# (I,N) = \# \{ 1 \leq n \leq N: x_n \in I \} \]
    %
    For each $n$, we can partition $\mathbf{T}^d$ into finitely many disjoint cubes $\{ I_n \}$ with sidelengths $1/n$. Since there are only finitely many such cubes, there is $N_n$ such that for $M \geq N_n$, $| \#(I_n,M)/M - |I_n|| \leq 1/n$. Now given any box $J$, we can find sets $J_1$ and $J_2$, each unions of the cubes $I_n$, with $J_1 \subset J \subset J_2$ and $|J - J_1|, |J_2 - J| \lesssim_d 1/n$. Thus
    %
    \[ |J| - \frac{2}{n} \leq |J_1| - \frac{1}{n} \leq \frac{\#(J_1,M)}{M} \leq \frac{\#(J,M)}{M} \leq \frac{\#(J_2,M)}{M} \leq |J_2| + \frac{1}{n} \leq |J| + \frac{2}{n} \]
    %
    which completes the proof, since $N_n$ is independent of $J$.
\end{proof}








\section{Results about Hypergraphs}

\begin{lemma}[Tur\'{a}n]
    For any $k$ uniform hypergraph $H = (V,E)$ with $|E| \leq |V|^\alpha$, $V$ contains an independant set of size $\Omega(|V|^{(k-\alpha)/(k-1)})$.
\end{lemma}
\begin{proof}
    We create an independant set $I$ by the following procedure. First, select a subset $S$ of vertices, including each independantly with probability $p$. Delete a single vertex from each edge in each hypergraph entirely contained in $S$, obtaining an independant set $I$. We find that each edge in $V$ is entirely included in $S$ with probability $p^k$, and $S$ has expected size $p |V|$, so $\mathbf{E}|I| = p |V| - p^k |E|$. If $|E| = |V|^\alpha$ for $\alpha \geq 1$, then setting $p = (1/2) |V|^{(1 - \alpha)/(k-1)}$ induces a set $I$ with size
    %
    \[ |V|^{(k - \alpha)/(k-1)}(1/2 - 1/2^k) \]

    We create an independant set $I$ by the following procedure. First, select a subset $S$ of vertices, including each vertex independantly with probability $p$. Delete a single vertex from each edge in each hypergraph which is entirely contained in $S$. Then $I$ is an independant set with respect to each hypergraph, and we shall show that for an appropriate choice of $p$, $\mathbf{E} |I| \geq h$.

    Trivially, we find $\mathbf{E}|S| = p |V|$. For any $i \geq 2$, the expected number of edges of $H_i$ falling entirely in $S$ is
    %
    \[ p^i |E_i| \leq \frac{p^i |V|^i}{c_k h^{i-1}} \]
    %
    therefore
    %
    \[ \mathbf{E}|I| = p|V| - \sum_{i = 2}^k \frac{p^i |V|^i}{c_k h^{i-1}} \]
    %
    Setting $p = 2h/|V|$ and $c_k = 2^{k+1}$ gives
    %
    \[ \mathbf{E}|I| = h \left( 2 - \sum_{i = 2}^k \frac{1}{2^{k+1-i}} \right) > h \]
    %
    which completes the proof.
\end{proof}

\section{Hyperdyadic Covers}

Recall the definition of the Hausdorff measure $H^\alpha(E) = \lim_{\delta \to 0} H^\alpha_\delta(E)$, where $H^\alpha_\delta(E)$ is the greatest lower bound of $\sum r_n^\alpha$, over all choices of covers of $E$ by cubes $I_1, I_2, \dots$, where $I_n$ has sidelengths $r_n$. We then define the Hausdorff dimension of $E$ to be the least upper bound of the scalars $\alpha$ such that $H^\alpha(E) = 0$, or alternatively, the greatest lower bound of $\alpha$ such that $H^\alpha(E) = \infty$.

To determine the Hausdorff dimension of $E$, it suffices to consider only dyadic cubes in the cover of $E$. Define $H^\alpha(E) = \lim_{\delta \to 0} H^\alpha_{D,\varepsilon}(E)$, where $H^\alpha_{D,\varepsilon}(E)$ is the greatest lower bound of $\sum r_n^\alpha$ over {\it dyadic} covers $I_1, I_2, \dots$, with $I_n \in \mathcal{B}(r_n)$. Then $H^\alpha_D$ is comparable with $H^\alpha$.

\begin{theorem}
    For any set $E$, $H^\alpha(E) \leq H^\alpha_D(E) \leq 2^{d + \alpha} H^\alpha(E)$.
\end{theorem}
\begin{proof}
    Given any not necessarily dyadic cover $I_1, I_2, \dots$, we can replace each sidelength $r_n$ cube $I_n$ with at most $2^d$ dyadic cubes with radius at most $2r_n$, which gives $H^\alpha_{D,\varepsilon}(E) \leq 2^{1 + \alpha} H^\alpha_\varepsilon(E)$, and taking the limit as $\varepsilon \to 0$ then gives the required upper bound for $H^\alpha_D$.
\end{proof}

If we are restricting ourselves to cubes lying at a series of discrete scales, it seems as if the dyadic sequence is about as fast as we can use so that the resultant Hausdorff measure is comparable to the usual Hausdorff measure. Nonetheless, using a weak type bound we can get results for a faster decreasing family of scales. This is necessary for our calculations. We fix a positive $\delta$, and consider a sequence of {\bf hyperdyadic scales} $H_N = 2^{- \lfloor (1 + \delta)^N \rfloor}$. A {\bf hyperdyadic cube} is then a cube in $\mathcal{B}(H_N)$ for some $N$.

%To measure the difference in decay rates between hyperdyadic and dyadic scales, we note that for any $n$, and $0 < A < 1$, the number of dyadic scales between $A$ and $A^n$ is comparable to $n \log(1/A)$, whereas the number of hyperdyadic scales is comparable to $\log(n) / \log(1 + \delta)$, which is completely independant of $A$. As is expected, a naive covering approach as in the last argument doesn't suffice to give results about dimensions and hyperdyadic coverings.

\begin{proof}
    For any sidelength $L$ cube, we can cover the cube by at most $2^d$ hyperdyadic cubes with sidelength at most $2L^{1 - \delta} \geq 2L^{(1+\delta)^{-1}}$. This is because
    %
    \[ 2 H_{N+1}^{(1 + \delta)^{-1}} = 2^{1 - (1 + \delta)^{-1} \lfloor (1 + \delta)^{N+1} \rfloor} \geq 2^{1 - (1 + \delta)^N} \geq 2^{\lfloor (1 + \delta)^N \rfloor} = H_N \]
    %
    If $E$ has Hausdorff dimension $\alpha$, for every $\varepsilon$ and $N$ we can find a collection of dyadic cubes $I_1, I_2, \dots$ covering $E$ with $I_k$ sidelength $L_k \leq H_N$, and $\sum L_{N,i}^{\alpha + \varepsilon} \lesssim_\varepsilon 1$. A weak type bound implies the number of cubes $I_k$ with $H_{N+1} \leq L_k \leq H_N$ is $O_\varepsilon(1/H_{N+1}^{\alpha + \varepsilon})$. But
    %
    \[ 1/H_{N+1}^{\alpha + \varepsilon} \leq (H_N/H_{N+1})^{\alpha + \varepsilon} 1/H_N^{\alpha + \varepsilon} \lesssim 1 / H_N^{\alpha + \varepsilon + \delta} \]
    %
    and so the cover of $E$ by hyperdyadic cubes contains $O_\varepsilon(1/H_N^{\alpha + \varepsilon + \delta})$ length $H_N$ cubes for each $N$.



    If we swap each cube $I_{N,i}$ with $2^d$ hyperdyadic cubes of length at most $2L^{1 - \delta}$, we obtain
    %
    \begin{align*}
        \sum 2^d (2 L_{N,i}^{1 - \delta})^{\alpha + \varepsilon} &= 2^{d + \alpha + \varepsilon} \sum L_{N,i}^{(1 - \delta)(\alpha + \varepsilon)} \lesssim_\varepsilon 1
    \end{align*}
    %
    Thus $H^{(1 - \delta)\alpha + \varepsilon}_{HD}(E) \lesssim_\varepsilon 1$.

    We can swap each cube $I_i$ with $2^d$ hyperdyadic cubes of length at most $2L^{(1 + \delta)^{-1}}$, without effecting the estimate too much.

    Then for every hyperdyadic number $H_N$, we can find a collection of cubes $I_{N,1}, I_{N,2}, \dots$ covering $E$ with $I_{N,i}$ sidelength $r_{N,i} \leq H_N$, and $\sum r_{N,i}^{\alpha + \varepsilon} \lesssim_\varepsilon 1$. Covering each cube by $2^d$ cubes with hyperdyadic sidelengths, which magnifies $r_{N,i}$ by at most
    %
    \[ 2 \cdot 2^{(1 + \delta)^{N+1} - (1 + \delta)^N} = 2 \cdot 2^{\delta (1 + \delta)^N} \lesssim 2 \cdot r_{N,i}^{- \delta} \]
    %
    We conclude that
    %
    \[ 2^{d+\alpha+\varepsilon} 2^{(\alpha + \varepsilon) \delta(1 + \delta)^N} C_\varepsilon \]
\end{proof}

We assume $\delta$ and $\varepsilon$ are some fixed parameters. If $A(\varepsilon, \delta)$ and $B(\varepsilon,\delta)$ are two quantities depending on $\varepsilon$ and $\delta$, we write $A \preccurlyeq B$ mean $A \lesssim_\varepsilon \delta^{-C \varepsilon} B$ for some $C$, and for every $\varepsilon$. We let $A \approx B$ mean $A \preccurlyeq B$ and $B \preccurlyeq A$ hold simultaneously. We say a union of balls is $\delta$ discretized if it is the union of balls with radius $\approx \delta$. Thus there exists $C_\varepsilon$ and $C$ such that for each ball $B_r$ of radius $r$, $|r - \delta| \leq C_\varepsilon \delta^{1-C \varepsilon}$. Thus
%
\[ \delta(1 - C_\varepsilon \delta^{-C \varepsilon}) \leq r \leq \delta(1 + C_\varepsilon \delta^{- C \varepsilon}) \]
%
In particular, the dyadic scales $2^{-\lfloor (1 + \varepsilon)^k \rfloor}$ are allowed in a discretization of a hyperdyadic scale $2^{-(1+\varepsilon)^k}$, since we can choose $C_\varepsilon$ and $C$ such that
%
\[ 1 - C_\varepsilon 2^{C (1 + \varepsilon)^k \varepsilon} \leq 1 \leq 2^{(1 + \varepsilon)^k -\lfloor (1 + \varepsilon)^k \rfloor} \leq 2 \leq 1 + C_\varepsilon 2^{(1 + \varepsilon)^k C \varepsilon} \]

\begin{theorem}
    Let $E$ be a compact subset of $\mathbf{R}^n$. If $0 < \alpha < n$, and $\dim(E) \leq \alpha$, then for each hyperdyadic number $\delta$, we can associate a $\delta$ discretized set $X_\delta$ with $|X_\delta \cap B(x,r)| \preccurlyeq \delta^n (r/\delta)^\alpha$ for all $\delta \leq r \leq 1$ and $x \in \mathbf{R}^n$, and every element of $E$ is contained in infinitely many of the $X_\delta$.
\end{theorem}
\begin{proof}
    Fix $E$. For every hyperdyadic $\delta$, we can find a cover of $E$ by balls $B(x_{\delta n}, r_{\delta n})$ such that $r_{\delta n} < \delta$, and
    %
    \begin{equation} \sum_n r_{\delta n}^{\alpha + C\varepsilon} \lesssim 1 \end{equation}
    %
    Choose $m_{\delta n}$ such that $2^{-(1 + \varepsilon)^{m_{\delta n}+1}} \leq r_{\delta n} \leq 2^{-(1 + \varepsilon)^{m_{\delta n}}}$. We calculate
    %
    \begin{align*}
        \frac{2^{-(\alpha + C'\varepsilon) (1 + \varepsilon)^{m_{\delta n}}}}{r_{\delta n}^{\alpha + C\varepsilon}} &\leq \frac{2^{- (\alpha + C'\varepsilon) (1 + \varepsilon)^{m_{\delta n}}}}{2^{- (\alpha + C\varepsilon) (1 + \varepsilon)^{m_{\delta n} + 1}}}\\
        &= \left( 2^{\varepsilon (1 + \varepsilon)^{m_{\delta n}}} \right)^{\alpha + (C (1 + \varepsilon) - C')}
    \end{align*}
    %
    Provided that $C' > \alpha + C(1 + \varepsilon)$, the quantity on the left is $\leq 1$, which is independant of $\varepsilon$ provided that $\varepsilon$ is bounded from above, and so we conclude
    %
    \[ \sum_n \left( 2^{-(1 + \varepsilon)^{m_{\delta n}}} \right)^{\alpha + C' \varepsilon} \leq \sum_n r_{\delta n}^{\alpha + C\varepsilon} \]
    %
    Thus we may assume by changing the value of $C$ that the quantities $r_{\delta n}$ are hyperdyadic from the outset. This means that at each hyperdyadic scale $\delta$, the number of hyperdyadic balls at the scale $\delta$ in each cover is $\lesssim (1/\delta)^{\alpha + C\varepsilon}$. STOP IS THIS ALL WE NEED, THEN COME BACK TO THE PROOF.


    For a pair of hyperdyadic numbers $\delta$ and $\gamma$ we set
    %
    \[ Y_{\delta \gamma} = \bigcup_{r_{\delta n} = \gamma} B(x_{\delta n}, r_{\delta n}) \]
    %
    Every element of $X$ is in infinitely many of the $Y_{\delta n}$. For each $\delta$ and $\gamma$, we let $Q_{\delta \gamma}$ be the collection of hyperdyadic cubes with sidelength at least $\gamma$ covering $Y_{\delta \gamma}$ and minimizing $\sum_{Q \in Q_{\delta \gamma}} l(Q)^\alpha$. From condition (1.1) we obtain that $Y_{\delta \gamma}$ can be covered by at most $r^{-\alpha - \varepsilon}$ sidelength $r$ cubes, so
    %
    \[ \sum_{Q \in Q_{\delta \gamma}} l(Q)^\alpha \leq Cr^{-\varepsilon} \]
    %
    and so $l(Q) \leq Cr^{-\varepsilon/\alpha}$ for all $Q \in Q_{\delta \gamma}$. From the construction of $Q_{\delta \gamma}$, we see that the $Q$ are all disjoint, and for any hyperdyadic cube $I$,
    %
    \[ \sum_{\substack{Q \in Q_{\delta \gamma}\\Q \subset I}} l(Q)^\alpha \leq l(I)^\alpha \]
    %
    since otherwise we could replace such elements of $Q$ in $Q_{\delta \gamma}$ by $I$ itself.
\end{proof}

\endinput
%% The following is a directive for TeXShop to indicate the main file
%%!TEX root = diss.tex

\chapter{Related Work}
\label{ch:RelatedWork}

Here, we discuss the main papers which influenced our results. In particular, the work of Keleti on translate avoiding sets, Fraser and Pramanik's work on sets avoiding smooth configurations, Math\'{e}'s result on sets avoiding algebraic varieties, and Schmerkin's result on sets with large Fourier dimension avoiding smooth configurations.

\section{Keleti: A Translate Avoiding Set}

Keleti's two page paper constructs a full dimensional set $X \subset [0,1]$ such that for each $t \neq 0$, $X$ intersects $t + X$ in at most one place. The set $X$ is then said to \emph{avoid translates}. This paper contains the core idea behind the \emph{interval dissection} method adapted in Fraser and Pramanik's paper. We also adapt this technique in our paper, which makes the result of interest.

It is often convenient to avoid certain configurations when they are expressed in terms of an equation, which is exploited in Fraser and Pramanik, 

\begin{lemma}
    Let $X$ be a set. Then $X$ avoids translates if and only if there do not exists values $x_1 < x_2 \leq x_3 < x_4$ in $X$ with $x_2 - x_1 = x_4 - x_3$.
\end{lemma}
\begin{proof}

    Suppose $(t + X) \cap X$ contains two points $a < b$. Without loss of generality, we may assume that $t > 0$. If $a \leq b - t$, then the equation
    %
    \[ a - (a - t) = t = b - (b - t) \]
    %
    satisfies the constraints, since $a - t < a \leq b - t < b$ are all elements of $X$. We also have
    %
    \[ (b - t) - (a - t) = b - a, \]
    %
    which satisfies the constraints if $a - t < b - t \leq a < b$. This covers all possible cases. Conversely, if there are $x_1 < x_2 \leq x_3 < x_4$ in $X$ with
    %
    \[ x_2 - x_1 = t = x_4 - x_3, \]
    %
    then $X + t$ contains $x_2 = x_1 + (x_2 - x_1)$ and $x_4 = x_3 + (x_4 - x_3)$.
\end{proof}

%\footnote{We always assume $L_n/L_{n+1}$ is an integer so that intervals in $\mathcal{B}(L_n)$ are either almost disjoint from intervals in $\mathcal{B}(L_{n+1})$ or contained completely within such an interval}

The basic, but fundamental idea to Keleti's technique is to introduce memory into Cantor set constructions. Keleti constructs a nested family of discrete sets $X_0 \supset X_1 \supset \dots$ converging to $X$, with each $X_k$ a union of disjoint intervals in $\DQ_k^d$, where the sequence $\{ N_k \}$ will be chosen later, but a multiple of 10. We initialize $X_0 = [0,1]$, and $l_0 = 1$. Furthermore, we consider a queue of intervals, initially just containining $[0,1]$. To construct the sequence $\{ X_k \}$, Keleti iteratively performs the following procedure:
%
\begin{algorithm}[H]
    \begin{algorithmic}%[1]
        \caption{Construction of the Sets $\{ X_k \}$:}
        \State{Set $k = 0$.}
        \MRepeat
            \State{Take off an interval $I$ from the front of the queue.}

            \MForAll{\ $J \in \DQ_k(X_k)$:}
                \State{Order the intervals in $\DQ_{k+1}(J)$ as $J_0, J_1, \dots, J_N$.}

                \State{{\bf If} $J \subset I$, add all intervals $J_i$ to $X_{k+1}$ with $i \equiv 0$ modulo 10.}
                \State{{\bf Else} add all $J_i$ with $i \equiv 5$ modulo 10.}
            \EndForAll
            \State{Add all intervals in $\DQ_{k+1}^d$ to the end of the queue.}
            \State{Increase $k$ by 1.}
        \EndRepeat   
    \end{algorithmic}
\end{algorithm}

Each iteration of the algorithm produces a new set $X_k$, and so leaving the algorithm to repeat infinitely produces a sequence $\{ X_k \}$ whose intersection is $X$.

\begin{lemma}
    The set $X$ is translate avoiding.
\end{lemma}
\begin{proof}
    If $X$ is not translate avoiding, there is $x_1 < x_2 \leq x_3 < x_4$ with $x_2 - x_1 = x_4 - x_3$. Since $l_k \to 0$, there is a suitably large integer $N$ such that $x_1$ is contained in an interval $I \in \DQ_N$ not containing $x_2,x_3$, or $x_4$. At stage $N$ of the algorithm, the interval $I$ is added to the end of the queue, and at a much later stage $M$, the interval $I$ is retrieved. Find the startpoints $x_1^\circ, x_2^\circ$, $x_3^\circ, x_4^\circ \in l_M \mathbf{Z}$ to the intervals in $\DQ_M$ containing $x_1$, $x_2$, $x_3$, and $x_4$. Then we can find $n$ and $m$ such that $x_4^\circ - x_3^\circ = (10n)l_M$, and $x_2^\circ - x_1^\circ = (10m + 5)l_M$. In particular, this means that $|(x_4^\circ - x_3^\circ) - (x_2^\circ - x_1^\circ)| \geq 5L_M$. But
    %
    \begin{align*}
        |(x_4^\circ - x_3^\circ) - (x_2^\circ - x_1^\circ)| &= |[(x_4^\circ - x_3^\circ) - (x_2^\circ - x_1^\circ)] - [(x_4 - x_3) - (x_2 - x_1)]|\\
        &\leq |x_1^\circ - x_1| + \dots + |x_4^\circ - x_4| \leq 4 L_M
    \end{align*}
    %
    which gives a contradiction.
\end{proof}

It is easy to see from the definition of the algorithm that
%
\[ \# (\DQ_k(X_k)) = (l_{k-1}/10l_k) \#(\DQ_{k-1}(X_{k-1})). \]
%
Thus closing the recursive definition shows
%
\[ \#(\DQ_k(X_k)) = \frac{1}{10^k l_k}. \]
%
In particular, this means $|X_k| = 1/10^k$, so $X$ has measure zero irrespective of our parameters. Nonetheless, the canonical measure $\mu$ on $X$ defined with respect to the decomposition $\{ X_k \}$ satisfies $\mu(I) = 10^k l_k$ for all $I \in \B(l_k,X)$. If $10^k l_k^\varepsilon \lesssim_\varepsilon 1$ for all $\varepsilon$, then we can establish the bounds $\mu(I) \lesssim_\varepsilon l_k^{1-\varepsilon}$ for all $\varepsilon$. In particular, this is true if $N_k = 10^{\lfloor \log(k + 2) \rfloor}$. And because this sequence does not rapidly decrease too fast, we can apply Lemma \ref{easyCoverTheorem} to show $\mu$ is a Frostman measure of dimension $1-\varepsilon$ for each $\varepsilon > 0$, so $X$ has full Hausdorff dimension.



\section{Generalizing Keleti's Argument}

Before we move onto other methods which developed Keleti's argument work, it is useful to dwell on what general properties this argument has:
%
\begin{itemize}
    \item \emph{Simplification to a Discrete Problem}: A major part of Keleti's argument is solving a discrete version of the configuration argument. We could summarize the result of Keleti's discrete argument in a lemma.
    %
    \begin{lemma} \label{KeletiDiscreteLemma}
        Let $T_1, T_2$ be disjoint, $\DQ_k$ discretized sets. Then we can find $S_1 \subset T_1$ and $S_2 \subset T_2$ such that
        %
        \begin{enumerate}
            \item[(i)] For each $k$, $S_k$ is a $\DQ_{k+1}$ discretized subset of $T_k$.
            \item[(ii)] If $x_1 \in S_1$ and $x_2,x_3,x_4 \in S_2$, then $x_2 - x_1 \neq x_4 - x_3$.
            \item[(iii)] For each cube $Q \in \DQ_k$ with $Q \subset T_k$,
            %
            \[ \#(\DQ_{k+1}(Q \cap S_k)) \geq (1/10) \cdot \#(\DQ_{k+1}(Q)). \]
        \end{enumerate}
    \end{lemma}

    \item \emph{Iterative Application of Discrete Solution}: Keleti then repeatedly applies his argument iteratively. In particular, Property (i) of Lemma \ref{KeletiDiscreteLemma} allows him to apply his argument iteratively. The reason why Keleti obtains a configuration avoiding set in the limit is because of Property (ii). Most importantly the reason why Keleti obtains a set with full Hausdorff dimension, if the sequence $\{ N_k \}$ decreases rapidly enough, is because of Property (iii).
\end{itemize}

Of course, it is not possible to extend the discrete solution of the configuration argument to general configurations; this part of Keleti's method strongly depends on the arithmetic structure of the configuration. The iterative application of a discrete solution, however, can be applied in generality.







\section{Fraser/Pramanik: Smooth Configurations}

Inspired by Keleti's result, Pramanik and Fraser obtained a generalization of the queue method which allows one to find sets avoiding $n+1$ point configurations given by the zero sets of smooth functions, i.e.
%
\[ \C = \{ (x_0, \dots, x_n) \in \RR^{dn} : f(x_0, \dots, x_n) = 0 \}, \]
%
under mild regularity conditions on the function $f: [0,1]^{dn} \to [0,1]^m$.

\begin{theorem}[Pramanik and Fraser]
    Consider a countable family of functions $\{ f_k : [0,1]^{dn} \to [0,1]^m \}$ each of which being $C^2$, and such that $Df_k$ has full rank at any $(x_1, \dots, x_n)$ with $f(x_1, \dots, x_n) = 0$, where the $(x_1, \dots, x_n)$ are distinct. Then there exists a set $X \subset \RR^d$ with Hausdorff dimension $m/(n-1)$ such that $X$ avoids the configuration
    %
    \[ \C = \bigcup_k \{ (x_1, \dots, x_n) \in \C^n(\RR^d): f(x_1, \dots, x_n) = 0 \}. \]
\end{theorem}

\begin{remark}
    For simplicity, we only prove the result for a single function, rather than a countable family.
\end{remark}

Pramanik and Fraser also simplify to a discrete version of their problem, which they then iteratively apply. In the discrete setting, rather than making a linear shift in one of the intervals we avoid as in Keleti's approach, one must use the smoothness properties of the function to find large segments of an interval avoiding. Corollary \ref{PramanikFraserBuildingBlockLemma} gives the discrete solution that Pramanik and Fraser utilizes.

\begin{lemma} \label{Lemma315091513}
    Fix $n > 1$. Let $T \subset [0,1]^d$ be $\DQ_k$ discretized, and $T' \subset [0,1]^{(n-1)d}$ be $\DQ_k$ discretized. Let $B \subset T \times T'$ be $\DQ_{k+1}$ discretized. Then there exists a $\DQ_{k+1}$ discretized set $S \subset T$, and a $\DQ_{k+1}$ discretized set $B' \subset T'$, such that
    %
    \begin{enumerate}
        \item \label{dimensionReductionProperty} $(S \times T') \cap B \subset S \times B'$.

        \item \label{bigProperty} For every $Q \in \DQ_k^d$, there exists $\mathcal{R}(Q) \subset \DR_{k+1}^d(Q)$, such that
        %
        \[ \#(\mathcal{R}(Q)) \geq (1/2) \cdot \#(\DR_{k+1}^d(Q)). \]
        %
        and for each $R \in \DR_{k+1}^d(Q)$,
        %
        \[ \#(\DQ_{k+1}(R)) = \begin{cases} 1 &: R \in \mathcal{R}(Q,) \\ 0 &: R \not \in \mathcal{R}(Q). \end{cases} \]

        \item \label{BBoundProperty} $\#(\DQ_{k+1}(B')) \leq 2 (N_1 \dots N_k)^d \left( M_{k+1}/N_{k+1} \right)^d \cdot \#(\DQ_{k+1}(B))$.
    \end{enumerate}
\end{lemma}
\begin{proof}
    Fix $Q_0 \in \DQ_k(T)$. For each $R \in \DR_{k+1}(Q_0)$, define a \emph{slab} $S[R] = R \times T'$, and for each $Q \in \DQ_{k+1}(Q_0)$, define a \emph{wafer} $W[Q] = Q \times T'$. We say a wafer $W[Q]$ is \emph{good} if
    %
    \begin{equation} \label{equation10291095429062}
        \#(\DQ_{k+1}(W[Q] \cap B)) \leq (2/N_{k+1}^d) \cdot \#(\DQ_{k+1}(B)),
    \end{equation}
    %
    Then at most $N_{k+1}^d/2$ wafers are bad. We call a slab \emph{good} if it contains a wafer which is good. Since a slab is the union of $(N_{k+1}/M_{k+1})^d$ wafers, at most $M_{k+1}^d /2 = (1/2) \cdot \#(\DR_{k+1}(Q_0))$ slabs are bad. Thus if we set
    %
    \[ \mathcal{R}(Q_0) = \{ R \in \DR_{k+1}(Q_0) : S[R]\ \text{is good} \}, \]
    %
    then
    %
    \begin{equation} \label{equation24016590369046}
        \#(\mathcal{R}(Q_0)) \geq (1/2) \cdot \#(\DR_{k+1}(Q_0)).
    \end{equation}
    %
    For each $R \in \mathcal{R}(Q_0)$, we pick $Q_R \in \DQ_{k+1}(R)$ such that $W[Q_R]$ is good, and define
    %
    \[ S = \bigcup \{ Q_R : R \in \mathcal{R}(Q_0) \}. \]
    %
    Equation \eqref{equation24016590369046} implies $S$ satisfies Property \ref{bigProperty}.

    Let $B'$ be the union of all cubes $Q' \in \DQ_{k+1}(T')$ such that there is $Q \in \DQ_{k+1}(S)$ with $Q \times Q' \in \DQ_{k+1}(B)$. By definition, Property \ref{dimensionReductionProperty} is then satisfied. For each $Q \in \DQ_{k+1}(S)$, $W[Q]$ is good, so \eqref{equation10291095429062} implies
    %
    \[ \# \{ Q' : Q \times Q' \in \DQ_{k+1}(B) \} \leq (2/N_{k+1}^d) \cdot \#(\DQ_{k+1}(B)). \]
    %
    But $\#(\DQ_{k+1}(S)) \leq \#(\DR_{k+1}(T)) \leq (1/r_{k+1})^d = (N_1 \dots N_k)^d M_{k+1}^d$, so
    %
    \begin{align*}
        \#(\DQ_{k+1}(B')) &\leq \#(\DQ_{k+1}(S))[(2/N_{k+1}^d) \cdot \#(\DQ_{k+1}(B))]\\
        &\leq 2(N_1 \dots N_k)^d (M_{k+1}/N_{k+1})^d \#(\DQ_{k+1}(B)),
    \end{align*}
    %
    which establishes Property \ref{BBoundProperty}.
\end{proof}

We apply the lemma recursively $n-1$ times to continually reduce the dimensionality of the avoidance problem we are considering. Eventually, we obtain the case where $n = 0$, and we are in need of a final technique.

\begin{lemma} \label{Lemma1209410535}
    Fix $n > 1$. Let $T \subset [0,1]^d$ be $\DQ_k$ discretized, and let $B \subset T$ be $\DQ_{k+1}$ discretized. Suppose
    %
    \[ \# \DQ_{k+1}(B) \leq \left[ C \cdot 2^{n-1} (N_1 \dots N_k)^{d(n-1)} (M_{k+1}/N_{k+1})^{d(n-1)} \right] (1/l_{k+1})^{dn - m}. \]
    %
    and
    %
    \begin{equation} \label{equation903103513095}
        N_{k+1} > \left[ C \cdot (1 + 2^d) 2^n (N_1 \dots N_k)^{2dn} \right]^{1/m} M_{k+1}^{d(n-1)/m}
    \end{equation}
    %
    Then there exists a $\DQ_{k+1}$ discretized set $S \subset T$ such that
    %
    \begin{enumerate}
        \item $S \cap B = \emptyset$.
        \item \label{badsetproperty5} For each $Q_0 \in \DQ_k(T)$, there is $\mathcal{R}(Q_0) \subset \DR_{k+1}(Q_0)$ with
        %
        \[ \#(\mathcal{R}(Q_0)) \geq (1/2) \#(\DR_{k+1}(Q_0)), \]
        %
        such that for each $R \in \DR_{k+1}(Q_0)$,
        %
        \[ \#(\mathcal{Q}_{k+1}(R \cap S)) = \begin{cases} 1 &: R \in \mathcal{R}(Q_0), \\ 0 &: R \not \in \mathcal{R}(Q_0). \end{cases} \]
    \end{enumerate}
\end{lemma}
\begin{proof}
    For each $Q_0 \in \DQ_k(T)$, we set
    %
    \[ \mathcal{R}(Q_0) = \{ R \in \DR_{k+1}(Q_0) : \#(\DQ_{k+1}(R \cap B)) \leq (2/M_{k+1}^d) \cdot \#(\DQ_{k+1}(B)) \}. \]
    %
    Since $\DR_{k+1}(Q_0) = M_{k+1}^d$,
    %
    \[ \#(\mathcal{R}(Q_0)) \geq \#(\DR_{k+1}(Q_0)) - (M_{k+1}^d/2) \geq (1/2) \cdot \#(\DR_{k+1}(T)). \]
    %
    Now \eqref{equation903103513095} implies that for each $R \in \mathcal{R}(Q_0)$,
    %
    \begin{align*}
        \#(\DQ_{k+1}(R \cap B)) &\leq (2/M_{k+1}^d) \cdot \#(\DQ_{k+1}(B))\\
        &\leq (2/M_{k+1}^d) \left(C \cdot 2^{n-1} (N_1 \dots N_k)^{2dn} (M_{k+1}/N_{k+1})^{d(n-1)} \right).\\
        &= \left[ 2^n C (N_1 \dots N_k)^{2dn} \right] \left( M_{k+1}^{d(n-2)} / N_{k+1}^{m-d} \right)\\
        &< \left( \frac{1}{1 + 2^d} \right) (N_{k+1}/M_{k+1})^d\\
        &= \left( \frac{1}{1 + 2^d} \right) \DQ_{k+1}(R)
    \end{align*}
    %
    Thus for each $R \in \mathcal{R}(Q_0)$, we can find $Q_R \in \DQ_{k+1}(R)$ such that $Q_R \cap B = \emptyset$. And so if we set
    %
    \[ S = \bigcup \{ Q_R : R \in \mathcal{R}(Q_0), Q_0 \in \DQ_k(T) \}, \]
    %
    Then (A) and (B) are satisfied.
\end{proof}

\begin{corollary} \label{PramanikFraserBuildingBlockLemma}
    Let $f: [0,1]^{dn} \to [0,1]^m$ be $C^2$, and have full rank at every point $(x_1, \dots, x_n)$ with all $x_1, \dots, x_n$ distinct, and $f(x_1, \dots, x_n) = 0$. Then there exists a universal constant $C$ depending only on $f$ such that, if \eqref{equation903103513095} is satisfied, then for any disjoint, $\DQ_k$ discretized sets $T_1, \dots, T_n \subset [0,1]^d$, we can find $\DQ_{k+1}$ discretized sets $S_1 \subset T_1, \dots, S_n \subset T_n$ such that
    %
    \begin{enumerate}
        \item If $x_1 \in S_1, \dots, x_n \in S_n$, then $f(x_1, \dots, x_n) \neq 0$.
        \item For each $k$, and for each $Q_0 \in \DQ_k(T_k)$, there is $\mathcal{R}(Q_0) \subset \mathcal{R}_{k+1}(Q_0)$ with
        %
        \[ \#(\mathcal{R}(Q_0)) \geq (1/2) \cdot \#(\mathcal{R}_{k+1}(Q_0)), \]
        %
        and for each $R \in \DR_{k+1}(Q_0)$,
        %
        \[ \#(\mathcal{Q}_{k+1}(R \cap S)) = \begin{cases} 1 &: R \in \mathcal{R}(Q_0), \\ 0 &: R \not \in \mathcal{R}(Q_0). \end{cases} \]
    \end{enumerate}
\end{corollary}
\begin{proof}
    Since $f$ is $C^2$ and has full rank on the set
    %
    \[ V(f) = \{ (x_1, \dots, x_n) \in \C^n(\RR^d) : f(x_1, \dots, x_n) = 0 \}, \]
    %
    the implicit function theorem implies $V(f)$ is a smooth manifold of dimension $nd - m$ in $\RR^{dn}$, and the coarea formula implies the existence of a constant $C$ such that for each $k$,
    %
    \[ \# \{ Q \in \DQ_k^{dn} : Q \cap V(f) \neq \emptyset \} \leq C/l_k^{dn-m}. \]
    %
    To apply Lemma \ref{Lemma315091513} and \ref{Lemma1209410535}, we set
    %
    \[ B = \# \{ Q \in \DQ_k^{dn} : Q \cap V(f) \neq \emptyset \}. \]
    %
    Applying Lemma \ref{Lemma315091513} iteratively $n-1$ times then finishing with Lemma $\ref{Lemma1209410535}$, constructs the sets $S_1, \dots, S_n$.
\end{proof}

Just like in Keleti's proof, Pramanik and Fraser's technique applies a discrete result, Corollary \ref{PramanikFraserBuildingBlockLemma}, iteratively at many scales to obtain a high dimensional set avoiding the zeroes of a function. We construct a nested family $\{ X_k : k \geq 0 \}$ of $\DQ_k$ discretized sets, converging to a set $X$, which we will show is translate avoiding. We intiailize $X_0 = [0,1]$. Our queue shall consist of $n$ tuples of disjoint intervals $(T_1, \dots, T_n)$, all of the same length, which initially consists of all possible tuples of intervals in $\DQ_1^d([0,1]^d)$. To construct the sequence $\{ X_k \}$, we perform the following iterative procedure:
%
\begin{algorithm}[H]
    \begin{algorithmic}
        \caption{Construction of the Sets $\{ X_k \}$}
        \State{Set $k = 0$}
        \MRepeat
            \State{Take off an $n$ tuple $(T_1', \dots, T_n')$ from the front of the queue}
            \State{Set $T_i = T_i' \cap X_k$ for each $i$}
            \State{Apply Corollary \ref{PramanikFraserBuildingBlockLemma} to the sets $T_0, \dots, T_d$, obtaining $\DQ_{k+1}$ discretized sets $S_1, \dots, S_n$ satisfying Properties (A), (B), and (C) of that Lemma.}
            \State{Set $X_{k+1} = X_k - \bigcup_{i = 1}^n T_i - S_i$.}
            \State{Add all $n$ tuples of disjoint cubes $(T_1', \dots, T_n')$ in $\DQ_{k+1}^d(X_{k+1})$ to the back of the queue.}
            \State{Increase $k$ by 1.}
        \EndRepeat   
    \end{algorithmic}
\end{algorithm}

\begin{lemma}
    The set $X$ constructed by the procedure avoids the configuration
    %
    \[ \C = \{ (x_1, \dots, x_n) \in \C^n(\RR^d) : f(x_1, \dots, x_n) = 0 \}. \]
\end{lemma}
\begin{proof}
Suppose $x_1, \dots, x_n \in X$ are distinct. Then at some stage $k$, $x_1, \dots, x_n$ lie in disjoint cubes $T_1', \dots, T_n' \in \DQ_k^d(X_k)$, for some large $k$. At this stage, $(T_1', \dots, T_n')$ is added to the back of the queue, and therefore, at some much later stage $N$, the tuple $(T_1', \dots, T_n')$ is taken off the front. Sets $S_1 \subset T_1', \dots, S_n \subset T_n'$ are constructed satisfying Property (A) of Corollary \ref{PramanikFraserBuildingBlockLemma}. Since $x_1, \dots, x_n \in X$, we must have $x_i \in S_i$ for each $i$, so $f(x_1, \dots, x_n) \neq 0$.
\end{proof}

What remains is to bound the Hausdorff dimension of $X$.

\begin{theorem}
    If $M_k = 2^{2^{C_0 k }}$, for a constant $C_0$ suitably large, depending only on $f$, the set $X$ has Hausdorff dimension exceeding $m/(n-1)$.
\end{theorem}
\begin{proof}
    First, we construct the canonical measure $\mu$ on $X$. Property (B) of Corollary \ref{PramanikFraserBuildingBlockLemma} implies that for each $Q \in \DQ_{k+1}^d$, $\mu(Q) \leq 2/M_{k+1}^d \mu(Q^*)$, which implies that for $Q \in \DQ_k^d$, $\mu(Q) \leq 2^k / (M_k \dots M_1)^d$. By Lemma \ref{uniformMassFrostman}, it suffices to show that for each $\varepsilon > 0$,
    %
    \begin{equation}\label{inequality7} \frac{2^k}{(M_{k+1} \dots M_1)^d} \lesssim_\varepsilon \frac{1}{(N_1 \dots N_{k+1})^{m/(n-1)(1 - \varepsilon)}} \end{equation}
    %
    Set
    %
    \[ N_{k+1} = A (N_1 \dots N_k)^{2dn/m} M_{k+1}^{d(n-1)/m}, \]
    %
    where $A$ is an arbitrary constant not depending on $k$ so that $\eqref{equation903103513095}$ holds. Then inequality \eqref{inequality7} is then implied if
    %
    \[ M_{k+1} \gtrsim_\varepsilon \left( (N_1 \dots N_k)^{6dn} [2^k A^d] \right)^{1/\varepsilon}. \]
    %
    This equation is satisfied, if $M_k = 2^{2^{C_0k}}$, where $C_0$ is a suitably large.
\end{proof}

\section{Math\'{e}: Polynomial Configurations}

Math\'{e}'s result constructs sets avoiding configurations specifiable as algebraic varieties.

\begin{theorem}[Math\'{e}]
    Let $\{ f_k: \RR^{nd} \to \mathbf{R} \}$ be a countable family of rational coefficient polynomials with degree at most $m$. Then there exists a set $X \subset [0,1]^d$ with Hausdorff dimension $d/m$ which avoids the configurations
    %
    \[ \C = \bigcup_k \{ (x_1, \dots, x_n) \in \RR^{nd} : f(x_1, \dots, x_n) = 0 \}. \]
\end{theorem}

Originally, Math\'{e}'s result does not explicitly use a discretization method analogous to Keleti and Pramanik and Fraser, but his proof strategy can be reconfigured to work in this setting. For the purpose of brevity, we do not carry out the complete argument, merely giving the discretization method below.

\begin{theorem}
    Let $f$ be a polynomial of degree $m$, and consider unions of length $1/M$ intervals $T_0, \dots, T_d \subset [0,1]$, with rational start-points. If $\partial_0 f$ is non-vanishing on $T_0 \times \dots \times T_d$, then there exists arbitrarily large integers $N$ and a constant $C$ not depending on $N$ and sets $S_n \subset T_n$ such that
    %
    \begin{itemize}
        \item $f(x) \neq 0$ for $x \in S_0 \times \dots \times S_d$.
        \item If $T_0, \dots, T_d$ are split into length $1/N$ intervals, then $S_n$ contains a length $C/N^d$ region of each interval.
    \end{itemize}
\end{theorem}
\begin{proof}
    Without loss of generality (by subdividing the initial intervals), let $M$ be the greatest common divisor of all of the startpoints of the intervals in $T_n$. Divide each interval $T_n$ into length $1/N$ intervals, and let $\mathbf{A} \subset (\mathbf{Z}/N)^d$ be the cartesian product of all startpoints of these length $1/N$ intervals. Since $f$ has degree $m$, $f(\mathbf{A}) \subset \mathbf{Z}/N^m$. If $A_0 \leq |\partial_0 f| \leq A_1$ on $T_0 \times \dots \times T_d$, then for any $a \in \mathbf{A}$, and $\delta_0$, there exists $\delta_1$ between $0$ and $\delta_0$ for which
    %
    \[ |f(a + \delta_0 e_0)| - f(a)| = \delta_0 |(\partial_0 f)(a + \delta_1)| \]
    %
    If $K$ is fixed such that $A_1 \leq (K-1)A_0$, so that we can choose
    %
    \[ \frac{1/K}{A_0N^m} \leq \delta_0 \leq \frac{\left( 1 - 1/K \right)}{A_1N^m} \]
    %
    Then
    %
    \[ \frac{1/K}{N^m} \leq |f(a + \delta_0 e_0) - f(a)| \leq \frac{1 - 1/K}{N^m} \]
    %
    Thus $d(f(\mathbf{A} + \delta_0 e_0), \mathbf{Z}/N^m) \geq 1/KN^m$. Thus if we thicken the coordinates of $\mathbf{A} + \delta_0$ to intervals of length $O(1/N^m)$, then we obtain sets $S_0, \dots, S_n$ avoiding solutions.
\end{proof}

\section{Schmerkin: Salem Sets Avoiding Arithmetic Progressions}

Schmerkin constructs sets with full Fourier dimension

\endinput
%% The following is a directive for TeXShop to indicate the main file
%%!TEX root = diss.tex

\chapter{Avoiding Rough Sets}
\label{ch:RoughSets}

In the last chapter, we saw that many authors have considered the pattern avoidance problem for configurations $\C$ which take the form of many general classes of smooth shapes; in Math\'{e}'s work, $\C$ can take the form of an algebraic variety of low degree, and in Pramanik and Fraser's work, $\C$ can take the form of a smooth manifold. In this chapter, we consider the pattern avoidance problem for an even more general class of `rough' patterns, that are the countable union of sets with controlled lower Minkowski dimension.
%
\begin{theorem}\label{mainTheorem}
	Let $\alpha \geq d$, and suppose $\C \subset \C^n(\RR^d)$ is the countable union of precompact sets, each with lower Minkowski dimension at most $\alpha$. Then there exists a set $X \subset [0,1]^d$ with Hausdorff dimension at least $(nd - \alpha)/(n-1)$ avoiding $\C$.
\end{theorem}

% CHANGE: Made Remarks section an AMSTHM class so that things are properly spaced / the code is neater. Also split up remarks that addressed two different points into two separate remarks. Also moved the proof discussion from the remarks since it is a bit too long for a remark.
\begin{remarks}
	\
	\begin{enumerate}
		\item[(1)] When $\alpha < d$, the pattern avoidance problem is trivial, since $X = [0,1)^d - \pi(Z)$ is full dimensional and solves the pattern avoidance problem, where $\pi(x_1, \dots, x_n) = x_1$ is a projection map from $\RR^{dn}$ to $\RR^d$. We will therefore assume that $\alpha \geq d$ in our proof of the theorem. Note that obtaining a full dimensional set in the case $\alpha = d$, however, is still interesting.

		\item[(2)] Theorem \ref{mainTheorem} is trivial when $\alpha = dn$, since we can set $X = \emptyset$. We will therefore assume that $\alpha < dn$ in our proof of the theorem.

		\item[(3)] When $Z$ is a countable union of smooth manifolds in $\RR^{nd}$ of co-dimension $m$, we have $\alpha = nd - m$. In this case Theorem \ref{mainTheorem} yields a set in $\RR^d$ with Hausdorff dimension at least $(nd - \alpha)/(n-1) = m/(n-1)$. This recovers Theorem 1.1 and 1.2 from \cite{MalabikaRob}, making Theorem \ref{mainTheorem} a generalization of these results.

		\item[(4)] Since Theorem \ref{mainTheorem} does not require any regularity assumptions on the set $Z$, it can be applied in contexts that cannot be addressed using previous methods. Two such applications, new to the best of our knowledge, have been recorded in Section \ref{applications}; see Theorems \ref{sumset-application} and \ref{C1IsoscelesThm} there.
	\end{enumerate}
\end{remarks}

The set $X$ in Theorem \ref{mainTheorem} is obtained as a limit of discretized sets $\{ X_k \}$, each of which avoids a certain discretization of the configuration $\C$ at the scale $l_k$. While this proof strategy is not new, our method for constructing the sets $\{X_k\}$ has several innovations that simplify the analysis of the resulting set $X=\bigcap X_k$. In particular, through a probabilistic selection process we are able to avoid the complicated queuing techniques used in \cite{KeletiDimOneSet} and \cite{MalabikaRob}, that required storage of data from each step of the iterated construction to be retrieved at a much later stage of the construction process.

At the same time, our construction shares certain features with \cite{MalabikaRob}, in particular, the strategy of iterative discretization. The details of a single step of this construction are described in Section \ref{discretesection}. In Section \ref{discretizationsection}, we explain how the length scales $l_k$ and $r_k$ for $X$ are chosen, and prove its avoidance property. In Section \ref{dimensionsection} we analyze the size of $X$ and show that it satisfies the conclusions of Theorem \ref{mainTheorem}.









\section{Avoidance at Discrete Scales}\label{discretesection}

In this section we describe a method for avoiding $Z$ at a single scale. We apply this technique in Section \ref{discretizationsection} at many scales to construct a set $X$ avoiding $Z$ at all scales. This single scale avoidance technique is the core building block of our construction, and the efficiency with which we can avoid $Z$ at a single scale has direct consequences on the Hausdorff dimension of the set $X$ constructed obtained in Theorem \ref{mainTheorem}.

% Change: Using B_s^{dn}(Z) is not correct here, i.e if Z is a dense set. Also reference definition when using strongly non diagonal cubes for the first time.
% Original: At a single scale, we solve a discretized version of the problem, where all sets are unions of cubes at two dyadic lengths $l \geq s$ (later, we will choose $l=l_n$ and $s=l_{n+1}$). Given a set $E \subseteq [0,1)^d$ that is a union of cubes in $\B_l^d$, our goal is to construct a set $F\subset E$ that is a union of cubes in $\B_s^d$ such that $F^n$ is disjoint from the strongly non-diagonal cubes of $\B_{s}^{dn}(Z)$.
At a single scale, we solve a discretized version of the problem, where all sets are unions of cubes at two dyadic lengths $l > s$. In this discrete setting, $Z$ is replaced by a discretized version of itself, a union of cubes in $\B^{dn}_s$ denoted by $G$. Given a set $E$, which is a union of cubes in $\B_l^d$, our goal is to construct a set $F \subset E$ that is a union of cubes in $\B_s^d$, such that $F^n$ is disjoint from strongly non-diagonal cubes (see Definition \ref{defStronglyNonDiagonal}) in $\B^{dn}_s(G)$. Using the setup introduced at the end of the introduction, we will later choose $l = l_k$, $s = l_{k+1}$, and $E = X_k$. The set $X_{k+1}$ will be defined as the set $F$ constructed.
%(see Definition \ref{defStronglyNonDiagonal}).
% DISCUSS: Don't we want to reference definitions when we first use them?

In order to ensure the final set $X$ obtained in Theorem \ref{mainTheorem} has large Hausdorff dimension regardless of the rapid decay of scales used in the construction of $X$, it is crucial that $F$ is uniformly distributed at intermediate scales between $l$ and $s$.
%
% CHANGE: This is a bad choice of language to use, because it is really a combination of non-concentration and large size which gives the uniform distribution (taking F to be empty would satisfy non concentration).
% ORIGINAL: This is the `non-concentration' property discussed below. The next lemma constructs a set $F$ with these properties. 
We achieve this by decomposing $E$ into sub-cubes in $\B^d_r$ for some intermediate scale $r \in [s,l]$, and distributing $F$ as evenly among these intermediate sub-cubes as possible. This is possible assuming a mild regularity condition on the number of cubes in $G$, i.e. Equation \eqref{ZsLarge}.
%

\begin{lemma} \label{discretelemma}
	Fix two distinct dyadic lengths $l$ and $s$, with $l > s$. Let $E \subseteq [0,1)^d$ be a nonempty union of cubes in $\B^d_l$, and let $G\subset\RR^{dn}$ be a nonempty union of cubes in $\B_s^{dn}$ such that
	%
	\begin{equation}\label{ZsLarge}
		(l/s)^d \leq \# \B^{dn}_s(G)  \leq \frac{1}{2}(l/s)^{dn}.
	\end{equation} 
	%
	Then there exists a dyadic length $r \in [s,l]$ of size
	%
	\begin{equation} \label{rBound}
	 	r \sim \left( l^{-d}s^{dn} \# \B^{dn}_s(G) \right)^{\frac{1}{d(n-1)}},
	\end{equation}
	%
	and a set $F \subset E$ that is a nonempty union of cubes in $\B^d_s(E)$ satisfying the following three properties:
	%
	\begin{enumerate}
		\item\label{avoidanceItem} \emph{Avoidance}: For any choice of distinct cubes $J_1, \dots, J_n \in \B^d_s(F)$, $J_1 \times \dots \times J_n \not \in \B_s^{dn}(G)$.

		\item\label{nonConcentrationItem} \emph{Non-Concentration}: For every $I' \in \B_r^d(E)$, there is at most one $J \in \B_s^d(F)$ with $J \subset I'$.

		\item\label{largeSizeItem} \emph{Large Size}: For every $I \in \B^d_l(E)$, $\# \B^d_s(F \cap I) \geq \# \B^d_r(I) / 2 = (l/r)^d / 2$.
	\end{enumerate}
\end{lemma}

\begin{remark}
	Property \ref{avoidanceItem} says that $F$ avoids strongly non-diagonal cubes in $\B^{dn}_s(G)$. Properties \ref{nonConcentrationItem} and \ref{largeSizeItem} together imply that for every $I \in \B^d_l(E)$, at least half the cubes $I'\in \B_r^d(I)$ contribute a single sub-cube of sidelength $s$ to $F$; the rest contribute none. 
	%contains a single cube of sizelength $s$ inside of $I$. 
\end{remark}

\begin{proof}
	Let $r$ be the smallest dyadic length at least as large as $R$, where
	%
	\begin{equation} \label{What-is-r}
		R = \big(2 l^{-d}s^{dn}\# \B^{dn}_s(G)\big)^{\frac{1}{d(n-1)}}.
		%r\geq\max\Big(s,\ \big(l^{-d}s^{dn}\# \B^{dn}_s(G)\big)^{\frac{1}{d(n-1)}}\Big).
	\end{equation} 
	%
	%By the first inequality in \eqref{ZsLarge}, 
	This choice of $r$ satisfies \eqref{rBound}. 
	%Define $A_l = (2^{1/d}/l)^{1/(n-1)}.$ By the second inequality in \eqref{ZsLarge}, we have
	%
	%	\[ A_l (s^{dn}\#\B^{dn}_s(G))^{1/d(n-1)} \leq A_l l^{n/(n-1)} / 2^{1/d(n-1)} = l. \]
	%Since $l$ is a dyadic length, we conclude that $r\leq l$ and thus 
	The inequalities in \eqref{ZsLarge} ensure that $r \in [s,l]$; more precisely, the left inequality in \eqref{ZsLarge} implies $R$ is bounded from below by $s$, and the right inequality implies $R$ is bounded from above by $l$. The minimality of $r$ ensures $s \leq r \leq l$.

	For each $I' \in \B_r^d(E)$, let $J_{I'}$ be an element of $\B^d_s(I)$ chosen uniformly at random; these choices are independent as $I'$ ranges over the elements of $\B_r^d(E)$. Define
	%
	\[ 	U = \bigcup \left\{ J_{I'} \setcolon I' \in \B_r^d(E) \right\}, \]
	%
	and
	%
	\[ \mathcal{K}(U) = \{ K \in \B^{dn}_s(G) \setcolon K \in U^n, \text{$K$ strongly non-diagonal} \}. \]
	%
	Note that the sets $U$ and $\mathcal{K}(U)$ are random sets, in the sense that they depend on the random variables $\{ J_{I'} \}$. Define
	%
	\begin{equation} \label{defnOfF}
		F(U) = U - \{ \pi(K) \setcolon K \in \mathcal{K}(U) \},
	\end{equation}
	%
	where $\pi \colon \RR^{dn} \to \RR^d$ is the projection map $(x_1, \dots, x_n) \mapsto x_1$, for $x_i \in \RR^d$. Thus $\pi$ sends the cube $J_1 \times \dots \times J_n\in \B^{dn}_s$ to the cube $J_1 \in \B^d_s$. Given any strongly non-diagonal cube $K = J_1 \times \cdots \times J_n \in \B_s^{dn}(G)$, either $K \not \in \B_s^{dn}(U^n)$, or $K \in \B_s^{dn}(U^n)$. If the former occurs then $K \not \in \B_s^{dn}(F(U)^n)$ since $F(U) \subset U$, so $\B_s^{dn}(F(U)^n) \subset \B_s^{dn}(U^n)$. If the latter occurs then $K \in \mathcal{K}(U)$, and since $\pi(K) = J_1$, $J_1 \not \in \B_s^d(F(U))$. In either case, $K \not \in \B_s^{dn}(F(U)^n)$, so $F(U)$ satisfies Property \ref{avoidanceItem}. By construction, $U$ contains at most one subcube $J \in \B^{dn}_s$ for each $I \in \B^{dn}_l(E)$. Since $F(U) \subset U$, $F(U)$ satisfies Property \ref{nonConcentrationItem}. Thus the set $F(U)$ satisfies Properties \ref{avoidanceItem} and \ref{nonConcentrationItem} regardless of which values are assumed by the random variables $\{ J_{I'} \}$. Next we will show that with non-zero probability, the set $F(U)$ satisfies Property \ref{largeSizeItem}. 

	For each cube $J \in \B_s^d(E)$, there is a unique `parent' cube $I' \in \B_r^d(E)$ such that $J \subset I'$. Since $I'$ contains $(r/s)^d$ cubes of sidelength $s$, and $J_{I'}$ is chosen uniformly at random from $\B^d_s(I')$,
	%
%	\begin{equation} \label{singleCubeProb}
	\[ \prob(J \subset U) = \prob(J_{I'} = J) = (s/r)^d. \]
%	\end{equation}
	%
	%Here the probability measure $\Prob(\cdot)$ is taken with respect to the randomly chosen set $U$ defined in \eqref{Udefinition}.
	The cubes $J_{I'}$ are chosen independently, so if $J_1, \dots, J_n$ are distinct cubes in $\B^d_s(E)$, then %the last calculation combined with Property \ref{nonConcentrationItem} shows that
	%
	\begin{equation}\label{jointprob}
	\prob(J_1, \dots, J_n \in U) = \begin{cases} (s/r)^{dn} & \text{if $J_1, \dots, J_n$ have distinct parents,} \\ 0 & \text{otherwise}. \end{cases} 
	\end{equation}
	%
	Let $K = J_1 \times \dots \times J_n \in \B^{dn}_s(G)$. If the cubes $J_1, \dots, J_n$ are distinct, we deduce from \eqref{jointprob} that
	%
	\begin{equation}\label{probaKSubsetUn}
		\prob(K \subset U^n) = \prob(J_1, \dots, J_n \in U) \leq (s/r)^{dn}.
	\end{equation}
	%
	By \eqref{probaKSubsetUn}, linearity of expectation, and \eqref{What-is-r},
	%
	\begin{align*}
		\expect(\# \mathcal{K}(U)) &= \sum_{K \in \B^{dn}_s(G)} \prob(K \subset U^n) \leq \# \B_s^{dn}(G) \cdot (s/r)^{dn}
		% (l/r)^{dn}/2
	%	&= \left[ s^{dn}\#\B^{dn}_s(G) r^{-d(n-1)} \right] r^{-d} \\
	%	& \leq \left[ s^{dn}\#\B^{dn}_s(G) (A_l (s^{dn}\#\B^{dn}_s(G))^{1/d(n-1)})^{-d(n-1)} \right] r^{-d} \\
		\leq 0.5 \cdot (l/r)^d.
		%& = (l/r)^d /2.
	\end{align*}
	%
	In particular, there exists at least one (non-random) set $U_0$ such that
	%
	\begin{equation}\label{KU0Small}
		\# \mathcal{K}(U_0) \leq \expect(\# \mathcal{K}(U)) \leq 0.5 \cdot (l/r)^d.
	\end{equation}
	%
	 In other words, $F(U_0) \subset U_0$ is obtained by removing at most $0.5 \cdot (l/r)^d$ cubes in $\B^d_s$ from $U_0$. For each $I \in \B_l^d(E)$, we know that $\# \B_{s}^d(I \cap U_0) = (l/r)^d$. Combining this with \eqref{KU0Small}, we arrive at the estimate 
	%
	% Change: Some of the manipulations of the old version of this inequality are not technically true using the notation provided.
	\begin{align*}
		\# \B_s^d(I \cap F(U_0)) &= \B_s^d(I \cap U_0) - \# \{ \pi(K) \setcolon K \in \mathcal{K}(U_0), \pi(K) \in F(U_0) \}\\
		&\geq \B_s^d(I \cap U_0) - \#(\mathcal{K}(U_0))\\
		&\geq (l/r)^d - 0.5 \cdot (l/r)^d \geq 0.5 \cdot (l/r)^d
%		\# \B_{s}^d(I \cap F_{U_0}) &= \# \B_{s}^d(I \cap F_{U_0}) - \# \B_{s}^d \bigl[ I \cap \pi(\mathcal K(U_0)) \bigr] \\  
%		&\geq \# \B_{s}^d(I \cap F_{U_0}) - \# \B_{s}^d (\pi(\mathcal K(U_0))) \\ 
%		&\geq \# \B_{s}^d(I \cap F_{U_0}) - \# \B_{s}^d (\mathcal K(U_0))\\
%		&\geq (l/r)^d - 0.5 \cdot (l/r)^d \geq 0.5 \cdot (l/r)^d.  
	\end{align*}  
	%
	In other words, $F(U_0)$ satisfies Property \ref{largeSizeItem}. Setting $F = F(U_0)$ completes the proof.
\end{proof}

\begin{remarks}
	\
	\begin{enumerate}
		\item[(1)] While Lemma \ref{discretelemma} uses probabilistic arguments, the conclusion of the lemma is not a probabilistic statement. In particular, one can find a suitable $F$ constructively by checking every possible choice of $U$ (there are finitely many) to find one particular choice $U_0$ which satisfies \eqref{KU0Small}, and then defining $F$ by \eqref{defnOfF}. Thus the set we obtain in Theorem \ref{mainTheorem} exists by purely constructive means.
		
		\item[(2)] At this point, it is possible to motivate the numerology behind the dimension bound $\dim(X) \geq (dn-\alpha)/(n-1)$ from Theorem \ref{mainTheorem}, albeit in the context of Minkowski dimension. We will pause to do so here before returning to the proof of Theorem \ref{mainTheorem}. For simplicity, let $\alpha > d$, and suppose that $Z \subset \RR^{dn}$ satisfies 
		%
		\begin{equation}\label{specialCase}
			\#\B_{s}^{dn}(Z) \sim s^{-\alpha} \quad \textrm{for every}\ s \in (0,1].
		\end{equation}
		%
		Let $l = 1$, $E = [0,1)^d$, and let $s > 0$ be a small parameter. If $s$ is chosen sufficiently small compared to $d,n$, and $\alpha$, then \eqref{ZsLarge} is satisfied with $G = \bigcup \B^{dn}_s(Z)$. We can then apply Lemma \ref{discretelemma} to find a dyadic scale $r \sim s^{(dn-\alpha)/d(n-1)}$ and a set $F$ that avoids the strongly non-diagonal cubes of $\B_{s}^{dn}(Z)$. The set $F$ is a union of approximately $r^{-d} \sim s^{-(dn-\alpha)/(n-1)}$ cubes of sidelength $s$. Thus informally, the set $F$ resembles a set with Minkowski dimension $\alpha$ when viewed at scale $s$. 

		The set $X$ constructed in Theorem \ref{mainTheorem} will be obtained by applying Lemma \ref{discretelemma} iteratively at many scales. At each of these scales, $X$ will resemble a set of Minkowski dimension $(dn - \alpha)/(n-1)$. A careful analysis of the construction (performed in Section \ref{dimensionsection}) shows that $X$ actually has Hausdorff dimension at least $(dn - \alpha)/(n-1)$.

		\item[(3)] Lemma \ref{discretelemma} is the core method in our avoidance technique. The remaining argument is fairly modular. If, for a special case of $Z$, one can improve the result of Lemma \ref{discretelemma} so that $r$ is chosen on the order of $s^{\beta/d}$, then the remaining parts of our paper can be applied near verbatim to yield a set $X$ with Hausdorff dimension $\beta$, as in Theorem \ref{mainTheorem}. 
	\end{enumerate} 
\end{remarks}









\section{Fractal Discretization}\label{discretizationsection}

% CHANGE: Now this section has been organized, you talk about the sets Z_k before you even introduce them. This needs to be reworded.
% ORIGINAL: In this section we will construct the set $X$ from Theorem \ref{mainTheorem} by applying Lemma \ref{discretelemma} at many scales. The goal is to find a nested decreasing family of discretized sets $\{ X_k \}$ and to set $X = \bigcap X_k$. One condition guaranteeing that $X$ avoids $Z$ is that $X_k^n$ is disjoint from {\it strongly non-diagonal} cubes in $Z_k$.
In this section we construct the set $X$ from Theorem \ref{mainTheorem} by applying Lemma \ref{discretelemma} at many scales. Let us start by fixing a strong cover $Z$ that we will work with in the sequel.

%Since $Z$ is a countable union of bounded sets with Minkowski dimension at most $\alpha$, there exists a strong cover (see Definition \ref{defStrongCover}) of $Z$ by cubes restricted to a sequence of dyadic lengths $\{ l_k \}$, with a quantitative bound on the number of cubes at each scale. We fix a cover so that the scales $l_k$ converge to $0$ very quickly.

\begin{lemma}\label{coveringLemma}
	Let $Z \subset \RR^{dn}$ be a countable union of bounded sets with Minkowski dimension at most $\alpha$, and let $\epsilon_k \searrow 0$ with $2\epsilon_k < dn - \alpha$ for all $k$. Then there exists a sequence of dyadic lengths $\{ l_k \}$ and a strong cover of $Z$ by a sequence of sets $\{ Z_k \}$, such that
	%
	\begin{enumerate}
		\item\label{DiscretenessProperty} \emph{Discreteness}: For all $k \geq 0$, $Z_k$ is a union of cubes in $\B^{dn}_{l_k}$.

		\item\label{SparsityProperty} \emph{Sparsity}: For all $k \geq 0$, $l_k^{-d} \leq \#\B^{dn}_{l_k}(Z_k) \leq l_k^{-\alpha-\epsilon_k}$.

		\item\label{RapidDecayProperty} \emph{Rapid Decay}: For all $k > 1$,
			\begin{align}
				l_k^{dn-\alpha-\varepsilon_k} & \leq 0.5 \cdot l_{k-1}^{dn} \label{coverBoundRequirement}, \\
				l_k^{\epsilon_k} & \leq l_{k-1}^{2d}\label{quadDecayRequirement}.
			\end{align}
	\end{enumerate}
\end{lemma}
\begin{proof}
	We can write $Z = \bigcup_{i = 1}^\infty Y_i$, with $\lowminkdim(Y_i) \leq \alpha$ for each $i$. Consider the $d$ dimensional hyperplane
	%
	\[ H = \{ (x_1,\dots, x_1) \setcolon x_1 \in [0,1)^d \}. \]
	%
	Let $Y_i' = Y_i \cup H$. In particular, this means for any $l$,
	%
	\begin{equation}\label{YPrimeLowerBound}
		\# \B^{nd}_l(Y_i') \geq \# \B^{nd}_l(H) = l^{-d}.
	\end{equation}
	%
	Note that $\lowminkdim(Y_i') \leq \alpha$ for each index $i$. Let $\{ i_k \}$ be a sequence of integers that repeats each integer infinitely often.

	The lengths $\{ l_k \}$ and sets $\{ Z_k \}$ are defined inductively. As a base case, set $l_0 = 1$ and $Z_0 = [0,1)^d$. Suppose that the lengths $l_0, \ldots, l_{k-1}$ have been chosen. Since $\lowminkdim(Y_{i_k}) \leq \alpha$, Definition \ref{defnMinkowskiDim} implies that there exists arbitrarily small lengths $l$ that satisfy
	%
\[ \# \B^{dn}_l(Y_{i_k}') \leq l^{-\alpha - \frac{\varepsilon_k}{4}}. \]
Since $\varepsilon_k>0$, this means that there exist arbitrarily small dyadic lengths $l$ that satisfy
	\begin{equation}\label{coveringOfBdnlZk}
		\# \B^{dn}_l(Y_{i_k}') \leq l^{-\alpha - \frac{\varepsilon_k}{2}}.
	\end{equation}
	%
	In particular, we can choose a dyadic length $l = l_k$ small enough to satisfy \eqref{coverBoundRequirement}, \eqref{quadDecayRequirement}, and \eqref{coveringOfBdnlZk}. With this choice of $l_k$, we have that Property \ref{RapidDecayProperty} is satisfied. Define $Z_k$ to be the union of the cubes in $\B^{dn}_{l_k}(Y_{i_k}')$.  This choice of $Z_k$ clearly satisfies Property \ref{DiscretenessProperty}, and Property \ref{SparsityProperty} is implied by \eqref{YPrimeLowerBound} and \eqref{coveringOfBdnlZk}.

	It remains to verify that the sets $\{Z_k\}$ strongly cover $Z$. Fix a point $z \in Z$. Then there exists an index $i$ such that $z \in Y_i$, and there is a subsequence $k_1, k_2, \dots$ such that $i_{k_j} = i$ for each $j$. But then $z \in Y_i \subset Y_i' \subset Z_{i_{k_j}}$, so $z$ is contained in each of the sets $Z_{i_{k_j}}$, and thus $z \in \limsup Z_i$.
\end{proof}

To construct $X$, we consider a nested, decreasing family of discretized sets $\{ X_k \}$, where $X_k$ is a union of cubes in $\B^d_{l_k}(X_k)$. We then set $X = \bigcap X_k$. The goal is to choose $X_k$ such that $X_k^n$ is disjoint from {\it strongly non diagonal} cubes in $Z_k$.

\begin{lemma} \label{stronglydiagonal}
	Let $Z \subset \RR^{dn}$, let $\{Z_k\}$ be a sequence of sets that strongly cover $Z$, and let $\{ l_k \}$ be a sequence of lengths converging to zero. For each index $k$, let $X_k$ be a union of cubes in $\B^d_{l_k}$. Suppose that for each $k$, $X_k^n$ avoids strongly non-diagonal cubes in $\B^{dn}_{l_k}(Z_k)$. If $X = \bigcap X_k$, then for any distinct $x_1, \dots, x_n \in X$, we have $(x_1, \dots, x_n) \not \in Z$.
\end{lemma}
\begin{proof}
	Let $z \in Z$ be a point with distinct coordinates $z_1, \dots, z_n$. Define
	%
	\[ \Delta = \{ (w_1, \dots, w_n) \in \RR^{dn} \setcolon \text{there exists $i \neq j$ such that $w_i = w_j$} \}. \]
	%
	Then $d(\Delta,z) > 0$, where $d$ is the Hausdorff distance between $\Delta$ and $z$. Since $\{ Z_k \}$ strongly covers $Z$, there is a subsequence $\{ k_m \}$ such that $z \in Z_{k_m}$ for every index $m$. Since $l_k$ converges to 0 and thus $l_{k_m}$ converges to $0$, if $m$ is sufficiently large then $\sqrt{dn} \cdot l_{k_m} < d(\Delta,z)$. Note that $\sqrt{dn} \cdot l_{km}$ is the diameter of a cube in $\B_{l_{k_m}}^{dn}$. For such a choice of $m$, if $I\in \B_{l_{k_m}}^{dn}(Z_{k_m})$ is the (unique) cube in $\B_{l_{k_m}}^{dn}$ containing $z$, then $I \cap \Delta = \emptyset$. But this means $I$ is strongly non-diagonal. Since $X_{k_m}$ avoids the strongly non-diagonal cubes of $Z_{k_m}$, we conclude that $z \not \in X_{k_m}^n$. In particular, this means $z \not\in X^n$.
\end{proof}

All that remains is to apply the discrete lemma to choose the sets $X_k$.

\begin{lemma} 
	Given a sequence of dyadic length scales $\{l_k\}$ obeying, \eqref{coverBoundRequirement}, \eqref{quadDecayRequirement}, and \eqref{coveringOfBdnlZk} as above, there exists a sequence of sets $\{X_k\}$ and a sequence of dyadic intermediate scales $\{ r_k \}$ with $l_k \leq r_k \leq l_{k-1}$ for each $k \geq 1$, such that each set $X_k$ is a union of cubes in $\B_{l_k}^d(X_{k-1})$ that avoids the strongly non-diagonal cubes of $\mathcal B_{l_k}^{dn}(Z_k)$. Furthermore, for each index $k\geq 1$ we have
	%
	\begin{align}
		& r_k \lesssim l_k^{(dn-\alpha -\epsilon_k)/d(n-1)},\label{rkSizeBound}\\
		& \# \B^d_{l_k}(X_k \cap I) \geq 0.5 \cdot (l_{k-1}/r_k)^d \quad \text{for each}\ I\in \B_{l_{k-1}}^d(X_{k-1}), \label{manyIkInIkm1}\\
		&\# \B^d_{l_k}(X_k \cap I') \leq 1 \quad \text{ for each}\ I' \in \B_{r_{k}}^d(X_{k-1}).\label{XkWellDistributed}
	\end{align}
\end{lemma}
\begin{proof}
	We construct $X_k$ by induction, using Lemma \ref{discretelemma} at each step. Set $X_0=[0,1)^d$. Next, suppose that the sets $X_0, \ldots, X_{k-1}$ have been defined. Our goal is to apply Lemma \ref{discretelemma} to $E = X_{k-1}$ and $G = Z_k$ with $l = l_{k-1}$ and $s = l_k$. This will be possible once we verify the hypothesis \eqref{ZsLarge}, which in this case takes the form
	%
	\begin{equation}
		(l_{k-1}/l_k)^d \leq \#\B_{l_k}^{dn}(Z_k) \leq 0.5 \cdot (l_{k-1}/l_k)^{dn}. \label{need-to-check}
	\end{equation}
	%
	The right hand side follows from Property \ref{SparsityProperty} of Lemma \ref{coveringLemma} and \eqref{quadDecayRequirement}. 
%imply that 
%$$
%\#\B_{l_k}^{dn}(Z_k)\leq\frac{1}{2}(l_{k-1}/l_k)^{dn}.
%$$
	On the other hand, Property \ref{SparsityProperty} and the fact that $l_{k-1} \leq 1$ implies that
	%
	\[ (l_{k-1}/l_k)^d\leq l_{k}^{-d}\leq \#\B_{l_k}^{dn}(Z_k), \]
	%
	establishing the left inequality in \eqref{need-to-check}. Applying Lemma \ref{discretelemma} as described above now produces a dyadic length
	%
	\begin{equation}\label{definOfr}
		r \sim \big(l_{k-1}^{-d}l_k^{dn} \# \B^{dn}_{l_k}(Z_k)\big)^{\frac{1}{d(n-1)}} 
	\end{equation}
	%
	and a set $F\subset X_{k-1}$ that is a union of cubes in $\B_{l_k}^{d}$. The set $F$ satisfies Properties \ref{avoidanceItem}, \ref{nonConcentrationItem}, and \ref{largeSizeItem} from the statement of Lemma \ref{discretelemma}. Define $r_k=r$ and $X_k=F$. The estimate  \eqref{rkSizeBound} on $r_k$ follows from \eqref{definOfr} using the known bounds \eqref{quadDecayRequirement} and \eqref{coveringOfBdnlZk}:
	%
	\[ r_k \lesssim \bigl( l_{k-1}^{-d}  l_k^{dn -\alpha - 0.5 \epsilon_k} \bigr)^{\frac{1}{d(n-1)}} = \bigl( l_{k-1}^{-d} l_k^{0.5 \epsilon_k} l_k^{dn -\alpha - \epsilon_k} \bigr)^{\frac{1}{d(n-1)}} = \bigl( l_{k-1}^{-2d} l_k^{\epsilon_k}\bigr)^{\frac{1}{2d(n-1)}} l_{k}^{\frac{dn-\alpha -\epsilon_k}{d(n-1)}} \lesssim l_{k}^{\frac{dn-\alpha -\epsilon_k}{d(n-1)}}. \]
	%
	The requirements \eqref{manyIkInIkm1} and \eqref{XkWellDistributed} follow from Properties \ref{nonConcentrationItem} and \ref{largeSizeItem} of Lemma \ref{discretelemma} respectively.
\end{proof} 

Now we have defined the sets $\{ X_k \}$, we set $X = \bigcap X_k$. Since $X_k$ avoids strongly non-diagonal cubes in $Z_k$, Lemma \ref{stronglydiagonal} implies that if $x_1, \dots, x_n \in X$ are distinct, then $(x_1, \dots, x_n) \not \in Z$. To finish the proof of Theorem \ref{mainTheorem}, we must show that $\hausdim(X) \geq (dn - \alpha)/(n - 1)$. This will be done in the next section. 







\section{Dimension Bounds}\label{dimensionsection}

To complete the proof of Theorem \ref{mainTheorem}, we must show that $\hausdim(X) \geq (dn - \alpha)/(n - 1)$.  %, where
%
% \[ \beta = \frac{dn - \alpha}{n - 1}. \]
In view of Definition \ref{defFrostmanItem}, we will do this by constructing a Frostman measure of appropriate dimension supported on $X$. 
%
% We begin with a rough outline of our proof strategy. Recall that from the previous section, we have a decreasing sequence of lengths $\{ l_k \}$. The most convenient way to examine the dimension of $X$ at various scales is to use Frostman's lemma (see Definition \ref{frostmanItem}). We construct a probability measure $\mu$ supported on $X$ such that for all $\varepsilon > 0$, for all dyadic lengths $l$, and for all $I \in \B^d_l$, $\mu(I) \lesssim_\varepsilon l^{\beta - \varepsilon}$. We begin by showing that for each $k\geq 1$,
% \begin{equation}\label{muIScalek}
% \mu(I) \lesssim l_k^{\beta - O(1/k)}\quad\textrm{for all}\ I \in \B^d_{l_k}.
% \end{equation}
% Heuristically, this inequality stays that $X$ looks like a set with dimension $\beta - O(1/k)$ at the scale $l_k$. Our next task will be understand the behavior of $\mu$ (and thus $X$) at scales between $l_{k-1}$ and $l_k$. This task is complicated by the fact that $l_{k}$ might be much smaller than $l_{k-1}$ (indeed, we have no effective control on how quickly the length scales $\{l_k\}$ converge to 0). Thankfully, however, the sets $X_k$ defined in the previous section are unions of cubes of sidelength $I_{l_k}$ that are somewhat uniformly distributed at scales larger than $l_k$ (this Property \ref{nonConcentrationItem} in Lemma \ref{discretelemma}); this fact will allow us to establish an analogue of \eqref{muIScalek} at intermediate scales between $l_k$ and $l_{k+1}$. 
%

We start by defining a premeasure on $\bigcup_{i = 1}^\infty \B^d_{l_i}[0,1)^d$. Set $\mu([0,1)^d) = 1$. Suppose now that $\mu(I)$ has been defined for all cubes in $\B^d_{l_{k-1}}[0,1)^d$, and let $J \in \B^d_{l_k}$. Consider the unique `parent cube' $I \in \B^d_{l_{k-1}}$ for which $J \subset I$. Define
%
\begin{equation} \label{muRecurse} 
	\mu(J) = \begin{cases} {\mu(I)}/{\# \B^d_{l_k}(X_k \cap I)} & \textrm{if}\ J \subset X_k,\\
0 & \textrm{otherwise}.
\end{cases}
\end{equation}
Observe that for each index $k\geq 1$ and each $I \in \B_{l_{k-1}}^d$, 
%
\begin{equation}\label{muBreakDown}
	\sum_{J \in \B_{l_k}^d(I)} \mu(J) = \sum_{J \in \B_{l_k}^d(X_k\cap I)} \mu(J) = \mu(I).
\end{equation}
In particular, for each index $k$ we have
%
\[ \sum_{I\in\B_{l_k}}\mu(I)=1. \]
%
By a standard argument involving the Caratheodory extension theorem \cite[Proposition 1.7]{Falconer}, the premeasure $\mu$ extends to a measure on the Borel subsets of $[0,1)^d$. Note that for each $k \geq 1$, $\mu$ is supported on $X_k$. Thus $\mu$ is supported on $\bigcap X_k = X$. To complete the proof of Theorem \ref{mainTheorem} we will show that $\mu$ is a Frostman measure of dimension $(dn - \alpha)/(n - 1)-\epsilon$ for every $\epsilon>0$. 



\begin{lemma}\label{massSomeScales}
	For each $k\geq 1$ and each $J \in \B^d_{l_k}(X)$, 
	%
	\[ \mu(J) \lesssim l_k^{\frac{dn-\alpha}{n-1}- \eta_k}, \quad \text{ where } \quad \eta_k = \frac{n+1}{2(n-1)} \cdot \epsilon_k \searrow 0 \text{ as } k \rightarrow \infty. \]
\end{lemma}
\begin{proof}
	Let $J \in \B^d_{l_k}$ and let $I \in \B^d_{l_{k-1}}$ be the parent cube of $J$. Since $\mu$ is a probability measure, we have $\mu(I) \leq 1$. Combining \eqref{muRecurse}, \eqref{manyIkInIkm1}, \eqref{rkSizeBound}, and \eqref{quadDecayRequirement} we obtain
	%
	\[ \mu(J)\leq \frac{2r_k^d}{l_{k-1}^d}\mu(I)\leq \frac{2r_k^d}{l_{k-1}^d}\lesssim \frac{l_{k}^{\frac{dn-\alpha - \epsilon_k}{n-1}}}{l_{k-1}^d}=l_k^{\frac{dn-\alpha}{n-1}-\eta_k}\big(l_k^{0.5 \epsilon_k}/l_{k-1}^d\big)\leq l_k^{\frac{dn-\alpha}{n-1}-\eta_k}.\qedhere \]
\end{proof}

\begin{corollary}\label{muAtScaleRk}
For each $k\geq 1$ and each $I' \in \B^d_{r_k} (X_{k-1})$, 
	\begin{equation} 
	\mu(I') \lesssim (r_k/l_{k-1})^d l_{k-1}^{\frac{dn-\alpha}{n-1}-\eta_{k-1}}. \label{mu-Rk}
	\end{equation} 
\end{corollary}
\begin{proof}
%Lemma \ref{massSomeScales} allows us to control $\mu$ at the scales $\{l_k\}$. 
Let us fix a cube $I' \in \B^d_{r_k}(X_{k-1})$, and let $I$ denote its unique parent cube in $\B_{l_{k-1}}^d (X_{k-1})$. According to \eqref{XkWellDistributed}, $I'$ contains at most one cube in $\B_{l_k}^d(I)$; let us denote this cube by $J$ if it exists. Then the mass distribution rule given by \eqref{muRecurse} dictates that zero
\begin{align*}
\mu(I') = \mu(X_k \cap I') = \begin{cases} \mu(J) = {\mu(I)}/{\# \B_{l_k}^d(X_k \cap I)}  &\text{ if } \# \B_{l_k}^d(X_k \cap I') = 1, \\ 0 &\text{ if } \# \B_{l_k}^d(X_k \cap I') = 0. \end{cases} 
\end{align*}
Using the estimate \eqref{manyIkInIkm1} and applying Lemma \ref{massSomeScales} to $I \in \mathcal B_{l_{k-1}}^d(X)$, we arrive at the claimed bound \eqref{mu-Rk}. 
\end{proof}
Lemma \ref{massSomeScales} and Corollary \ref{muAtScaleRk} allow us to control the behavior of $\mu$ at all scales. %To understand the behavior of $\mu$ at other scales, we will obtain a Frostman measure bound at {\it all} scales, we need to apply a covering argument. This is where the uniform mass assignment technique comes into play. Because $\mu$ behaves like a full dimensional set between the scales $l_k$ and $r_{k+1}$, we won't be penalized for making the gap between $l_k$ and $r_{k+1}$ arbitrarily large. This is essential to our argument, because $l_k$ decays faster than $2^{-k^m}$ for any $m > 0$.

\begin{lemma} \label{frostmanBound}
For every $\alpha \in [d, dn)$, and for each $\epsilon>0$, there is a constant $C_\epsilon$ so that for all dyadic lengths $l\in (0,1]$ and all $I \in \B_l^d$, we have
	\begin{equation} 
	\mu(I) \leq C_{\epsilon} l^{\frac{dn - \alpha}{n - 1} - \epsilon}. \label{mu-ball-condition} 
	\end{equation} 
\end{lemma}
\begin{proof}
	Fix $\epsilon > 0$. Since $\eta_k \searrow 0$ as $k\to\infty$, there is a constant $C_{\epsilon}$ so that $l_k^{-\eta_k}\leq C_{\epsilon}l_k^{-\epsilon}$ for each $k \geq 1$. For instance, if $\varepsilon_k$ is decreasing, we could choose $C_{\epsilon}=l_{k_0}^{-\eta_{k_0}}$, where $k_0$ is the largest integer for which $\eta_{k_0} \geq \epsilon$. Next, let $k$ be the (unique) index so that $l_{k+1}\leq l < l_{k}$. We will split the proof of \eqref{mu-ball-condition} into two cases, depending on the position of  $l$ within $[l_{k+1}, l_k]$. 
	%We now consider several cases. 
	%\begin{itemize}
	%\item If $k<k_0$, then $l\geq l_{k_0}$ and thus 
	%$$
	%\mu(I)\leq 1 = \big(l^{\frac{dn - \alpha}{n - 1} - \epsilon}\big)^{-1}\big(l^{\frac{dn - \alpha}{n - 1} - \epsilon}\big)\leq C_{\epsilon}\big(l^{\frac{dn - \alpha}{n - 1} - \epsilon}\big).
	%$$

	{\em{Case 1: }} If $r_{k+1} \leq l \leq l_k$, 
	we can cover $I$ by $(l/r_{k+1})^d$ cubes in $\B^d_{r_{k+1}}$. By Corollary \ref{muAtScaleRk},
	\begin{equation}
	\begin{split}
	\mu(I) & \lesssim (l/r_{k+1})^d (r_{k+1}/l_k)^d l_k^{\frac{dn-\alpha}{n-1}-\eta_k} \\
	& = (l/l_k)^d l_k^{\frac{dn-\alpha}{n-1}-\eta_{k}}\\
	& = l^{\frac{dn-\alpha}{n-1}} (l/l_k)^{\frac{\alpha - d}{n-1}} l_k^{-\eta_k}\\
	& \leq l^{\frac{dn-\alpha}{n-1} - \eta_k}  \\
	& \leq C_{\epsilon}l^{\frac{dn-\alpha}{n-1}-\epsilon}.
	\end{split}
	\end{equation}
The penultimate inequality is a consequence of our assumption $\alpha \geq d$. 

	{\em{Case 2: }} If $l_{k+1} \leq l \leq r_{k+1},$ we can cover $I$ by a single cube in $\B^d_{r_{k+1}}$. By \eqref{XkWellDistributed}, each cube in $\B^d_{r_{k+1}}$ contains at most one cube $I_0 \in \B^d_{l_{k+1}}(X_{k+1})$, so by Lemma \ref{massSomeScales},
	%
	\[ 
		\mu(I) \leq \mu(I_0) \lesssim l_{k+1}^{\frac{dn - \alpha}{n - 1} - \eta_{k+1}} 
		% \lesssim l_{k+1}^{\frac{dn - \alpha}{n - 1}}r_{k+1}^{-\eta_{k+1}\frac{d(n-1)}{dn-\alpha-\epsilon_{k+1}}}
		% \leq C_{\epsilon}l_{k+1}^{\frac{dn - \alpha}{n - 1}}r_{k+1}^{-\epsilon}
		\leq C_{\epsilon}l_{k+1}^{\frac{dn - \alpha}{n - 1} - \epsilon}
		\leq C_{\epsilon}l^{\frac{dn - \alpha}{n - 1} - \epsilon}.\qedhere
	\]
	%\end{itemize}

\end{proof}

Applying Frostman's lemma to Lemma \ref{frostmanBound} gives $\hausdim(X) \geq \frac{dn - \alpha}{n - 1} - \epsilon$ for every $\epsilon>0$, which concludes the proof of Theorem \ref{mainTheorem}.









\section{Applications}\label{applications}

As discussed in the introduction, Theorem \ref{mainTheorem} generalizes Theorems 1.1 and 1.2 from \cite{MalabikaRob}. In this section, we present two applications of Theorem \ref{mainTheorem} in settings where previous methods do not yield any results.

\subsection{Sum-sets avoiding specified sets}

\begin{theorem} \label{sumset-application} 
	Let $Y \subset \RR^d$ be a countable union of sets of Minkowski dimension at most $\beta < d$. Then there exists a set $X \subset \RR^d$ with Hausdorff dimension at least $d - \beta$ such that $X + X$ is disjoint from $Y$.
\end{theorem}
\begin{proof}
	Define $Z = Z_1 \cup Z_2$, where
	%
	\[ Z_1 = \{ (x,y) \setcolon x + y \in Y \} \quad \text{and} \quad Z_2 = \{ (x,y) \setcolon y \in Y/2 \}. \]
	%
	Since $Y$ is a countable union of sets of Minkowski dimension at most $\beta$, $Z$ is a countable union of sets with lower Minkowski dimension at most $d + \beta$. Applying Theorem \ref{mainTheorem} with $n = 2$ and $\alpha = d + \beta$ produces a set $X \subset \RR^d$ with Hausdorff dimension $2d  - (d + \beta) = d - \beta$ such that $(x,y) \not \in Z$ for all $x,y \in X$ with $x \neq y$. We claim that $X+ X$ is disjoint from $Y$. To see this, first suppose $x, y \in X$, $x \ne y$. Since $X$ avoids $Z_1$, we conclude that $x + y \not \in Y$. Suppose now that $x = y \in X$. Since $X$ avoids $Z_2$, we deduce that $X \cap (Y/2) = \emptyset$, and thus for any $x \in X$, $x + x = 2x \not \in Y$. This completes the proof.
\end{proof}


\subsection{Subsets of Lipschitz curves avoiding isosceles triangles}

\subsection{Subsets of Lipschitz curves avoiding isosceles triangles}

In \cite{MalabikaRob}, Fraser and the second author prove that there exists a set $S \subset [0,1]$ with dimension $\log_3 2$ such that for any simple $C^2$ curve $\gamma \colon [0,1] \to \RR^n$ with bounded non-vanishing curvature, $\gamma(S)$ does not contain the vertices of an isosceles triangle. Our method enables us to obtain a result that works for a Lipschitz curve. Currently, we are able to provide a slightly worse dimensional bound ($1/2$ instead of $\log_3 2$), and are unable to ensure uniformity across all Lipschitz curves with a given Lipschitz constant.

\begin{theorem}\label{C1IsoscelesThm}
	Let $f\colon [0,1] \to \RR^{n-1}$ be Lipschitz with $\| f \|_{\text{Lip}} < 1$. Then there is a set $X \subset [0,1]$ of Hausdorff dimension $1/2$ so that the set $\{(t,f(t)) \setcolon t\in X\}$ does not contain the vertices of an isosceles triangle.
\end{theorem}
\begin{proof}
	Set
	%
	\[ Z = \left\{ (x_1,x_2,x_3) \in [0,1]^3\setcolon \begin{array}{c} (x_1,f(x_1)), (x_2,f(x_2)), (x_3,f(x_3))\\
		\textrm{form the vertices of an isosceles triangle} \end{array} \right\}. \]
	%
	In the next lemma, we show $Z$ has lower Minkowski dimension at most two. By Theorem \ref{mainTheorem}, there is a set $X_1\subset[0,1]$ of Hausdorff dimension $1/2$ so that for each distinct $x_1,x_2,x_3\in X_1$, we have $(x_1,x_2,x_3)\not\in Z$. This is precisely the statement that for each $x_1,x_2,x_3\in X$, the points $(x_1,f(x_1)),\ (x_2,f(x_2))$, and $(x_3,f(x_3))$ do not form the vertices of an isosceles triangle. To complete the proof, let $X = \{ (x,f(x)) : x \in X_1 \}$.
\end{proof}

\begin{lemma}
	Let $f\colon [0,1] \to \RR^{n-1}$ be Lipschitz with $\| f \|_{\text{Lip}} < 1$. Then the set
	%
	\[ Z = \left\{ (x_1,x_2,x_3) \in [0,1]^3\setcolon \begin{array}{c} (x_1,f(x_1)), (x_2,f(x_2)), (x_3,f(x_3))\\
		\textrm{form the vertices of an isosceles triangle} \end{array} \right\} \]
	%
	has Minkowski dimension at most two.
\end{lemma}
\begin{proof}
	First, notice that three points $p,q,r \in \RR^n$ form an isosceles triangle, with $r$ as the apex, if and only if $r \in H_{p,q}$, where
	%
	\[ H_{p,q} = \left\{ x \in \RR^n : \left( x - \frac{p + q}{2} \right) \cdot (q - p) = 0 \right\}. \]
	%
	To prove $Z$ has Minkowski has dimension at most two, it suffices to show that the set
	%
	\[ W = \left\{ x \in [0,1]^3 : (x_3,f(x_3)) \in H_{(x_1,f(x_1)), (x_2,f(x_2))} \right\} \]
	%
	has Minkowski dimension at most 2, because $Z$ is the union of three copies of $W$, obtained by permuting coordinates.

	Fix $0 < \delta < 1$, and consider $I_1, I_2 \in \B^1_\delta[0,1]$, together with an integer $k > 0$ such that $d(I_1,I_2) = k \cdot \delta$. Let $x_1$ be the midpoint of $I_1$, and $x_2$ the midpoint of $I_2$. Suppose $(y_1,y_2,y_3) \in W \cap I_1 \times I_2 \times [0,1]$. Then
	%
	\[ \left( y_3 - \frac{y_1 + y_2}{2} \right) \cdot (y_2 - y_1) + \left( f(y_3) - \frac{f(y_2) + f(y_1)}{2} \right) \cdot (f(y_2) - f(y_1)) = 0. \]
	%
	We know $|x_1 - y_1|, |x_2 - y_2| \leq \delta/2$, so
	%
	\begin{align} \label{xyDiff}
		&\left| \left( y_3 - \frac{y_1 + y_2}{2} \right) (y_2 - y_1) - \left( y_3 - \frac{x_1 + x_2}{2} \right) (x_2 - x_1) \right| \nonumber\\
		&\ \ \ \ \ \leq \frac{|y_1 - x_1| + |y_2 - x_2|}{2} |y_2 - y_1| + \Big( |y_1 - x_1| + |y_2 - x_2| \Big) \left| y_3 - \frac{x_1 + x_2}{2} \right|\\
		&\ \ \ \ \ \leq (\delta/2) \cdot 1 + \delta \cdot 1 \leq 3\delta/2. \nonumber
	\end{align}
	%
	Conversely, we know $|f(x_1) - f(y_1)|, |f(x_2) - f(y_2)| \leq \delta/2$ because $\| f \|_{\text{Lip}} < 1$, and a similar calculation yields
	%
	\begin{align} \label{fnDiff}
	\begin{split}
		&\Big| \left( f(y_3) - \frac{f(y_1) + f(y_2)}{2} \right) \cdot (f(y_2) - f(y_1))\\
		\\&\ \ \ \ \ - \left( f(y_3) - \frac{f(x_1) + f(x_2)}{2} \right) \cdot (f(x_2) - f(x_1)) \Big|\leq 3\delta/2.
	\end{split}
	\end{align}
	%
	Putting \eqref{xyDiff} and \eqref{fnDiff} together, we conclude that
	%
	\begin{equation} \label{hyperplanethick}
		\left| \left( y_3 - \frac{x_1 + x_2}{2} \right) (x_2 - x_1) + \left( f(y_3) - \frac{f(x_2) + f(x_1)}{2} \right) \cdot (f(x_2) - f(x_1)) \right| \leq 3\delta.
	\end{equation}
	%
	Since $|(x_2-x_1,f(x_2)-f(x_1))| \geq k\delta$, we can interpret \eqref{hyperplanethick} as saying $(y_1,y_2,y_3)$ is contained in a $3/k$ thickening of the hyperplane $H_{(x_1,f(x_1)), (x_2,f(x_2))}$. Given another $y'$ contained in a $3/k$ thickening of the hyperplane, it satisfies a variant of the inequality \eqref{hyperplanethick}, we can subtract the difference between the two inequalities to conclude
	%
	\begin{equation} \label{diffinequality}
		\left| \left( y_3 - y_3' \right) (x_2 - x_1) + (f(y_3) - f(y_3')) \cdot (f(x_2) - f(x_1)) \right| \leq 6\delta.
	\end{equation}
	%
	The triangle difference inequality applied with \eqref{diffinequality} implies
	%
	\begin{align} \label{yylowbound}
	\begin{split}
		(f(y_3) - f(y_3')) \cdot (f(x_2) - f(x_1)) &\geq |y_3 - y_3'||x_2-x_1| - 6\delta \geq k\delta \cdot |y_3 - y_3'| - 6 \delta.
	\end{split}
	\end{align}
	%
	Conversely,
	%
	\begin{align} \label{yyupbound}
	\begin{split}
		(f(y_3) - f(y_3')) \cdot (f(x_2) - f(x_1)) &\leq \| f \|_{\text{Lip}}^2 |y_3 - y_3'| |x_2 - x_1| \leq \| f \|_{\text{Lip}}^2 (k+1) \delta |y_3 - y_3'|.
	\end{split}
	\end{align}
	%
	Combining \eqref{yylowbound} and \eqref{yyupbound} and rearranging gives that if $k$ is sufficiently large, depending only on $\| f \|_{\text{Lip}}$, %$k \geq 2|f|^2/(1 - |f|^2)$,
	%
	\begin{equation} |y_3 - y_3'| \leq \frac{6}{k - (k+1) \| f \|_{\text{Lip}}^2} \lesssim 1/k \end{equation} %\leq \frac{12}{k (1 - |f|^2)}. \]
	%
	so $\# \B^3_\delta(W \cap I_1 \times I_2 \times [0,1]) \lesssim 1/k$. If $k$ is too small to use this bound, we just conclude the trivial bound that $\# \B^3_\delta(W \cap I_1 \times I_2 \times [0,1]) \lesssim 1/\delta$.

	For each value of $k$, there are at most $O(1/\delta)$ pairs $(I_1,I_2)$ with $d(I_1,I_2) = k \delta$. And the maximum value of $k$ for any pairs $I_1,I_2$ is $O(1/\delta)$. Thus
	%
	\begin{align*}
		\# \B^3_\delta(W) &= \sum_{k = 0}^{O(1)} \sum_{d(I_1,I_2) = k\delta} \# \B^3_\delta(W \cap I_1 \times I_2 \times [0,1])\\
		&\ \ \ \  + \sum_{k = O(1)}^{O(1/\delta)} \sum_{d(I_1,I_2) = k\delta} \# \B^3_\delta(W \cap I_1 \cap I_2 \times [0,1])\\
		&= O(1) \cdot O(1/\delta) \cdot O(1/\delta) + O \left( \sum_{k = O(1)}^{O(1/\delta)} 1/k \right) \cdot O(1/\delta) \cdot O(1/\delta)\\
		&= O(\log(1/\delta) \cdot (1/\delta^2)).
	\end{align*}
	%
	This shows $W$ has upper Minkowski dimension at most 2.
\end{proof}

\endinput
%% The following is a directive for TeXShop to indicate the main file
%%!TEX root = diss.tex

\chapter{Extensions to Low Rank Configurations}
\label{ch:LowRank}



\section{Boosting the Dimension of Pattern Avoiding Sets by Low Rank Coordinate Changes}

We now consider finding subsets of $[0,1]$ avoiding solutions to the equation $y = f(Tx)$, where $T$ is a rank $k$ linear transformation with integer coefficients with respect to standard coordinates, and $f$ is real-valued and Lipschitz continuous. Fix a constant $A$ bounding the operator norm of $T$, in the sense that $|Tx| \leq A|x|$ for all $x \in \mathbf{R}^n$, and a constant $B$ such that $|f(x+y) - f(x)| \leq B|y|$ for all $x$ and $y$ for which the equation makes sense (if $f$ is $C^1$, this is equivalent to a bound $\| \nabla f \|_\infty \leq B$). Consider sets $J_0, J_1, \dots, J_n, \subset [0,1]$, which are unions of intervals of length $1/M$, with startpoints lying on integer multiples of $1/M$. The next theorem works as a `building block lemma' used in our algorithm for constructing a set avoiding solutions to the equation with Hausdorff dimension $k$ and full Minkowski dimension.

\begin{theorem}
    For infinitely many integers $N$, there exists $S_i \subset J_i$ avoiding solutions to $y = f(Tx)$ with $y \in S_0$ and $x_n \in S_n$, such that
    %
    \begin{itemize}
        \item For $n \neq 0$, if we decompose each $J_i$ into length $1/N$ consecutive intervals, $S_i$ contains an initial portion $\Omega(1/N^k)$ of each length $1/N$ interval. This part of the decomposition gives the Hausdorff dimension $1/k$ bound for the set we will construct.

        \item If we decompose $J_i$ into length $1/N$ intervals, and then subdivide these intervals into length $\Omega(1/N^k)$ intervals, then $S_0$ contains a subcollection of these $1/N^k$ intervals which contains a total length $\Omega(1/N)$ of a fraction $1 - 1/M$ of the length $1/N$ intervals. This property gives that our resultant set will have full Minkowski dimension.
    \end{itemize}
    %
    The implicit constants in these bounds depend only on $A$, $B$, $n$, and $k$.
\end{theorem}
\begin{proof}
Split each interval of $J_a$ into length $1/N$ intervals, and then set
%
\[ \mathbf{A} = \{ x : x_a\ \text{is a startpoint of a $1/N$ interval in $J_a$} \} \]
%
Since the startpoints of the intervals are integer multiples of $1/N$, $T(\mathbf{A})$ is contained with a rank $k$ sublattice of $(\mathbf{Z}/N)^m$. The operator norm also guarantees $T(\mathbf{A})$ is contained within the ball $B_A$ of radius $A$ in $\mathbf{R}^m$. Because of the lattice structure of the image, $| x - y | \gtrsim_n 1/N$ for each distinct pair $x,y \in T(\mathbf{A})$. For any $R$, we can cover $\Sigma \cap B_A$ by $O_{n,k}((A/R)^k)$ balls of radius $R$. If $R \gtrsim_n 1/N$, then each ball can contain only a single element of $T(\mathbf{A})$, so we conclude that $|T(\mathbf{A})| \lesssim_{n,k} (AN)^k$. If we define the set of `bad points' to be
%
\[ \mathbf{B} = \{ y \in [0,1] : \text{there is $x \in \mathbf{A}$ such that $y = f(T(x))$} \} \]
%
Then
%
\[ |\mathbf{B}| = |f(T(\mathbf{A}))| \leq |T(\mathbf{A})| = O_{A,n,k}(N^k) \]
%
For simplicity, we now introduce an integer constant $C_0 = C_0(A,n,k,M)$ such that $|\mathbf{B}| \leq (C_0/M^2) N^k$. We now split each length $1/M$ interval in $J_0$ into length $1/N$ intervals, and filter out those intervals containing more than $C_0N^{k-1}$ elements of $\mathbf{B}$. Because of the cardinality bound we have on $\mathbf{B}$ there can be at most $N/M^2$ such intervals, so we discard at most a fraction $1/M$ of any particular length $1/M$ interval in $J_0$. If we now dissect the remaining intervals into $4C_0N^{k-1}$ intervals of length $1/4C_0N^k$, and discard any intervals containing an element of $\mathbf{B}$, or adjacent to such an interval, then the remaining such intervals $I$ satisfy $d(I,\mathbf{B}) \geq 1/4C_0N^k \gtrsim_{A,n,M,k}(1/N^k)$, and because of our bound on the number of elements of $\mathbf{B}$ in these intervals, there are at least $C_0N^{k-1}$ intervals remaining, with total length exceeding $C_0N^{k-1}/4C_0N^k = \Omega(1/N)$. If $f$ is $C^1$ with $\| \nabla f \|_\infty \leq B$, or more generally, if $f$ is Lipschitz continuous of magnitude $B$, then
%
\[ | f(Tx) - f(Tx')| \leq AB |x - x'| \]
%
and so we may choose $S_i \subset J_i$ by thickening each startpoint $x \in J_i$ to a length $O(1/N^k)$ interval while still avoiding solutions to the equation $y = f(T(x))$.
\end{proof}

\begin{remark}
    If $T$ is a rank $k$ linear transformation with rational coefficients, then there is some number $a$ such that $aT$ has integer coefficients, and then the equation $y = f(Tx)$ is the same as the equation $y = f_0((aT)(x))$, where $f_0(x) = f(x)/a$. Since $f_0$ is also Lipschitz continuous, we conclude that we still get the dimension $1/k$ bound if $T$ has rational rather than integral coefficients. More generally, this trick shows the result applies unperturbed if all coefficients of $T$ are integer multiples of some fixed real number. More generally, by varying the lengths of our length $1/N$ decomposition by a constant amount, we can further generalize this to the case where each column of $T$ are integers multiples of some fixed real number.
\end{remark}

\begin{remark}
    To form $\mathbf{A}$, we take startpoints lying at equal spaced $1/N$ points. However, by instead taking startpoints at varying points in the length $1/N$ intervals, we might be able to make points cluster more than in the original algorithm. Maybe the probabalistic method would be able to guarantee the existence of a choice of startpoints whose images are tightly clustered together.
\end{remark}

\begin{remark}
    Since the condition $y = f(Tx)$ automatically assumes a kind of `non-vanishing derivative' condition on our solutions, we do not need to assume the regularity of $f$, and so the theorem extends naturally to a more general class of functions than Rob's result, i.e. the Lipschitz continuous functions.
\end{remark}

Using essentially the same approach as the last argument shows that we can avoid solutions to $y = f(Tx)$, where $y$ and $x$ are now vectors in some $\mathbf{R}^m$, and $T$ has rank $k$. If we consider unions of $1/M$ cubes $J_0, \dots, J_n$. If we fix startpoints of each $x_k$ forming lattice spaced apart by $\Omega(1/N)$, and consider the space $\mathbf{A}$ of products, then there are $O(N^k)$ points in $T(\mathbf{A})$, and so there are $O(N^k)$ elements in $\mathbf{B}$. We now split each $1/M$ cube in $J_0$ into length $1/N$ cubes, and discard those cubes which contain more than $O(N^{k-m})$ bad points, then we discard at most $1 - 1/M$ of all such cubes. We can dissect the remaining length $1/N$ cubes into $O(N^{k-m})$ length $\Omega(1/N^{k/m})$ cubes, and as in the previous argument, the cubes not containing elements of $\mathbf{B}$ nor adjacent to an element have total volume $\Omega(1/N^m)$, which we keep. The startpoints in the other intervals $T_i$ may then be thickened to a length $\Omega(1/N^{m/k})$ portion while still avoiding solutions. This gives a set with full Minkowski dimension and Hausdorff dimension $m/k$ avoiding solutions to $y = f(Tx)$. (I don't yet understand Minkowski dimension enough to understand this, but the techniques of the appendix make proving the Hausdorff dimension $1/k$ bound easy)

\section{Extension to Well Approximable Numbers}

If the coefficients of the linear transformation $T$ in the equation $y = f(Tx)$ are non-rational, then the images of startpoints under the action of $T$ do not form a lattice, and so points may not overlap so easily when avoiding solutions to the equations $y = f(Tx)$. However, if $T$ is `very close' to a family of rational coefficient linear transformations, then we can show the images of the startpoints are `very close' to a lattice, which will still enable us to find points avoiding solutions by replacing the direct combinatorial approach in the argument for integer matrices with a covering argument.

Suppose that $T$ is a real-coefficient linear transformation with the property that for each coefficient $x$ there are infinitely many rational numbers $p/q$ with $|x - p/q| \leq 1/q^\alpha$, for some fixed $\alpha$. For infinitely many $K$, we can therefore find a linear transformation $S$ with coefficients in $\mathbf{Z}/K$ with each coefficient of $T$ differing from the corresponding coefficient in $S$ by at most $1/K^\alpha$. Then for each $x$, we find
%
\[ \| (T - S)(x) \|_\infty \leq (n/K^\alpha) \| x \|_\infty \]
%
If we now consider $T_0, \dots, T_n$, splitting $T_1, \dots, T_n$ into length $1/N$ intervals, and considering $\mathbf{A}$ as in the last section, then $S(\mathbf{A})$ lie in a $k$ dimensional sublattice of $(\mathbf{Z}/KN)^m$, hence containing at most $(2A)^k (KN)^k = O_{T,n}((KN)^k)$ points. By our error term calculation of $T-S$, the elements of $T(\mathbf{A})$ are contained in cubes centered at these lattice points with side-lengths $2n/K^\alpha$, or balls centered at these points with radius $n^{3/2}/K^\alpha$. If $\| \nabla f \| \leq B$, then the images of the radius $n^{3/2}/K^\alpha$ balls under the action of $f$ are contained in length $Bn^{3/2}/K^\alpha$ intervals. Thus the total length of the image of all these balls under $f$ is $(Bn^{3/2}/K^\alpha)(2A)^k(KN)^k = (2A)^k Bn^{3/2} K^{k-\alpha} N^k$. If $k < \alpha$, then we can take $K$ arbitrarily large, so that there exists intervals with $\text{dist}(I,\mathbf{B}) = \Omega_{A,k,M}(1/N^k)$. But I believe that, after adding the explicit constants in, we cannot let $k = \alpha$.

\begin{remark}
    One problem is that, if $T$ has rank $k$, we might not be able to choose $S$ to be rank $k$ as well. Is this a problem? If $T$ has full rank, then the set of all such matrices is open so if $T$ and $S$ are close enough, $S$ also has rank $k$, but this need not be true if $T$ does not have full rank.
\end{remark}

\begin{example}
    If $T$ has rank 1, then Dirichlet's theorem says that every irrational number $x$ can be approximated by infinitely many $p/q$ with $|x - p/q| < 1/q^2$, so every real-valued rank 1 linear transformation can be avoided with a dimension one bound.
\end{example}

\section{Equidistribution and Real Valued Matrices}

If $T$ is a non-invertible matrix containing irrational coefficients, then the values $Tx$, for $x \in \mathbf{Z}^n$, do not form a lattice, and therefore we cannot use the direct combinatorial arguments of the past section to obtain the decomposition lemma. However, without loss of generality, we can write $T(x) = S(x_1) + U(x_2)$, where $x = (x_1,x_2)$, $x_1 \in \mathbf{R}^k$, $x_2 \in \mathbf{R}^{n-k}$, and $S$ has full rank $k$. Then $S$ is an embedding of $\mathbf{R}^k$ into $\mathbf{R}^n$, so $\Gamma = S((\mathbf{Z}/N)^k)$ forms a lattice with points spaced apart by a distance on the order of $\Omega(1/N)$. Since $T$ has rank $k$, the image of $U$ is contained within the image of $S$. We let $\mathbf{T}$ denote the torus obtained by quotienting the $k$ dimensional subplane forming the image of $T$ by $\Gamma$. Then the image of $S$ in $\mathbf{T}$ is contained within an $\alpha$ dimensional subtorus of $\mathbf{T}$. Note that $\alpha = 0$ precisely when $T$ still has rank $k$ over the rational numbers, so that in a suitable basis $T$ is an integer valued matrix. If $\alpha = 1$, then by an appropriate scaling in the values $x_2$ we can still make the values of $S$ lie at lattice points, which should give a Hausdorff dimension one set. When $\alpha = 2$, we run into problems.

If $\pi: \mathbf{R}^k \to T$ is the homomorphism obtained by composing the quotient map onto the torus with the linear map $T$, then $\pi(x_1,0) = 0$ for all $x_1 \in (\mathbf{Z}/N)^k$. On the other hand, Ratner's theorem implies that for each $x_2 \in \mathbf{R}^{n-k}$, there is some $M$ such that the sequence $\pi(0, nMx_2) = Mn \pi(0, x_2)$ is equidistributed on a subtorus of $T$. Equidistribution may be useful in extending the rational matrix result to all real matrices with some dimension loss, since a matrix is rational if and only if $n M \pi(0, x_2)$ is equidistributed on a zero dimensional lattice -- it may ensure that points are closely clustered to lattice points.

Lets consider the simplest case, where $n = k+1$, so $x_2 \in \mathbf{R}$. Consider our setup, with intervals $J_0, \dots, J_n$, and an equation $y = f(Tx)$, where $f$ is Lipschitz with Lipschitz norm bounded by $B$. Now if $T = S + U$, then $S(\mathbf{Z}^k)$ forms a rank $k$ lattice, and $U(\mathbf{Z})$ equidistributes over an $\alpha$ dimensional subtorus of the torus generated over the lattice. In particular, since the set $\mathbf{Z} \cap NJ_n$ contains $\Omega(N |J_n|) = \Omega(N)$ consecutive points, for any $\varepsilon$ and suitably large $N$, $\mathbf{Z} \cap NJ_n$ contains $\Omega(Nr^\alpha)$ points $x$ such that $S(x)$ is within a distance $r$ from a lattice point, for any $r$. Dividing by $N$ tells us that we have $O(Nr^\alpha)$ points $x$ in $\mathbf{Z}/N \cap J_n$ such that $U(x)$ is at a distance $r/N$ from a lattice point in $S((\mathbf{Z}/N)^k)$. If $r = 1/N^\beta$, then  we have $O(N^{1-\alpha\beta})$ points at a distance $1/N^{\beta + 1}$ from a lattice point. There are $O(N^k)$ points in the lattice, and so provided that $k < 1 + \beta$, we can find a large subset avoiding the images of these startpoints, and we should be able to thicken the startpoints to lengt h $\Omega(1/N^k)$ intervals, hence we should expect the set we construct to have Hausdorff dimension $1/k(k-1)$ if this process is repeated to construct our solution avoiding set.

\section{Applications of Low Rank Coordinate Changes}

\begin{example}
Our initial exploration of low rank coordinate changes was inspired by trying to find solutions to the equation
%
\[ y - x = (u - w)^2 \]
%
Our algorithm gives a Hausdorff dimension $1/2$ set avoiding solutions to this equation. This equals Math\'{e}'s result. But this dimension for us now depends on the shifts involved in the equation, not on the exponent, so we can actually avoid solutions to the equation
%
\[ y - x = (u - w)^n \]
%
for any $n$, in a set of Hausdorff dimension $1/2$. More generally, if $X$ is a set, then given a smooth function $f$ of $n$ variables, we can find a set $X$ of Hausdorff dimension $1/n$ such that there is no $x \in X$, and $y_1, \dots, y_n \in X - X$ such that $x = f(y_1, \dots, y_n)$. This is better than the $1/2n$ bound that is obtained by Malabika and Fraser's result.
\end{example}

\begin{example}
For any fixed $m$, we can find a set $X \subset \mathbf{R}^n$ of full Hausdorff dimension which contains no solutions to
%
\[ a_1x_1 + \dots + a_nx_n = 0 \]
%
for {\it any} rational numbers $a_n$ which are not all zero. Since Malabika/Fraser's technique's solutions are bounded by the number of variables, they cannot let $n \to \infty$ to obtain a linearly independant set over the rational numbers. But since the Hausdorff dimension of our sets now only depends on the rank of $T$, rather than the total number of variables in $T$, we can let $n \to \infty$ to obtain full sets linearly independant over the rationals. More generally, for any Lipschitz continuous function $f: \mathbf{R} \to \mathbf{R}$, we can find a full Hausdorff dimensional set such that there are no solutions
%
\[ f(a_1x_1 + \dots + a_nx_n, y) \]
%
for any $n$, and for any rational numbers $a_n$ that are not all zero.
\end{example}

\begin{example}
The easiest applications of the low rank coordinate change method are probably involving configuration problems involving pairwise distances between $m$ points in $\mathbf{R}^n$, where $m \ll n$, since this can best take advantage of our rank condition. Perhaps one way to encompass this is to avoid $m$ vertex polyhedra in $n$ dimensional space, where $m \ll n$. In order to distinguish this problem from something that can be solved from Math\'{e}'s approach, we can probably find a high dimensional set avoiding $m$ vertex polyhedra on a parameterized $n$ dimensional manifold, where $m \ll n$. There is a result in projective geometry which says that every projectively invariant property of $m$ points in $\mathbf{RP}^d$ is expressible as a function in the ${m \choose d}$ bracket polynomials with respect to these $m$ points. In particular, our result says that we can avoid a countable collection of such invariants in a dimension $1/{m \choose d}$ set. This is a better choice of coordinates than Euclidean coordinates if ${m \choose d} \leq m$. Update: I don't think this is ever the case.
\end{example}

\begin{remark}
    Because of how we construct our set $X$, we can find a dimension $1/k$ set avoiding solutions to $y = f(Tx)$ for {\it all} rank $k$ rational matrices $T$, without losing any Hausdorff dimension. Maybe this will help us avoid solutions to more general problems?
\end{remark}

\begin{example}
Given a smooth curve $\Gamma$ in $\mathbf{R}^n$, can we find a subset $E$ with high Hausdorff dimension avoiding isoceles triangles. That is, if the curve is parameterized by $\gamma: [0,1] \to \mathbf{R}^n$, can we find $E \subset [0,1]$ such that for any $t_1, t_2, t_3$, $\gamma(t_1)$, $\gamma(t_2)$, and $\gamma(t_3)$ do not form the vertices of an isoceles triangle. This is, in a sense, a non-linear generalization of sets avoiding arithmetic progressions, since if $\Gamma$ is a line, an isoceles triangle is given by arithmetic progressions. Assuming our curve is simple, we must avoid zeroes of the function
%
\[ |\gamma(t_1) - \gamma(t_2)|^2 = |\gamma(t_2) - \gamma(t_3)|^2 \]
%
If we take a sufficiently small segment of this curve, and we assume the curve has non-zero curvature on this curve, we can assume that $t_1 < t_2 < t_3$ in our dissection method.

If the coordinates of $\gamma$ are given by polynomials with maximum degree $d$, then the equation
%
\[ |\gamma(t_1) - \gamma(t_2)|^2 - |\gamma(t_2) - \gamma(t_3)|^2 \]
%
is a polynomial of degree $2d$, and so Math\'{e}'s result gives a set of dimension $1/2d$ avoiding isoceles triangles. In the case where $\Gamma$ is a line, then the function $f(t_1,t_2,t_3) = \gamma(t_1) + \gamma(t_3) - 2\gamma(t_2)$ avoids arithmetic progresions, and Math\'{e}'s result gives a dimension one set avoiding such progressions. Rob and Malabika's algorithm easily gives a set with dimension $1/2$ for any curve $\Gamma$. Our algorithm doesn't seem to be able to do much better here.
\end{example}

\begin{example}
    What is the largest dimension of a set in Euclidean space such that for any value $\lambda$, there is at most one pair of points $x,y$ in the set such that $|x - y| = \lambda$.
\end{example}

\begin{example}
    What is the largest dimension of a set which avoids certain angles, i.e. for which a triplet $x,y,z$ avoids certain planar configurations.
\end{example}

\begin{example}
    A set of points $x_0, \dots, x_d \in \mathbf{R}^d$ lie in a hyperplane if and only if the determinant formed by the vectors $x_n - x_0$, for $n \in \{ 1, \dots, d \}$, is zero. This is a degree $d$ polynomial, hence Math\'{e}'s result gives a dimension one set with no set of $d+1$ points lying in a hyperplane. On the other hand, a theorem of Mattila shows that every analytic set $E$ with dimension exceeding one contains $d + 1$ points in a hyperplane. Can we generalize this to a more general example avoiding points on a rotational, translation invariant family of manifolds using our results?
\end{example}

\begin{example}
    Given a set $F$ not containing the origin, what is the largest Hausdorff dimension of a set $E$ such that for any for any distinct rational $a_1, \dots, a_N$, the sum $a_1E + \dots + a_NE$ does not contain any elements of $F$. Thus the vector space over the rationals generated by $E$ does not contain any elements of $F$. We can also take the non-linear values $f(a_1E + \dots + a_NE)$ avoiding elements of $F$. $F$ must have non-empty interior for the problem to be interesting. Then can we find a smooth function $f$ with non-nanishing derivative which vanishes over $F$, or a family of smooth functions with non-vanishing derivative around $F$.
\end{example}

\section{Idea: Generalizing This Problem to low rank smooth functions}

Suppose we are able to find dimension $1/k$ sets avoiding configurations $y = g(f(x))$, where $f$ is a smooth function from $\mathbf{R}^n \to \mathbf{R}^m$ of rank $k$. Then given any function $g(f(x))$, where $f$ has rank $k$, if $g(f(x)) = 0$, then the implicit function theorem guarantees that there is a cover $U_\alpha$ and functions $h_\alpha: U^k \to \mathbf{R}^{n-k}$ such that for each if $g(f(x)) = 0$, for $x \in U_\alpha$, then there is a subset of $k$ indices $I$ such that $x_{I^c} = h_\alpha(x_I)$.

Then given any function $f(x)$ with rank $k$, we can use the implicit function theorem to find sets $U_\alpha$, indices $n_\alpha$, and functions $g_\alpha$ such that if $f(x) = 0$, for $x \in U_\alpha$, then $x_{n_\alpha} = g_\alpha(x_1, \dots, \widehat{x_{n_\alpha}}, \dots, x_n)$. Thus we need only avoid this type of configuration to avoid configurations of a general low rank function. If we don't believe that we are able to get dimension $1/(k-1)$ sets for rank $k$ configurations, then we shouldn't be able to find sets of Hausdorff dimension $1/k$ avoiding configurations of the form $y = f(x)$, where $f$ has rank $k$. 

\begin{remark}
    If the functions $g_\alpha$ are only partially defined, this makes the problem easier than if the functions were globally defined, because the constraint condition is now smaller than the original constraint.
\end{remark}

\begin{remark}
    This would solve our problem of avoiding $y - x = (u - v)^2$, since if $f(x,y,u,v) = y - x - (u - v)^2$, then
    %
    \[ \nabla f \]
\end{remark}

\section{Idea: Algebraic Number Fields}

If $\mathbf{Q}(\omega)$ is a quadratic extension of the rational numbers, then the ring of integers in this field form a lattice. Perhaps we can use this to generalize our approach to avoiding configurations $y = f(Tx)$, where all coefficients of the matrix $T$ lie in some common quadratic extension of the rational numbers.

\section{A Scheme for Avoiding Configurations}

Math\'{e}'s result can be reconfigured in terms of a building block strategy for implementation in our algorithm.

\begin{theorem}
    Let $f$ be a polynomial of degree $m$, and consider unions of length $1/M$ intervals $T_0, \dots, T_d \subset [0,1]$, with rational start-points. If $\partial_0 f$ is non-vanishing on $T_0 \times \dots \times T_d$, then there exists arbitrarily large integers $N$ and a constant $C$ not depending on $N$ and sets $S_n \subset T_n$ such that
    %
    \begin{itemize}
        \item $f(x) \neq 0$ for $x \in S_0 \times \dots \times S_d$.
        \item If $T_0, \dots, T_d$ are split into length $1/N$ intervals, then $S_n$ contains a length $C/N^d$ region of each interval.
    \end{itemize}
\end{theorem}
\begin{proof}
    Without loss of generality (by subdividing the initial intervals), let $M$ be the greatest common divisor of all of the startpoints of the intervals in $T_n$. Divide each interval $T_n$ into length $1/N$ intervals, and let $\mathbf{A} \subset (\mathbf{Z}/N)^d$ be the cartesian product of all startpoints of these length $1/N$ intervals. Since $f$ has degree $m$, $f(\mathbf{A}) \subset \mathbf{Z}/N^m$. If $A_0 \leq |\partial_0 f| \leq A_1$ on $T_0 \times \dots \times T_d$, then for any $a \in \mathbf{A}$, and $\delta_0$, there exists $\delta_1$ between $0$ and $\delta_0$ for which
    %
    \[ |f(a + \delta_0 e_0)| - f(a)| = \delta_0 |(\partial_0 f)(a + \delta_1)| \]
    %
    If $K$ is fixed such that $A_1 \leq (K-1)A_0$, so that we can choose
    %
    \[ \frac{1/K}{A_0N^m} \leq \delta_0 \leq \frac{\left( 1 - 1/K \right)}{A_1N^m} \]
    %
    Then
    %
    \[ \frac{1/K}{N^m} \leq |f(a + \delta_0 e_0) - f(a)| \leq \frac{1 - 1/K}{N^m} \]
    %
    Thus $d(f(\mathbf{A} + \delta_0 e_0), \mathbf{Z}/N^m) \geq 1/KN^m$. Thus if we thicken the coordinates of $\mathbf{A} + \delta_0$ to intervals of length $O(1/N^m)$, then we obtain sets $S_0, \dots, S_n$ avoiding solutions.
\end{proof}

TODO: CAN WE USE THE COMBINATORIAL NULLSTELLENSATZ TO COME UP WITH AN ALTERNATE BUILDING BLOCK LEMMA FOR ARBITRARY FIELDS?

\section{Square Free Sets}

We now look at avoiding solutions to the equation $x - y = (u - v)^2$. We consider two sets $I$ and $J$. Suppose that we can select a subset $\mathbf{S}$ from $\mathbf{Z} \cap N^2I$ such that if $x,y \in \mathbf{S}$ are distinct, $x - y$ is not a perfect square, and $|\mathbf{S}| \gtrsim |\mathbf{Z} \cap N^2I|^\alpha$. Then for any distinct $u,v \in \mathbf{Z} \cap NJ$, $(u - v)^2 \not \in \mathbf{S} - \mathbf{S}$. But this means that if we thicken the points in $\mathbf{S}/N^2$ to length $O(1/N^2)$ intervals, and the points in $\mathbf{Z}/N \cap J$ into length $O(1/N)$ intervals, then the resultant set will avoid solutions to $x - y = (u - v)^2$. This should give a dimension $\alpha$ set.

\endinput
%% The following is a directive for TeXShop to indicate the main file
%%!TEX root = diss.tex

%\chapter{Constructing Squarefree Sets}
%\label{ch:Squarefree}

\endinput
%% The following is a directive for TeXShop to indicate the main file
%%!TEX root = diss.tex

\chapter{Future Work}
\label{ch:Conclusions}

To conclude this thesis, we sketch some ideas developing the theory of `rough sets avoiding patterns', which we introduced in Chapter \ref{ch:RoughSets}. Section 6.1 attempts to exploit additional geometric information about certain rough configurations to find sets with large Hausdorff dimension avoiding patterns, and Section 6.2 finds configuration avoiding sets supported a measure with large Fourier decay.

\section{Low Rank Avoidance}

One way we can extend the results of Chapter \ref{ch:RoughSets} is to utilize addition geometric structure of the rough configuration $\C$ to obtain larger avoiding sets. Recall that in Chapter \ref{ch:RoughSets}, we studied the avoidance problem for configurations with low Minkowski dimension. This condition means precisely that these configurations are efficiently covered by cubes at all scales. The idea of this section is to study configurations which are efficiently covered by other families of geometric objects at all scales. Here, we study the simple setting where our set is efficiently covered by families of thickened hyperplanes or thickened lines. We note that a set $E$ is efficiently covered by a family of thickened parallel hyperplanes at all scales if and only, for a linear transformation $M$ with that hyperplane as a kernel, $M(E)$ has low Minkowski dimension.

\begin{theorem} \label{theorem9063909014901}
    Let $\C \subset \C(\RR)$ be the countable union of sets $\{ \C_i \}$ such that
    %
    \begin{itemize}
        \item For each $i$, there exists $n_i$ such that $\C_i$ is a pre-compact subset of $\C^{n_i}(\RR)$.

        \item There exists an integer $m_i > 0$ and $s_i \in [0,m_i)$, together with a full-rank rational-coefficient linear transformation $M_i: \RR^{n_i} \to \RR^{m_i}$ such that $M_i(\C_i)$ has lower Minkowski dimension at most $s_i$.
    \end{itemize}
    %
    Then there exists a set $X \subset [0,1]$ avoiding $\C$ with Hausdorff dimension at least
    %
    \[ \inf_i \left( \frac{m_i - s_i}{m_i} \right). \]
\end{theorem}

\begin{remarks}
    \
    \begin{enumerate}
        \item[1.] A useful feature of this method is that the resulting set does not depend on the number of points in a configuration. This is a feature only shared by Math\'{e}'s result, Theorem \ref{mathemainresult} in Section 3.3. We exploit this feature later on in this section to find large subsets avoiding a countable family of equations with arbitrarily many variables.

        \item[2.] It might be expected, based on the result of Theorem \ref{mainTheorem}, that one should be able to obtain a set $X \subset [0,1]$ avoiding $\C$ with Hausdorff dimension
        %
        \[ \inf_i \left( \frac{m_i - s_i}{m_i - 1} \right), \]
        %
        whenever $s_i \geq 1$ for all $i$. We plan to pursue whether this conjecture is true in further research.

        \item[3.] Compared to Theorem \ref{mainTheorem}, this result only applies in the one-dimensional configuration avoidance setting. We also plan to find higher dimensional analogues to this theorem, when $d > 1$, in the near future.
    \end{enumerate}
\end{remarks}

For purpose of brevity, here we only describe a solution to the discretized version of the problem. This can be fleshed out into a full proof of Theorem \ref{theorem9063909014901} by techniques analogous to those given in Chapters \ref{ch:RelatedWork} and \ref{ch:RoughSets}. Thus we discuss a single linear transformation $M: \RR^{dn} \to \RR^m$, and try to avoid a discretized version of a low dimensional set.

Before we describe the discretized result, let us simplify the problem slightly. Since our transformation $M$ has full rank, we may find indices
%
\[ i_1, \dots, i_m \in \{ 1, \dots, n \} \]
%
such that the transformation $M$ is invertible when restricted to the span of $\{ e_{i_1}, \dots, e_{i_m} \}$. By an affine change of coordinates in the range of $M$, which preserves the Minkowski dimension of any set, we may assume without loss of generality that $M(e_{i_j}) = e_j$ for each $1 \leq j \leq m$.

\begin{theorem} \label{theorem059891891829}
    Fix $s \in [0,m)$ and $\varepsilon \in [0, (m-s)/2)$. Let $T_1, \dots, T_n \subset [0,1]$ be disjoint, $\DQ_k$ discretized sets, and let $B \subset \RR^m$ be a $\DQ_{k+1}$ discretized set such that
    %
    \begin{equation} \label{equation6091904232093}
        \#(\DQ_{k+1}(B)) \leq N_{k+1}^{s + \varepsilon}.
    \end{equation}
    %
    Then there exists a constant $C(n,m,M) > 0$, and an integer constant $A(M) > 0$, such that if $A(M) \divides N_{k+1}$, and 
    %
    \begin{equation} \label{equation19024u1298352389}
        N_{k+1} > C(n,m,M) \cdot M_{k+1}^{\frac{m}{m - (s + \varepsilon)}}.
    \end{equation}
    %
    then there exists $\DQ_{k+1}$ discretized sets $S_1 \subset T_1$, \dots, $S_n \subset T_n$ such that
    %
    \begin{enumerate}
        \item For any collection of $n$ distinct cubes $Q_i \in \DQ_{k+1}(S_i)$,
        %
        \[ Q_1 \times \dots \times Q_n \not \in \DQ_{k+1}(B). \]

        \item For each $i$, and for each $Q \in \DQ_k(T_i)$, there exists $\DR_Q \subset \DR_{k+1}(Q)$ such that
        %
        \[ \#(\DR_Q) \geq \frac{\#(\DR_{k+1}(Q))}{A(M)}, \]
        %
        and if $R \in \DR_{k+1}(Q)$,
        %
        \[ \#(\DQ_{k+1}(R \cap S_i)) = \begin{cases} 1 &: R \in \DR_Q, \\ 0 &: R \not \in \DR_Q. \end{cases} \]
    \end{enumerate}
\end{theorem}
\begin{proof}
    For each $i \not \in \{ i_1, \dots, i_m \}$, there are rational numbers $a_{ij} = p_{ij}/q_{ij} \in \mathbf{Q}$ such that $M(e_i) = \sum a_{ij} e_j$. Set $A(M) = \prod_{ij} q_{ij}$. For each interval $R \in \DR_{k+1}(T_i)$, we let
    %
    \[ a(R) \in \{ 0, \dots, N_1 \dots N_k M_{k+1} - 1 \} \]
    %
    be the unique integer such that
    %
    \[ R = \left[ \frac{a(R)}{N_1 \dots N_k M_{k+1}}, \frac{a(R) + 1}{N_1 \dots N_k M_{k+1}} \right]. \]
    %
    Let $X \in \{ 0, \dots, N_{k+1}/M_{k+1} - 1 \}^m$. For each $1 \leq j \leq m$, define
    %
    \[ S_{i_j}(X) = \bigcup_{R \in \DR_{k+1}(T_{i_j})} \left[ \frac{a(R)}{N_1 \dots N_k M_{k+1}} + \frac{X_j}{N_1 \dots N_{k+1}}, \frac{a(R)}{N_1 \dots N_k M_{k+1}} + \frac{X_j + 1}{N_1 \dots N_{k+1}} \right]. \]
    %
    For $i \not \in \{ i_1, \dots, i_m \}$, define
    %
    \[ S_i(X) = \bigcup_{\substack{R \in \DR_{k+1}(T_i)\\ \prod q_{ij} \divides a(R)}} \left[ \frac{a(R)}{N_1 \dots N_k M_{k+1}}, \frac{a(R)}{N_1 \dots N_k M_{k+1}} + \frac{1}{N_1 \dots N_{k+1}} \right] \]
    %
    For each $i$, we let $\mathcal{S}_i(X)$ denote the set of startpoints to intervals in $S_i$. Then
    %
    \[ \mathcal{S}_{i_j}(X) \subset \frac{\ZZ}{N_1 \dots N_k M_{k+1}} + \frac{X_j}{N_1 \dots N_{k+1}} \]
    %
    and for $i \not \in \{ i_1, \dots, i_m \}$,
    %
    \[ \mathcal{S}_i(X) \subset \frac{\prod q_{ij} \ZZ}{N_1 \dots N_k M_{k+1}}. \]
    %
    It therefore follows that if
    %
    \[ \mathcal{A}(X) = M(\mathcal{S}_1(X) \times \dots \times \mathcal{S}_n(X)), \]
    %
    then
    %
    \[ \mathcal{A}(X) \subset \frac{\ZZ^m}{N_1 \dots N_k M_{k+1}} + \frac{X}{N_1 \dots N_{k+1}}. \]
    %
    In particular, if $X \neq X'$, $\mathcal{A}(X)$ and $\mathcal{A}(X')$ are disjoint. Equation \eqref{equation19024u1298352389} implies there is a constant $C(n,m,M)$, such that
    %
    \begin{equation} \label{equation69129319031209}
    \begin{split}
        \#& \left\{ n \in \mathbf{Z}^m : d \left( \frac{n}{N_1 \dots N_{k+1}}, B \right) \leq \frac{2}{\sqrt{d} \cdot \| M \|} \frac{1}{N_1 \dots N_{k+1}} \right\}\\
        &\ \ \ \ \ \ \ \ \ \ \ \ \ \ \ \ \ \ \ \ \ \ \ \ \ \ \ \ \ \ \ \ \ \ \leq C(n,m,M)^{m - (s + \varepsilon)} \cdot N_{k+1}^{s + \varepsilon}.
    \end{split}
    \end{equation}
    %
    Applying the pigeonhole principle, \eqref{equation6091904232093}, and \eqref{equation69129319031209}, there exists some value $X_0$ such that
    %
    \begin{align*}
        \# \left\{ n \in \mathbf{A}(X_0) : d(n,B) \leq \frac{2}{\sqrt{d} \cdot \| M \|} \frac{1}{N_1 \dots N_{k+1}} \right\} &\leq \frac{C(n,m,M)^{m - (s + \varepsilon)} \cdot N_{k+1}^{s + \varepsilon}}{(N_{k+1}/M_{k+1})^m}\\
        &\leq \frac{C(n,m,M)^{m - (s + \varepsilon)} \cdot M_{k+1}^m}{N_{k+1}^{m - (s + \varepsilon)}} < 1.
    \end{align*}
    %
    In particular, this set is actually empty. But this means that the set
    %
    \[ M(S_1(X_0) \times \dots \times S_n(X_0)) \]
    %
    is disjoint from $B$. Taking $S_i = S_i(X_0)$ for each $i$ completes the proof.
\end{proof}

Before we move on, consider one application of Theorem \ref{theorem9063909014901}, which gives an extension of Theorem \ref{sumset-application} to arbitrarily large sums.

\begin{theorem}
    Let $Y \subset \RR$ be a countable union of pre-compact sets with lower Minkowski dimension at most $t$. Then there exists a set $X \subset \RR$ with Hausdorff dimension at least $1 - t$ such that for any integer $n > 0$, for any $a_1, \dots, a_n \in \QQ$, and for any $x_1, \dots, x_n \in X$,
    %
    \[ (a_1X + \dots + a_n X) \cap Y \subset (0). \]
\end{theorem}
\begin{proof}
    Let $Y = \bigcup_{i = 1}^\infty Y_i$, where each $Y_i$ has lower Minkowski dimension at most $t$. For each $n$, $i$, and $a = (a_1, \dots, a_n) \in \QQ^n$ with $a \neq 0$, let
    %
    \[ \C_{n,a,i} = \{ (x_1, \dots, x_n) \in \C^n : a_1x_1 + \dots + a_nx_n \in Y_i \}, \]
    %
    and let $\C = \bigcup \C_{n,a,i}$. Let $T_{n,a}(x_1,\dots,x_n)$ be the linear map given by
    %
    \[ T_{n,a}(x_1,\dots,x_n) = a_1x_1 + \dots + a_nx_n. \]
    %
    Then $T_{n,a}$ is nonzero, and $T_{n,a}(\C_{n,i,a})$ how lower Minkowski dimension at most $t$. Applying Theorem \ref{theorem9063909014901}, we obtain a set $X \subset [0,1]$ avoiding $\C$ with Hausdorff dimension at least $1 - t$.

    We prove $X$ satisfies the conclusions of this theorem by induction on $n$. Consider the case $n = 1$, and fix $a \in \QQ$. If $a \neq 0$, then because $X$ avoids $\C_{n,a,i}$ for each $i$, if $x \in X$, $ax \not \in Y$, so $aX \cap Y = \emptyset$. If $a = 0$, then $aX = 0$, so $(aX) \cap Y \subset (0)$.

    In general, consider $a = (a_1, \dots, a_{n+1}) \in \QQ^{n+1}$. If $a \neq 0$, then because $X$ avoids $\C_{n,a,i}$ for each $i$, we know if $x_1, \dots, x_{n+1} \in X$ are distinct, then $a_1 x_1 + \dots + a_{n+1} x_{n+1} \not \in Y$. If the values $x_1, \dots, x_{n+1} \in X$ are not distinct, then by rearranging both the values $\{ x_i \}$ and $\{ a_i \}$, we may without loss of generality assume that $x_n = x_{n+1}$. Then
    %
    \begin{align*}
        a_1 x_1 + \dots + a_{n+1} x_{n+1} &= a_1 x_1 + \dots + a_{n-1} x_{n-1} + (a_n + a_{n+1}) x_n\\
        &\subset (a_1 X + \dots + a_{n-1} X + (a_n + a_{n+1}) X).
    \end{align*}
    %
    By induction,
    %
    \[ (a_1 X + \dots + a_{n-1} X + (a_n + a_{n+1}) X) \cap Y \subset (0), \]
    %
    so we conclude that either $a_1 x_1 + \dots + a_{n+1} x_{n+1} \not \in Y$, or $a_1x_1 + \dots + a_{n+1} x_{n+1} = 0$. The only remaining case we have not covered is if $a \in \QQ^{n+1}$ is equal to zero. But in this case,
    %
    \[ (a_1 X + \dots + a_n X) = (0 + \dots + 0) = 0, \]
    %
    and so it is trivial that $(a_1 X + \dots + a_n X) \cap Y \subset (0)$.
\end{proof}

\section{Fourier Dimension}

Recently, there has been much interest in determining whether sets with large Fourier dimension can avoid configurations. Results published recently in the literature include \cite{PramanikLaba} and \cite{Shmerkin}. In this Section, we attempt to modify the procedure of Theorem \ref{mainTheorem} to obtain a set with large Fourier dimension. We obtain such a result, though with a suboptimal dimension to what we expect from Theorem \ref{mainTheorem}. Furthermore, we restrict ourself to $d = 1$. We are currently researching methods which may give the improved bound, and apply to configuration avoidance problems where $d > 1$.

\begin{theorem} \label{FourierTheorem}
    Suppose $\C$ is a configuration on $\RR$, formed from the countable union of pre-compact sets, each with lower Minkowski dimension at most $s$. Then there exists a set $X \subset [0,1]$ with Fourier dimension at least $(n - s)/n$ avoiding $\C$.
\end{theorem}

We begin with a lemma which uses the Poisson summation theorem to restrict the analysis of the Fourier decay of the probability measures we study to the analysis of frequencies in $\ZZ$.

\begin{lemma} \label{discretefouriermeasures}
    Fix $s \in [0,d]$. Suppose $\mu$ is a compactly supported finite Borel measure on $\RR^d$. Then there exists a constant $A \geq 1$, depending only on the dimension of $d$ and the radius of the support of $\mu$, such that
    %
    \[ \sup_{\xi \in \RR^d} |\xi|^{s/2} |\widehat{\mu}(\xi)| \leq 1 + A \left( \sup_{m \in \ZZ^d} |m|^{s/2} |\widehat{\mu}(m)| \right). \]
\end{lemma}
\begin{proof}
    Without loss of generality, we may assume that $\mu$ is supported on a compact subset of $[1/3,2/3)^d$, since every compactly supported measure is a finite sum of translates of measures of this form. Let
    %
    \[ C = \sup_{m \in \ZZ^d} |m|^{s/2} |\widehat{\mu}(m)|, \]
    %
    which we may assume, without loss of generality, to be finite. Consider the distribution $\Lambda = \sum_{m \in \mathbf{Z}^d} \delta_m$, where $\delta_m$ is the Dirac delta distribution at $m$. Then the Poisson summation formula says that the Fourier transform of $\Lambda$ is itself. If $\psi \in C_c(\RR^d)$ is a bump function supported on $[0,1)^d$, with $\psi(x) = 1$ for $x \in [1/3,2/3)^d$, then $\mu = \psi (\Lambda * \mu)$, so
    %
    \begin{equation} \label{mubounded}
    \begin{split}
        |\widehat{\mu}(\xi)| &= \left| \left[ \widehat{\psi} * (\Lambda \widehat{\mu}) \right](\xi) \right|\\
        &= \left| \sum_{m \in \mathbf{Z}^d} \widehat{\mu}(m)(\widehat{\psi} * \delta_m)(\xi) \right|\\
        &= \left| \sum_{m \in \mathbf{Z}^d} \widehat{\mu}(m) \widehat{\psi}(\xi - m) \right|.
%       &\lesssim \sum_{n \in \mathbf{Z}^d} |\widehat{\mu}(n)| \prod_{i = 1}^d \frac{1}{1 + |n_i - \xi_i|}
    \end{split}
    \end{equation}
    %
    Since $\psi$ is smooth, we know that for all $\eta \in \RR^d$, $|\widehat{\psi}(\eta)| \lesssim 1/|\eta|^{d+1}$. If we perform a dyadic decomposition, we find
    %
    \begin{equation}
        \label{calculation1}
    \begin{split}
        \sum_{1 \leq |m - \xi| \leq |\xi|/2} |\widehat{\mu}(m)| |\widehat{\psi}(\xi - m)| &\leq C \sum_{1 \leq |m - \xi| \leq |\xi|/2} |\xi|^{-s/2} |\widehat{\psi}(\xi - m)|\\
        &\lesssim C \sum_{k = 1}^{\log |\xi|} \sum_{\frac{|\xi|}{2^{k+1}} \leq |m - \xi| \leq \frac{|\xi|}{2^{k}}} |\xi|^{-s/2} \left( 2^k/|\xi| \right)^{d+1}\\
        &\lesssim C \sum_{k = 1}^{\log |\xi|} |\xi|^{-s/2} (2^k / |\xi| ) \lesssim C |\xi|^{-s/2}.
    \end{split}
    \end{equation}
    %
    There are $O_d(1)$ points $m \in \mathbf{Z}^d$ with $|m - \xi| \leq 1$, so if $|\xi| \geq 2$,
    %
    \begin{equation} \label{calculation2}
        \sum_{|m - \xi| \leq 1} |\widehat{\mu}(m)| |\widehat{\psi}(m - \xi)| \lesssim C |\xi|^{-s/2}.
    \end{equation}
    %
    We can also perform another dyadic decomposition, using the fact that for all $\eta \in \RR^d$, $|\widehat{\psi}(\eta)| \lesssim 1/|\eta|^{2d}$, to find that
    %
    \begin{equation} \label{calculation3}
    \begin{split}
        \sum_{|m - \xi| \geq |\xi|/2} |\widehat{\mu}(m)| |\widehat{\psi}(m - \xi)| &\lesssim \sum_{k = 0}^\infty \sum_{|\xi| 2^{k-1} \leq |m - \xi| \leq |\xi| 2^k} \frac{|\widehat{\mu}(m)|}{|\xi|^{2d} 2^{2dk}}\\
        &\lesssim C \sum_{k = 0}^\infty |\xi|^{-d} 2^{-dk} \lesssim C |\xi|^{-d}.
    \end{split}
    \end{equation}
    %
    Combining \eqref{calculation1}, \eqref{calculation2}, and \eqref{calculation3} with \eqref{mubounded}, we conclude that there exists a constant $A \geq 1$ depending only on the dimension $d$ such that if $|\xi| \geq 2$,
    %
    \begin{equation} \label{endequation53}
        |\widehat{\mu}(\xi)| \leq A \cdot C \cdot |\xi|^{-s/2}.
    \end{equation}
    %
    Since $\| \widehat{\mu} \|_{L^\infty(\RR^d)} \leq 1$, \eqref{endequation53} actually holds for all $\xi \in \RR^d$, provided $C \geq 1$.
\end{proof}

Our goal now is to carefully modify the discrete selection strategy and discretized probability measures we use to obtain have sharp control over the Fourier transform of these measures at each scale of our construction. A key strategy is to obtain high probability bounds controlling the Fourier transform of functions on the sets we choose using Hoeffding's inequality.

\begin{theorem}[Hoeffding's Inequality]
    Let $\{ X_i \}$ be an independent family of $N$ mean-zero random variables, and let $A > 0$ be a constant such that $\| X_i \|_\infty \leq A$ for each $i$. Then for each $t > 0$,
    %
    \[ \PP \left( \left| \frac{1}{N} \sum_{i = 1}^N X_i \right| \geq t \right) \leq 2 \exp \left( (N/A^2) \cdot (- t^2) \right). \]
\end{theorem}

As with Theorem \ref{mainTheorem}, we perform a multi-scale analysis, using the notations introduced in Section \ref{sec:Dyadics}. Lemma \ref{discretefouriermeasures} implies that we only need control over integer-valued frequencies. The discretized measures $\{ \nu_k \}$ we select are, for each $k$, a sum of point mass distributions at the points $\ZZ/N_1 \dots N_k$. Therefore, $\widehat{\nu_k}$ will be $N_1 \dots N_k$ periodic, in the sense that for any $m \in (N_1 \dots N_k) \ZZ$ and $\xi \in \RR$, $\widehat{\nu_k}(\xi + m) = \widehat{\nu_k}(\xi)$. Since we are only concerned with integer valued frequencies, it will therefore suffice to control the Fourier transform of $\nu_k$ on frequencies lying in $\{ 1, \dots, N_1 \dots N_k \}$.

In the discrete lemma below, we rely on a variant of the proof strategy of Theorem 2.1 of \cite{Shmerkin}, but modified so that we can allow the branching factors $\{ N_k \}$ to increase arbitrarily fast. For each $\DQ_k$ discretized set $E \subset [0,1]$, we define a probability measure
%
\[ \nu_E = \frac{1}{\#(\DQ_k(E))} \sum_{Q \in \DQ_k(E)} \delta(a(Q)), \]
%
where for each $x \in \RR$, $\delta(x)$ is the Dirac delta measure at $x$, and for each $Q \in \DQ_k$, $a(Q)$ is the startpoint of the interval $Q$. Also, for each $k$, we define a probability measure
%
\[ \eta_k = \frac{1}{N_k} \sum_{i_1, \dots, i_d = 0}^{N_k - 1} \delta \left( \frac{i}{N_1 \dots N_k} \right).  \]
%
The purpose of introducing $\eta_k$ is so that, given a measure $\mu$ which is a sum of point mass distributions in $\ZZ/N_1 \dots N_k$, the probability measure $\mu * \eta_{k+1}$ is a sum of point mass distributions in $\ZZ/N_1 \dots N_{k+1}$, uniformly distributed at the scale $1/N_1 \dots N_{k+1}$.

%Our goal now is now to carefully modify the discrete selection strategy and discretized probability measures we use, so that with high probability, the measures have the appropriate Fourier decay for the Fourier dimension bound we wish to obtain. Surprisingly, here we only need to perform a single scale analysis with the family of cubes $\DQ^d$, rather than a multi scale analysis involving the cubes $\DQ^d$ and $\DR^d$ as in Chapter \ref{ch:RoughSets}.

\begin{lemma} \label{discreteFourierBuildingBlock}
    Fix $s \in [1,dn)$, and $\varepsilon \in [0,(n-s)/4)$. Let $T \subset \RR$ be a nonempty, $\DQ_k$ discretized set, and let $B \subset \RR^n$ be a nonempty $\DQ_{k+1}$ discretized set such that
    %
    \begin{equation} \label{equation982589128942189}
    \begin{split}
        \#(\DQ_{k+1}(B)) \leq N_{k+1}^{s + \varepsilon}.
    \end{split}
    \end{equation}
    %  \leq N_{k+1}^d
    %
    Provided that
    %
    \begin{equation} \label{equation5523786128439}
        M_{k+1} \leq N_{k+1}^{\frac{n-s-2\varepsilon}{n}} \leq 2 M_{k+1},
    \end{equation}
    %
    %\begin{equation} \label{equation5523786128439}
    %    M_{k+1}^{\frac{n}{n - s - 2\varepsilon}} \leq N_{k+1} \leq 2 M_{k+1}^{\frac{n}{n - s - 2\varepsilon}},
    %\end{equation}
    %
    \begin{equation} \label{equation189248914891}
        \quad N_{k+1} \geq 3^{1/\varepsilon},
    \end{equation}
    %
    \begin{equation} \label{equation8941894189238912}
        N_{k+1} \geq \exp \left( \left( \frac{4n}{n-s} \right)^4 N_1 \dots N_k \right),
    \end{equation}
    %
    and
    %
    \begin{equation} \label{equation77871247817841278}
        N_{k+1} \geq (1/\varepsilon)^{1/\varepsilon},
    \end{equation}
    %
    there exists a universal constant $A(n,s)$ and a $\DQ_{k+1}$ discretized set $S \subset T$, satisfying the following properties:
    %
    \begin{enumerate}
        \item[(A)] For any collection of $n$ distinct cubes $Q_1, \dots, Q_n \in \DQ_{k+1}(S)$,
        %
        \[ Q_1 \times \dots \times Q_n \not \in \DQ_{k+1}(B). \]

        \item[(B)] For any $m \in \ZZ$,
        %
        \[ |\widehat{\nu_S}(m) - \widehat{\eta_{k+1}}(m) \widehat{\nu_T}(m)| \leq A(n,s) \cdot (N_1 \dots N_{k+1})^{-\frac{n - s}{2n} + 2\varepsilon}. \]
    \end{enumerate}
\end{lemma}
\begin{proof}
    For each $R \in \DR_{k+1}(T)$, let $Q_R$ be randomly selected from $\DQ_{k+1}(R)$, such that the collection $\{ Q_R \}$ forms an independent family of random variables. Then, set $S = \bigcup \{ Q_R: R \in \DR_{k+1}(T) \}$. We then have
    %
    \begin{equation} \label{equation6900921094190290}
        \#(\DQ_{k+1}(S)) = \#(\DR_{k+1}(T)) = M_{k+1} \cdot \DQ_k(T).
    \end{equation}
    %
    Without loss of generality, removing cubes from $B$ if necessary, we may assume that for every cube $Q_1 \times \dots \times Q_n \in \DQ_{k+1}(B)$, the values $Q_1, \dots, Q_n$ occur in distinct intervals in $\DR_{k+1}(T)$. In particular, given any such cube, just as in Lemma \ref{discretelemma}, we have
    %
    \begin{equation} \label{equation12043910293120909}
        \mathbf{P}(Q_1 \times \dots Q_n \in \DQ_{k+1}(S^n)) = (M_{k+1}/N_{k+1})^n.
    \end{equation}
    %
    Thus \eqref{equation982589128942189}, \eqref{equation5523786128439}, and \eqref{equation12043910293120909} imply
    %
    \begin{equation} \label{equation999992482}
        \mathbf{E} \left[ \#(\DQ_{k+1}(B) \cap \DQ_{k+1}(S^n)) \right] \leq M_{k+1}^n/N_{k+1}^{n - (s + \varepsilon)} \leq 1/N_{k+1}^\varepsilon.
    \end{equation}
    %
    Markov's inequality, together with \eqref{equation189248914891} and \eqref{equation999992482}, imply
    %
    \begin{equation} \label{fourierdim2}
    \begin{split}
        \mathbf{P}(\DQ_{k+1}(B) \cap \DQ_{k+1}(S^n) \neq \emptyset) &= \mathbf{P}(\# (\DQ_{k+1}(B) \cap \DQ_{k+1}(S^n)) \geq 1)\\
        &\leq 1/N_{k+1}^\varepsilon \leq 1/3.
    \end{split}
    \end{equation}
    %
    Thus $\DQ_{k+1}(S^n)$ is disjoint from $\DQ_{k+1}(B)$ with high probability.

    Now we analyze the Fourier transform of the measures $\nu_S$. For each cube $R \in \DR_{k+1}(T)$, and for each $m \in \ZZ$, let
    %
    \[ A_R(m) = e^{\frac{-2 \pi i m \cdot a(Q_R)}{N_1 \dots N_{k+1}}} - \frac{1}{N_{k+1}} \sum_{l = 0}^{N_{k+1} - 1} e^{\frac{-2 \pi i m \cdot [N_{k+1} a(Q) + l]}{N_1 \dots N_{k+1}}}. \]
    %
    Then $\EE[A_R(m)] = 0$, $|A_R(m)| \leq 2$ for each $m$, and
    %
    \[ \widehat{\nu_S}(m) - \widehat{\eta_{k+1}}(m) \widehat{\nu_T}(m) = \frac{1}{\#(\DR_{k+1}(T))} \sum_{R \in \DR_{k+1}(T)} A_R(m). \]
    %
    Fix a particular value of $m$. Since the random variables $\{ A_R(m) : R \in \DR_{k+1}(T) \}$ are bounded and independent from one another, we can apply Hoeffding's inequality with \eqref{equation6900921094190290} to conclude that for each $t > 0$,
    %
    \begin{equation} \label{equation5551902402919120}
    \begin{split}
        \PP \left( |\widehat{\nu_S}(m) - \widehat{\eta_{k+1}}(m) \widehat{\nu_T}(m)| \geq t \right) &\leq 2 \exp \left( \frac{- \#(\DR_{k+1}(T)) t^2}{4} \right)\\
        &= 2 \exp \left( \frac{- \#(\DQ_k(T)) M_{k+1} t^2}{4} \right).
    \end{split}
    \end{equation}
    %
    The function $\widehat{\nu_S} - \widehat{\eta_{k+1}} \widehat{\nu_T}$ is $N_1 \dots N_{k+1}$ periodic. Thus, to uniformly bound this function, we need only bound the function over $N_1 \dots N_{k+1}$ values of $m$. Applying a union bound with \eqref{equation5551902402919120}, we conclude
    %
    \begin{equation} \label{equation6662410242191209}
        \PP \left( \| \widehat{\nu_S} - \widehat{\eta_{k+1}} \widehat{\nu_T} \|_{L^\infty(\ZZ)} \geq t \right) \leq 2 N_1 \dots N_{k+1} \exp \left( \frac{- \#(\DQ_k(T)) M_{k+1} t^2}{4} \right).
    \end{equation}
    %
    In particular, \eqref{equation5523786128439}, applied to \eqref{equation6662410242191209}, shows
    %
    \begin{align*}
        \PP & \left( \| \widehat{\nu_S} - \widehat{\eta_{k+1}} \widehat{\nu_T} \|_{L^\infty(\ZZ)} \geq (N_1 \dots N_k M_{k+1})^{-1/2} \log(M_{k+1}) \right)\\
        &\ \ \ \ \leq 2N_1 \dots N_{k+1} \exp \left( - \frac{\#(\DQ_k(T)) \log(M_{k+1})^2}{4 N_1 \dots N_k} \right)\\
        &\ \ \ \ = 2 N_1 \dots N_k \exp \left( \log(N_{k+1}) - \frac{\log(M_{k+1})^2}{4 N_1 \dots N_k} \right)\\
        %&\ \ \ \ \leq 2 N_1 \dots N_k \exp \left( \log(N_{k+1}) - \frac{\log \left( N_{k+1}^{\frac{n-s-2\varepsilon}{2n}}/2 \right)^2}{4 N_1 \dots N_k} \right)\\
        &\ \ \ \ \leq 2 N_1 \dots N_k \exp \left( \log(N_{k+1}) - \left[ \left( \frac{n - s}{4n} \right) \log(N_{k+1}) - \log(2) \right]^2 \frac{1}{N_1 \dots N_k} \right).
    \end{align*}
    % A = 2N_1 ... N_k
    % D = 1/N_1 ... N_k
    % B = (n-s/4n)
    % C = log(2)
    %
    % A e(X - D (BX + C)^2) <= 3
    % X - D(BX + C)^2 <= log(3/A)
    % X - (B^2D)X^2 - 2BCDX - C^2D <= log(3/A)
    % (B^2D) X^2 + (2BCD - 1)X + [log(3/A) - C^2D] >= 0
    % X >= (1/2B^2D - C/B) + sqrt((2BCD - 1)^2 - 4(B^2D)(log(3/A) - C^2D))/2B^2D
    % X >= (4n/n-s)^4[N_1 ... N_k]
    % N_{k+1} \geq \exp \left( (4n/n-s)^4 [N_1 ... N_k] \right)
    % 
    Thus \eqref{equation8941894189238912} implies
    %
    \begin{equation} \label{equation90120931902390190}
        \PP \left( \| \widehat{\nu_S} - \widehat{\eta_{k+1}} \widehat{\nu_T} \|_{L^\infty(\ZZ)} \geq (N_1 \dots N_k M_{k+1})^{-1/2} \log(M_{k+1}) \right) \leq 1/3.
    \end{equation}
    %
    Taking a union bound over \eqref{fourierdim2} and \eqref{equation90120931902390190}, we conclude that there is a non-zero probability that the set $S$ satisfies Property (A), and
    %
    \[ \| \widehat{\nu_S} - \widehat{\eta_{k+1}} \widehat{\nu_T} \|_{L^\infty(\ZZ)} \leq (N_1 \dots N_k M_{k+1})^{-1/2} \log(M_{k+1}). \]
    %
    Since \eqref{equation77871247817841278} holds,
    % \log(x) \leq x^\varepsilon
    % 1/\varepsilon^{1/\varepsilon} \leq x
    % N_{k+1} \geq 1/\varepsilon^{1/\varepsilon}
    %
    \begin{align*}
        (N_1 \dots N_k M_{k+1})^{-1/2} \log(M_{k+1}) &\lesssim_{n,s} \log(N_{k+1}) (N_1 \dots N_{k+1})^{- \frac{n-s-2\varepsilon}{2n}}\\
        &\leq (N_1 \dots N_{k+1})^{- \frac{n-s}{2n} + 2\varepsilon}.
    \end{align*}
    %
    Thus the set $S$ also satisfies Property (B) with an appropriately chosen constant $A(n,s)$.
\end{proof}

\begin{comment}

Let us describe the measures we construct. For each $k$, we let
%
\[ \psi_k = N_1 \dots N_k \cdot \mathbf{I}_{\left[ 0, \frac{1}{N_1 \dots N_k} \right]}. \]
%
Then for each $\xi \in \RR^d$,
%
\[ |\widehat{\psi_k}(\xi)| = \left| N_1 \dots N_k \cdot \frac{e^{- \frac{2 \pi i \xi}{N_1 \dots N_k}} - 1}{- 2\pi i \xi} \right| \lesssim \frac{N_1 \dots N_k}{1 + |\xi|}. \]
%
Thus $\widehat{\psi_k}$ has fast decay for $|\xi| \geq N_1 \dots N_k$. For any $\DQ_k$ discretized set $E$, we let $\mu_E$ by the absolutely continuous probability measure with density function
%
\[ \frac{d\mu_E}{dx} = \frac{1}{\#(\DQ_k(E))} \sum_{Q \in \DQ_k(E)} \mathbf{I}_Q. \]
%
Then $\mu_E$ is supported on $E$. Since $\mu_E$ can be viewed as a convolution of $\psi_k$ with a discrete probability measure at the left-hand edges of the intervals in $\DQ_k(E)$, for each $\xi \in \RR^d$,
%
\[ |\widehat{\mu_E}(\xi)| \leq |\widehat{\psi_k}(\xi)| \lesssim \frac{N_1 \dots N_k}{1 + |\xi|}. \]
%
Thus $\widehat{\mu_E}$ has large decay for frequencies of large magnitude, and it suffices to construct $E$ such that we can control the Fourier transform of $\mu_E$ on low magnitude frequencies. To prove this, we take $E$ to be a random set, and apply Hoeffding's inequality to obtain tail bounds on the magnitude of the Fourier transform.

\begin{theorem}[Hoeffding's Inequality]
    Let $\{ X_i \}$ be a family of $N$ independant, mean zero random variables, such that $|X_i| \leq B$ for all $i$. Then
    %
    \[ \PP \left( \sum X_i \geq t \right) \leq 2 \exp \left( \frac{-t^2}{2B^2 N} \right). \]
\end{theorem}

\begin{lemma} \label{discreteFourierBuildingBlock}
    Fix $s \in [1,n)$, $\varepsilon_1 \in [0,(n-s)/4)$, and $\varepsilon_2 \in (0,\infty)$. Let $T \subset \RR$ be a nonempty, $\DQ_k$ discretized set, and let $B \subset \RR^n$ be a nonempty $\DQ_{k+1}$ discretized set such that
    %
    \begin{equation} \label{equation982589128942189}
    \begin{split}
        \#(\DQ_{k+1}(B)) \leq N_{k+1}^{s + \varepsilon_1}.
    \end{split}
    \end{equation}
    %  \leq N_{k+1}^d
    %
    Then there exists a constant $A(d,n,s)$ such that, provided
    %
    \begin{equation} \label{equation5523786128439}
        M_{k+1}^{\frac{dn}{dn - s - 2\varepsilon}} \leq N_{k+1} \leq 2 M_{k+1}^{\frac{dn}{dn - s - 2\varepsilon}},
    \end{equation}
    %
    \begin{equation} \label{equation189248914891}
        \quad N_{k+1} \geq 3^{1/\varepsilon},
    \end{equation}
    %
    and
    %
    \begin{equation} \label{equation77871247817841278}
        N_{k+1} \geq 3^{1/d} \left( \left\lceil \frac{A(d,n,s)}{\varepsilon} \right\rceil! \right)^{3/d} (N_1 \dots N_k),
    \end{equation}
    %
    then there exists a $\DQ_{k+1}$ discretized set $S \subset T$, satisfying the following properties:
    %
    \begin{enumerate}
        \item[(A)] For any collection of $n$ distinct cubes $Q_1, \dots, Q_n \in \DQ_{k+1}(S)$,
        %
        \[ Q_1 \times \dots \times Q_n \not \in \DQ_{k+1}(B). \]

        \item[(B)] For any $m \in \ZZ$,
        %
        \[ |\widehat{\mu_S}(m) - \widehat{\mu_T}(m)| \leq N_{k+1}^{-(1 - \varepsilon_2) \frac{n - s}{2n}}. \]
    \end{enumerate}
\end{lemma}
\begin{proof}
    For each $R \in \DR_{k+1}(T)$, let $Q_R$ be randomly selected from $\DQ_{k+1}(R)$, independently from all other selections $Q_{R'}$. Then, set $S = \bigcup \{ Q_R: R \in \DR_{k+1}(T) \}$. We then have
    %
    \[ \#(\DQ_{k+1}(S)) = \#(\DR_{k+1}(T)) = M_{k+1}^d \DQ_k(T). \]
    %
    Without loss of generality, removing cubes from $B$ if necessary, we may assume that every cube $Q_1 \times \dots \times Q_n \in \DQ_{k+1}(B)$, the values $Q_1, \dots, Q_n$ are distinct. In particular, given any such cube, just as in Lemma \ref{discretelemma}, we have
    %
    \[ \mathbf{P}(Q_1 \times \dots Q_n \in \DQ_{k+1}(S^n)) = (M_{k+1}/N_{k+1})^n. \]
    %
    Thus \eqref{equation982589128942189} and \eqref{equation5523786128439} imply
    %
    \[ \mathbf{E}(\#(\DQ_{k+1}(B) \cap \DQ_{k+1}(S^n))) \leq M_{k+1}^n/N_{k+1}^{n - (s + \varepsilon_1)} \leq 1/N_{k+1}^{\varepsilon_1}. \]
    %
    Markov's inequality, together with \eqref{equation189248914891} implies
    %
    \begin{equation} \label{fourierdim2}
    \begin{split}
        \mathbf{P}(\DQ_{k+1}(B) \cap \DQ_{k+1}(S^n) \neq \emptyset) &= \mathbf{P}(\# (\DQ_{k+1}(B) \cap \DQ_{k+1}(S^n)) \geq 1)\\
        &\leq 1/N_{k+1}^\varepsilon \leq 1/3.
    \end{split}
    \end{equation}
    %
    Thus $\DQ_{k+1}(S^n)$ is disjoint from $\DQ_{k+1}(B)$ with high probability.

    Now we analyze the Fourier transform of the measure $\mu_S$. For each $R \in \DR_{k+1}(T)$, and $m \in \ZZ$, let
    %
    \[ A_R(m) = \int_R \left( \mu_S(x) - \mu_T(x) \right) e^{-2 \pi i m \cdot x}\; dx. \]
    %
    We note that for each $Q \in \DQ_{k+1}(R)$, and $x \in Q^\circ$,
    %
    \begin{align*}
        \EE(\mu_S(x) - \mu_T(x)) &= (N_1 \dots N_{k+1}) \frac{\PP(Q_R = 1)}{\#(\DQ_{k+1}(S))} - (N_1 \dots N_k) \frac{1}{\#(\DQ_k(T)}\\
        &= (N_1 \dots N_{k+1}) \frac{(M_{k+1}/N_{k+1})}{M_{k+1} \cdot \#(\DQ_k(T))} - (N_1 \dots N_k) \frac{1}{\#(\DQ_k(T))} = 0.
    \end{align*}
    %
    Thus
    %
    \[ \EE[A_R(m)] = \int_R \EE \left[ \mu_S(x) - \mu_T(x) \right] e^{-2 \pi i m \cdot x}\; dx = 0. \]
    %
    Notice that for each fixed $m$, the family $\{ A_R(m) \}$ are a family of $M_{k+1} \cdot \DQ_k(T)$ independant random variables, and
    %
    \[ |A_R(m)| \leq \int_R |\mu_S(x) - \mu_T(x)|\; dx = \frac{2}{M_{k+1}}. \]
    %
    We note that
    %
    \[ \widehat{\mu_S}(m) - \widehat{\mu_T}(m) = \sum_{R \in \DR_{k+1}(T)} A_R(m). \]
    %
    Applying Hoeffding's inequality, we conclude that for each $t > 0$,
    %
    \[ \PP \left( \left| \widehat{\mu_S}(m) - \widehat{\mu_T}(m) \right| \geq t \right) \leq 2 \exp \left( \frac{- M_{k+1}}{8 \cdot \#(\DQ_k(T))} \cdot t^2 \right). \]
    %
    In particular,
    %
    \begin{equation} \label{equation68994812893189}
        \PP \left( \left| \widehat{\mu_S}(m) - \widehat{\mu_T}(m) \right| \geq N_{k+1}^{-(1 - \varepsilon_2) \frac{n-s}{2n}} \right) \leq 2 \exp \left( \frac{-M_{k+1} N_{k+1}^{-(1 - \varepsilon_2) \frac{n-s}{n}}}{8 \cdot \#(\DQ_k(T))} \right).
    \end{equation}
    %
    Applying a union bound with \eqref{equation68994812893189}, we conclude that if
    %
    \[ I = \{ m \in \ZZ : |m| \leq (N_1 \dots N_k)^{A} \}, \]
    %
    then
    %
    \[ \PP \left( \left\| \widehat{\mu_S} - \widehat{\mu_T} \right\|_{L^\infty(I)} \geq N_{k+1}^{- \left(1 - \varepsilon_2 \right) \frac{n-s}{2n}} \right) \leq 2 (N_1 \dots N_k)^A \exp \left( \frac{-M_{k+1} N_{k+1}^{-(1 - \varepsilon_2) \frac{n-s}{n}}}{8 \cdot \#(\DQ_k(T))} \right). \]
    %
    But if $|m| \geq (N_1 \dots N_k)^A$, then
    %
    \[ \PP \left( \widehat{\mu_S}(m) \right) \]



    For each cube $R \in \DR_{k+1}(T)$, and for each $m$, let
    %
    \[ A_R(m) = e^{\frac{-2 \pi i m \cdot a(Q_R)}{N_1 \dots N_{k+1}}} - \frac{1}{N_{k+1}^d} \sum_{k_1, \dots, k_d = 0}^N e^{\frac{-2 \pi i m \cdot [N_{k+1} a(Q) + k]}{N_1 \dots N_{k+1}}}. \]
    %
    Then $\EE[A_R(m)] = 0$, $|A_R(m)| \leq 2$ for each $m$, and
    %
    \[ \widehat{\nu_S}(m) - \widehat{\eta_{k+1}}(m) \widehat{\nu_T}(m) = \frac{1}{\#(\DR_{k+1}(T))} \sum_{R \in \DR_{k+1}(T)} A_R(m). \]
    %
    Now fix a particular value of $m$. Since the random variables $A_R(m)$ are independant from one another as $R$ ranges over $\DR_{k+1}(T)$, we can apply Hoeffding's inequality to conclude that for each $t > 0$,
    %
    \[ \PP \left( |\widehat{\nu_S}(m) - \widehat{\eta_{k+1}}(m) \widehat{\nu_T}(m)| \geq t \right) \leq e^{-\#(\DR_{k+1}(T)) t^2/2} = e^{-\#(\DQ_k(T)) M_{k+1}^d t^2/2}. \]
    %
    In particular,
    %
    \[ \PP \left( |\widehat{\nu_S}(m) - \widehat{\eta_{k+1}}(m) \widehat{\nu_T}(m)| \geq M_{k+1}^{-d/2 - \varepsilon} \right) \leq \exp(- \#(\DQ_k(T)) M_{k+1}^\varepsilon / 2 ). \]
    %
    The function $\widehat{\nu_S} - \widehat{\eta_{k+1}}$ is $N_1 \dots N_{k+1}$ periodic. Thus, to uniformly bound $\widehat{\nu_S}(m) - \widehat{\eta_{k+1}}(m)$, we need only bound the function over $(N_1 \dots N_{k+1})^d$ values. Applying a union bound with \eqref{equation5523786128439}, we find that
    %
    \begin{equation} \label{equation81298398120412}
    \begin{split}
        \PP \left( \| \widehat{\nu_S} - \widehat{\eta_{k+1}} \widehat{\nu_T} \|_{L^\infty(\ZZ^d)} \geq M_{k+1}^{- d/2 - \varepsilon} \right) &\leq (N_1 \dots N_{k+1})^d \exp \left( - \#(\DQ_k(T)) M_{k+1}^\varepsilon / 2 \right)\\
        &\leq 2^{\lceil \frac{4d^2n}{\varepsilon(dn - s)} \rceil} \left\lceil \frac{4d^2n}{\varepsilon(dn-s)} \right\rceil! \frac{(N_1 \dots N_{k+1})^d}{M_{k+1}^{4d^2n/(dn - s)}}\\
        &\leq 2^{\lceil 8d^2n/(dn - s) \rceil} \left\lceil \frac{4d^2n}{\varepsilon(dn-s)} \right\rceil! \frac{(N_1 \dots N_k)^d}{N_{k+1}^d}\\
        &\leq \left( \left\lceil \frac{4d^2n}{\varepsilon(dn-s)} \right\rceil! \right)^3 \frac{(N_1 \dots N_k)^d}{N_{k+1}^d}.
    \end{split}
    \end{equation}
    %
    Note that
    %
    \begin{equation} \label{equation8998724714871}
        M_{k+1}^{-d/2 - \varepsilon} \lesssim N_{k+1}^{-(1 - 2\varepsilon/d) \frac{dn - s - 2\varepsilon}{2n}} \leq N_{k+1}^{-(1 - \varepsilon) \frac{dn - s - 2\varepsilon}{2n}}.
    \end{equation}
    %
    If we set $A(d,n,s) = (4d^2n/(dn - s))$, then \eqref{equation77871247817841278}, \eqref{equation81298398120412}, and \eqref{equation8998724714871} allow us to conclude that
    %
    \begin{equation} \label{equation1241751}
        \PP \left( \| \widehat{\nu_S} - \widehat{\eta_{k+1}} \widehat{\nu_T} \|_{L^\infty(\ZZ^d)} \geq N_{k+1}^{-(1 - \varepsilon)} \right) \leq 1/3.
    \end{equation}
    %
    Taking a union bound over \eqref{fourierdim2} and \eqref{equation1241751}, we find there is a non-zero probability that a set $S$ exists satisfying Property (A) and (B).
    \begin{comment}

    %
    Then for each $m \in \mathbf{Z}^d$,
    %
    \[ \widehat{\nu_S}(m) - \widehat{\eta_{k+1}}(m) \widehat{\nu_T}(m) = \sum_{Q \in \DQ_{k+1}(T)} A_Q e^{-\frac{2 \pi i m \cdot a(Q)}{N_1 \dots N_{k+1}}}. \]
    %
    We calculate that for each $Q \in \DQ_{k+1}(T)$,
    %
    \begin{align*}
        \EE[A_Q|\#(\DQ_k(S))] &= \frac{\PP (X_Q = 1 | \#(\DQ_k(S)))}{\#(\DQ_{k+1}(S))} - \frac{1}{N_{k+1}^d \#(\DQ_{k+1}(T))}\\
        &= \frac{\#(\DQ_{k+1}(S)) / N_{k+1}^d \#(\DQ_{k+1}(T))}{\#(\DQ_{k+1}(S))} - \frac{1}{N_{k+1}^d \#(\DQ_{k+1}(T))} = 0.
    \end{align*}
    %
    In particular, $\EE[A_Q] = 0$. Now for each $q \geq 1$,
    %
    \begin{align*}
        \EE[A_Q^q|\DQ_{k+1}(S)] &= \frac{\PP(X_Q = 1 | \DQ_{k+1}(S))}{\#(\DQ_{k+1}(S))^q} &= \frac{1}{\#(\DQ_{k+1}(T)) \cdot \#(\DQ_{k+1}(S))^{q-1}}.
    \end{align*}
    %
    Thus
    %
    \begin{align*}
        \EE[A_Q^q] &= \frac{1}{\#(\DQ_{k+1}(T))} \EE \left[ \frac{1}{\#(\DQ_{k+1}(S))^{q-1}} \right]\\
        &\leq s
    \end{align*}


    Now fix $m \in \{ -N_1 \dots N_{k+1}, N_1 \dots N_{k+1} \}^d$.



    We can then apply Hoeffding's inequality to conclude that for each $t > 0$,
    %
    \begin{align*}
        \PP \left( |\widehat{\nu_S}(m)| \geq \frac{t p \cdot \#(\DQ_{k+1}(T))^{1/2}}{\#(\DQ_{k+1}(S))} \right) &= \PP \left( \sum \left| \#(\DQ_{k+1}(S)) A_Q e^{-\frac{2 \pi i m \cdot a(Q)}{N_1 \dots N_{k+1}}} \right| \geq A \right) \\
        &\leq 2 \cdot \exp \left( - 2 t^2) \right)
    \end{align*}
    %
    If we now take a union bound over all $m \in \{ -N_1 \dots N_{k+1}, \dots, N_1 \dots N_{k+1} \}^d$, we can guarantee that
    %
    \begin{equation} \label{fourierdim3}
        \mathbf{P} \left( |\widehat{f}(m)| \leq \log(N_{k+1})/S\ \text{for all $m \in \{ -N, \dots, N\}^d$} \right) \geq 1 - 2^{d+1}/N^{c \log N - d}.
    \end{equation}
    %
    Since $\widehat{\nu_S}$ is $N_1 \dots N_{k+1}$ periodic, this means we can control all integer values of $\widehat{\nu_S}$ with high probability.

    Combining \eqref{fourierdim1}, \eqref{fourierdim2}, and \eqref{fourierdim3}, we conclude that there exists a constant $C$ such that with probability at least
    %
    \[ 1 - 2 \exp \left( \frac{-N^{d-s/n}}{A^{1/n} (\log N)^{1/n}} \right) - 1/\log N - \frac{2^{d+1}}{N^{c \log N - d}} \geq 1 - C / \log N, \]
    %
    the set $X$ avoids $K$, and for all $m \in \{ -N, \dots, N \}^d$,
    %
    \[ |\widehat{f}(m)| \leq \frac{C (\log N)^{1-1/n}}{N^{d-s/n}}. \qedhere \]



    Let
    %
    \[ p = \frac{1}{(N_{k+1}^{s + \varepsilon} \log(N_{k+1}))^{1/n}}, \]
    %
    and let $\{ X_Q \}$ be a family of independent and identically distributed $\{ 0, 1 \}$ valued Bernoulli random variables, for each $Q \in \DQ_{k+1}(T)$, such that $\PP(X_Q = 1) = p$. Then, define $S = \bigcup \{ Q : X_Q = 1 \}$. Then $\#(\DQ_{k+1}(S)) = \sum_Q X_Q$ is the sum of $\#(\DQ_k(T)) \cdot N_{k+1}^d$ independant and identically distributed random variables, and so Chernoff's inequality implies that
    %
    \[ \PP \left( \left| \#(\DQ_{k+1}(S)) - p \cdot \DQ_k(T) \cdot N_{k+1}^d \right| \leq \frac{p \cdot \# \DQ_k(T) \cdot N_{k+1}^d}{2} \right) \leq 10 e^{- p \DQ_k(T) N_{k+1}^d}. \]
    %
    Substituting in the value of $p$, we conclude
    %
    \begin{equation} \label{fourierdim1}
    \begin{split}
        \PP& \left( \left| \#(\DQ_{k+1}(S)) - \frac{\#(\DQ_k(T)) N_{k+1}^{\frac{dn - (s + \varepsilon)}{n}}}{\log(N_{k+1})^{1/n}} \right| \leq \frac{\#(\DQ_k(T)) N_{k+1}^{\frac{dn - (s + \varepsilon)}{n}}}{2 \log(N_{k+1})^{1/n}} \right)\\
        &\ \ \ \ \ \ \ \ \ \ \leq 10 \exp \left( \frac{- N_{k+1}^{\frac{dn - (s + \varepsilon)}{n}} \cdot \DQ_k(T)}{\log(N_{k+1})^{1/n}} \right)\\
        &\ \ \ \ \ \ \ \ \ \ \leq 10 \exp \left( \frac{-N_{k+1}^{\frac{dn - (s + \varepsilon)}{n}}}{\log(N_{k+1})^{1/n}} \right)
    \end{split}
    \end{equation}
    %
    Thus $S$ is the union of a large number of cubes, with high probability.

    Without loss of generality, removing cubes from $B$ if necessary, we may assume that every cube $Q_1 \times \dots \times Q_n \in \DQ_{k+1}(B)$, the values $Q_1, \dots, Q_n$ are distinct. Just as in Lemma

    In particular, given any such cube, we have
    %
    \[ \mathbf{P}(Q_1 \times \dots Q_n \subset S) = \mathbf{P}(X_{Q_1} = 1, \dots, X_{Q_n} = 1) = p^n. \]
    %
    Thus
    %
    \[ \mathbf{E}(\#(\DQ_{k+1}(B) \cap \DQ_{k+1}(S^n))) \leq N_{k+1}^{s+\varepsilon} p^n = \log(N_{k+1})^{-1}. \]
    %
    Markov's inequality implies
    %
    \begin{equation} \label{fourierdim2}
    \begin{split}
        \mathbf{P}(\DQ_{k+1}(B) \cap \DQ_{k+1}(S^n) \neq \emptyset) &= \mathbf{P}(\# (\DQ_{k+1}(B) \cap \DQ_{k+1}(S^n)) \geq 1)\\
        &\leq \log(N_{k+1})^{-1}.
    \end{split}
    \end{equation}
    %
    Thus $\DQ_{k+1}(S^n)$ is disjoint from $\DQ_{k+1}(B)$ with high probability.

    Now we analyze the Fourier transform of the measure $\nu_S$. For each cube $Q \in \DR_{k+1}(T)$, we can define
    %
    \[ A_Q = \nu_S(a(Q)) - (\eta_{k+1} * \nu_T)(a(Q)) = \begin{cases} \frac{1}{\#(\DQ_{k+1}(S))} - \frac{1}{N_{k+1}^d \#(\DQ_k(T))} &: X_Q = 1, \\ - \frac{1}{ N_{k+1}^d \#(\DQ_k(T))} &: X_Q = 0. \end{cases} \]
    %
    Then for each $m \in \mathbf{Z}^d$,
    %
    \[ \widehat{\nu_S}(m) - \widehat{\eta_{k+1}}(m) \widehat{\nu_T}(m) = \sum_{Q \in \DQ_{k+1}(T)} A_Q e^{-\frac{2 \pi i m \cdot a(Q)}{N_1 \dots N_{k+1}}}. \]
    %
    We calculate that for each $Q \in \DQ_{k+1}(T)$,
    %
    \begin{align*}
        \EE[A_Q|\#(\DQ_k(S))] &= \frac{\PP (X_Q = 1 | \#(\DQ_k(S)))}{\#(\DQ_{k+1}(S))} - \frac{1}{N_{k+1}^d \#(\DQ_{k+1}(T))}\\
        &= \frac{\#(\DQ_{k+1}(S)) / N_{k+1}^d \#(\DQ_{k+1}(T))}{\#(\DQ_{k+1}(S))} - \frac{1}{N_{k+1}^d \#(\DQ_{k+1}(T))} = 0.
    \end{align*}
    %
    In particular, $\EE[A_Q] = 0$. Now for each $q \geq 1$,
    %
    \begin{align*}
        \EE[A_Q^q|\DQ_{k+1}(S)] &= \frac{\PP(X_Q = 1 | \DQ_{k+1}(S))}{\#(\DQ_{k+1}(S))^q} &= \frac{1}{\#(\DQ_{k+1}(T)) \cdot \#(\DQ_{k+1}(S))^{q-1}}.
    \end{align*}
    %
    Thus
    %
    \begin{align*}
        \EE[A_Q^q] &= \frac{1}{\#(\DQ_{k+1}(T))} \EE \left[ \frac{1}{\#(\DQ_{k+1}(S))^{q-1}} \right]\\
        &\leq s
    \end{align*}


    Now fix $m \in \{ -N_1 \dots N_{k+1}, N_1 \dots N_{k+1} \}^d$.



    We can then apply Hoeffding's inequality to conclude that for each $t > 0$,
    %
    \begin{align*}
        \PP \left( |\widehat{\nu_S}(m)| \geq \frac{t p \cdot \#(\DQ_{k+1}(T))^{1/2}}{\#(\DQ_{k+1}(S))} \right) &= \PP \left( \sum \left| \#(\DQ_{k+1}(S)) A_Q e^{-\frac{2 \pi i m \cdot a(Q)}{N_1 \dots N_{k+1}}} \right| \geq A \right) \\
        &\leq 2 \cdot \exp \left( - 2 t^2) \right)
    \end{align*}
    %
    If we now take a union bound over all $m \in \{ -N_1 \dots N_{k+1}, \dots, N_1 \dots N_{k+1} \}^d$, we can guarantee that
    %
    \begin{equation} \label{fourierdim3}
        \mathbf{P} \left( |\widehat{f}(m)| \leq \log(N_{k+1})/S\ \text{for all $m \in \{ -N, \dots, N\}^d$} \right) \geq 1 - 2^{d+1}/N^{c \log N - d}.
    \end{equation}
    %
    Since $\widehat{\nu_S}$ is $N_1 \dots N_{k+1}$ periodic, this means we can control all integer values of $\widehat{\nu_S}$ with high probability.

    Combining \eqref{fourierdim1}, \eqref{fourierdim2}, and \eqref{fourierdim3}, we conclude that there exists a constant $C$ such that with probability at least
    %
    \[ 1 - 2 \exp \left( \frac{-N^{d-s/n}}{A^{1/n} (\log N)^{1/n}} \right) - 1/\log N - \frac{2^{d+1}}{N^{c \log N - d}} \geq 1 - C / \log N, \]
    %
    the set $X$ avoids $K$, and for all $m \in \{ -N, \dots, N \}^d$,
    %
    \[ |\widehat{f}(m)| \leq \frac{C (\log N)^{1-1/n}}{N^{d-s/n}}. \qedhere \]
%\end{comment}
\end{proof}
\end{comment}

The construction of the set $X$ follows essentially the construction of the configuration avoiding set in Chapter \ref{ch:RoughSets}. We choose a decreasing sequence of parameters $\{ \varepsilon_k \}$ such that $\varepsilon_k < (n-s)/4$ for each $k$, as well as parameters $\{ N_k \}$ such that
%
\[ N_k \geq 3^{1/\varepsilon_k}, \]
%
\[ N_k \geq \exp \left( \left( \frac{4n}{n-s} \right)^4 N_1 \dots N_{k-1} \right), \]
%
\[ N_k \geq (1/\varepsilon_k)^{1/\varepsilon_k}, \]
%
\begin{equation} \label{equation13895891489132}
    N_k \geq (N_1 \dots N_{k-1})^{2/\varepsilon_k}.
\end{equation}
%
The choice of $N_k$ is also chosen sufficiently large that we can find a $\DQ_k$ discretized set $B_k$ such that
%
\[ \#(\DQ_k(B_k)) \leq (N_1 \dots N_k)^{s + \varepsilon_k/2} \leq N_k^{s + \varepsilon_k}, \]
%
and such that the collection $\{ B_k \}$ forms a strong cover of the configuration $\C$. We then choose a sequence $\{ M_k \}$  such that for each $k$,
%
\[ M_k \leq N_k^{\frac{n-s-2\varepsilon_k}{n}} \leq 2 M_{k+1}. \]
%
Just as was done in Chapter \ref{ch:RoughSets}, this choice of parameters enables us to find a nested family of sets $\{ X_k \}$, obtained by setting $X_0 = [0,1]$, and letting $X_{k+1}$ be obtained from $X_k$ by applying Lemma \ref{discreteFourierBuildingBlock} with $\varepsilon = \varepsilon_{k+1}$, $T = X_k$, and $B = B_{k+1}$. We set $X = \bigcap X_k$. Since Property (A) of Lemma \eqref{discreteFourierBuildingBlock} is true at each step of the process, this is sufficient to guarantee that $X$ avoids the configuration $\C$. The remainder of this section is devoted to showing that Property (B) of Lemma \ref{discreteFourierBuildingBlock} is sufficient to obtain the Fourier dimension bound on $X$ guaranteed by Theorem \ref{FourierTheorem}.

Let $\nu_k = \nu_{X_k}$ for each $k$. Property (B) of Lemma \ref{discreteFourierBuildingBlock} implies that for each $k$,
%
\begin{equation} \label{equation77770123091293120}
    \left\| \widehat{\nu_{k+1}} - \widehat{\eta_{k+1}} \widehat{\nu_k} \right\|_{L^\infty(\ZZ)} \leq A(n,s) \cdot (N_1 \dots N_{k+1})^{- \frac{n-s}{2n} + 2\varepsilon_{k+1}}.
\end{equation}
%
We shall form a sequence of measures $\{ \mu_k \}$ by convolving the measures $\{ \nu_k \}$ with an appropriate family of mollifiers, which will be sufficient to obtain the required asymptotic bound.

\begin{lemma}
    There exists a sequence of probability measures $\{ \mu_k \}$, with $\mu_k$ supported on $X_k$ for each $k$, such that for each $\varepsilon > 0$,
    %
    \[ \sup_{k > 0} \sup_{m \in \ZZ} |m|^{\frac{n-s}{2n} - \varepsilon} |\widehat{\mu_k}(m)| < \infty. \]
\end{lemma}
\begin{proof}
    For each $k$, let
    %
    \[ \psi_k(x) = (N_1 \dots N_k) \cdot \mathbf{I}_{\left[ 0, \frac{1}{N_1 \dots N_k} \right]}. \]
    %
    Then it is easy to calculate that
    %
    \begin{equation} \label{equation901418294891481792}
        |\widehat{\psi_k}(m)| \lesssim \min \left( 1, \frac{N_1 \dots N_k}{|m|} \right).
    \end{equation}
    %
    Note that the measures $\mu_k = \nu_k * \psi_k$ are still supported on $X_k$, and
    %
    \[ \widehat{\mu_k}(\xi) = \widehat{\nu_k}(\xi) \widehat{\psi_k}(\xi). \]
    %
    Also note that $\psi_k = \psi_{k+1} * \eta_{k+1}$. If $\varepsilon > 0$, then we can apply \eqref{equation77770123091293120} with \eqref{equation901418294891481792} to conclude
    %
    \begin{equation} \label{equation6892489214781278}
    \begin{split}
        &|\widehat{\mu_{k+1}}(m) - \widehat{\mu_k}(m)|\\
        &\ \ \ \ = |\widehat{\psi_{k+1}}(m)| |\widehat{\nu_k}(m) - \widehat{\eta_{k+1}}(m) \widehat{\nu_k}(m)|\\
        &\ \ \ \ \lesssim \min \left( 1, \frac{N_1 \dots N_{k+1}}{|m|} \right) (N_1 \dots N_{k+1})^{-\frac{n-s}{2n} + 2\varepsilon_{k+1}}.\\
        &\ \ \ \ = \min \left( \frac{|m|^{\frac{n-s}{2n} - \varepsilon}}{(N_1 \dots N_{k+1})^{\frac{n-s}{2n} - 2\varepsilon_{k+1}}}, \frac{(N_1 \dots N_k)^{1 - \frac{n-s}{2n} + 2\varepsilon_{k+1}}}{|m|^{1 - \frac{n-s}{2n} + \varepsilon}} \right) |m|^{- \frac{n-s}{2n} + \varepsilon}.
    \end{split}
    \end{equation}
    %
    The minima is maximized when $|m| = N_1 \dots N_{k+1}$, which gives
    %
    \[ \min \left( \frac{|m|^{\frac{n-s}{2n} - \varepsilon}}{(N_1 \dots N_{k+1})^{\frac{n-s}{2n} - 2\varepsilon_{k+1}}}, \frac{(N_1 \dots N_k)^{1 - \frac{n-s}{2n} + 2\varepsilon_{k+1}}}{|m|^{1 - \frac{n-s}{2n} + \varepsilon}} \right) \leq (N_1 \dots N_{k+1})^{2\varepsilon_{k+1} - \varepsilon}. \]
    %
    Thus, for all $k$, for all $m \in \ZZ$, and for all $\varepsilon > 0$,
    %
    \begin{equation} \label{equation11020404120}
    \begin{split}
        |\widehat{\mu_{k+1}}(m) - \widehat{\mu_k}(m)| \lesssim \frac{(N_1 \dots N_{k+1})^{2\varepsilon_{k+1} - \varepsilon}}{|m|^{\frac{n-s}{2n} - \varepsilon}}.
    \end{split}
    \end{equation}
    %
    For each $k$, let
    %
    \[ A_k = \sup_{m \in \ZZ} |\widehat{\mu_k}(m)| |m|^{\frac{n-s}{2n} - \varepsilon}. \]
    %
    Then \eqref{equation11020404120} implies that
    %
    \[ A_{k+1} = A_k + O \left( (N_1 \dots N_{k+1})^{2\varepsilon_{k+1} - \varepsilon} \right). \]
    %
    Thus for all $k > 0$,
    %
    \[ A_k = O \left( \sum_{k = 1}^\infty (N_1 \dots N_k)^{2\varepsilon_{k+1} - \varepsilon} \right). \]
    %
    Provided the sum on the right hand side converges for each $\varepsilon > 0$, this gives a uniform bound of $A_k$ in $k$ for each $\varepsilon > 0$, completing the proof. But for suitably large $k$, depending on $\varepsilon$, it is eventually true that $\varepsilon_{k+1} \leq \varepsilon/8$, and so
    %
    \begin{align*}
        A_k &= O_\varepsilon(1) + \sum_{k = 1}^\infty (N_1 \dots N_k)^{-\varepsilon/4} = O_\varepsilon(1) + \sum_{k = 1}^\infty 2^{-k\varepsilon/4} = O_\varepsilon(1). \qedhere
    \end{align*}
\end{proof}

Just as for the sequence of measures in Theorem \ref{massdistributionprinciplelem}, the sequence $\{ \mu_k \}$ is a Cauchy sequence of probability measures, and therefore converges weakly to some measure $\mu$. Because for each $k$, $\mu_k$ is supported on $X_k$, $\mu$ is supported on $\bigcap X_k = X$. Furthermore, the Fourier transform of each $\mu_k$ converges pointwise to the Fourier transform of $\mu$. Thus we find that for each $\varepsilon > 0$,
%
\[ \sup_{m \in \ZZ} |m|^{\frac{n-s}{2n} - \varepsilon} |\widehat{\mu}(m)| \leq \sup_{k > 0} \sup_{m \in \ZZ} |m|^{\frac{n-s}{2n} - \varepsilon} |\widehat{\mu_k}(m)| < \infty. \]
%
Combined with Lemma \ref{discretefouriermeasures}, this implies $X$ has Fourier dimension $(nd - s)/n$.

\endinput

% 3. Notes
% 4. Footnotes

% 5. Bibliography
\begin{singlespace}
\raggedright
\bibliographystyle{abbrvnat}
\bibliography{biblio}
\end{singlespace}

%\appendix
% 6. Appendices (including copies of all required UBC Research
% Ethics Board's Certificates of Approval)
% \include{reb-coa}	% pdfpages is useful here
%\include{appendix}

\nocite{*}
\backmatter
% 7. Index
% See the makeindex package: the following page provides a quick overview
% <http://www.image.ufl.edu/help/latex/latex_indexes.shtml>

\end{document}
