%% The following is a directive for TeXShop to indicate the main file
%%!TEX root = diss.tex

%% https://www.grad.ubc.ca/current-students/dissertation-thesis-preparation/preliminary-pages
%% 
%% LAY SUMMARY Effective May 2017, all theses and dissertations must
%% include a lay summary.  The lay or public summary explains the key
%% goals and contributions of the research/scholarly work in terms that
%% can be understood by the general public. It must not exceed 150
%% words in length.

%% The lay or public summary explains the key goals and contributions of
%% the research\slash{}scholarly work in terms that can be understood by the
%% general public. It must not exceed 150 words in length.

\chapter{Lay Summary}

%Imagine a patch of carpet under a microscope. Zooming in, we see the carpet is really a collection of twines tied together. On closer inspection, those twines break off into smaller tufts of fabric. The carpet is rough at all scales. Shapes like circles or polygons do not have complexity at arbitrarily small scales. To model the roughness of a carpet, you'd need a fractal: a shape with complex structure at all small scales. Such models are often useful in small-scale physics or computer graphics.

%It is easy to construct polyhedra with geometric properties at a single scale. But it is non-trivial to construct fractals with properties at many scales. For instance, how do we construct a fractal which intersects any line in at most two points? In this thesis, we begin with an exposition on previous constructions in the literature, and then provide new construction techniques utilizing randomness.

In geometry, we are often interested in whether shapes with certain properties exist. For instance, given three points, can one find a circle connecting these three points? Most questions of this type involving classical shapes have been answered. But in the 20th century, mathematicians began discussing a new class of geometric shapes known as fractals: shapes with complex structure at all scales. Popular examples include the Koch snowflake, or Sierpinski triangle. Such shapes often occur in small scale physics and computer graphics.

Many open questions remain as to whether one can construct fractals with various properties. For instance, can one construct a large fractal so that one cannot form an isosceles triangle from three points lying on the fractal. This thesis focuses on fractal construction problems. We begin with an exposition on previous constructions in the literature, and then provide new construction techniques utilizing randomness.