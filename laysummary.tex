%% The following is a directive for TeXShop to indicate the main file
%%!TEX root = diss.tex

%% https://www.grad.ubc.ca/current-students/dissertation-thesis-preparation/preliminary-pages
%% 
%% LAY SUMMARY Effective May 2017, all theses and dissertations must
%% include a lay summary.  The lay or public summary explains the key
%% goals and contributions of the research/scholarly work in terms that
%% can be understood by the general public. It must not exceed 150
%% words in length.

%% The lay or public summary explains the key goals and contributions of
%% the research\slash{}scholarly work in terms that can be understood by the
%% general public. It must not exceed 150 words in length.

\chapter{Lay Summary}

Imagine looking at a patch of carpet with a microscope. Zooming in, we see the carpet is really a collection of twines tied together. On closer inspection, those twines break off into smaller twines. The carpet is rough at all scales. If you want to model the carpet mathematically such that all small features are accounted for, normal shapes like polyhedra wouldn't suffice. Instead, you need a \emph{fractal}: shapes with structure at small scales. Such models are often useful in applied mathematics and small-scale physics.

It is often easy to construct polyhedra with geometric properties at a single scale, but non-trivial to describe how to construct fractals with properties at many scales, such as not containing the vertices of an isoceles triangles, or not containing any three collinear points. In this thesis, we provide an exposition of previous work in this topic, as well as new random fractal avoidance techniques.