%% The following is a directive for TeXShop to indicate the main file
%%!TEX root = diss.tex

%% https://www.grad.ubc.ca/current-students/dissertation-thesis-preparation/preliminary-pages
%% 
%% LAY SUMMARY Effective May 2017, all theses and dissertations must
%% include a lay summary.  The lay or public summary explains the key
%% goals and contributions of the research/scholarly work in terms that
%% can be understood by the general public. It must not exceed 150
%% words in length.

%% The lay or public summary explains the key goals and contributions of
%% the research\slash{}scholarly work in terms that can be understood by the
%% general public. It must not exceed 150 words in length.

\chapter{Lay Summary}

Imagine looking at a patch of carpet with a microscope. Zooming in, we see the carpet is really a collection of twines tied together. On closer inspection, those twines break off into smaller twines. The carpet is rough at all scales. Classical geometric shapes like polyhedra do not model objects with complexity at arbitrarily small scales. Instead, you need a fractal: shapes with structure at small scales. Such models are often useful in applied mathematics and small-scale physics.

It is easy to construct polyhedra with geometric properties at a single scale. But it is non-trivial to construct fractals with properties at many scales. For instance, how do we construct a fractal which intersects any line in at most two points? In this thesis, we begin with an exposition on previous construction, and then provide new random fractal construction techniques.