%% The following is a directive for TeXShop to indicate the main file
%%!TEX root = diss.tex

\chapter{Extensions to Low Rank Configurations}
\label{ch:LowRank}



\section{Boosting the Dimension of Pattern Avoiding Sets by Low Rank Coordinate Changes}

We now consider finding subsets of $[0,1]$ avoiding solutions to the equation $y = f(Tx)$, where $T$ is a rank $k$ linear transformation with integer coefficients with respect to standard coordinates, and $f$ is real-valued and Lipschitz continuous. Fix a constant $A$ bounding the operator norm of $T$, in the sense that $|Tx| \leq A|x|$ for all $x \in \mathbf{R}^n$, and a constant $B$ such that $|f(x+y) - f(x)| \leq B|y|$ for all $x$ and $y$ for which the equation makes sense (if $f$ is $C^1$, this is equivalent to a bound $\| \nabla f \|_\infty \leq B$). Consider sets $J_0, J_1, \dots, J_n, \subset [0,1]$, which are unions of intervals of length $1/M$, with startpoints lying on integer multiples of $1/M$. The next theorem works as a `building block lemma' used in our algorithm for constructing a set avoiding solutions to the equation with Hausdorff dimension $k$ and full Minkowski dimension.

\begin{theorem}
    For infinitely many integers $N$, there exists $S_i \subset J_i$ avoiding solutions to $y = f(Tx)$ with $y \in S_0$ and $x_n \in S_n$, such that
    %
    \begin{itemize}
        \item For $n \neq 0$, if we decompose each $J_i$ into length $1/N$ consecutive intervals, $S_i$ contains an initial portion $\Omega(1/N^k)$ of each length $1/N$ interval. This part of the decomposition gives the Hausdorff dimension $1/k$ bound for the set we will construct.

        \item If we decompose $J_i$ into length $1/N$ intervals, and then subdivide these intervals into length $\Omega(1/N^k)$ intervals, then $S_0$ contains a subcollection of these $1/N^k$ intervals which contains a total length $\Omega(1/N)$ of a fraction $1 - 1/M$ of the length $1/N$ intervals. This property gives that our resultant set will have full Minkowski dimension.
    \end{itemize}
    %
    The implicit constants in these bounds depend only on $A$, $B$, $n$, and $k$.
\end{theorem}
\begin{proof}
Split each interval of $J_a$ into length $1/N$ intervals, and then set
%
\[ \mathbf{A} = \{ x : x_a\ \text{is a startpoint of a $1/N$ interval in $J_a$} \} \]
%
Since the startpoints of the intervals are integer multiples of $1/N$, $T(\mathbf{A})$ is contained with a rank $k$ sublattice of $(\mathbf{Z}/N)^m$. The operator norm also guarantees $T(\mathbf{A})$ is contained within the ball $B_A$ of radius $A$ in $\mathbf{R}^m$. Because of the lattice structure of the image, $| x - y | \gtrsim_n 1/N$ for each distinct pair $x,y \in T(\mathbf{A})$. For any $R$, we can cover $\Sigma \cap B_A$ by $O_{n,k}((A/R)^k)$ balls of radius $R$. If $R \gtrsim_n 1/N$, then each ball can contain only a single element of $T(\mathbf{A})$, so we conclude that $|T(\mathbf{A})| \lesssim_{n,k} (AN)^k$. If we define the set of `bad points' to be
%
\[ \mathbf{B} = \{ y \in [0,1] : \text{there is $x \in \mathbf{A}$ such that $y = f(T(x))$} \} \]
%
Then
%
\[ |\mathbf{B}| = |f(T(\mathbf{A}))| \leq |T(\mathbf{A})| = O_{A,n,k}(N^k) \]
%
For simplicity, we now introduce an integer constant $C_0 = C_0(A,n,k,M)$ such that $|\mathbf{B}| \leq (C_0/M^2) N^k$. We now split each length $1/M$ interval in $J_0$ into length $1/N$ intervals, and filter out those intervals containing more than $C_0N^{k-1}$ elements of $\mathbf{B}$. Because of the cardinality bound we have on $\mathbf{B}$ there can be at most $N/M^2$ such intervals, so we discard at most a fraction $1/M$ of any particular length $1/M$ interval in $J_0$. If we now dissect the remaining intervals into $4C_0N^{k-1}$ intervals of length $1/4C_0N^k$, and discard any intervals containing an element of $\mathbf{B}$, or adjacent to such an interval, then the remaining such intervals $I$ satisfy $d(I,\mathbf{B}) \geq 1/4C_0N^k \gtrsim_{A,n,M,k}(1/N^k)$, and because of our bound on the number of elements of $\mathbf{B}$ in these intervals, there are at least $C_0N^{k-1}$ intervals remaining, with total length exceeding $C_0N^{k-1}/4C_0N^k = \Omega(1/N)$. If $f$ is $C^1$ with $\| \nabla f \|_\infty \leq B$, or more generally, if $f$ is Lipschitz continuous of magnitude $B$, then
%
\[ | f(Tx) - f(Tx')| \leq AB |x - x'| \]
%
and so we may choose $S_i \subset J_i$ by thickening each startpoint $x \in J_i$ to a length $O(1/N^k)$ interval while still avoiding solutions to the equation $y = f(T(x))$.
\end{proof}

\begin{remark}
    If $T$ is a rank $k$ linear transformation with rational coefficients, then there is some number $a$ such that $aT$ has integer coefficients, and then the equation $y = f(Tx)$ is the same as the equation $y = f_0((aT)(x))$, where $f_0(x) = f(x)/a$. Since $f_0$ is also Lipschitz continuous, we conclude that we still get the dimension $1/k$ bound if $T$ has rational rather than integral coefficients. More generally, this trick shows the result applies unperturbed if all coefficients of $T$ are integer multiples of some fixed real number. More generally, by varying the lengths of our length $1/N$ decomposition by a constant amount, we can further generalize this to the case where each column of $T$ are integers multiples of some fixed real number.
\end{remark}

\begin{remark}
    To form $\mathbf{A}$, we take startpoints lying at equal spaced $1/N$ points. However, by instead taking startpoints at varying points in the length $1/N$ intervals, we might be able to make points cluster more than in the original algorithm. Maybe the probabalistic method would be able to guarantee the existence of a choice of startpoints whose images are tightly clustered together.
\end{remark}

\begin{remark}
    Since the condition $y = f(Tx)$ automatically assumes a kind of `non-vanishing derivative' condition on our solutions, we do not need to assume the regularity of $f$, and so the theorem extends naturally to a more general class of functions than Rob's result, i.e. the Lipschitz continuous functions.
\end{remark}

Using essentially the same approach as the last argument shows that we can avoid solutions to $y = f(Tx)$, where $y$ and $x$ are now vectors in some $\mathbf{R}^m$, and $T$ has rank $k$. If we consider unions of $1/M$ cubes $J_0, \dots, J_n$. If we fix startpoints of each $x_k$ forming lattice spaced apart by $\Omega(1/N)$, and consider the space $\mathbf{A}$ of products, then there are $O(N^k)$ points in $T(\mathbf{A})$, and so there are $O(N^k)$ elements in $\mathbf{B}$. We now split each $1/M$ cube in $J_0$ into length $1/N$ cubes, and discard those cubes which contain more than $O(N^{k-m})$ bad points, then we discard at most $1 - 1/M$ of all such cubes. We can dissect the remaining length $1/N$ cubes into $O(N^{k-m})$ length $\Omega(1/N^{k/m})$ cubes, and as in the previous argument, the cubes not containing elements of $\mathbf{B}$ nor adjacent to an element have total volume $\Omega(1/N^m)$, which we keep. The startpoints in the other intervals $T_i$ may then be thickened to a length $\Omega(1/N^{m/k})$ portion while still avoiding solutions. This gives a set with full Minkowski dimension and Hausdorff dimension $m/k$ avoiding solutions to $y = f(Tx)$. (I don't yet understand Minkowski dimension enough to understand this, but the techniques of the appendix make proving the Hausdorff dimension $1/k$ bound easy)

\section{Extension to Well Approximable Numbers}

If the coefficients of the linear transformation $T$ in the equation $y = f(Tx)$ are non-rational, then the images of startpoints under the action of $T$ do not form a lattice, and so points may not overlap so easily when avoiding solutions to the equations $y = f(Tx)$. However, if $T$ is `very close' to a family of rational coefficient linear transformations, then we can show the images of the startpoints are `very close' to a lattice, which will still enable us to find points avoiding solutions by replacing the direct combinatorial approach in the argument for integer matrices with a covering argument.

Suppose that $T$ is a real-coefficient linear transformation with the property that for each coefficient $x$ there are infinitely many rational numbers $p/q$ with $|x - p/q| \leq 1/q^\alpha$, for some fixed $\alpha$. For infinitely many $K$, we can therefore find a linear transformation $S$ with coefficients in $\mathbf{Z}/K$ with each coefficient of $T$ differing from the corresponding coefficient in $S$ by at most $1/K^\alpha$. Then for each $x$, we find
%
\[ \| (T - S)(x) \|_\infty \leq (n/K^\alpha) \| x \|_\infty \]
%
If we now consider $T_0, \dots, T_n$, splitting $T_1, \dots, T_n$ into length $1/N$ intervals, and considering $\mathbf{A}$ as in the last section, then $S(\mathbf{A})$ lie in a $k$ dimensional sublattice of $(\mathbf{Z}/KN)^m$, hence containing at most $(2A)^k (KN)^k = O_{T,n}((KN)^k)$ points. By our error term calculation of $T-S$, the elements of $T(\mathbf{A})$ are contained in cubes centered at these lattice points with side-lengths $2n/K^\alpha$, or balls centered at these points with radius $n^{3/2}/K^\alpha$. If $\| \nabla f \| \leq B$, then the images of the radius $n^{3/2}/K^\alpha$ balls under the action of $f$ are contained in length $Bn^{3/2}/K^\alpha$ intervals. Thus the total length of the image of all these balls under $f$ is $(Bn^{3/2}/K^\alpha)(2A)^k(KN)^k = (2A)^k Bn^{3/2} K^{k-\alpha} N^k$. If $k < \alpha$, then we can take $K$ arbitrarily large, so that there exists intervals with $\text{dist}(I,\mathbf{B}) = \Omega_{A,k,M}(1/N^k)$. But I believe that, after adding the explicit constants in, we cannot let $k = \alpha$.

\begin{remark}
    One problem is that, if $T$ has rank $k$, we might not be able to choose $S$ to be rank $k$ as well. Is this a problem? If $T$ has full rank, then the set of all such matrices is open so if $T$ and $S$ are close enough, $S$ also has rank $k$, but this need not be true if $T$ does not have full rank.
\end{remark}

\begin{example}
    If $T$ has rank 1, then Dirichlet's theorem says that every irrational number $x$ can be approximated by infinitely many $p/q$ with $|x - p/q| < 1/q^2$, so every real-valued rank 1 linear transformation can be avoided with a dimension one bound.
\end{example}

\section{Applications of Low Rank Coordinate Changes}

\begin{example}
Our initial exploration of low rank coordinate changes was inspired by trying to find solutions to the equation
%
\[ y - x = (u - w)^2 \]
%
Our algorithm gives a Hausdorff dimension $1/2$ set avoiding solutions to this equation. This equals Math\'{e}'s result. But this dimension for us now depends on the shifts involved in the equation, not on the exponent, so we can actually avoid solutions to the equation
%
\[ y - x = (u - w)^n \]
%
for any $n$, in a set of Hausdorff dimension $1/2$. More generally, if $X$ is a set, then given a smooth function $f$ of $n$ variables, we can find a set $X$ of Hausdorff dimension $1/n$ such that there is no $x \in X$, and $y_1, \dots, y_n \in X - X$ such that $x = f(y_1, \dots, y_n)$. This is better than the $1/2n$ bound that is obtained by Malabika and Fraser's result.
\end{example}

\begin{example}
For any fixed $m$, we can find a set $X \subset \mathbf{R}^n$ of full Hausdorff dimension which contains no solutions to
%
\[ a_1x_1 + \dots + a_nx_n = 0 \]
%
for {\it any} rational numbers $a_n$ which are not all zero. Since Malabika/Fraser's technique's solutions are bounded by the number of variables, they cannot let $n \to \infty$ to obtain a linearly independant set over the rational numbers. But since the Hausdorff dimension of our sets now only depends on the rank of $T$, rather than the total number of variables in $T$, we can let $n \to \infty$ to obtain full sets linearly independant over the rationals. More generally, for any Lipschitz continuous function $f: \mathbf{R} \to \mathbf{R}$, we can find a full Hausdorff dimensional set such that there are no solutions
%
\[ f(a_1x_1 + \dots + a_nx_n, y) \]
%
for any $n$, and for any rational numbers $a_n$ that are not all zero.
\end{example}

\begin{example}
The easiest applications of the low rank coordinate change method are probably involving configuration problems involving pairwise distances between $m$ points in $\mathbf{R}^n$, where $m \ll n$, since this can best take advantage of our rank condition. Perhaps one way to encompass this is to avoid $m$ vertex polyhedra in $n$ dimensional space, where $m \ll n$. In order to distinguish this problem from something that can be solved from Math\'{e}'s approach, we can probably find a high dimensional set avoiding $m$ vertex polyhedra on a parameterized $n$ dimensional manifold, where $m \ll n$. There is a result in projective geometry which says that every projectively invariant property of $m$ points in $\mathbf{RP}^d$ is expressible as a function in the ${m \choose d}$ bracket polynomials with respect to these $m$ points. In particular, our result says that we can avoid a countable collection of such invariants in a dimension $1/{m \choose d}$ set. This is a better choice of coordinates than Euclidean coordinates if ${m \choose d} \leq m$. Update: I don't think this is ever the case.
\end{example}

\begin{remark}
    Because of how we construct our set $X$, we can find a dimension $1/k$ set avoiding solutions to $y = f(Tx)$ for {\it all} rank $k$ rational matrices $T$, without losing any Hausdorff dimension. Maybe this will help us avoid solutions to more general problems?
\end{remark}

\begin{example}
Given a smooth curve $\Gamma$ in $\mathbf{R}^n$, can we find a subset $E$ with high Hausdorff dimension avoiding isoceles triangles. That is, if the curve is parameterized by $\gamma: [0,1] \to \mathbf{R}^n$, can we find $E \subset [0,1]$ such that for any $t_1, t_2, t_3$, $\gamma(t_1)$, $\gamma(t_2)$, and $\gamma(t_3)$ do not form the vertices of an isoceles triangle. This is, in a sense, a non-linear generalization of sets avoiding arithmetic progressions, since if $\Gamma$ is a line, an isoceles triangle is given by arithmetic progressions. Assuming our curve is simple, we must avoid zeroes of the function
%
\[ |\gamma(t_1) - \gamma(t_2)|^2 = |\gamma(t_2) - \gamma(t_3)|^2 \]
%
If we take a sufficiently small segment of this curve, and we assume the curve has non-zero curvature on this curve, we can assume that $t_1 < t_2 < t_3$ in our dissection method.

If the coordinates of $\gamma$ are given by polynomials with maximum degree $d$, then the equation
%
\[ |\gamma(t_1) - \gamma(t_2)|^2 - |\gamma(t_2) - \gamma(t_3)|^2 \]
%
is a polynomial of degree $2d$, and so Math\'{e}'s result gives a set of dimension $1/2d$ avoiding isoceles triangles. In the case where $\Gamma$ is a line, then the function $f(t_1,t_2,t_3) = \gamma(t_1) + \gamma(t_3) - 2\gamma(t_2)$ avoids arithmetic progresions, and Math\'{e}'s result gives a dimension one set avoiding such progressions. Rob and Malabika's algorithm easily gives a set with dimension $1/2$ for any curve $\Gamma$. Our algorithm doesn't seem to be able to do much better here.
\end{example}

\begin{example}
    What is the largest dimension of a set in Euclidean space such that for any value $\lambda$, there is at most one pair of points $x,y$ in the set such that $|x - y| = \lambda$.
\end{example}

\begin{example}
    What is the largest dimension of a set which avoids certain angles, i.e. for which a triplet $x,y,z$ avoids certain planar configurations.
\end{example}

\begin{example}
    A set of points $x_0, \dots, x_d \in \mathbf{R}^d$ lie in a hyperplane if and only if the determinant formed by the vectors $x_n - x_0$, for $n \in \{ 1, \dots, d \}$, is zero. This is a degree $d$ polynomial, hence Math\'{e}'s result gives a dimension one set with no set of $d+1$ points lying in a hyperplane. On the other hand, a theorem of Mattila shows that every analytic set $E$ with dimension exceeding one contains $d + 1$ points in a hyperplane. Can we generalize this to a more general example avoiding points on a rotational, translation invariant family of manifolds using our results?
\end{example}

\begin{example}
    Given a set $F$ not containing the origin, what is the largest Hausdorff dimension of a set $E$ such that for any for any distinct rational $a_1, \dots, a_N$, the sum $a_1E + \dots + a_NE$ does not contain any elements of $F$. Thus the vector space over the rationals generated by $E$ does not contain any elements of $F$. We can also take the non-linear values $f(a_1E + \dots + a_NE)$ avoiding elements of $F$. $F$ must have non-empty interior for the problem to be interesting. Then can we find a smooth function $f$ with non-nanishing derivative which vanishes over $F$, or a family of smooth functions with non-vanishing derivative around $F$.
\end{example}

\endinput