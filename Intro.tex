%% The following is a directive for TeXShop to indicate the main file
%%!TEX root = diss.tex

\chapter{Introduction}
\label{ch:Introduction}

In this thesis, we study a simple question:
%
\begin{center}
	{\it How large can Euclidean sets be not containing geometric patterns?}.
\end{center}
%
The patterns manifest as certain configurations of point tuples, often affine invariant. For instance, we might like to find large subsets $X$ of $\RR^d$ that contain no three collinear points, or do not contain the vertices of any isoceles triangle. Aside from purely geometric interest, these problems provide useful scenarios in which to test the methods of ergodic theory and harmonic analysis.

The Lebesgue density theory shows any subset of $\RR^d$ with positive measure must contain a copy of the vertices of any suitably small isosceles triangle. Similar techniques show sets avoiding any of the patterns considered in this thesis must have measure zero. Thus non-trivial pattern avoiding sets must be highly irregular, and require techniques from geometric measure theory to be understood.

In particular, the Lebesgue measure cannot quantify the `maximal size' of a pattern avoiding set. Thus a `second-order' quantification of the size of a measure zero must be introduced, and this is satisfied by the fractional dimension of a set, first introduced by Felix Hausdorff in 1918. We shall take the Hausdorff dimension he introduced as a primary measure of a set's size.%, as well as various other more modern notions of fractional dimension which give more structural information about a set.

At the time of the writing of this thesis, many fundamental questions about the relation between the geometric arrangement of a set and it's fractional dimension remain unsolved. It might be expected that sets with sufficiently large Hausdorff dimension contain patterns, but this is not always true. For example, Theorem 6.8 of \cite{Matilla} shows that any set $X \subset \RR^d$ with Hausdorff dimension exceeding one must contain three colinear points. On the other hand, \cite{Maga} constructs a set $X \subset \RR^d$ with full Hausdorff dimension such that no four points in $X$ form the vertices of a parallelogram. For most geometric configurations, it remains unknown at what threshold patterns are guaranteed, or whether such a threshold exists at all. So finding sets with large Hausdorff dimension avoiding patterns is an important topic to determine for which classes of patterns our intuition remains true.

Our goal in this thesis is to derive new methods for constructing sets avoiding patterns. And rather than studying particular instances of the configuration avoidance problem, we choose to study general methods for finding large subsets of space avoiding any pattern. In particular, we expand on a number of general {\it pattern dissection methods} which have proven useful in the area, originally developed by Keleti but also studied notably by Math\'{e}, and Pramanik/Fraser.

The main contribution we give to pattern dissection methods is using random dissections to avoid patterns without much structual information. Thus we can expand the utility of interval dissection methods from regular patterns to a set of {\it fractal avoidance problems}, that previous methods were completely unavailable to address. Such problems include finding large $X$ such that the angles formed by any three distinct points of $X$ avoid a specified set $Y$ of angles, where $Y$ has a fixed Hausdorff dimension. For instance, $Y$ could be the set of all angles expressable as $q + x$ radians, where $q$ is rational, and $x$ lies in the Cantor set.

\endinput

Any text after an \endinput is ignored.
You could put scraps here or things in progress.