%% The following is a directive for TeXShop to indicate the main file
%%!TEX root = diss.tex

\chapter{Introduction}
\label{ch:Introduction}

In this thesis, we study a simple family of questions:
%
\begin{center}
	{\it How large can Euclidean sets be not containing geometric patterns?}.
\end{center}
%
For instance, what is the maximal size of a set $X \subset \RR^d$ that contains no three collinear points? Or what is the maximal size of a set not containing the vertices of an isosceles triangle? Aside from pure geometric interest, these problems provide useful settings to test methods of ergodic theory, additive combinatorics and harmonic analysis.

Sets which avoid the patterns we consider are highly irregular, in many ways behaving like a fractal set. Our understanding of their structure requires techniques from geometric measure theory. In particular, we use various \emph{fractal dimensions} to measure the size of a pattern avoiding set.

%In particular, we use the Hausdorff dimension and Minkowski dimension, as well as various other more modern variants of fractal dimension which give more structural information about a set.

%The Lebesgue density theory shows any subset of $\RR^d$ with positive measure must contain a copy of the vertices of any suitably small isosceles triangle. Similar techniques show sets avoiding any of the patterns considered in this thesis must have measure zero. Thus non-trivial pattern avoiding sets must be highly irregular, and require techniques from geometric measure theory to be understood.

%In particular, the Lebesgue measure cannot quantify the `maximal size' of a pattern avoiding set. Thus a `second-order' quantification of the size of a measure zero must be introduced, and this is satisfied by the fractional dimension of a set, first introduced by Felix Hausdorff in 1918. We shall take the Hausdorff dimension he introduced as a primary measure of a set's size.%, as well as various other more modern notions of fractional dimension which give more structural information about a set.

At present, many fundamental questions about geometric structure and its relation to fractal dimension remain unsolved. One might expect sets with sufficiently large fractal dimension must contain a given pattern. For example, Theorem 6.8 of \cite{Matilla} shows that any set $X \subset \RR^d$ with Hausdorff dimension exceeding one must contain three collinear points. On the other hand, Theorem 2.3 of \cite{Maga} constructs a set $X \subset \RR^2$ with full Hausdorff dimension such that no four points in $X$ form the vertices of a parallelogram. Thus the conjecture that sufficiently large sets must contain a given pattern depends on the particular patterns involved. For most geometric configurations, it remains unknown at what threshold patterns are guaranteed, or whether such a threshold exists at all.

%So finding sets with large Hausdorff dimension avoiding patterns is an important topic to determine for which classes of patterns our intuition remains true.

Our goal in this thesis is to derive new methods for constructing sets avoiding patterns. In particular, we expand on a number of general {\it pattern dissection methods} which have proven useful in the area, originally developed by Keleti but also studied notably by Fraser, Math\'{e}, and Pramanik. In Chapter \ref{ch:RelatedWork}, we give an exposition of some of their methods, after we establish some background in Chapter \ref{ch:Background}. Our main contribution is to avoid new classes of patterns using random dissection methods. This enables us to expand the utility of pattern dissection methods from regular families of patterns to a family of {\it fractal avoidance problems}, that previous methods were completely unavailable to address. Such problems include finding large sets $X$ such that the angles formed by any three distinct points of $X$ avoid a specified set $Y$ of angles, where $Y$ has a fixed fractal dimension. This method is described in Chapter \ref{ch:RoughSets}. Applications are then described in Chapter \ref{ch:Applications}. Chapter \ref{ch:Conclusions} describes various improvements to our results which we hope to develop in the future.

\endinput

Any text after an \endinput is ignored.
You could put scraps here or things in progress.