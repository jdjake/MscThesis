%% The following is a directive for TeXShop to indicate the main file
%%!TEX root = diss.tex

\chapter{Constructing Squarefree Sets}
\label{ch:Squarefree}

\section{Ideas For New Work}

A continuous formulation of the squarefree difference problem is not so clear to formulate, because every positive real number has a square root. Instead, we consider a problem which introduces a similar structure to avoid in the continuous domain rather than the discrete. Unfortunately, there is no direct continuous anology to the squarefree subset problem on the interval $[0,1]$, because there is no canonical subset of $[0,1]$ which can be identified as `perfect squares', unlike in $\mathbf{Z}$. If we only restrict ourselves to perfect squares of a countable set, like perfect squares of rational numbers, a result of Keleti gives us a set of full Hausdorff dimension avoiding this set. Thus, instead, we say a set $X \subset [0,1]$ is (continuously) {\bf squarefree} if there are no nontrivial solutions to the equation $x - y = (u - v)^2$, in the sense that there are no $x,y,u,v \in X$ satisfying the equation for $x \neq y, u \neq v$. In this section we consider some blue sky ideas that might give us what we need.

How do we adopt Rusza's power series method to this continuous formulation of the problem? We want to scale up the problem exponentially in a way we can vary to give a better control of the exponentials. Note that for a fixed $m$, every elements $x \in [0,1]$ has an essentially unique $m$-ary expansion
%
\[ x = \sum_{n = 1}^\infty \frac{x_n}{m^n} \]
%
and the pullback to the Haar measure on $\mathbf{F}_m^\infty$ is measure preserving (with respect to the natural Haar measure on $\mathbf{F}_m^\infty$), so perhaps there is a way to reformulate the problem natural as finding nice subsets of $\mathbf{F}_m^\infty$ avoiding squares. In terms of this expansion, the equation $x - y = (u - v)^2$ can be rewritten as
%
\[ \sum_{n = 1}^\infty \frac{x_n - y_n}{m^n} = \left( \sum_{k = 1}^\infty \frac{u_n - v_n}{m^n} \right)^2 = \sum_{n = 1}^\infty \left( \sum_{k = 1}^{n-1} (u_k - v_k)(u_{n-k} - v_{n-k}) \right) \frac{1}{m^n} \]
%
One problem with this expansion is that the sums of the differences of each element do not remain in $\{ 0, \dots, m-1 \}$, so the sum on the right cannot be considered an equivalent formal expansion to the expansion on the left. Perhaps $\mathbf{F}_m^\infty$ might be a simpler domain to explore the properties of squarefree subsets, in relation to Ruzsa's discrete strategy. What if we now consider the problem of finding the largest subset $X$ of $\mathbf{F}_m^\infty$ such that there do not exist $x,y,u,v \in \mathbf{F}_m^\infty$ such that if $x,y,u,v \in X$, $x \neq y$, $u \neq v$, then for any $n$
%
\[ x_n - y_n \neq \sum_{k = 1}^{n-1} (u_k - v_k)(u_{n-k} - v_{n-k}) \]

What if we consider the problem modulo $m$, so that the convolution is considered modulo $m$, and we want to avoid such differences modulo $m$. So in particular, we do not find any solutions to the equation
%
\begin{align*}
    x_2 - y_2 &= (u_1 - v_1)^2\\
    x_3 - y_3 &= 2 (u_1 - v_1)(u_2 - v_2)\\
    x_4 - y_4 &= (u_1 - v_1)(u_3 - v_3) + (u_2 - v_2)^2\\
    &\ \ \vdots
\end{align*}
%
which are considered modulo $m$. The topology of the $p$-adic numbers induces a power series relationship which `goes up' and might be useful to our analysis, if the measure theory of the $p$-adic numbers agrees with the measure theory of normal numbers in some way, or as an alternate domain to analyze the squarefree problem as with $\mathbf{F}_m^\infty$.

The problem with the squarefree subset problem is that we are trying to optimize over two quantities. We want to choose a set $X$ such that the number of distinct differences $x - y$ as small as possible, while keeping the set as large as possible. This double optimization is distinctly different from the problem of finding squarefree difference subsets of the integers. Perhaps a more natural analogy is to fix a set $V$, and to find the largest subset $X$ of $[0,1]$ such that $x - y = (u - v)^2$, where $x \neq y \in X$, and $u \neq v \in V$. Then we are just avoiding subsets of $[0,1]$ which avoid a particular set of differences, and I imagine this subset has a large theory. But now we can solve the general subset problem by finding large subsets $X$ such that $(X - X)^2 \subset V$ and $X$ containing no differences in $V$. Does Rusza's method utilize the fact that the problem is a single optimization? Can we adapt Rusza's method work to give better results about finding subsets $X$ of the integers such that $X - X$ is disjoint from $(X - X)^2$?

\section{Squarefree Sets Using Modulus Techniques}

We now try to adapt Ruzsa's idea of applying congruences modulo $m$ to avoid squarefree differences on the integers to finding high dimensional subsets of $[0,1]$ which satisfy a continuous analogy of the integer constraint. One problem with the squarefree problem is that solutions are non-scalable, in the sense that if $X \subset [N]$ is squarefree, $\alpha X$ may not be squarefree. This makes sense, since avoiding solutions to $\alpha (x - y) = \alpha^2 (u - v)^2$ is clearly not equivalent to the equation $x - y = (u - v)^2$. As an example, $X = \{ 0, 1/2 \}$ is squarefree, but $2X = \{ 0, 1 \}$ isn't. On the other hand, if $X$ avoids squarefree differences modulo $N$, it {\it is} scalable by a number congruent to 1 modulo $N$. More generally, if $\alpha$ is a rational number of the form $p/q$, then $\alpha X$ will avoid nontrivial solutions to $q (x - y) = p (u - v)^2$, and if $p$ and $q$ are both congruent to 1 modulo $N$, then $X$ is squarefree, so modulo arithmetic enables us to scale down. Since the set of rational numbers with numerator and denominator congruent to 1 is dense in $\mathbf{R}$, {\it essentially} all scales of $X$ are continuously squarefree. Since $X$ is discrete, it has Hausdorff dimension zero, but we can `fatten' the scales of $X$ to obtain a high dimension continuously squarefree set. To initially simplify the situation, we now choose to avoid nontrivial solutions to $y - x = (z - x)^2$, removing a single degree of freedom from the domain of the equation.

So we now fix a subset $X$ of $\{ 0, \dots, m-1 \}$ avoiding squares modulo $m$. We now ask how large can we make $\varepsilon$ such that nontrivial solutions to $x - y = (x - z)^2$ in the set
%
\[ E = \bigcup_{x \in X} [\alpha x, \alpha x + \varepsilon) \]
%
occur in a common interval, if $\alpha$ is just short of $1/m^n$. This will allow us to recursively place a scaled, `fattened' version of $X$ in every interval, and then consider a limiting process to obtain a high dimensional continuously squarefree set. If we have a nontrivial solution triple, we can write it as $\alpha x + \delta_1, \alpha y + \delta_2$, and $\alpha z + \delta_3$, with $\delta_1, \delta_2, \delta_3 < \varepsilon$. Expanding the solution leads to
%
\[ \alpha (x - y) + (\delta_1 - \delta_2) = \alpha^2 (x - z)^2 + 2\alpha(x - z)(\delta_1 - \delta_3) + (\delta_1 - \delta_3)^2 \]
%
If $x$, $y$, and $z$ are all distinct, then, as we have discussed, we cannot have $\alpha (x - y) = \alpha^2 (x - z)^2$. if $\alpha$ is chosen close enough to $1/m^n$, then we obtain an approximate inequality
%
\[ |\alpha (x - y) - \alpha^2 (x - z)^2| \geq \alpha^2 \]
%
(we require $\alpha$ to be close enough to $1/n$ for some $n$ to guarantee this). Thus we can guarantee at least two of $x$, $y$, and $z$ are equal to one another if
%
\[ |2\alpha(x - z)(\delta_1 - \delta_3) + (\delta_1 - \delta_3)^2 - (\delta_1 - \delta_2)| < \frac{1}{m^{2n}} \]
%
We calculate that
%
\[ 2\alpha(x - z)(\delta_1 - \delta_3) + (\delta_1 - \delta_3)^2 - (\delta_1 - \delta_2) < 2\alpha(m-1)\varepsilon + \varepsilon^2 + \varepsilon \]
\[ (\delta_1 - \delta_2) - 2\alpha(x - z)(\delta_1 - \delta_3) - (\delta_1 - \delta_3)^2 \leq \varepsilon + 2\alpha(m-1)\varepsilon \]
%
So it suffices to choose $\varepsilon$ such that
%
\[ \varepsilon^2 + [2\alpha(m-1) + 1]\varepsilon \leq \alpha^2 \]
%
This is equivalent to picking
%
\[ \varepsilon \leq \sqrt{\left( \frac{2 \alpha(m - 1) + 1}{2} \right)^2 + \alpha^2} - \frac{2\alpha(m-1) + 1}{2} \approx \frac{\alpha^2}{2\alpha(m-1) + 1} \]

We split the remaining discussion of the bound we must place on $\varepsilon$ into the three cases where two of $x$, $y$, and $z$ are equal, but one is distinct, to determine how small $\varepsilon$ must be to prevent this from happening. Now
%
\begin{itemize}
    \item If $y = z$, but $x$ is distinct, then because we know $\alpha(x - y) = \alpha^2(x - y)^2$ has no solution in $X$, we obtain that (provided $\alpha$ is close enough to $1/m^n$),
    %
    \[ |\alpha(x - y) - \alpha^2(x-y)^2| \geq \alpha^2 \]
    %
    and the same inequality that worked for the case where the three equations are distinct now applies for this case.

    \item If $x = y$, but $z$ is distinct, we are left with the equation
    %
    \[ \delta_1 - \delta_2 = \alpha^2(x - z)^2 + 2\alpha(x - z)(\delta_1 - \delta_3) + (\delta_1 - \delta_3)^2 \]
    %
    Now $\alpha^2(x - z)^2 \geq \alpha^2$, and
    %
    \[ \delta_1 - \delta_2 - 2\alpha(x-z)(\delta_1 - \delta_3) - (\delta_1 - \delta_3)^2 < \varepsilon + 2\alpha(m-1)\varepsilon \]
    %
    so we need the additional constraint $\varepsilon + 2\alpha(m-1)\varepsilon \leq \alpha^2$, which is equivalent to saying
    %
    \[ \varepsilon \leq \frac{\alpha^2}{1 + 2\alpha(m-1)} \]

    \item If $x = z$, but $y$ is distinct, we are left with the equation
    %
    \[ \alpha(x - y) + (\delta_1 - \delta_2) = (\delta_1 - \delta_3)^2 \]
    %
    Now $|\alpha(x-y)| \geq \alpha$, and
    %
    \[ (\delta_1 - \delta_3)^2 - (\delta_1 - \delta_2) < \varepsilon^2 + \varepsilon \]
    \[ (\delta_1 - \delta_2) - (\delta_1 - \delta_3)^2 < \varepsilon \]
    %
    so to avoid this case, we need $\varepsilon^2 + \varepsilon \leq \alpha$, or
    %
    \[ \varepsilon \leq \frac{\sqrt{1 + 4\alpha} - 1}{2} \approx \alpha \]
\end{itemize}
%
Provided $\varepsilon$ is chosen as above, all solutions in $E$ must occur in a common interval. Thus, if we now replace the intervals with a recursive fattened scaling of $X$, all solutions must occur in smaller and smaller intervals. If we choose the size of these scalings to go to zero, these solutions are required to lie in a common interval of length zero, and thus the three values must be equal to one another. Rigorously, we set $\varepsilon \approx 1/m^2$, and $\alpha \approx 1/m$, we can define a recursive construction by setting
%
\[ E_1 = \bigcup_{x \in X} [\alpha x, \alpha x + \varepsilon_1) \]
%
and if we then set $X_n$ to be the set of startpoints of the intervals in $E_n$, then
%
\[ E_{n+1} = \bigcup_{x \in X_n} (x + \alpha^2 E_n) \]
%
Then $\bigcap E_n$ is a continuously squarefree subset. But what is it's dimension?

\section{Idea; Delaying Swaps}

By delaying the removing in the pattern removal queue, we may assume in our dissection methods that we are working with sets with certain properties, i.e. we can swap an interval with a dimension one set avoiding translates.

\section{Squarefree Subsets Using Interval Dissection Methods}

The main idea of Keleti's proof was that, for a function $f$, given a method that takes a sequence of disjoint unions of sets $J_1, \dots, J_N$, each a union of almost disjoint closed intervals of the same length, and gives large subsets $J_n' \subset J_n$, each a union of almost disjoint intervals of a much smaller length, such that $f(x_1, \dots, x_n) \neq 0$ for $x_n \in J_n'$. Then one can find high dimensional subsets $K$ of the real line such that $f(x_1, \dots, x_n) \neq 0$ for a sequence of distinct $x_1, \dots, x_n \in K$. The larger the subsets $J_n'$ are compared to $J_n$, the higher the Hausdorff dimension of $K$. We now try and apply this method to construct large subsets avoiding solutions to the equation $f(x,y,z) = (x - y) - (x - z)^2$. In this case, since solutions to the equation above satisfy $y = x - (x-z)^2$, given $J_1, J_2, J_3$, finding $J_1', J_2', J_3'$ as in the method above is the same as choosing $J_1'$ and $J_3'$ such that the image of $J_1' \times J_3'$ under the map $g(x,z) = x - (x-z)^2$ is small in $J_2$. We begin by discretizing the problem, splitting $J_1$ and $J_3$ into unions of smaller intervals, and then choosing large subsets of these intervals, and finding large intervals of $J_2$ avoiding the images of the startpoints to these intervals.

So suppose that $J_1,J_2$, and $J_3$ are unions of intervals of length $1/M$, for which we may find subsets $A,B \subset [M]$ of the integers such that
%
\[ J_1 = \bigcup_{a \in A} \left[\frac{a}{M}, \frac{a + 1}{M} \right]\ \ \ \ \ J_3 = \bigcup_{b \in B} \left[ \frac{b}{M} , \frac{b + 1}{M} \right] \]
%
If we split $J_1$ and $J_3$ into intervals of length $1/NM$, for some $N \gg M$ to be specified later (though we will assume it is a perfect square), then
%
\[ J_1 = \bigcup_{\substack{a \in A\\0 \leq k < N}} \left[ \frac{Na + k}{NM}, \frac{Na + k}{NM} + \frac{1}{NM} \right]\ \ \ \ \ J_3 = \bigcup_{\substack{b \in A\\0 \leq l < N}} \left[ \frac{Nb + l}{NM}, \frac{Nb + l}{NM} + \frac{1}{NM} \right] \]
%
We now calculate $g$ over the startpoints of these intervals, writing
%
\begin{align*}
    g \left( \frac{Na + k}{NM}, \frac{Nb + l}{NM} \right) &= \frac{Na + k}{NM} - \left( \frac{N(a - b) + (k-l)}{NM} \right)^2\\
    &= \frac{a}{M} - \frac{(a-b)^2}{M^2} + \frac{k}{NM} - \frac{2(a-b)(k-l)}{NM^2} + \frac{(k-l)^2}{(NM)^2}
\end{align*}
%
which splits the terms into their various scales. If we write $m = k - l$, then $m$ can range on the integers in $(-N,N)$, and so, ignoring the first scale of the equation, we are motivated to consider the distribution of the set of points of the form
%
\[ \frac{k}{NM} - \frac{2(a-b)m}{NM^2} + \frac{m^2}{(NM)^2} \]
%
where $k$ is an integer in $[0,N)$, and $m$ an integer in $(-N,N)$. To do this, fix $\varepsilon > 0$. Suppose that we find some value $\alpha \in [0,1]$ such that $S$ intersects
%
\[ \left[ \alpha , \alpha + \frac{1}{N^{1 + \varepsilon}} \right] \]
%
Then there is $k$ and $m$ such that
%
\[ 0 \leq \frac{kNM - 2N(a-b)m + m^2}{(NM)^2} - \alpha \leq \frac{1}{N^{1 + \varepsilon}} \]
%
Write $m = q \sqrt{N} + r$ (remember that we chose $N$ so it's square root is an integer), with $0 \leq r < \sqrt{N}$. Then $m^2 = qN + 2qr \sqrt{N} + r^2$, and if $2qr = Q\sqrt{N} + R$, where $0 \leq R < \sqrt{N}$, then we find
%
\[ -\frac{R}{M^2 N^{3/2}} - \frac{r^2}{(NM)^2} \leq \frac{kM - 2(a-b)m + q + Q}{NM^2} - \alpha \leq \frac{1}{N^{1 + \varepsilon}} - \frac{R}{M^2 N^{3/2}} - \frac{r^2}{(NM)^2} \]
%
Thus
%
\[ d(\alpha, \mathbf{Z}/NM^2) \leq \max \left( \frac{1}{N^{1+\varepsilon}} - \frac{R}{\sqrt{N}} - \frac{r^2}{N}, \frac{R}{M^2 N^{3/2}} + \frac{r^2}{(NM)^2} \right) \]
%
If we now restrict our attention to the set $S$ consisting of the expressions we are studying where $R \leq (\delta_0/2) \sqrt{N}$, $r \leq \sqrt{\delta_0 N/2}$, then if the interval corresponding to $\alpha$ intersects $S$, then
%
\[ d(\alpha, \mathbf{Z}/NM^2) \leq \max \left( \frac{1}{N^{1+\varepsilon}} , \frac{\delta_0}{NM^2} \right) \]
%
If $N^\varepsilon \geq M^2/\delta_0$, then we can force $d(\alpha, \mathbf{Z}/NM^2) \leq \delta_0/NM^2$ for all $\alpha$ intersecting $S$. Thus, if we split $J_2$ into intervals starting at points of the form
%
\[ \frac{k + 1/2}{NM^2} \]
%
each of length $1/N^{1+\varepsilon}$, then provided $\delta_0 < 1/2$, we conclude that these intervals do not contain any points in $S$, since
%
\[ d \left( \frac{k + 1/2}{NM^2}, \mathbf{Z}/NM^2 \right) = \frac{1}{2NM^2} > \frac{\delta_0}{NM^2} \]
%
So we're well on our way to using Pramanik and Fraser's recursive result, since this argument shows that, provided points in $J_1$ and $J_3$ are chosen carefully, we can keep $O_M(1/N^{1 + \varepsilon})$ of each interval in $J_2$, which should lead to a dimension bound arbitrarily close to one.

\section{Finding Many Startpoints of Small Modulus}

To ensure a high dimension corresponding to the recursive construction, it now suffices to show $J_1$ and $J_3$ contain many startpoints corresponding to points in $S$, so that the refinements can be chosen to obtain $O_M(1/N)$ of each of the original intervals. Define $T$ to be the set of all integers $m \in (-N,N)$ with $m = q \sqrt{N} + r$ and $r \leq \sqrt{\delta_0 N/2}$ and $2qr = Q\sqrt{N} + R$ with $R \leq (\delta_0/2) \sqrt{N}$. Because of the uniqueness of the division decomposition, we find $T$ is in one to one correspondence with the set $T'$ of all pairs of integers $(q,r)$, with $q \in (-\sqrt{N},\sqrt{N})$ and $r \in [0,\sqrt{N})$, with $r \leq \sqrt{\delta_0 N/2}$, $2qr = Q \sqrt{N} + R$, and $R \leq (\delta_0/2) \sqrt{N}$. Thus we require some more refined techniques to better upper bound the size of this set.

Let's simplify notation, generalizing the situation. Given a fixed $\varepsilon$, We want to find a large number of integers $n \in (-N,N)$ with a decomposition $n = qr$, where $r \leq \varepsilon \sqrt{N}$, and $q \leq \sqrt{N}$. The following result reduces our problem to understanding the distribution of the smooth integers.

\begin{lemma}
    Fix constants $A,B$, and let $n \leq AN$ be an integer. If all prime factors of $n$ are $\leq BN^{1-\delta}$, then $n$ can be decomposed as $qr$ with $r \leq \varepsilon \sqrt{N}$ and $q \leq \sqrt{N}$.
\end{lemma}
\begin{proof}
    Order the prime factors of $n$ in increasing order as $p_1 \leq p_2 \leq \dots \leq p_K$. Let $r = p_1 \dots p_m$ denote the largest product of the first prime factors such that $r \leq \varepsilon \sqrt{N}$. If $r = n$, we can set $q = 1$, and we're finished. Otherwise, we know $r p_{m+1} > \varepsilon \sqrt{N}$, hence
    %
    \[ r > \frac{\varepsilon \sqrt{N}}{p_{m+1}} \geq \frac{\varepsilon \sqrt{N}}{B N^{1-\delta}} = \frac{\varepsilon}{B} N^{\delta - 1/2} \]
    %
    And if we set $q = n/r$, the inequality above implies
    %
    \[ q < \frac{nB}{\varepsilon} N^{1/2 - \delta} \leq \frac{AB}{\varepsilon} N^{3/2-\delta} \]
    %
    But now we run into a problem, because the only way we can set $q < \sqrt{N}$ while keeping $A$, $B$, and $\varepsilon$ fixed constants is to set $\delta = 1$, and $AB/\varepsilon \leq 1$.
\end{proof}

\begin{remark}
    Should we expect this method to work? Unless there's a particular reason why values of $(q,r)$ should accumulate near $Q = 0$, we should expect to lose all but $N^{-1/2}$ of the $N$ values we started with, so how can we expect to get $\Omega(N)$ values in our analysis. On the other hand, if a number $n$ is suitably smooth, in a linear amount of cases we should be able to divide up primes into two numbers $q$ and $r$ such that $r$ is small and $q$ fits into a suitable value of $Q$, so maybe this method will still work.
\end{remark}

Regardless of whether the lemma above actually holds through, we describe an asymptotic formula for perfect numbers which might come in handy. If $\Psi(N,M)$ denotes the number of integers $n \leq N$ with no prime factor exceeding $M$, then Karl Dickman showed
%
\[ \Psi(N,N^{1/u}) = N \rho(u) + O \left( \frac{u N}{\log N} \right) \]
%
This is essentially linear for a fixed $u$, which could show the set of $(q,r)$ is $\Omega_\varepsilon(N)$, which is what we want. Additional information can be obtained from Hildebrand and Tenenbaum's survey paper ``Integers Without Large Prime Factors''.

\section{A Better Approach}

Remember that we can write a general value in our set as
%
\[ x = \frac{-2(a-b)m}{NM^2} + \frac{q}{NM^2} + \frac{Q}{NM^2} + \frac{R}{N^{3/2} M^2} + \frac{r^2}{N^2M^2} \]
%
with the hope of guaranteeing the existence of many points, rather than forcing $R$ to be small, we now force $R$ to be close to some scaled value of $\sqrt{N}$, 
%
\[ |R - n \varepsilon \sqrt{N}| = \delta \sqrt{N} \leq \varepsilon \sqrt{N} \]
%
Then
%
\[ x = \frac{-2(a-b)m + q + Q + n\varepsilon + \delta}{NM^2} + \frac{r^2}{N^2M^2} \]
%
So
%
\[ d \left( x, \mathbf{Z}/NM^2 + n\varepsilon / NM^2 \right) \leq \frac{\varepsilon}{NM^2} + \frac{1}{4NM^2} = \frac{\varepsilon + 1/4}{NM^2} \]
%
By the pidgeonhole principle, since $R < \sqrt{N}$, there are $1/\varepsilon$ choices for $n$, whereas there are


The choice has the benefit of automatically possessing a lot of points by the pidgeonhole principle,


\endinput