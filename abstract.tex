%% The following is a directive for TeXShop to indicate the main file
%%!TEX root = diss.tex

\chapter{Abstract}

% MAXIMUM 350 WORDS!

The pattern avoidance problem seeks to construct a set $X\subset \RR^d$ with large dimension that avoids a prescribed pattern such as three term arithmetic progressions, or more general patterns such as avoiding points $x_1, \dots, x_n$ such that $f(x_1, \dots, x_n) = 0$ (three term arithmetic progressions are specified by the pattern $x_1 - 2x_2 + x_3 = 0$). Previous work on the subject has considered patterns described by polynomials, or by functions $f$ satisfying certain regularity conditions. We consider the case of `rough patterns.
% DISCUSS: SHOULD WE INCLUDE THE FACT THAT WE USE HAUSDORFF DIMENSION?
There are several problems that fit into the framework of rough pattern avoidance. As a first application, if $Y\subset[0,1]$ is a set with Minkowski dimension $\alpha$, we construct a set $X\subset[0,1]$ with Hausdorff dimension $1-\alpha$ so that $X+X$ is disjoint from $Y$. As a second application, given a set $Y$ of dimension close to one, we can construct a subset $X\subset Y$ of dimension $1/2$ that avoids isosceles triangles.