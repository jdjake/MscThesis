%% The following is a directive for TeXShop to indicate the main file
%%!TEX root = diss.tex

\chapter{Abstract}

% MAXIMUM 350 WORDS!

The pattern avoidance problem seeks to construct a set $X \subset \RR^d$ with large fractal dimension that avoids a prescribed pattern, such as three term arithmetic progressions, or more general patterns such as avoiding points $x_1, \dots, x_n \in \RR^d$ such that $f(x_1, \dots, x_n) = 0$ for a given function $f$. Previous work on the subject has considered patterns described by polynomials, or functions $f$ satisfying certain regularity conditions. We provide an exposition of some results in this setting, as well as considering new strategies to avoid `rough patterns. There are several problems that fit into the framework of rough pattern avoidance. As a first application, if $Y\subset[0,1]$ is a set with Minkowski dimension $s$, we construct a set $X\subset[0,1]$ with Hausdorff dimension $1-s$ so that $X+X$ is disjoint from $Y$. As a second application, given a set $Y$ with dimension close to one, we can construct a set $X\subset Y$ of dimension $1/2$ that avoids isosceles triangles.