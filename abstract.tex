%% The following is a directive for TeXShop to indicate the main file
%%!TEX root = diss.tex

\chapter{Abstract}

% MAXIMUM 350 WORDS!

The pattern avoidance problem seeks to construct a set with large fractal dimension that avoids a prescribed pattern, such as three term arithmetic progressions, or more general patterns such as finding a set whose Cartesian product avoids the zero set of a given function. Previous work on the subject has considered patterns described by polynomials, or functions satisfying certain regularity conditions. We provide an exposition of some results in this setting, as well as considering new strategies to avoid `rough patterns'. There are several problems that fit into the framework of rough pattern avoidance. For instance, we prove that for any set $X$ with lower Minkowski dimension $s$, there exists a set $Y$ with Hausdorff dimension $1-s$ such that for any rational numbers $a_1, \dots, a_N$, $a_1Y + \dots + a_NY$ is disjoint from $X$, or intersects solely at the origin. As a second application, we construct subsets of Lipschitz curves with dimension $1/2$ not containing the vertices of any isosceles triangle.