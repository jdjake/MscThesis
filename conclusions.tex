%% The following is a directive for TeXShop to indicate the main file
%%!TEX root = diss.tex

\chapter{Future Work}
\label{ch:Conclusions}

To conclude this thesis, we describe a rough sketch of some ideas which we hope will lead to future developments from the ideas of `rough sets avoiding patterns' we detailed in Chapter \ref{ch:RoughSets}. Section 6.1 attempts to explit additional geometric information about rough configurations to find sets with large Hausdorff dimension avoiding patterns, and section 6.2 studies the problem of finding a set avoiding a rough configuration which supports a measure with large Fourier decay.

\section{Low Rank Avoidance}

Another way we can attempt to generalize the results of Chapter \ref{ch:RoughSets} is by attempting to improve the result if the rough configuration $\C$ is given additional geometric structure. In Chapter \ref{ch:RoughSets}, we showed a technique to find large sets avoid configurations with low Minkowski dimension. This means precisely that these configurations are efficiently covered by cubes at all scales. Here, we discuss whether one can obtain large sets avoiding configurations efficiently covered by other families of geometric shapes, e.g. by thickened lines, thickened planes, or more generally, by thickened variants of any family of algebraic hypersurfaces.

\begin{theorem}
    Let $\C \subset \C(\RR)$ be the countable union of sets $\{ \C_i \}$ such that
    %
    \begin{itemize}
        \item For each $i$, there exists $n_i$ such that $\C_i$ is a pre-compact subset of $\C^{n_i}(\RR)$.

        \item There exists an integer $k_i > 0$ and $s \in [0,k_i)$, together with a rational-coefficient linear transformation $M: \RR^{n_i} \to \RR^{k_i}$ such that $M(\C_i)$ has Minkowski dimension $s$.
    \end{itemize}
    %
    Then there exists a set $X \subset [0,1]$ avoiding $\C$ with Hausdorff dimension at least
    %
    \[ \inf_i \left( \frac{k_i - s_i}{k_i} \right). \]
\end{theorem}

\begin{remark}
    Rescaling the problem by a constant does not affect any of the quantities in the problem. In particular, in our proof we may assume that the matrix we study is integer valued, which can be obtained by scaling down by a large integer.
\end{remark}

Here, we only describe a discrete building block lemma, which can be fleshed out into a full proof by techniques analogous to that given in Chapters \ref{ch:RelatedWork} and \ref{ch:RoughSets}. But first, we must `change coordinates' in the problem, so our description of the lemma is simpler. First, since our transformation $M$ has full rank, we may find indices
%
\[ i_1, \dots, i_k \in \{ 1, \dots, n \} \]
%
such that the transformation $M$ is invertible when restricted to these indices the span of $\{ e_{i_1}, \dots, e_{i_k} \}$. By an affine change of coordinates in the range of $M$, which preserves the Minkowski dimension of any set, we may assume that $M(e_{i_j}) = e_j$ for each $1 \leq j \leq k$.

\begin{theorem} \label{theorem059891891829}
    Fix $s \in [0,k)$ and $\varepsilon \in [0, (k-s)/2)$. Let $T_1, \dots, T_n \subset [0,1]$ be disjoint, $\DQ_k$ discretized sets, and let $B \subset \RR^k$ be a $\DQ_{k+1}$ discretized set such that
    %
    \[ \#(\DQ_{k+1}(B)) \leq N_{k+1}^{s + \varepsilon}. \]
    %
    Then there exists a constant $C(s,n,k,M) > 0$, and an integer constant $A(M) > 0$, such that if $A(M) \divides N_{k+1}$, and 
    %
    \begin{equation} \label{equation19024u1298352389}
        N_{k+1} \geq C(s,n,k,M) \cdot M_{k+1}^{\frac{k}{s - k}},
    \end{equation}
    %
    then there exists sets $S_1 \subset T_1$, \dots, $S_n \subset T_n$ such that
    %

\end{theorem}
\begin{proof}
    For each $i \not \in \{ i_1, \dots, i_k \}$, there are rational numbers $a_{ij} = p_{ij}/q_{ij} \in \mathbf{Q}$ such that $M(e_i) = \sum a_{ij} e_j$. For each interval $R \in \DR_{k+1}(T_i)$, we let $a(R) \in \{ 0, \dots, N_1 \dots N_k M_{k+1} - 1 \}$ be the integer such that
    %
    \[ R = \left[ \frac{a(R)}{N_1 \dots N_k M_{k+1}}, \frac{a(R) + 1}{N_1 \dots N_k M_{k+1}} \right]. \]
    %
    Let $X \in \{ 0, \dots, N_{k+1}/M_{k+1} - 1 \}^k$. For each $1 \leq j \leq k$, define
    %
    \[ S_{i_j}(X) = \bigcup_{R \in \DR_{k+1}(T_{i_j})} \left[ \frac{a(R)}{N_1 \dots N_k M_{k+1}} + \frac{X_j}{N_1 \dots N_{k+1}}, \frac{a(R)}{N_1 \dots N_k M_{k+1}} + \frac{X_j + 1}{N_1 \dots N_{k+1}} \right]. \]
    %
    For $i \not \in \{ i_1, \dots, i_k \}$, define
    %
    \[ S_i(X) = \bigcup_{\substack{R \in \DR_{k+1}(T_i)\\ \prod q_{ij} \divides a(R)}} \left[ \frac{a(R)}{N_1 \dots N_k M_{k+1}}, \frac{a(R)}{N_1 \dots N_k M_{k+1}} + \frac{1}{N_1 \dots N_{k+1}} \right] \]
    %
    For each $i$, we let $\mathcal{S}_i(X)$ denote the set of startpoints to intervals in $S_i$, and
    %
    \[ \mathcal{A}(X) = M(\mathcal{S}_1(X) \times \dots \times \mathcal{S}_n(X)). \]
    %
    Then $\mathcal{A}(X) \subset (\ZZ^k + X)/(N_1 \dots N_{k+1})$. In particular, this implies that for $X \neq X'$, $d(\mathcal{A}(X), \mathcal{A}(X')) \geq 1/N_1 \dots N_{k+1}$. Applying the pigeonhole principle and applying \eqref{equation19024u1298352389}, this means that there exists a constant $C$ depending only on $M$ and $k$, and some value of $X_0$ such that
    %
    \[ \# \left\{ a \in \mathcal{A}(X_0) : d(a,B) \leq \frac{2}{\sqrt{d} \| M \|} \frac{1}{N_1 \dots N_{k+1}} \right\} \leq C \cdot \frac{N_{k+1}^{s + \varepsilon}}{(N_{k+1}/M_{k+1})^k} = C \cdot \frac{M_{k+1}^k}{N_{k+1}^{k - (s + \varepsilon)}} < 1. \]
    %
    In particular, this means that there is some $X_0$ such that
    %
    \[ d(a,B) \geq \frac{2}{\sqrt{d} \| M \|} \frac{1}{N_1 \dots N_{k+1}}. \]
    %
    for all $a \in \mathcal{A}(X_0)$. But this means that $M(S_1(X_0) \times \dots \times S_n(X_0))$ is disjoint from $B$. Taking $S_i = S_i(X_0)$ for each $i$ completes the proof.
\end{proof}

Much remains on research related to this result. For instance, it is unclear whether one can improve the bound in Theorem \ref{theorem059891891829} to
%
\[ \inf \frac{k_i - s}{k_i - 1}, \]
%
which would match the bound in Theorem \ref{mainTheorem}, and how this theorem generalizes to sets avoiding patterns in $\RR^d$, instead of just in $\RR^1$.

\section{Fourier Dimension}

A popular technique in current research in geometric measure theory is exploiting Fourier analysis to obtain additional structural information about configurations in sets. A key insight to this technique is that the Frostman dimension of any finite Borel measure $\mu$ is equal to
%
\[ \sup \left\{ s > 0 : \int \frac{|\widehat{\mu}(\xi)|^2}{|\xi|^{d-s}} d\xi < \infty \right\}. \]
%
For brevity, we leave the proof to other sources, e.g. \cite[Section 3.5]{Matilla}. Note that if
%
\[ \int \frac{|\widehat{\mu}(\xi)|^2}{|\xi|^{d-s}}\; d\xi < \infty, \]
%
then there exists a constant $C$ such that for \emph{most} values $\xi \in \RR^d$,
%
\begin{equation} \label{fourierdimensioncondition}
    |\widehat{\mu}(\xi)| \leq C |\xi|^{-s/2}.
\end{equation}
%
We obtain a strengthening of the Frostman measure condition if we require \eqref{fourierdimensioncondition} to hold for \emph{all} values $\xi$. In particular, we say a measure $\mu$ has \emph{Fourier dimension} $s$ if \eqref{fourierdimensioncondition} holds for all $\xi \in \RR^d$. If this is true, then for all $t < s$,
%
\[ \int \frac{|\widehat{\mu}(\xi)|^2}{|\xi|^{d-t}}\; d\xi < \infty \]
%
so $\mu$ has Frostman dimension $t$ for all $t < s$. Thus if we define the \emph{Fourier dimension} of a set $E$ as
%
\[ \fordim(E) = \sup \left\{ s : \begin{array}{c} \text{there is a Borel probability measure $\mu$}\\ \text{supported on $E$ with Fourier dimension $s$} \end{array} \right\}, \]
%
then $\fordim(E) \leq \hausdim(E)$. We view the Fourier dimension as a refinement of the Hausdorff dimension which gives greater structural control on the set in the `frequency domain'.

There is much interest in determining whether sets with large Fourier dimension can avoid configurations. Results published recently include \cite{PramanikLaba} and \cite{Shmerkin}. Most classical examples of fractals, like the middle-thirds Cantor set, have Fourier dimension zero. Nonetheless, it is a general principle that \emph{random} families of sets tend to have Fourier dimension equal to the Hausdorff dimension. Since the main technique of Theorem \ref{mainTheorem} involves a random selection strategy, we might hope that a modification of the procedure produces a set with large Fourier dimension.

\begin{theorem} \label{FourierTheorem}
    Let $s \geq d$, and suppose $\C$ is the countable union of pre-compact sets, each with lower Minkowski dimension at most $s$. Then there exists a set $X \subset [0,1]^d$ with Fourier dimension at least $(nd - s)/n$ avoiding $\C$.
\end{theorem}

\begin{remark}
    Comparing Theorem \ref{FourierTheorem} with Theorem \ref{mainTheorem}, we see that our result here is slightly suboptimal, giving a dimension $(nd - s)/n$ set rather than the $(nd - s)/(n-1)$ dimensional set we obtained in the Hausdorff dimension setting. The hope to obtain a more optimal result in the near future.
\end{remark}

We begin with a lemma which simplifies the analysis of the Fourier decay of the probability measures we study to the analysis of frequencies in $\ZZ^d$.

\begin{lemma} \label{discretefouriermeasures}
    Suppose $\mu$ is a compactly supported finite Borel measure, consider $s > 0$, and suppose there exists a constant $C$ such that for each $m \in \ZZ^d$,
    %
    \begin{equation} \label{mufourierbound1}   
        |\widehat{\mu}(m)| \leq C |m|^{-s/2}.
    \end{equation} 
    %
    Then there exists a constant $C'$ such that for each $|\xi| \in \RR^d$,
    %
    \begin{equation} \label{mufourierbound2}
        |\widehat{\mu}(\xi)| \lesssim C' |\xi|^{-s/2}.
    \end{equation}
    %
    In particular, $\mu$ has Fourier dimension $s$.
\end{lemma}
\begin{proof}
    Without loss of generality, we may assume that $\mu$ is supported on a compact subset of $[1/3,2/3)^d$, since every compactly supported measure is a finite sum of translates of measures of this form. Consider the distribution $\Lambda = \sum_{m \in \mathbf{Z}^d} \delta_m$, where $\delta_m$ is the Dirac delta distribution at $m$. Then the Poisson summation formula says that the Fourier transform of $\Lambda$ is itself. If $\psi \in C_c(\RR^d)$ is a bump function supported on $[0,1)^d$, with $\psi(x) = 1$ for $x \in [1/3,2/3)^d$, then $\mu = \psi (\Lambda * \mu)$, so
    %
    \begin{equation} \label{mubounded}
    \begin{split}
        |\widehat{\mu}(\xi)| &= \left| \left[ \widehat{\psi} * (\Lambda \widehat{\mu}) \right](\xi) \right|\\
        &= \left| \sum_{m \in \mathbf{Z}^d} \widehat{\mu}(m)(\widehat{\psi} * \delta_m)(\xi) \right|\\
        &= \left| \sum_{m \in \mathbf{Z}^d} \widehat{\mu}(m) \widehat{\psi}(\xi - m) \right|.
%       &\lesssim \sum_{n \in \mathbf{Z}^d} |\widehat{\mu}(n)| \prod_{i = 1}^d \frac{1}{1 + |n_i - \xi_i|}
    \end{split}
    \end{equation}
    %
    Since $\psi$ is smooth, we know that for all $\eta \in \RR^d$, $|\widehat{\psi}(\eta)| \lesssim 1/|\eta|^{d+1}$. If we perform a dyadic decomposition, we find
    %
    \begin{equation}
        \label{calculation1}
    \begin{split}
        \sum_{1 \leq |m - \xi| \leq |\xi|/2} |\widehat{\mu}(m)| |\widehat{\psi}(\xi - m)| &\lesssim \sum_{1 \leq |m - \xi| \leq |\xi|/2} |\xi|^{-s/2} |\widehat{\psi}(\xi - m)|\\
        &\lesssim \sum_{k = 1}^{\log |\xi|} \sum_{\frac{|\xi|}{2^{k+1}} \leq |m - \xi| \leq \frac{|\xi|}{2^{k}}} |\xi|^{-s/2} \left( 2^k/|\xi| \right)^{d+1}\\
        &\lesssim \sum_{k = 1}^{\log |\xi|} |\xi|^{-s/2} (2^k / |\xi| ) \lesssim |\xi|^{-s/2}.
    \end{split}
    \end{equation}
    %
    There are $O(1)$ points $m \in \mathbf{Z}^d$ with $|m - \xi| \leq 1$, so if $|\xi| \geq 2$,
    %
    \begin{equation} \label{calculation2}
        \sum_{|m - \xi| \leq 1} |\widehat{\mu}(m)| |\widehat{\psi}(m - \xi)| \lesssim |\xi|^{-s/2}.
    \end{equation}
    %
    We can also perform another dyadic decomposition, using the fact that for all $\eta \in \RR^d$, $|\widehat{\psi}(\eta)| \lesssim 1/|\eta|^{2d}$, to find that
    %
    \begin{equation} \label{calculation3}
    \begin{split}
        \sum_{|m - \xi| \geq |\xi|/2} |\widehat{\mu}(m)| |\widehat{\psi}(m - \xi)| &\lesssim \sum_{k = 0}^\infty \sum_{|\xi| 2^{k-1} \leq |m - \xi| \leq |\xi| 2^k} \frac{|\widehat{\mu}(m)|}{|\xi|^{2d} 2^{2dk}}\\
        &\lesssim \sum_{k = 0}^\infty |\xi|^{-d} 2^{-dk} \lesssim |\xi|^{-d}.
    \end{split}
    \end{equation}
    %
    Combining \eqref{calculation1}, \eqref{calculation2}, and \eqref{calculation3} with \eqref{mubounded}, we conclude that if $|\xi| \geq 2$,
    %
    \begin{equation} \label{endequation53}
        |\widehat{\mu}(\xi)| \lesssim |\xi|^{-s/2},
    \end{equation}
    %
    and since $\widehat{\mu}$ is bounded, \eqref{endequation53} actually holds for all $\xi \in \RR^d$.
\end{proof}

Our goal now is to carefully modify the discrete selection strategy and discretized probability measures we use so that with high probability, we have sharp control over the Fourier transform of these measures. Lemma \ref{discretefouriermeasures} implies that we only need control over integer-valued frequencies. The discretized measures $\{ \nu_k \}$ we select are, for each $k$, a sum of point mass distributions at the points $(\ZZ/N_1 \dots N_k)^d$. Therefore, $\widehat{\nu_k}$ will be $N_1 \dots N_k$ periodic, in the sense that for any $m \in (N_k \ZZ)^d$ and $\xi \in \RR^d$, $\widehat{\nu_k}(\xi + m) = \widehat{\nu_k}(\xi)$. Since we are only concerned with integer-valued frequencies, it will therefore suffice to obtain control of $\widehat{\nu_k}$ on frequencies lying in $\{ 1, \dots, N_1 \dots N_k \}^d$.

In the discrete lemma below, we rely on a variant of the proof strategy of Theorem 2.1 of \cite{Shmerkin}, but modified so that we can allow the branching factors $\{ N_k \}$ to increase arbitrarily fast. We still use the families $\DQ_k^d$ and $\DR_k^d$, together with parameters $\{ N_k \}$ and $\{ M_k \}$, in order to perform a multi-scale analysis. For each $\DQ_k$ discretized set $E \subset [0,1]^d$, we define a probability measure
%
\[ \nu_E = \frac{1}{\#(\DQ_k(E))} \sum_{Q \in \DQ_k(E)} \delta(a(Q)), \]
%
where for each $x \in \RR^d$, $\delta(x)$ is the Dirac delta measure at $x$, and for each $Q \in \DQ_k^d$, $a(Q)$ is the left-hand corner of the cube $Q$. Also, for each $k$, we define a probability measure
%
\[ \eta_k = \frac{1}{N_{k+1}^d} \sum_{i_1, \dots, i_d = 0}^{N_{k+1}} \delta \left( \frac{i}{N_1 \dots N_{k+1}} \right).  \]
%
The purpose of introducing $\eta_k$ is so that, given a measure $\mu$ which is a sum of point mass distributions in $(\ZZ/N_1 \dots N_k)^d$, the probability measure $\mu * \eta_k$ is a sum of point mass distributions in $(\ZZ/N_1 \dots N_{k+1})^d$, uniformly distributed at the scale $1/N_1 \dots N_{k+1}$.

%Our goal now is now to carefully modify the discrete selection strategy and discretized probability measures we use, so that with high probability, the measures have the appropriate Fourier decay for the Fourier dimension bound we wish to obtain. Surprisingly, here we only need to perform a single scale analysis with the family of cubes $\DQ^d$, rather than a multi scale analysis involving the cubes $\DQ^d$ and $\DR^d$ as in Chapter \ref{ch:RoughSets}.

\begin{lemma} \label{discreteFourierBuildingBlock}
    Fix $s \in [1,dn)$, and $\varepsilon \in [0,(dn-s)/4)$. Let $T \subset \RR^d$ be a nonempty, $\DQ_k$ discretized set, and let $B \subset \RR^{dn}$ be a nonempty $\DQ_{k+1}$ discretized set such that
    %
    \begin{equation} \label{equation982589128942189}
        \#(\DQ_{k+1}(B)) \leq N_{k+1}^{s + \varepsilon}.
    \end{equation}
    %
    Then there exists a constant $A(d,n,s)$ such that, provided
    %
    \begin{equation} \label{equation5523786128439}
        M_{k+1}^{\frac{dn}{dn - s - 2\varepsilon}} \leq N_{k+1} \leq 2 M_{k+1}^{\frac{dn}{dn - s - 2\varepsilon}},
    \end{equation}
    %
    there exists a random $\DQ_k$ discretized set $S \subset T$, such that with probability
    %
    \[ 1 - \frac{1}{N_{k+1}^\varepsilon} - \left( \left\lceil \frac{A(d,n,s)}{\varepsilon} \right \rceil! \right)^3 \frac{(N_1 \dots N_k)^d}{N_{k+1}^d}, \]
    %
    the set $S$ satisfies the following two properties:
    %
    \begin{enumerate}
        \item[(A)] For any collection of $n$ distinct cubes $Q_1, \dots, Q_n \in \DQ_{k+1}(S)$,
        %
        \[ Q_1 \times \dots \times Q_n \not \in \DQ_{k+1}(B). \]

        \item[(B)] For any $m \in \ZZ^d$,
        %
        \[ |\widehat{\nu_S}(m) - \widehat{\eta_{k+1}}(m) \widehat{\nu_T}(m)| \leq N_{k+1}^{-(1 - \varepsilon) \frac{dn - s - 2\varepsilon}{2n}}. \]
    \end{enumerate}
\end{lemma}
\begin{proof}
    For each $R \in \DR_{k+1}(T)$, let $Q_R$ be randomly selected from $\DQ_{k+1}(R)$, independently from all other selections $Q_{R'}$. Then, set $S = \bigcup \{ Q_R: R \in \DR_{k+1}(T) \}$. We then have
    %
    \[ \#(\DQ_{k+1}(S)) = \#(\DR_{k+1}(T)) = M_{k+1}^d \DQ_k(T). \]
    %
    Without loss of generality, removing cubes from $B$ if necessary, we may assume that every cube $Q_1 \times \dots \times Q_n \in \DQ_{k+1}(B)$, the values $Q_1, \dots, Q_n$ are distinct. In particular, given any such cube, just as in Lemma \ref{discretelemma}, we have
    %
    \[ \mathbf{P}(Q_1 \times \dots Q_n \in \DQ_{k+1}(S^n)) = (M_{k+1}/N_{k+1})^{dn}. \]
    %
    Thus \eqref{equation982589128942189} and \eqref{equation5523786128439} imply
    %
    \[ \mathbf{E}(\#(\DQ_{k+1}(B) \cap \DQ_{k+1}(S^n))) \leq M_{k+1}^{dn}/N_{k+1}^{dn - (s + \varepsilon)} \leq 1/N_{k+1}^\varepsilon. \]
    %
    Markov's inequality implies
    %
    \begin{equation} \label{fourierdim2}
    \begin{split}
        \mathbf{P}(\DQ_{k+1}(B) \cap \DQ_{k+1}(S^n) \neq \emptyset) &= \mathbf{P}(\# (\DQ_{k+1}(B) \cap \DQ_{k+1}(S^n)) \geq 1)\\
        &\leq 1/N_{k+1}^\varepsilon.
    \end{split}
    \end{equation}
    %
    Thus $\DQ_{k+1}(S^n)$ is disjoint from $\DQ_{k+1}(B)$ with high probability.

    Now we analyze the Fourier transform of the measure $\nu_S$. For each cube $R \in \DR_{k+1}(T)$, and for each $m$, let
    %
    \[ A_R(m) = e^{\frac{-2 \pi i m \cdot a(Q_R)}{N_1 \dots N_{k+1}}} - \frac{1}{N_{k+1}^d} \sum_{k_1, \dots, k_d = 0}^N e^{\frac{-2 \pi i m \cdot [N_{k+1} a(Q) + k]}{N_1 \dots N_{k+1}}}. \]
    %
    Then $\EE[A_R(m)] = 0$, $|A_R(m)| \leq 2$ for each $m$, and
    %
    \[ \widehat{\nu_S}(m) - \widehat{\eta_{k+1}}(m) \widehat{\nu_T}(m) = \frac{1}{\#(\DR_{k+1}(T))} \sum_{R \in \DR_{k+1}(T)} A_R(m). \]
    %
    Now fix a particular value of $m$. Since the random variables $A_R(m)$ are independant from one another as $R$ ranges over $\DR_{k+1}(T)$, we can apply Hoeffding's inequality to conclude that for each $t > 0$,
    %
    \[ \PP \left( |\widehat{\nu_S}(m) - \widehat{\eta_{k+1}}(m) \widehat{\nu_T}(m)| \geq t \right) \leq e^{-\#(\DR_{k+1}(T)) t^2/2} = e^{-\#(\DQ_k(T)) M_{k+1}^d t^2/2}. \]
    %
    In particular,
    %
    \[ \PP \left( |\widehat{\nu_S}(m) - \widehat{\eta_{k+1}}(m) \widehat{\nu_T}(m)| \geq M_{k+1}^{-d/2 - \varepsilon} \right) \leq \exp(- \#(\DQ_k(T)) M_{k+1}^\varepsilon / 2 ). \]
    %
    The function $\widehat{\nu_S} - \widehat{\eta_{k+1}}$ is $N_1 \dots N_{k+1}$ periodic. Thus, to uniformly bound $\widehat{\nu_S}(m) - \widehat{\eta_{k+1}}(m)$, we need only bound the function over $(N_1 \dots N_{k+1})^d$ values. Applying a union bound and applying \eqref{equation5523786128439}, we therefore find that
    %
    \begin{equation} \label{equation81298398120412}
    \begin{split}
        \PP \left( \| \widehat{\nu_S} - \widehat{\eta_{k+1}} \widehat{\nu_T} \|_{L^\infty(\ZZ^d)} \geq M_{k+1}^{- d/2 - \varepsilon} \right) &\leq (N_1 \dots N_{k+1})^d \exp \left( - \#(\DQ_k(T)) M_{k+1}^\varepsilon / 2 \right)\\
        &\leq 2^{\lceil \frac{4d^2n}{\varepsilon(dn - s)} \rceil} \left\lceil \frac{4d^2n}{\varepsilon(dn-s)} \right\rceil! \frac{(N_1 \dots N_{k+1})^d}{M_{k+1}^{4d^2n/(dn - s)}}\\
        &\leq 2^{\lceil 8d^2n/(dn - s) \rceil} \left\lceil \frac{4d^2n}{\varepsilon(dn-s)} \right\rceil! \frac{(N_1 \dots N_k)^d}{N_{k+1}^d}\\
        &\leq \left( \left\lceil \frac{4d^2n}{\varepsilon(dn-s)} \right\rceil! \right)^3 \frac{(N_1 \dots N_k)^d}{N_{k+1}^d}.
    \end{split}
    \end{equation}
    %
    Note that
    %
    \[ M_{k+1}^{-d/2 - \varepsilon} \lesssim N_{k+1}^{-(1 - 2\varepsilon/d) \frac{dn - s - 2\varepsilon}{2n}} \leq N_{k+1}^{-(1 - \varepsilon) \frac{dn - s - 2\varepsilon}{2n}}. \]
    %
    If we set $A(d,n,s) = (4d^2n/(dn - s))$, then \eqref{fourierdim2} and \eqref{equation81298398120412} allow us to conclude that with probability at least
    %
    \[ 1 - \frac{1}{N_{k+1}^\varepsilon} - \left[ \left( \frac{A(d,n,s)}{\varepsilon} \right)! \right]^3 \frac{(N_1 \dots N_k)^d}{N_{k+1}^d}, \]
    %
    the sets $\DQ_{k+1}(S^n)$ and $\DQ_{k+1}(B)$ are disjoint, and
    %
    \[ \| \widehat{\nu_S} - \widehat{\eta_{k+1}} \widehat{\nu_T} \|_{L^\infty(\ZZ^d)} \leq N_{k+1}^{-(1 - \varepsilon) \frac{dn - s - 2\varepsilon}{2n}}. \qedhere \]

    \begin{comment}

    %
    Then for each $m \in \mathbf{Z}^d$,
    %
    \[ \widehat{\nu_S}(m) - \widehat{\eta_{k+1}}(m) \widehat{\nu_T}(m) = \sum_{Q \in \DQ_{k+1}(T)} A_Q e^{-\frac{2 \pi i m \cdot a(Q)}{N_1 \dots N_{k+1}}}. \]
    %
    We calculate that for each $Q \in \DQ_{k+1}(T)$,
    %
    \begin{align*}
        \EE[A_Q|\#(\DQ_k(S))] &= \frac{\PP (X_Q = 1 | \#(\DQ_k(S)))}{\#(\DQ_{k+1}(S))} - \frac{1}{N_{k+1}^d \#(\DQ_{k+1}(T))}\\
        &= \frac{\#(\DQ_{k+1}(S)) / N_{k+1}^d \#(\DQ_{k+1}(T))}{\#(\DQ_{k+1}(S))} - \frac{1}{N_{k+1}^d \#(\DQ_{k+1}(T))} = 0.
    \end{align*}
    %
    In particular, $\EE[A_Q] = 0$. Now for each $q \geq 1$,
    %
    \begin{align*}
        \EE[A_Q^q|\DQ_{k+1}(S)] &= \frac{\PP(X_Q = 1 | \DQ_{k+1}(S))}{\#(\DQ_{k+1}(S))^q} &= \frac{1}{\#(\DQ_{k+1}(T)) \cdot \#(\DQ_{k+1}(S))^{q-1}}.
    \end{align*}
    %
    Thus
    %
    \begin{align*}
        \EE[A_Q^q] &= \frac{1}{\#(\DQ_{k+1}(T))} \EE \left[ \frac{1}{\#(\DQ_{k+1}(S))^{q-1}} \right]\\
        &\leq s
    \end{align*}


    Now fix $m \in \{ -N_1 \dots N_{k+1}, N_1 \dots N_{k+1} \}^d$.



    We can then apply Hoeffding's inequality to conclude that for each $t > 0$,
    %
    \begin{align*}
        \PP \left( |\widehat{\nu_S}(m)| \geq \frac{t p \cdot \#(\DQ_{k+1}(T))^{1/2}}{\#(\DQ_{k+1}(S))} \right) &= \PP \left( \sum \left| \#(\DQ_{k+1}(S)) A_Q e^{-\frac{2 \pi i m \cdot a(Q)}{N_1 \dots N_{k+1}}} \right| \geq A \right) \\
        &\leq 2 \cdot \exp \left( - 2 t^2) \right)
    \end{align*}
    %
    If we now take a union bound over all $m \in \{ -N_1 \dots N_{k+1}, \dots, N_1 \dots N_{k+1} \}^d$, we can guarantee that
    %
    \begin{equation} \label{fourierdim3}
        \mathbf{P} \left( |\widehat{f}(m)| \leq \log(N_{k+1})/S\ \text{for all $m \in \{ -N, \dots, N\}^d$} \right) \geq 1 - 2^{d+1}/N^{c \log N - d}.
    \end{equation}
    %
    Since $\widehat{\nu_S}$ is $N_1 \dots N_{k+1}$ periodic, this means we can control all integer values of $\widehat{\nu_S}$ with high probability.

    Combining \eqref{fourierdim1}, \eqref{fourierdim2}, and \eqref{fourierdim3}, we conclude that there exists a constant $C$ such that with probability at least
    %
    \[ 1 - 2 \exp \left( \frac{-N^{d-s/n}}{A^{1/n} (\log N)^{1/n}} \right) - 1/\log N - \frac{2^{d+1}}{N^{c \log N - d}} \geq 1 - C / \log N, \]
    %
    the set $X$ avoids $K$, and for all $m \in \{ -N, \dots, N \}^d$,
    %
    \[ |\widehat{f}(m)| \leq \frac{C (\log N)^{1-1/n}}{N^{d-s/n}}. \qedhere \]



    Let
    %
    \[ p = \frac{1}{(N_{k+1}^{s + \varepsilon} \log(N_{k+1}))^{1/n}}, \]
    %
    and let $\{ X_Q \}$ be a family of independent and identically distributed $\{ 0, 1 \}$ valued Bernoulli random variables, for each $Q \in \DQ_{k+1}(T)$, such that $\PP(X_Q = 1) = p$. Then, define $S = \bigcup \{ Q : X_Q = 1 \}$. Then $\#(\DQ_{k+1}(S)) = \sum_Q X_Q$ is the sum of $\#(\DQ_k(T)) \cdot N_{k+1}^d$ independant and identically distributed random variables, and so Chernoff's inequality implies that
    %
    \[ \PP \left( \left| \#(\DQ_{k+1}(S)) - p \cdot \DQ_k(T) \cdot N_{k+1}^d \right| \leq \frac{p \cdot \# \DQ_k(T) \cdot N_{k+1}^d}{2} \right) \leq 10 e^{- p \DQ_k(T) N_{k+1}^d}. \]
    %
    Substituting in the value of $p$, we conclude
    %
    \begin{equation} \label{fourierdim1}
    \begin{split}
        \PP& \left( \left| \#(\DQ_{k+1}(S)) - \frac{\#(\DQ_k(T)) N_{k+1}^{\frac{dn - (s + \varepsilon)}{n}}}{\log(N_{k+1})^{1/n}} \right| \leq \frac{\#(\DQ_k(T)) N_{k+1}^{\frac{dn - (s + \varepsilon)}{n}}}{2 \log(N_{k+1})^{1/n}} \right)\\
        &\ \ \ \ \ \ \ \ \ \ \leq 10 \exp \left( \frac{- N_{k+1}^{\frac{dn - (s + \varepsilon)}{n}} \cdot \DQ_k(T)}{\log(N_{k+1})^{1/n}} \right)\\
        &\ \ \ \ \ \ \ \ \ \ \leq 10 \exp \left( \frac{-N_{k+1}^{\frac{dn - (s + \varepsilon)}{n}}}{\log(N_{k+1})^{1/n}} \right)
    \end{split}
    \end{equation}
    %
    Thus $S$ is the union of a large number of cubes, with high probability.

    Without loss of generality, removing cubes from $B$ if necessary, we may assume that every cube $Q_1 \times \dots \times Q_n \in \DQ_{k+1}(B)$, the values $Q_1, \dots, Q_n$ are distinct. Just as in Lemma

    In particular, given any such cube, we have
    %
    \[ \mathbf{P}(Q_1 \times \dots Q_n \subset S) = \mathbf{P}(X_{Q_1} = 1, \dots, X_{Q_n} = 1) = p^n. \]
    %
    Thus
    %
    \[ \mathbf{E}(\#(\DQ_{k+1}(B) \cap \DQ_{k+1}(S^n))) \leq N_{k+1}^{s+\varepsilon} p^n = \log(N_{k+1})^{-1}. \]
    %
    Markov's inequality implies
    %
    \begin{equation} \label{fourierdim2}
    \begin{split}
        \mathbf{P}(\DQ_{k+1}(B) \cap \DQ_{k+1}(S^n) \neq \emptyset) &= \mathbf{P}(\# (\DQ_{k+1}(B) \cap \DQ_{k+1}(S^n)) \geq 1)\\
        &\leq \log(N_{k+1})^{-1}.
    \end{split}
    \end{equation}
    %
    Thus $\DQ_{k+1}(S^n)$ is disjoint from $\DQ_{k+1}(B)$ with high probability.

    Now we analyze the Fourier transform of the measure $\nu_S$. For each cube $Q \in \DR_{k+1}(T)$, we can define
    %
    \[ A_Q = \nu_S(a(Q)) - (\eta_{k+1} * \nu_T)(a(Q)) = \begin{cases} \frac{1}{\#(\DQ_{k+1}(S))} - \frac{1}{N_{k+1}^d \#(\DQ_k(T))} &: X_Q = 1, \\ - \frac{1}{ N_{k+1}^d \#(\DQ_k(T))} &: X_Q = 0. \end{cases} \]
    %
    Then for each $m \in \mathbf{Z}^d$,
    %
    \[ \widehat{\nu_S}(m) - \widehat{\eta_{k+1}}(m) \widehat{\nu_T}(m) = \sum_{Q \in \DQ_{k+1}(T)} A_Q e^{-\frac{2 \pi i m \cdot a(Q)}{N_1 \dots N_{k+1}}}. \]
    %
    We calculate that for each $Q \in \DQ_{k+1}(T)$,
    %
    \begin{align*}
        \EE[A_Q|\#(\DQ_k(S))] &= \frac{\PP (X_Q = 1 | \#(\DQ_k(S)))}{\#(\DQ_{k+1}(S))} - \frac{1}{N_{k+1}^d \#(\DQ_{k+1}(T))}\\
        &= \frac{\#(\DQ_{k+1}(S)) / N_{k+1}^d \#(\DQ_{k+1}(T))}{\#(\DQ_{k+1}(S))} - \frac{1}{N_{k+1}^d \#(\DQ_{k+1}(T))} = 0.
    \end{align*}
    %
    In particular, $\EE[A_Q] = 0$. Now for each $q \geq 1$,
    %
    \begin{align*}
        \EE[A_Q^q|\DQ_{k+1}(S)] &= \frac{\PP(X_Q = 1 | \DQ_{k+1}(S))}{\#(\DQ_{k+1}(S))^q} &= \frac{1}{\#(\DQ_{k+1}(T)) \cdot \#(\DQ_{k+1}(S))^{q-1}}.
    \end{align*}
    %
    Thus
    %
    \begin{align*}
        \EE[A_Q^q] &= \frac{1}{\#(\DQ_{k+1}(T))} \EE \left[ \frac{1}{\#(\DQ_{k+1}(S))^{q-1}} \right]\\
        &\leq s
    \end{align*}


    Now fix $m \in \{ -N_1 \dots N_{k+1}, N_1 \dots N_{k+1} \}^d$.



    We can then apply Hoeffding's inequality to conclude that for each $t > 0$,
    %
    \begin{align*}
        \PP \left( |\widehat{\nu_S}(m)| \geq \frac{t p \cdot \#(\DQ_{k+1}(T))^{1/2}}{\#(\DQ_{k+1}(S))} \right) &= \PP \left( \sum \left| \#(\DQ_{k+1}(S)) A_Q e^{-\frac{2 \pi i m \cdot a(Q)}{N_1 \dots N_{k+1}}} \right| \geq A \right) \\
        &\leq 2 \cdot \exp \left( - 2 t^2) \right)
    \end{align*}
    %
    If we now take a union bound over all $m \in \{ -N_1 \dots N_{k+1}, \dots, N_1 \dots N_{k+1} \}^d$, we can guarantee that
    %
    \begin{equation} \label{fourierdim3}
        \mathbf{P} \left( |\widehat{f}(m)| \leq \log(N_{k+1})/S\ \text{for all $m \in \{ -N, \dots, N\}^d$} \right) \geq 1 - 2^{d+1}/N^{c \log N - d}.
    \end{equation}
    %
    Since $\widehat{\nu_S}$ is $N_1 \dots N_{k+1}$ periodic, this means we can control all integer values of $\widehat{\nu_S}$ with high probability.

    Combining \eqref{fourierdim1}, \eqref{fourierdim2}, and \eqref{fourierdim3}, we conclude that there exists a constant $C$ such that with probability at least
    %
    \[ 1 - 2 \exp \left( \frac{-N^{d-s/n}}{A^{1/n} (\log N)^{1/n}} \right) - 1/\log N - \frac{2^{d+1}}{N^{c \log N - d}} \geq 1 - C / \log N, \]
    %
    the set $X$ avoids $K$, and for all $m \in \{ -N, \dots, N \}^d$,
    %
    \[ |\widehat{f}(m)| \leq \frac{C (\log N)^{1-1/n}}{N^{d-s/n}}. \qedhere \]
\end{comment}
\end{proof}

The remainder of the construction follows essentially the construction of the configuration avoiding set in Chapter \ref{ch:RoughSets}, obtaining a family of discretized sets $\{ X_k \}$ with $X = \bigcap X_k$. The only difference is we do not `terminate randomness' at each application of the discrete lemma, instead just choosing intervals randomly at each stage and not worrying whether the sets we choose actually satisfy Properties (A) and (B) at each stage. Nonetheless, we know that these properties are satisfied with large probability, and so provided the parameters $\{ N_k \}$ and $\varepsilon_k$ satisfy
%
\[ \sum_{k = 1}^\infty \frac{1}{N_{k+1}^\varepsilon} + \left( \left\lceil \frac{A(d,n,s)}{\varepsilon_k} \right\rceil ! \right)^3 \frac{(N_1 \dots N_{k-1})^d}{N_k^d} < \infty, \]
%
the Borel-Cantelli lemma implies that almost surely, Properties (A) and (B) of Lemma \ref{discreteFourierBuildingBlock} hold at all sufficiently large stages of the construction. This is sufficient to guarantee that the set $X$ avoids the configuration $\C$. We will show that under the additional constraint that for each $\varepsilon > 0$,
%
\begin{equation} \label{equation41028694692} (N_1 \dots N_k) \lesssim_\varepsilon N_{k+1}^\varepsilon, \end{equation}
%
the set $X$ has Fourier dimension $(nd - s)/n$.

We must now construct a measure on $X$ with the appropriate Fourier decay. The measures $\nu_S$ do not `quite' suffice for this purpose, since their Fourier transforms do not actually have any decay. We will fix this by convolving $\nu_S$ with an appropriate bump function. Let $\psi$ be a compactly supported, positive, smooth function supported on $[0,1]^d$ with
%
\[ \int \psi(x)\; dx = 1. \]
%
For each $k$, let
%
\[ \psi_k(x) = (N_1 \dots N_k)^d \psi(N_1 \dots N_k x). \]
%
Then $\psi_k$ is supported on $[0,1/N_1 \dots N_k]^d$, $\int \psi_k(x)\; dx = 1$, and for any $\xi \in \RR^d$,
%
\[ \widehat{\psi_k}(\xi) = \widehat{\psi}(\xi/N_1 \dots N_k). \]
%
Note that, because $\psi$ is smooth and compactly supported, for any $m > 0$,
%
\[ |\widehat{\psi_k}(\xi) \lesssim_m (N_1 \dots N_k)^m / |\xi|^m. \]
%
An important heuristic, a variant of the Heisenberg uncertainty principle, is that $\widehat{\psi_k}$ is `essentially' supported on the box $[-N_k,N_k]^d$.

\begin{lemma} \label{lemma09679009341}
    Suppose that \eqref{equation41028694692} holds. For any $\DQ_k$ discretized set $E$, the measure $\mu_E = \nu_E * \psi_k$ is supported on $E$, and for each $\varepsilon > 0$, there is a universal constant $C_\varepsilon$, independant of $k$ and $E$, such that if $|\xi| \geq N_k^{1 + \varepsilon}$, then
    %
    \[ |\widehat{\mu_E}(\xi)| \leq C_\varepsilon |\xi|^{-d/2}. \]
\end{lemma}
\begin{proof}
    It is simple to see that $\mu_E$ is supported on $E$, since by general properties of convolution it is supported on the sum of the support of $\psi_k$, which is $[0,1/N_k]^d$, and the support of $\nu_E$, which is the union of the left-hand corners of the cubes in $\DQ_k(E)$. We also note that
    %
    \[ \widehat{\mu_E}(\xi) = \widehat{\nu_E}(\xi) \widehat{\psi}(\xi/N_k). \]
    %
    Since $\nu_E$ is a probability measure, $\| \widehat{\nu_E} \|_{L^\infty(\RR^d)} = 1$. Now pick $m \geq d(1 + 1/\varepsilon)$. If $|\xi| \geq N_k^{1 + \varepsilon}$, we calculate, using \eqref{equation41028694692}, to conclude
    %
    \begin{align*}
        |\widehat{\mu_E}(\xi)| &\leq |\widehat{\psi}(\xi/N_k)| \lesssim_m \frac{(N_1 \dots N_k)^m}{|\xi|^m} \lesssim_\varepsilon \frac{N_k^{m(1 + \varepsilon/2)}}{|\xi|^{m - d/2}} \frac{1}{|\xi|^{d/2}} \leq \frac{1}{|\xi|^{d/2}}. \qedhere
    \end{align*}
\end{proof}

\begin{lemma}
    Almost surely, for each $\varepsilon > 0$, there exists a constant $C_\varepsilon > 0$ and an integer $K > 0$, such that for any integer $k \geq K$, and any $m \in \ZZ^d$,
    %
    \[ |\widehat{\mu_{X_k}}(m)| \leq C_\varepsilon k \cdot |m|^{- \left( \frac{dn - s}{2n} - \varepsilon \right)}. \]
\end{lemma}
\begin{proof}
    By the Borel-Cantelli lemma, almost surely, there exists some $K$ such that Property (A) and (B) of Lemma \ref{discreteFourierBuildingBlock} is satisfied at every stage of the construction past the $K$th stage. Without loss of generality, by choosing a larger $K$ if necessary, that $\varepsilon_k \leq \varepsilon [2n/(dn - s + 2)]$. Because $\mu_{X_K}$ is smooth and compactly supported, there is certainly a constant $C_\varepsilon$ such that for any $m \in \ZZ^d$,
    %
    \[ |\widehat{\mu_{X_K}}(m)| \leq C_\varepsilon |m|^{- \left( \frac{dn - s}{2n} - \varepsilon \right)} \leq C_\varepsilon K \cdot |m|^{- \left( \frac{dn - s}{2n} - \varepsilon \right)}. \]
    %
    This constant should also be chosen larger than the constant in Lemma \ref{lemma09679009341}. The remainder of the argument is established by induction. Suppose that for each $m \in \ZZ^d$,
    %
    \[ |\widehat{\mu_{X_l}}(m)| \leq C_\varepsilon l \cdot |m|^{- \left( \frac{dn - s}{2n} - \varepsilon \right)}. \]
    %
    For $|m| \leq N_{l+1}^{1 + \varepsilon}$,
    % (1-\varepsilon_k) \frac{dn - s - 2\varepsilon_k}{2n} \geq (dn - s)/(2n) - \varepsilon
    \begin{align*}
        |\widehat{\mu_{X_{l+1}}}(m)| &\leq N_{l+1}^{-(1-\varepsilon_{l+1}) \frac{dn - s - 2\varepsilon_{l+1}}{2n}} + C_\varepsilon l \cdot |m|^{- \left( \frac{dn - s}{2n} - \varepsilon \right)}\\
        &\leq (l + 1) |m|^{- \left( \frac{dn - s}{2n} - \varepsilon \right)}.
    \end{align*}
    %
    If $|m| \geq N_{k+1}^{1 + \varepsilon}$, we can apply Lemma \ref{lemma09679009341} to conclude that
    %
    \[ |\widehat{\mu_{X_{l+1}}}(m)| \leq C_\varepsilon |m|^{- \left( \frac{dn - s}{2n} - \varepsilon \right)} \leq C_\varepsilon (l+1) |m|^{\left( \frac{dn - s}{2n} - \varepsilon \right)}. \]
    %
    This completes the proof.
\end{proof}

The same argument as in Theorem \ref{massdistributionprinciplelem} shows that the sequence $\{ \mu_{X_k} \}$ is a Cauchy sequence under the weak $*$ topology. If we form the limit $\mu_X = \lim_{k \to \infty} \mu_{X_k}$, then the Fourier transforms $\widehat{\mu_{X_k}}$ converge pointwise to $\widehat{\mu_X}$. In particular, for each $\varepsilon > 0$ and $n \in \ZZ^d$, we conclude that
%
\[ |\widehat{\mu_X}(n)| \leq \limsup |\widehat{\mu_{X_k}}(n)| \leq C_\varepsilon |n|^{- \left( \frac{nd - s}{n} - \varepsilon \right)}. \]
%
Combined with Lemma \ref{discretefouriermeasures}, this implies $X$ almost surely has Fourier dimension $(nd - s)/n$.

\endinput