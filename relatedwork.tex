%% The following is a directive for TeXShop to indicate the main file
%%!TEX root = diss.tex

\chapter{Related Work}
\label{ch:RelatedWork}

Here, we discuss the main papers which influenced our results. In particular, the work of Keleti on translate avoiding sets, Fraser and Pramanik's work on sets avoiding smooth configurations, and Math\'{e}'s result on sets avoiding algebraic varieties. 

\begin{comment}
and Schmerkin's result on sets with large Fourier dimension avoiding smooth configurations.
\end{comment}

\section{Keleti: A Translate Avoiding Set}

Keleti's two page paper constructs a full dimensional set $X \subset [0,1]$ such that for each $t \neq 0$, $X$ intersects $t + X$ in at most one place. The set $X$ is then said to \emph{avoid translates}. This paper contains the core idea behind the \emph{discretization} method adapted in Fraser and Pramanik's paper. We also adapt this technique in our paper, which makes the result of interest.

\begin{lemma}
    Let $X$ be a set. Then $X$ avoids translates if and only if there do not exists values $x_1 < x_2 \leq x_3 < x_4$ in $X$ with $x_2 - x_1 = x_4 - x_3$. In particular, a set $X$ avoids translates if and only if it avoids the four point configuration
    %
    \[ \C = \{ (x_1, x_2, x_3, x_4): x_1 < x_2 \leq x_3 < x_4,\ x_2 - x_1 = x_4 - x_3 \}. \]
\end{lemma}
\begin{proof}

    Suppose $(t + X) \cap X$ contains two points $a < b$. Without loss of generality, we may assume that $t > 0$. If $a \leq b - t$, then the equation
    %
    \[ a - (a - t) = t = b - (b - t) \]
    %
    satisfies the constraints, since $a - t < a \leq b - t < b$ are all elements of $X$. We also have
    %
    \[ (b - t) - (a - t) = b - a, \]
    %
    which satisfies the constraints if $a - t < b - t \leq a < b$. This covers all possible cases. Conversely, if there are $x_1 < x_2 \leq x_3 < x_4$ in $X$ with
    %
    \[ x_2 - x_1 = t = x_4 - x_3, \]
    %
    then $X + t$ contains $x_2 = x_1 + (x_2 - x_1)$ and $x_4 = x_3 + (x_4 - x_3)$.
\end{proof}

%\footnote{We always assume $L_n/L_{n+1}$ is an integer so that intervals in $\mathcal{B}(L_n)$ are either almost disjoint from intervals in $\mathcal{B}(L_{n+1})$ or contained completely within such an interval}

The basic, but fundamental idea of the interval dissection technique is to introduce memory into Cantor set constructions. Keleti constructs a nested family of discrete sets $\{ X_k \}$, with $X_k$ a $\DQ_k$ discretized set, and with $X = \bigcap X_k$. The sequence $\{ N_k \}$ will be specified later, but each $N_k$ will be a multiple of 10. We initialize $X_0 = [0,1]$. The novel feature of the argument is to add a queue of intervals to the construction, initially just containing $[0,1]$. To construct the sequence $\{ X_k \}$, Keleti iteratively performs the following procedure:
%
\begin{algorithm}[H]
    \begin{algorithmic}%[1]
        \caption{Construction of the Sets $\{ X_k \}$:}
        \State{Set $k = 0$.}
        \MRepeat
            \State{Take off an interval $I$ from the front of the queue.}

            \MForAll{\ $J \in \DQ_k(X_k)$:}
                \State{Order the intervals in $\DQ_{k+1}(J)$ as $J_1, \dots, J_N$.}

                \State{{\bf If} $J \subset I$, add all intervals $J_i$ to $X_{k+1}$ with $i \equiv 0$ modulo 10.}
                \State{{\bf Else} add all $J_i$ with $i \equiv 5$ modulo 10.}
            \EndForAll
            \State{Add all intervals in $\DQ_{k+1}^d$ to the end of the queue.}
            \State{Increase $k$ by 1.}
        \EndRepeat   
    \end{algorithmic}
\end{algorithm}

Each iteration of the algorithm produces a new set $X_k$, and so leaving the algorithm to repeat infinitely produces a sequence $\{ X_k \}$ whose intersection is $X$.

\begin{lemma}
    The set $X$ is translate avoiding.
\end{lemma}
\begin{proof}
    If $X$ is not translate avoiding, there is $x_1 < x_2 \leq x_3 < x_4$ with $x_2 - x_1 = x_4 - x_3$. Since $l_k \to 0$, there is a suitably large integer $N$ such that $x_1$ is contained in an interval $I \in \DQ_N$ not containing $x_2,x_3$, or $x_4$. At stage $N$ of the algorithm, the interval $I$ is added to the end of the queue, and at a much later stage $M$, the interval $I$ is retrieved. Find the startpoints $x_1^\circ, x_2^\circ$, $x_3^\circ, x_4^\circ \in l_M \mathbf{Z}$ to the intervals in $\DQ_M$ containing $x_1$, $x_2$, $x_3$, and $x_4$. Then we can find $n$ and $m$ such that $x_4^\circ - x_3^\circ = (10n)l_M$, and $x_2^\circ - x_1^\circ = (10m + 5)l_M$. In particular, this means that $|(x_4^\circ - x_3^\circ) - (x_2^\circ - x_1^\circ)| \geq 5L_M$. But
    %
    \begin{align*}
        |(x_4^\circ - x_3^\circ) - (x_2^\circ - x_1^\circ)| &= |[(x_4^\circ - x_3^\circ) - (x_2^\circ - x_1^\circ)] - [(x_4 - x_3) - (x_2 - x_1)]|\\
        &\leq |x_1^\circ - x_1| + \dots + |x_4^\circ - x_4| \leq 4 L_M
    \end{align*}
    %
    which gives a contradiction.
\end{proof}

It is easy to see from the algorithm that
%
\[ \# (\DQ_k(X_k)) = (N_k/10) \cdot \#(\DQ_{k-1}(X_{k-1})). \]
%
Closing the recursive definition shows
%
\[ \#(\DQ_k(X_k)) = \frac{1}{10^k l_k}. \]
%
In particular, this means $|X_k| = 1/10^k$, so $|X|$ has Lebesgue measure zero regardless of how we choose the parameters $\{ N_k \}$. 

\begin{theorem}
    For some sequence $\{ N_k \}$, the set $X$ has full Hausdorff dimension.
\end{theorem}
\begin{proof}
    Set $N_k = 10 M_k$ for each $k \geq 1$, then one sees that for each $R \in \DR_k(X_k)$, $\#(\DQ_{k+1}(R \cap X_{k+1})) = 1$. Thus Property \ref{SingleSelection} of Theorem \ref{TheConstructionTheorem} is satisfied. Furthermore, Property \ref{ChangeofScales} of Theorem \ref{TheConstructionTheorem} is satisfied with $s = 1$. We conclude that if $N_1 \dots N_k \lesssim_\varepsilon N_{k+1}^\varepsilon$, then all assumptions of Theorem \ref{TheConstructionTheorem} is satisfied, and we conclude $X$ is a set with full Hausdorff dimension. This is true, for instance, if we set $N_k = 2^{2^{\lfloor k \log k \rfloor}}$.
\end{proof}

The most important feature of Keleti's argument is his reduction of a non-discrete configuration avoidance problem to a sequence of discrete avoidance problems on cubes. Let us summarize the result of Keleti's discrete argument in a lemma.

\begin{lemma} \label{KeletiDiscreteLemma}
    Let $T_1, T_2 \subset \RR$ be disjoint, $\DQ_k$ discretized sets. If $M_{k+1} = N_{k+1}/10$, then we can find $S_1 \subset T_1$ and $S_2 \subset T_2$ such that
    %
    \begin{enumerate}
        \item[(i)] For each $k$, $S_k$ is a $\DQ_{k+1}$ discretized subset of $T_k$.
        \item[(ii)] If $x_1 \in S_1$ and $x_2,x_3,x_4 \in S_2$, then $x_2 - x_1 \neq x_4 - x_3$.
        \item[(iii)] For each $i$, and each cube $R \in \DR_{k+1}(S_i)$, $\#(\DQ_{k+1}(R \cap S_i)) = 1$.
    \end{enumerate}
\end{lemma}
%
Property (i) of Lemma \ref{KeletiDiscreteLemma} allows Keleti to apply his argument iteratively at each Dyadic scale. Property (ii) implies Keleti obtains a configuration avoiding set by iterative the argument infinitely many times. And the reason why Keleti obtains a set with full Hausdorff dimension, relating back to Property \ref{ChangeofScales} in Theorem \ref{TheConstructionTheorem}, is because of Property (iii), which allows us to select $N_{k+1} \lesssim M_{k+1}$.

Of course, it is not possible to extend the discrete solution of the configuration argument to general configurations; this part of Keleti's method strongly depends on the arithmetic structure of the configuration. The iterative application of a discrete solution, however, can be applied in generality. Combined with Theorem \ref{TheConstructionTheorem}, this technique gives a powerful method to reduce configuration avoidance problems about Hausdorff dimension to discrete avoidance problems on cubes.







\section{Fraser/Pramanik: Smooth Configurations}

Inspired by Keleti's result, Pramanik and Fraser obtained a generalization of the queue method which allows one to find sets avoiding $n+1$ point configurations given by the zero sets of smooth functions, i.e.
%
\[ \C = \{ (x_1, \dots, x_n) \in \C^n[0,1]^d : f(x_0, \dots, x_n) = 0 \}, \]
%
under mild regularity conditions on the function $f: \C^n[0,1]^d \to [0,1]^m$. Fraser and Pramanik implement a simple pidgeonholing strategy.

\begin{theorem}[Pramanik and Fraser] \label{pramanikandfrasertheorem}
    Fix $m \leq d(n-1)$. Consider a countable family of $C^2$ functions $\{ f_k : [0,1]^{dn} \to [0,1]^m \}$ such that for each $k$, $Df_k$ has full rank at any $(x_1, \dots, x_n) \in \C^n[0,1]^d$ where $f_k(x_1, \dots, x_n) = 0$. Then there exists a set $X \subset \RR^d$ with Hausdorff dimension $m/(n-1)$ such that $X$ avoids the configuration
    %
    \[ \C = \bigcup_k \{ (x_1, \dots, x_n) \in \C^n(\RR^d): f_k(x_1, \dots, x_n) = 0 \}. \]
\end{theorem}

\begin{remark}
    For simplicity, we only prove the result for a single function, rather than a countable family of functions. The only major difference between the two approaches is the choice of scales we must choose later on in the argument, and a slight modification of the queueing argument.
\end{remark}

Pramanik and Fraser also simplify to a discrete version of their problem, which they then iteratively apply. In the discrete setting, rather than making a linear shift in one of the intervals we avoid as in Keleti's approach, one must use the smoothness properties of the function to find large segments of an interval avoiding. Corollary \ref{PramanikFraserBuildingBlockLemma} gives the discrete solution that Pramanik and Fraser utilizes.

\begin{lemma} \label{Lemma315091513}
    Fix $n > 1$. Let $T \subset [0,1]^d$ and $T' \subset [0,1]^{(n-1)d}$ be $\DQ_k$ discretized sets. Let $B \subset T \times T'$ be $\DQ_{k+1}$ discretized. Then there exists a $\DQ_{k+1}$ discretized set $S \subset T$, and a $\DQ_{k+1}$ discretized set $B' \subset T'$, such that
    %
    \begin{enumerate}
        \item \label{dimensionReductionProperty} $(S \times T') \cap B \subset S \times B'$.

        \item \label{bigProperty} For every $Q \in \DQ_k^d$, there exists $\mathcal{R}(Q) \subset \DR_{k+1}^d(Q)$, such that
        %
        \[ \#(\mathcal{R}(Q)) \geq (1/2) \cdot \#(\DR_{k+1}^d(Q)). \]
        %
        and for each $R \in \DR_{k+1}^d(Q)$,
        %
        \[ \#(\DQ_{k+1}(R)) = \begin{cases} 1 &: R \in \mathcal{R}(Q), \\ 0 &: R \not \in \mathcal{R}(Q). \end{cases} \]

        \item \label{BBoundProperty} $\#(\DQ_{k+1}(B')) \leq 2 (N_1 \dots N_k)^d \left( M_{k+1}/N_{k+1} \right)^d \cdot \#(\DQ_{k+1}(B))$.
    \end{enumerate}
\end{lemma}
\begin{proof}
    Fix $Q_0 \in \DQ_k(T)$. For each $R \in \DR_{k+1}(Q_0)$, define a \emph{slab} $S[R] = R \times T'$, and for each $Q \in \DQ_{k+1}(Q_0)$, define a \emph{wafer} $W[Q] = Q \times T'$. We say a wafer $W[Q]$ is \emph{good} if
    %
    \begin{equation} \label{equation10291095429062}
        \#(\DQ_{k+1}(W[Q] \cap B)) \leq (2/N_{k+1}^d) \cdot \#(\DQ_{k+1}(B)),
    \end{equation}
    %
    Then at most $N_{k+1}^d/2$ wafers are bad. We call a slab \emph{good} if it contains a wafer which is good. Since a slab is the union of $(N_{k+1}/M_{k+1})^d$ wafers, at most $M_{k+1}^d /2 = (1/2) \cdot \#(\DR_{k+1}(Q_0))$ slabs are bad. Thus if we set
    %
    \[ \mathcal{R}(Q_0) = \{ R \in \DR_{k+1}(Q_0) : S[R]\ \text{is good} \}, \]
    %
    then
    %
    \begin{equation} \label{equation24016590369046}
        \#(\mathcal{R}(Q_0)) \geq (1/2) \cdot \#(\DR_{k+1}(Q_0)).
    \end{equation}
    %
    For each $R \in \mathcal{R}(Q_0)$, we pick $Q_R \in \DQ_{k+1}(R)$ such that $W[Q_R]$ is good, and define
    %
    \[ S = \bigcup \{ Q_R : R \in \mathcal{R}(Q_0) \}. \]
    %
    Equation \eqref{equation24016590369046} implies $S$ satisfies Property \ref{bigProperty}.

    Let $B'$ be the union of all cubes $Q' \in \DQ_{k+1}(T')$ such that there is $Q \in \DQ_{k+1}(S)$ with $Q \times Q' \in \DQ_{k+1}(B)$. By definition, Property \ref{dimensionReductionProperty} is then satisfied. For each $Q \in \DQ_{k+1}(S)$, $W[Q]$ is good, so \eqref{equation10291095429062} implies
    %
    \[ \# \{ Q' : Q \times Q' \in \DQ_{k+1}(B) \} \leq (2/N_{k+1}^d) \cdot \#(\DQ_{k+1}(B)). \]
    %
    But $\#(\DQ_{k+1}(S)) \leq \#(\DR_{k+1}(T)) \leq (1/r_{k+1})^d = (N_1 \dots N_k)^d M_{k+1}^d$, so
    %
    \begin{align*}
        \#(\DQ_{k+1}(B')) &\leq \#(\DQ_{k+1}(S))[(2/N_{k+1}^d) \cdot \#(\DQ_{k+1}(B))]\\
        &\leq 2(N_1 \dots N_k)^d (M_{k+1}/N_{k+1})^d \#(\DQ_{k+1}(B)),
    \end{align*}
    %
    which establishes Property \ref{BBoundProperty}.
\end{proof}

We apply the lemma recursively $n-1$ times to continually reduce the dimensionality of the avoidance problem we are considering. Eventually, we obtain the case where $n = 0$, and then avoiding the configuration is easy.

\begin{lemma} \label{Lemma1209410535}
    Fix $n > 1$. Let $T \subset [0,1]^d$ be $\DQ_k$ discretized, and let $B \subset T$ be $\DQ_{k+1}$ discretized. Suppose
    %
    \[ \# \DQ_{k+1}(B) \leq \left[ C \cdot 2^{n-1} (N_1 \dots N_k)^{d(n-1)} (M_{k+1}/N_{k+1})^{d(n-1)} \right] (1/l_{k+1})^{dn - m}. \]
    %
    and
    %
    \begin{equation} \label{equation903103513095}
        N_{k+1} \geq \left[ C \cdot 2^n (N_1 \dots N_k)^{2dn} \right]^{1/m} M_{k+1}^{d(n-1)/m}
    \end{equation}
    %
    Then there exists a $\DQ_{k+1}$ discretized set $S \subset T$ such that
    %
    \begin{enumerate}
        \item $S \cap B = \emptyset$.
        \item \label{badsetproperty5} For each $Q_0 \in \DQ_k(T)$, there is $\mathcal{R}(Q_0) \subset \DR_{k+1}(Q_0)$ with
        %
        \[ \#(\mathcal{R}(Q_0)) \geq (1/2) \#(\DR_{k+1}(Q_0)), \]
        %
        such that for each $R \in \DR_{k+1}(Q_0)$,
        %
        \[ \#(\mathcal{Q}_{k+1}(R \cap S)) = \begin{cases} 1 &: R \in \mathcal{R}(Q_0), \\ 0 &: R \not \in \mathcal{R}(Q_0). \end{cases} \]
    \end{enumerate}
\end{lemma}
\begin{proof}
    For each $Q_0 \in \DQ_k(T)$, we set
    %
    \[ \mathcal{R}(Q_0) = \{ R \in \DR_{k+1}(Q_0) : \#(\DQ_{k+1}(R \cap B)) \leq (2/M_{k+1}^d) \cdot \#(\DQ_{k+1}(B)) \}. \]
    %
    Since $\DR_{k+1}(Q_0) = M_{k+1}^d$,
    %
    \[ \#(\mathcal{R}(Q_0)) \geq \#(\DR_{k+1}(Q_0)) - (M_{k+1}^d/2) \geq (1/2) \cdot \#(\DR_{k+1}(T)). \]
    %
    Now \eqref{equation903103513095} implies that for each $R \in \mathcal{R}(Q_0)$,
    %
    \begin{align*}
        \#(\DQ_{k+1}(R \cap B)) &\leq (2/M_{k+1}^d) \cdot \#(\DQ_{k+1}(B))\\
        &\leq (2/M_{k+1}^d) \left(C \cdot 2^{n-1} (N_1 \dots N_k)^{2dn} (M_{k+1}/N_{k+1})^{d(n-1)} \right).\\
        &= \left[ 2^n C (N_1 \dots N_k)^{2dn} \right] \left( M_{k+1}^{d(n-2)} / N_{k+1}^{m-d} \right)\\
        &< (N_{k+1}/M_{k+1})^d\\
        &= \DQ_{k+1}(R)
    \end{align*}
    %
    Thus for each $R \in \mathcal{R}(Q_0)$, we can find $Q_R \in \DQ_{k+1}(R)$ such that $Q_R \cap B = \emptyset$. And so if we set
    %
    \[ S = \bigcup \{ Q_R : R \in \mathcal{R}(Q_0), Q_0 \in \DQ_k(T) \}, \]
    %
    then (A) and (B) are satisfied.
\end{proof}

\begin{corollary} \label{PramanikFraserBuildingBlockLemma}
    Let $f: [0,1]^{dn} \to [0,1]^m$ be $C^2$, and have full rank at every point $(x_1, \dots, x_n) \in \C^n(\RR^d)$ such that $f(x_1, \dots, x_n) = 0$. Then there exists a universal constant $C$ depending only on $f$ such that, if \eqref{equation903103513095} is satisfied, then for any disjoint, $\DQ_k$ discretized sets $T_1, \dots, T_n \subset [0,1]^d$, we can find $\DQ_{k+1}$ discretized sets $S_1 \subset T_1, \dots, S_n \subset T_n$ such that
    %
    \begin{enumerate}
        \item If $x_1 \in S_1, \dots, x_n \in S_n$, then $f(x_1, \dots, x_n) \neq 0$.
        \item For each $k$, and for each $Q_0 \in \DQ_k(T_k)$, there is $\mathcal{R}(Q_0) \subset \mathcal{R}_{k+1}(Q_0)$ with
        %
        \[ \#(\mathcal{R}(Q_0)) \geq (1/2) \cdot \#(\mathcal{R}_{k+1}(Q_0)), \]
        %
        and for each $R \in \DR_{k+1}(Q_0)$,
        %
        \[ \#(\mathcal{Q}_{k+1}(R \cap S)) = \begin{cases} 1 &: R \in \mathcal{R}(Q_0), \\ 0 &: R \not \in \mathcal{R}(Q_0). \end{cases} \]
    \end{enumerate}
\end{corollary}
\begin{proof}
    Since $f$ is $C^2$ and has full rank on the set
    %
    \[ V(f) = \{ (x_1, \dots, x_n) \in \C^n(\RR^d) : f(x_1, \dots, x_n) = 0 \}, \]
    %
    the implicit function theorem implies $V(f)$ is a smooth manifold of dimension $nd - m$ in $\RR^{dn}$, and the coarea formula implies the existence of a constant $C$ such that for each $k$,
    %
    \[ \# \{ Q \in \DQ_k^{dn} : Q \cap V(f) \neq \emptyset \} \leq C/l_k^{dn-m}. \]
    %
    To apply Lemma \ref{Lemma315091513} and \ref{Lemma1209410535}, we set
    %
    \[ B = \# \{ Q \in \DQ_k^{dn} : Q \cap V(f) \neq \emptyset \}. \]
    %
    Applying Lemma \ref{Lemma315091513} iteratively $n-1$ times, then finishing with an application of Lemma $\ref{Lemma1209410535}$ constructs the required sets $S_1, \dots, S_n$.
\end{proof}

Just like in Keleti's proof, Pramanik and Fraser's technique applies a discrete result, Corollary \ref{PramanikFraserBuildingBlockLemma}, iteratively at many scales, with the help of a queueing process, to obtain a high dimensional set avoiding the zeroes of a function. We construct a nested family $\{ X_k : k \geq 0 \}$ of $\DQ_k$ discretized sets, converging to a set $X$, which we will show is translate avoiding. We initialize $X_0 = [0,1]$. Our queue shall consist of $n$ tuples of disjoint intervals $(T_1, \dots, T_n)$, all of the same length, which initially consists of all possible tuples of intervals in $\DQ_1^d([0,1]^d)$. To construct the sequence $\{ X_k \}$, we perform the following iterative procedure:
%
\begin{algorithm}[H]
    \begin{algorithmic}
        \caption{Construction of the Sets $\{ X_k \}$}
        \State{Set $k = 0$}
        \MRepeat
            \State{Take off an $n$ tuple $(T_1', \dots, T_n')$ from the front of the queue}
            \State{Set $T_i = T_i' \cap X_k$ for each $i$}
            \State{Apply Corollary \ref{PramanikFraserBuildingBlockLemma} to the sets $T_1, \dots, T_d$, obtaining $\DQ_{k+1}$ discretized sets $S_1, \dots, S_n$ satisfying Properties (A), (B), and (C) of that Lemma.}
            \State{Set $X_{k+1} = X_k - \bigcup_{i = 1}^n T_i - S_i$.}
            \State{Add all $n$ tuples of disjoint cubes $(T_1', \dots, T_n')$ in $\DQ_{k+1}^d(X_{k+1})$ to the back of the queue.}
            \State{Increase $k$ by 1.}
        \EndRepeat   
    \end{algorithmic}
\end{algorithm}

\begin{lemma}
    The set $X$ constructed by the procedure avoids the configuration
    %
    \[ \C = \{ (x_1, \dots, x_n) \in \C^n(\RR^d) : f(x_1, \dots, x_n) = 0 \}. \]
\end{lemma}
\begin{proof}
    Suppose $x_1, \dots, x_n \in X$ are distinct. Then at some stage $k$, $x_1, \dots, x_n$ lie in disjoint cubes $T_1', \dots, T_n' \in \DQ_k^d(X_k)$, for some large $k$. At this stage, $(T_1', \dots, T_n')$ is added to the back of the queue, and therefore, at some much later stage $N$, the tuple $(T_1', \dots, T_n')$ is taken off the front. Sets $S_1 \subset T_1', \dots, S_n \subset T_n'$ are constructed satisfying Property (A) of Corollary \ref{PramanikFraserBuildingBlockLemma}. Since $x_1, \dots, x_n \in X$, we must have $x_i \in S_i$ for each $i$, so $f(x_1, \dots, x_n) \neq 0$.
\end{proof}

What remains is to show that for some sequence of parameters $\{ N_k \}$ and $\{ M_k \}$ satisfying \eqref{equation903103513095}, we can apply Theorem \ref{TheConstructionTheorem}. The conclusion of Corollary \ref{PramanikFraserBuildingBlockLemma} shows Property \ref{SingleSelection} of Theorem \ref{TheConstructionTheorem} is always satisfied. If we set
%
\[ N_{k+1} = \left\lceil \left[ C \cdot 2^n (N_1 \dots N_k)^{2dn} \right]^{1/m} M_{k+1}^{d(n-1)/m} \right\rceil, \]
%
then \eqref{equation903103513095} holds, so we can apply Lemma \ref{pramanikandfrasertheorem} at each scale. Properties \ref{RapidDecrease} and \ref{ChangeofScales} of Theorem \ref{TheConstructionTheorem} are satisfied with $s = m/d(n-1)$ if $N_1 \dots N_k \lesssim_\varepsilon M_{k+1}^\varepsilon$ for each $\varepsilon > 0$. This is true, for instance, if we set $M_k = 2^{\lfloor 2^{k \log k} \rfloor}$, and Theorem \ref{TheConstructionTheorem} then shows the resultant set $X$ has Hausdorff dimension $m/(n-1)$.

\section{Math\'{e}: Polynomial Configurations}

Math\'{e}'s result constructs sets avoiding algebraic varieties.

\begin{theorem}[Math\'{e}]
    Let $\{ f_k: \RR^{nd} \to \mathbf{R} \}$ be a countable family of rational coefficient polynomials with degree at most $m$. Then there exists a set $X \subset [0,1]^d$ with Hausdorff dimension $d/m$ which avoids the configuration
    %
    \[ \C = \bigcup_k \{ (x_1, \dots, x_n) \in \C^n(\RR^d) : f_k(x_1, \dots, x_n) = 0 \}. \]
\end{theorem}

Originally, Math\'{e}'s result does not explicitly use a discretization method analogous to Keleti and Pramanik and Fraser, but his proof strategy can be reconfigured to work in this setting. For the purpose of brevity, we do not carry out the complete argument, merely giving the discretization method below. By first trying to avoid the zero sets of the partial derivatives of the function $f$, one can reduce to the case where a partial derivative of $f$ is non-vanishing on the $\DQ_k$ discretized sets we start with. The key technique is that $f$ maps discrete lattices of points to a discrete, `one dimensional lattice' in $\RR^d$, and the degree of the polynomial gives us the difference in lengths between the two lattices. This idea was present in Keleti's work, and one can view Math\'{e}'s result as a generalization along these lines. We revisit this idea in Chapter BLAH.

\begin{theorem}
    Let $f: [0,1]^{dn} \to \RR$ be a rational coefficient polynomial of degree $m$, and disjoint, $\DQ_k$ discretized sets $T_1, \dots, T_n \subset [0,1]^d$, such that $\inf |\partial_1 f| \neq 0$ on $T_1 \times \dots \times T_n$. Then there exists a constant $C$, depending only on $f$, such that if
    %
    \begin{equation} \label{equation124426034990370935} N_{k+1} \geq C \cdot (N_1 \dots N_k)^{m-1} \cdot M_{k+1}^m, \end{equation}
    %
    then there exists $\DQ_{k+1}$ discretized sets $S_1 \subset T_1, \dots, S_n \subset T_n$ such that
    %
    \begin{enumerate}
        \item $f(x) \neq 0$ for $x \in S_1 \times \dots \times S_n$.
        \item For each $i$, and for each $R \in \DR_{k+1}^d(T_i)$, $\#(\DQ_{k+1}^d(R \cap S_i)) = 1$.
    \end{enumerate}
\end{theorem}
\begin{proof}
    Without loss of generality, by considering an appropriate integer multiple of $f$, we may assume $f$ has integer coefficients. Let $\mathbf{A} \subset (r_{k+1} \cdot \mathbf{Z})^d$. Since $f$ has degree $m$, $f(\mathbf{A}) \subset r_{k+1}^m \cdot \mathbf{Z}$. Suppose that $c_0 \leq |\partial_1 f| \leq C_0$ on $T_1 \times \dots \times T_n$. Then the mean value theorem guarantees that if $\delta \leq r_{k+1}$, then
    %
    \[ c_0 \delta \leq |f(a + \delta e_1) - f(a)| \leq C_0 \delta. \]
    %
    Pick $\varepsilon \in (0,1)$ small enough that $\varepsilon/c_0 < (1 - \varepsilon)/C_0$. If
    %
    \begin{equation} \label{equation920340923}
        l_{k+1} \leq [(1-\varepsilon)/C_0 - \varepsilon/c_0] r_{k+1}^m
    \end{equation}
    %
    we can find $\delta \in l_{k+1} \cdot \mathbf{Z}$ is such that
    %
    \[ [\varepsilon/c_0] r_{k+1}^m \leq \delta \leq [(1 - \varepsilon)/C_0] r_{k+1}^m. \]
    %
    Equation \eqref{equation920340923} is guaranteed by \eqref{equation124426034990370935} for a sufficiently large constant $C$. We then find $d(f(\mathbf{A} + \delta e_1), r_{k+1}^m \cdot \ZZ) \geq \varepsilon r_{k+1}^m$. For $i > 1$, define
    %
    \[ S_i = \bigcup [a_1, a_1 + l_{k+1}] \times \dots \times [a_d, a_d + l_{k+1}], \]
    %
    where $a = (a_1, \dots, a_d)$ ranges over all startpoints to intervals
    %
    \[ [a_1,a_1 + r_{k+1}] \times \dots \times [a_d,a_d + r_{k+1}] \in \DR_{k+1}^d(T_i). \]
    %
    Similarily, define
    %
    \[ S_1 = \bigcup [a_1 + \delta e_1, a_1 + \delta e_1 + s] \times [a_2, a_2 + l_{k+1}] \times \dots \times [a_d, a_d + l_{k+1}], \]
    %
    where $a$ ranges over all startpoints to intervals
    %
    \[ [a_1,a_1 + r_{k+1}] \times \dots \times [a_d,a_d + r_{k+1}] \in \DR_{k+1}^d(T_1). \]
    %
    %     % [(1 - \varepsilon)/C_1 - \varepsilon/C_0] r_{k+1}^m > l_{k+1}
    %
    Because $f$ is $C^1$, we can find $C_2$ such that if $x, y \in \RR^d$ satisfy $|x_i - y_i| \leq t$ for all $i$, then $|f(x) - f(y)| \leq C_2 t$. But then for a sufficiently large constant $C$, \eqref{equation124426034990370935} implies that
    %
    \[ d(f(S_1 \times \dots \times S_n), r_{k+1}^m \cdot \ZZ) \geq \varepsilon r_{k+1}^m - C_2 l_{k+1} \geq (\varepsilon/2) r_{k+1}^m. \]
    %
    In particular, this implies Property (A).
\end{proof}

\begin{comment}

\section{Schmerkin: Salem Sets Avoiding APs}

Schmerkin is the first of the papers we mention to construct \emph{Salem sets} avoiding configurations.

\begin{theorem}
    There exists a set $X \subset [0,1]$ with $\fordim(X) = 1$, which avoids
    %
    \[ \C = \{ (x,y,z) \in \C^3 : x < y < z,\; z - y = y - x \}. \]
    %
    In other words, $X$ contains no nontrivial three term arithmetic progressions.
\end{theorem}

\begin{remark}
    For simplicity we prove here that one can find $X$ with $\fordim(X) = 1 - \varepsilon$, for each $\varepsilon > 0$. The dimension one case can be obtained from the proof we give by a careful manipulation of parameters at each step of the construction.
\end{remark}

Schmerkin implements two main ideas. First, he works with a discretization modulo $m$, exploiting Behrend's construction of subsets $K \subset \{ 1, \dots, m \}$ of cardinality $M^{1-o(1)}$ avoiding arithmetic progressions modulo $M$. Then, for any integers $n_k$, the set $\{ k + n_k M : k \in K \}$ also avoids arithmetic progressions. In particular, if each $n_k$ is chosen at random, then this induces enough random translations into the set construction to get a Salem set.

\begin{lemma} \label{SchmerkinLemma}
    Let $X_k \subset [0,1]$ be $\DQ_k$ discretized, such that for any non-trivial three term arithmetic progression $x_1,x_2,x_3 \in X_k$, the values $x_1,x_2,x_3$ occur in a common interval in $\DQ_k(X_k)$. Then there exists a constant $C$, depending only on $\varepsilon$, such that if $N_{k+1}$ is even, and $N_{k+1} \geq C$, then there exists a $\DQ_{k+1}$ discretized set $X_{k+1}$ such that
    %
    \begin{enumerate}
        \item If $x_1, x_2, x_3 \in X_{k+1}$ form an arithmetic progression, then $x_1, x_2, x_3$ occur in a common interval in $\DQ_{k+1}(X_{k+1})$.

        \item If $\mu_{k+1}$ is the probability measure on $X_{k+1}$ equal to a constant multiple of $dx|_{X_{k+1}}$, and $\mu_k$ the probability measure on $X_k$ equal to a constant multiple of $dx|_{X_k}$, then with probability $1 - 8 N_{k+1} \exp(-\log(N_{k+1})^2/16)$, for all $|m| \leq N_{k+1}$,
        %
        \[ \left| \widehat{\mu_{k+1}}(m) - \widehat{\mu_k}(m) \right| \leq \frac{\log(N_{k+1})}{[\#(\DQ_k(X_k))]^{1/2}}. \]
    \end{enumerate}
\end{lemma}
\begin{proof}
    Set $M = N_{k+1}/2$. If $N_{k+1}$ is sufficently large, Behrend's construction allows us to find $K_0 \subset \mathbf{Z}/M\mathbf{Z}$ with $|K| \geq N_{k+1}^{1-\varepsilon}$ which avoids arithmetic progressions modulo $M$. Then, for each integer $n \in \{ 1, \dots, M \}$, the set
    %
    \[ K(n) = \{ 2m \in \{ 0, \dots, N_{k+1} \} : m \in \{ 0, \dots, M-1 \}, m + M \mathbf{Z} \in 2K_0 + n \} \]
    %
    avoids arithmetic progressions. For each $I = [a_I, a_I + l_k] \in \DQ_k(X_k)$, pick $n_I \in \{ 1, \dots, N_{k+1} \}$ uniformly at random, and define
    %
    \[ X_{k+1} = \bigcup_{I \in \DQ_k(X_k)} \bigcup_{2m \in K(n_I)}\; [a_I + 2m l_{k+1}, a_I + (2m + 1) l_{k+1}) \]
    %
    Suppose $x_1,x_2,x_3 \in X_{k+1}$ form an arithmetic progression, i.e. $x_2 = (x_1 + x_3)/2$. Then by assumption, there is $I \in \DQ_k(X_k)$ such that $x_1,x_2,x_3 \in I$. We can find integers $2m_1, 2m_2, 2m_3 \in K(n_I)$ such that $x_i = a_I + 2m_i l_{k+1} + \delta_i$, where $0 \leq \delta_i < l_{k+1}$. But this means that
    %
    \[ \left| 2m_2 - (m_1 + m_3) \right| = \left| \frac{\delta_1 + \delta_3}{2 l_{k+1}} - \frac{\delta_2}{l_{k+1}} \right| < 1. \]
    %
    Since $m_1, m_2$, and $m_3$ are integers, this implies $m_2 = (m_1 + m_3)/2$, which implies $m_1 = m_2 = m_3$. Thus Property (A) is satisfied. 

    For each $I \in \DQ_k(X_k)$, we let $\nu_I = \mu_{k+1}|_I$, and $\eta_I = \mu_k|_I$. Then, pointwise, we have $\mathbf{E}[\nu_I(x)] = \eta_I(x)$, and this implies that for each integer $m$,
    %
    \[ \mathbf{E}[\widehat{\nu_I}(m)] = \mathbf{E} \left( \int \nu_I(x) e^{-2 \pi i xm} \right) = \int \mathbf{E}[\nu_I(x)] e^{-2 \pi i xm} =  \widehat{\eta_I}(m). \]
    %
    Moreover,
    %
    \[ |\widehat{\nu_I}(m) - \widehat{\eta_I}(m)| \leq \| \nu_I - \mu_I \| \leq \frac{2}{\#(\DQ_k(X_k))}. \]
    %
    Applying Hoeffding's inequality, since
    %
    \[ \sum_I \widehat{\nu_I}(m) - \widehat{\mu_I}(m) = \widehat{\mu_{k+1}}(m) - \widehat{\mu_k}(m), \]
    %
    we conclude that
    %
    \[ \mathbf{P} \left( \left| \widehat{\mu_{k+1}}(m) - \widehat{\mu_k}(m) \right| \geq t \right) \leq 4 \exp \left( \frac{- \# (\DQ_k(X_k)) t^2}{16} \right). \]
    %
    Setting $t = \log(N_{k+1}) / ( \#(\DQ_k(X_k)))^{1/2}$, and applying a union bound, we conclude that with probability $1 - 8 N_{k+1} \exp(-\log(N_{k+1})^2/16)$, for all $|m| \leq N_{k+1}$,
    %
    \[ \left| \widehat{\mu_{k+1}}(m) - \widehat{\mu_k}(m) \right| \leq \frac{\log(N_{k+1})}{[\#(\DQ_k(X_k))]^{1/2}}. \]
    %
    This gives Property (B).
\end{proof}

\begin{theorem}
    If $N_k = 2Ck$, where $C$ is as in Lemma \ref{SchmerkinLemma}, then almost surely, the set $X$ is a Salem set of dimension $1 - \varepsilon$.
\end{theorem}
\begin{proof}
    It is easy to verify that
    %
    \begin{align*}
        \sum_{k = 1}^\infty 8N_k \exp(- \log(N_k)^2/16) < \infty.
    \end{align*}
    %
    The Borel Cantelli lemma therefore implies that, almost surely, for sufficiently large $k$, and for all $|m| \leq N_{k+1}$, we have
    %
    \[ \left| \widehat{\mu_{k+1}}(m) - \widehat{\mu_k}(m) \right| \leq \frac{\log(N_{k+1})}{[\#(\DQ_k(X_k))]^{1/2}}. \]
    %
    Now for each $k$, we know $\#(\DQ_{k+1}(X_{k+1})) \geq N_{k+1}^{1-\varepsilon} \#(\DQ_k(X_k))$, which gives that $\#(\DQ_k(X_k)) \geq N_1^{1-\varepsilon} \dots N_k^{1-\varepsilon}$. Thus
    %
    \[ \left| \widehat{\mu_{k+1}}(m) - \widehat{\mu_k}(m) \right| \leq \frac{\log(N_{k+1})}{N_{k+1}^{1-\varepsilon} \dots N_1^{1-\varepsilon}} \]
\end{proof}

\end{comment}

\endinput