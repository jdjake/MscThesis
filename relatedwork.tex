%% The following is a directive for TeXShop to indicate the main file
%%!TEX root = diss.tex

\chapter{Related Work}
\label{ch:RelatedWork}

Here, we discuss the main papers which influenced our results. In particular, the work of Keleti on translate avoiding sets, Fraser and Pramanik's work on sets avoiding smooth configurations, Math\'{e}'s result on sets avoiding algebraic varieties, and Schmerkin's result on sets with large Fourier dimension avoiding smooth configurations.

\section{Keleti: A Translate Avoiding Set}

Keleti's two page paper constructs a full dimensional set $X \subset [0,1]$ such that for each $t \neq 0$, $X$ intersects $t + X$ in at most one place. The set $X$ is then said to \emph{avoid translates}. This paper contains the core idea behind the \emph{interval dissection} method adapted in Fraser and Pramanik's paper. We also adapt this technique in our paper, which makes the result of interest.

It is often convenient to avoid certain configurations when they are expressed in terms of an equation, which is also exploited in Fraser and Pramanik's work, and in Math\'{e}'s.

\begin{lemma}
    Let $X$ be a set. Then $X$ avoids translates if and only if there do not exists values $x_1 < x_2 \leq x_3 < x_4$ in $X$ with $x_2 - x_1 = x_4 - x_3$.
\end{lemma}
\begin{proof}

    Suppose $(t + X) \cap X$ contains two points $a < b$. Without loss of generality, we may assume that $t > 0$. If $a \leq b - t$, then the equation
    %
    \[ a - (a - t) = t = b - (b - t) \]
    %
    satisfies the constraints, since $a - t < a \leq b - t < b$ are all elements of $X$. We also have
    %
    \[ (b - t) - (a - t) = b - a, \]
    %
    which satisfies the constraints if $a - t < b - t \leq a < b$. This covers all possible cases. Conversely, if there are $x_1 < x_2 \leq x_3 < x_4$ in $X$ with
    %
    \[ x_2 - x_1 = t = x_4 - x_3, \]
    %
    then $X + t$ contains $x_2 = x_1 + (x_2 - x_1)$ and $x_4 = x_3 + (x_4 - x_3)$.
\end{proof}

%\footnote{We always assume $L_n/L_{n+1}$ is an integer so that intervals in $\mathcal{B}(L_n)$ are either almost disjoint from intervals in $\mathcal{B}(L_{n+1})$ or contained completely within such an interval}

The basic, but fundamental idea of the interval dissection technique is to introduce memory into Cantor set constructions. Keleti constructs a nested family of discrete sets $\{ X_k \}$, with $X_k$ a $\DQ_k$ discretized set, and with $X = \bigcap X_k$. The sequence $\{ N_k \}$ will be specified later, but each $N_k$ will be a multiple of 10. We initialize $X_0 = [0,1]$. The novel feature of the argument is to add a queue of intervals to the construction, initially just containing $[0,1]$. To construct the sequence $\{ X_k \}$, Keleti iteratively performs the following procedure:
%
\begin{algorithm}[H]
    \begin{algorithmic}%[1]
        \caption{Construction of the Sets $\{ X_k \}$:}
        \State{Set $k = 0$.}
        \MRepeat
            \State{Take off an interval $I$ from the front of the queue.}

            \MForAll{\ $J \in \DQ_k(X_k)$:}
                \State{Order the intervals in $\DQ_{k+1}(J)$ as $J_1, \dots, J_N$.}

                \State{{\bf If} $J \subset I$, add all intervals $J_i$ to $X_{k+1}$ with $i \equiv 0$ modulo 10.}
                \State{{\bf Else} add all $J_i$ with $i \equiv 5$ modulo 10.}
            \EndForAll
            \State{Add all intervals in $\DQ_{k+1}^d$ to the end of the queue.}
            \State{Increase $k$ by 1.}
        \EndRepeat   
    \end{algorithmic}
\end{algorithm}

Each iteration of the algorithm produces a new set $X_k$, and so leaving the algorithm to repeat infinitely produces a sequence $\{ X_k \}$ whose intersection is $X$.

\begin{lemma}
    The set $X$ is translate avoiding.
\end{lemma}
\begin{proof}
    If $X$ is not translate avoiding, there is $x_1 < x_2 \leq x_3 < x_4$ with $x_2 - x_1 = x_4 - x_3$. Since $l_k \to 0$, there is a suitably large integer $N$ such that $x_1$ is contained in an interval $I \in \DQ_N$ not containing $x_2,x_3$, or $x_4$. At stage $N$ of the algorithm, the interval $I$ is added to the end of the queue, and at a much later stage $M$, the interval $I$ is retrieved. Find the startpoints $x_1^\circ, x_2^\circ$, $x_3^\circ, x_4^\circ \in l_M \mathbf{Z}$ to the intervals in $\DQ_M$ containing $x_1$, $x_2$, $x_3$, and $x_4$. Then we can find $n$ and $m$ such that $x_4^\circ - x_3^\circ = (10n)l_M$, and $x_2^\circ - x_1^\circ = (10m + 5)l_M$. In particular, this means that $|(x_4^\circ - x_3^\circ) - (x_2^\circ - x_1^\circ)| \geq 5L_M$. But
    %
    \begin{align*}
        |(x_4^\circ - x_3^\circ) - (x_2^\circ - x_1^\circ)| &= |[(x_4^\circ - x_3^\circ) - (x_2^\circ - x_1^\circ)] - [(x_4 - x_3) - (x_2 - x_1)]|\\
        &\leq |x_1^\circ - x_1| + \dots + |x_4^\circ - x_4| \leq 4 L_M
    \end{align*}
    %
    which gives a contradiction.
\end{proof}

It is easy to see from the algorithm that
%
\[ \# (\DQ_k(X_k)) = (l_{k-1}/10l_k) \cdot \#(\DQ_{k-1}(X_{k-1})). \]
%
Closing the recursive definition shows
%
\[ \#(\DQ_k(X_k)) = \frac{1}{10^k l_k}. \]
%
In particular, this means $|X_k| = 1/10^k$, so $X$ has measure zero irrespective of our parameters. Nonetheless, the canonical measure $\mu$ on $X$ defined with respect to the decomposition $\{ X_k \}$ satisfies $\mu(I) = 10^k l_k$ for all $I \in \B(l_k,X)$. If $10^k l_k^\varepsilon \lesssim_\varepsilon 1$ for all $\varepsilon$, then we can establish the bounds $\mu(I) \lesssim_\varepsilon l_k^{1-\varepsilon}$ for all $\varepsilon$. In particular, this is true if $N_k = 10^{\lfloor \log(k + 2) \rfloor}$. And because this sequence does not rapidly decrease too fast, we can apply Lemma \ref{easyCoverTheorem} to show $\mu$ is a Frostman measure of dimension $1-\varepsilon$ for each $\varepsilon > 0$, so $X$ has full Hausdorff dimension.



\section{Generalizing Keleti's Argument}

Before we move onto other methods which developed Keleti's argument work, it is useful to dwell on what general properties this argument has:
%
\begin{itemize}
    \item \emph{Simplification to a Discrete Problem}: A major part of Keleti's argument is solving a discrete version of the configuration argument. We could summarize the result of Keleti's discrete argument in a lemma.
    %
    \begin{lemma} \label{KeletiDiscreteLemma}
        Let $T_1, T_2$ be disjoint, $\DQ_k$ discretized sets. Then we can find $S_1 \subset T_1$ and $S_2 \subset T_2$ such that
        %
        \begin{enumerate}
            \item[(i)] For each $k$, $S_k$ is a $\DQ_{k+1}$ discretized subset of $T_k$.
            \item[(ii)] If $x_1 \in S_1$ and $x_2,x_3,x_4 \in S_2$, then $x_2 - x_1 \neq x_4 - x_3$.
            \item[(iii)] For each cube $Q \in \DQ_k$ with $Q \subset T_k$,
            %
            \[ \#(\DQ_{k+1}(Q \cap S_k)) \geq (1/10) \cdot \#(\DQ_{k+1}(Q)). \]
        \end{enumerate}
    \end{lemma}

    \item \emph{Iterative Application of Discrete Solution}: Keleti then repeatedly applies his argument iteratively. In particular, Property (i) of Lemma \ref{KeletiDiscreteLemma} allows him to apply his argument iteratively. The reason why Keleti obtains a configuration avoiding set in the limit is because of Property (ii). Most importantly the reason why Keleti obtains a set with full Hausdorff dimension, if the sequence $\{ N_k \}$ decreases rapidly enough, is because of Property (iii).
\end{itemize}

Of course, it is not possible to extend the discrete solution of the configuration argument to general configurations; this part of Keleti's method strongly depends on the arithmetic structure of the configuration. The iterative application of a discrete solution, however, can be applied in generality. In various senses, this technique is applied in all the results we review in this chapter.







\section{Fraser/Pramanik: Smooth Configurations}

Inspired by Keleti's result, Pramanik and Fraser obtained a generalization of the queue method which allows one to find sets avoiding $n+1$ point configurations given by the zero sets of smooth functions, i.e.
%
\[ \C = \{ (x_0, \dots, x_n) \in \RR^{dn} : f(x_0, \dots, x_n) = 0 \}, \]
%
under mild regularity conditions on the function $f: [0,1]^{dn} \to [0,1]^m$. Here, a simple pidgeonholing strategy suffices.

\begin{theorem}[Pramanik and Fraser]
    Fix $m \leq d(n-1)$. Consider a countable family of functions $\{ f_k : [0,1]^{dn} \to [0,1]^m \}$ each of which being $C^2$, and such that for each $k$, $Df_k$ has full rank at any $(x_1, \dots, x_n)$ with $f(x_1, \dots, x_n) = 0$, where the $(x_1, \dots, x_n)$ are distinct. Then there exists a set $X \subset \RR^d$ with Hausdorff dimension $m/(n-1)$ such that $X$ avoids the configuration
    %
    \[ \C = \bigcup_k \{ (x_1, \dots, x_n) \in \C^n(\RR^d): f(x_1, \dots, x_n) = 0 \}. \]
\end{theorem}

\begin{remark}
    For simplicity, we only prove the result for a single function, rather than a countable family of functions. The only major difference between the two approaches is the choice of scales we must choose later on in the argument.
\end{remark}

Pramanik and Fraser also simplify to a discrete version of their problem, which they then iteratively apply. In the discrete setting, rather than making a linear shift in one of the intervals we avoid as in Keleti's approach, one must use the smoothness properties of the function to find large segments of an interval avoiding. Corollary \ref{PramanikFraserBuildingBlockLemma} gives the discrete solution that Pramanik and Fraser utilizes.

\begin{lemma} \label{Lemma315091513}
    Fix $n > 1$. Let $T \subset [0,1]^d$ be $\DQ_k$ discretized, and $T' \subset [0,1]^{(n-1)d}$ be $\DQ_k$ discretized. Let $B \subset T \times T'$ be $\DQ_{k+1}$ discretized. Then there exists a $\DQ_{k+1}$ discretized set $S \subset T$, and a $\DQ_{k+1}$ discretized set $B' \subset T'$, such that
    %
    \begin{enumerate}
        \item \label{dimensionReductionProperty} $(S \times T') \cap B \subset S \times B'$.

        \item \label{bigProperty} For every $Q \in \DQ_k^d$, there exists $\mathcal{R}(Q) \subset \DR_{k+1}^d(Q)$, such that
        %
        \[ \#(\mathcal{R}(Q)) \geq (1/2) \cdot \#(\DR_{k+1}^d(Q)). \]
        %
        and for each $R \in \DR_{k+1}^d(Q)$,
        %
        \[ \#(\DQ_{k+1}(R)) = \begin{cases} 1 &: R \in \mathcal{R}(Q,) \\ 0 &: R \not \in \mathcal{R}(Q). \end{cases} \]

        \item \label{BBoundProperty} $\#(\DQ_{k+1}(B')) \leq 2 (N_1 \dots N_k)^d \left( M_{k+1}/N_{k+1} \right)^d \cdot \#(\DQ_{k+1}(B))$.
    \end{enumerate}
\end{lemma}
\begin{proof}
    Fix $Q_0 \in \DQ_k(T)$. For each $R \in \DR_{k+1}(Q_0)$, define a \emph{slab} $S[R] = R \times T'$, and for each $Q \in \DQ_{k+1}(Q_0)$, define a \emph{wafer} $W[Q] = Q \times T'$. We say a wafer $W[Q]$ is \emph{good} if
    %
    \begin{equation} \label{equation10291095429062}
        \#(\DQ_{k+1}(W[Q] \cap B)) \leq (2/N_{k+1}^d) \cdot \#(\DQ_{k+1}(B)),
    \end{equation}
    %
    Then at most $N_{k+1}^d/2$ wafers are bad. We call a slab \emph{good} if it contains a wafer which is good. Since a slab is the union of $(N_{k+1}/M_{k+1})^d$ wafers, at most $M_{k+1}^d /2 = (1/2) \cdot \#(\DR_{k+1}(Q_0))$ slabs are bad. Thus if we set
    %
    \[ \mathcal{R}(Q_0) = \{ R \in \DR_{k+1}(Q_0) : S[R]\ \text{is good} \}, \]
    %
    then
    %
    \begin{equation} \label{equation24016590369046}
        \#(\mathcal{R}(Q_0)) \geq (1/2) \cdot \#(\DR_{k+1}(Q_0)).
    \end{equation}
    %
    For each $R \in \mathcal{R}(Q_0)$, we pick $Q_R \in \DQ_{k+1}(R)$ such that $W[Q_R]$ is good, and define
    %
    \[ S = \bigcup \{ Q_R : R \in \mathcal{R}(Q_0) \}. \]
    %
    Equation \eqref{equation24016590369046} implies $S$ satisfies Property \ref{bigProperty}.

    Let $B'$ be the union of all cubes $Q' \in \DQ_{k+1}(T')$ such that there is $Q \in \DQ_{k+1}(S)$ with $Q \times Q' \in \DQ_{k+1}(B)$. By definition, Property \ref{dimensionReductionProperty} is then satisfied. For each $Q \in \DQ_{k+1}(S)$, $W[Q]$ is good, so \eqref{equation10291095429062} implies
    %
    \[ \# \{ Q' : Q \times Q' \in \DQ_{k+1}(B) \} \leq (2/N_{k+1}^d) \cdot \#(\DQ_{k+1}(B)). \]
    %
    But $\#(\DQ_{k+1}(S)) \leq \#(\DR_{k+1}(T)) \leq (1/r_{k+1})^d = (N_1 \dots N_k)^d M_{k+1}^d$, so
    %
    \begin{align*}
        \#(\DQ_{k+1}(B')) &\leq \#(\DQ_{k+1}(S))[(2/N_{k+1}^d) \cdot \#(\DQ_{k+1}(B))]\\
        &\leq 2(N_1 \dots N_k)^d (M_{k+1}/N_{k+1})^d \#(\DQ_{k+1}(B)),
    \end{align*}
    %
    which establishes Property \ref{BBoundProperty}.
\end{proof}

We apply the lemma recursively $n-1$ times to continually reduce the dimensionality of the avoidance problem we are considering. Eventually, we obtain the case where $n = 0$, and we are in need of a final technique.

\begin{lemma} \label{Lemma1209410535}
    Fix $n > 1$. Let $T \subset [0,1]^d$ be $\DQ_k$ discretized, and let $B \subset T$ be $\DQ_{k+1}$ discretized. Suppose
    %
    \[ \# \DQ_{k+1}(B) \leq \left[ C \cdot 2^{n-1} (N_1 \dots N_k)^{d(n-1)} (M_{k+1}/N_{k+1})^{d(n-1)} \right] (1/l_{k+1})^{dn - m}. \]
    %
    and
    %
    \begin{equation} \label{equation903103513095}
        N_{k+1} > \left[ C \cdot (1 + 2^d) 2^n (N_1 \dots N_k)^{2dn} \right]^{1/m} M_{k+1}^{d(n-1)/m}
    \end{equation}
    %
    Then there exists a $\DQ_{k+1}$ discretized set $S \subset T$ such that
    %
    \begin{enumerate}
        \item $S \cap B = \emptyset$.
        \item \label{badsetproperty5} For each $Q_0 \in \DQ_k(T)$, there is $\mathcal{R}(Q_0) \subset \DR_{k+1}(Q_0)$ with
        %
        \[ \#(\mathcal{R}(Q_0)) \geq (1/2) \#(\DR_{k+1}(Q_0)), \]
        %
        such that for each $R \in \DR_{k+1}(Q_0)$,
        %
        \[ \#(\mathcal{Q}_{k+1}(R \cap S)) = \begin{cases} 1 &: R \in \mathcal{R}(Q_0), \\ 0 &: R \not \in \mathcal{R}(Q_0). \end{cases} \]
    \end{enumerate}
\end{lemma}
\begin{proof}
    For each $Q_0 \in \DQ_k(T)$, we set
    %
    \[ \mathcal{R}(Q_0) = \{ R \in \DR_{k+1}(Q_0) : \#(\DQ_{k+1}(R \cap B)) \leq (2/M_{k+1}^d) \cdot \#(\DQ_{k+1}(B)) \}. \]
    %
    Since $\DR_{k+1}(Q_0) = M_{k+1}^d$,
    %
    \[ \#(\mathcal{R}(Q_0)) \geq \#(\DR_{k+1}(Q_0)) - (M_{k+1}^d/2) \geq (1/2) \cdot \#(\DR_{k+1}(T)). \]
    %
    Now \eqref{equation903103513095} implies that for each $R \in \mathcal{R}(Q_0)$,
    %
    \begin{align*}
        \#(\DQ_{k+1}(R \cap B)) &\leq (2/M_{k+1}^d) \cdot \#(\DQ_{k+1}(B))\\
        &\leq (2/M_{k+1}^d) \left(C \cdot 2^{n-1} (N_1 \dots N_k)^{2dn} (M_{k+1}/N_{k+1})^{d(n-1)} \right).\\
        &= \left[ 2^n C (N_1 \dots N_k)^{2dn} \right] \left( M_{k+1}^{d(n-2)} / N_{k+1}^{m-d} \right)\\
        &< \left( \frac{1}{1 + 2^d} \right) (N_{k+1}/M_{k+1})^d\\
        &= \left( \frac{1}{1 + 2^d} \right) \DQ_{k+1}(R)
    \end{align*}
    %
    Thus for each $R \in \mathcal{R}(Q_0)$, we can find $Q_R \in \DQ_{k+1}(R)$ such that $Q_R \cap B = \emptyset$. And so if we set
    %
    \[ S = \bigcup \{ Q_R : R \in \mathcal{R}(Q_0), Q_0 \in \DQ_k(T) \}, \]
    %
    Then (A) and (B) are satisfied.
\end{proof}

\begin{corollary} \label{PramanikFraserBuildingBlockLemma}
    Let $f: [0,1]^{dn} \to [0,1]^m$ be $C^2$, and have full rank at every point $(x_1, \dots, x_n)$ with all $x_1, \dots, x_n$ distinct, and $f(x_1, \dots, x_n) = 0$. Then there exists a universal constant $C$ depending only on $f$ such that, if \eqref{equation903103513095} is satisfied, then for any disjoint, $\DQ_k$ discretized sets $T_1, \dots, T_n \subset [0,1]^d$, we can find $\DQ_{k+1}$ discretized sets $S_1 \subset T_1, \dots, S_n \subset T_n$ such that
    %
    \begin{enumerate}
        \item If $x_1 \in S_1, \dots, x_n \in S_n$, then $f(x_1, \dots, x_n) \neq 0$.
        \item For each $k$, and for each $Q_0 \in \DQ_k(T_k)$, there is $\mathcal{R}(Q_0) \subset \mathcal{R}_{k+1}(Q_0)$ with
        %
        \[ \#(\mathcal{R}(Q_0)) \geq (1/2) \cdot \#(\mathcal{R}_{k+1}(Q_0)), \]
        %
        and for each $R \in \DR_{k+1}(Q_0)$,
        %
        \[ \#(\mathcal{Q}_{k+1}(R \cap S)) = \begin{cases} 1 &: R \in \mathcal{R}(Q_0), \\ 0 &: R \not \in \mathcal{R}(Q_0). \end{cases} \]
    \end{enumerate}
\end{corollary}
\begin{proof}
    Since $f$ is $C^2$ and has full rank on the set
    %
    \[ V(f) = \{ (x_1, \dots, x_n) \in \C^n(\RR^d) : f(x_1, \dots, x_n) = 0 \}, \]
    %
    the implicit function theorem implies $V(f)$ is a smooth manifold of dimension $nd - m$ in $\RR^{dn}$, and the coarea formula implies the existence of a constant $C$ such that for each $k$,
    %
    \[ \# \{ Q \in \DQ_k^{dn} : Q \cap V(f) \neq \emptyset \} \leq C/l_k^{dn-m}. \]
    %
    To apply Lemma \ref{Lemma315091513} and \ref{Lemma1209410535}, we set
    %
    \[ B = \# \{ Q \in \DQ_k^{dn} : Q \cap V(f) \neq \emptyset \}. \]
    %
    Applying Lemma \ref{Lemma315091513} iteratively $n-1$ times, then finishing with Lemma $\ref{Lemma1209410535}$, constructs the sets $S_1, \dots, S_n$.
\end{proof}

\begin{remark}
    The values of $N_1, \dots, N_k$, exponentials $2^k$, and constants, are roughly insignificant in comparison to $M_{k+1}$ and $N_{k+1}$, when these numbers are hyperdyadic for a suitably large constant, or grow faster than hyperdyadic. In particular, we can roughly read off the Hausdorff dimension of the set $m/(n-1)$ we construct from the inverse power of $M_{k+1}$ in \eqref{equation903103513095}, i.e. that $N_{k+1} \gtrapprox M_{k+1}^{d(n-1)/m}$.
\end{remark}

Just like in Keleti's proof, Pramanik and Fraser's technique applies a discrete result, Corollary \ref{PramanikFraserBuildingBlockLemma}, iteratively at many scales to obtain a high dimensional set avoiding the zeroes of a function. We construct a nested family $\{ X_k : k \geq 0 \}$ of $\DQ_k$ discretized sets, converging to a set $X$, which we will show is translate avoiding. We intiailize $X_0 = [0,1]$. Our queue shall consist of $n$ tuples of disjoint intervals $(T_1, \dots, T_n)$, all of the same length, which initially consists of all possible tuples of intervals in $\DQ_1^d([0,1]^d)$. To construct the sequence $\{ X_k \}$, we perform the following iterative procedure:
%
\begin{algorithm}[H]
    \begin{algorithmic}
        \caption{Construction of the Sets $\{ X_k \}$}
        \State{Set $k = 0$}
        \MRepeat
            \State{Take off an $n$ tuple $(T_1', \dots, T_n')$ from the front of the queue}
            \State{Set $T_i = T_i' \cap X_k$ for each $i$}
            \State{Apply Corollary \ref{PramanikFraserBuildingBlockLemma} to the sets $T_0, \dots, T_d$, obtaining $\DQ_{k+1}$ discretized sets $S_1, \dots, S_n$ satisfying Properties (A), (B), and (C) of that Lemma.}
            \State{Set $X_{k+1} = X_k - \bigcup_{i = 1}^n T_i - S_i$.}
            \State{Add all $n$ tuples of disjoint cubes $(T_1', \dots, T_n')$ in $\DQ_{k+1}^d(X_{k+1})$ to the back of the queue.}
            \State{Increase $k$ by 1.}
        \EndRepeat   
    \end{algorithmic}
\end{algorithm}

\begin{lemma}
    The set $X$ constructed by the procedure avoids the configuration
    %
    \[ \C = \{ (x_1, \dots, x_n) \in \C^n(\RR^d) : f(x_1, \dots, x_n) = 0 \}. \]
\end{lemma}
\begin{proof}
Suppose $x_1, \dots, x_n \in X$ are distinct. Then at some stage $k$, $x_1, \dots, x_n$ lie in disjoint cubes $T_1', \dots, T_n' \in \DQ_k^d(X_k)$, for some large $k$. At this stage, $(T_1', \dots, T_n')$ is added to the back of the queue, and therefore, at some much later stage $N$, the tuple $(T_1', \dots, T_n')$ is taken off the front. Sets $S_1 \subset T_1', \dots, S_n \subset T_n'$ are constructed satisfying Property (A) of Corollary \ref{PramanikFraserBuildingBlockLemma}. Since $x_1, \dots, x_n \in X$, we must have $x_i \in S_i$ for each $i$, so $f(x_1, \dots, x_n) \neq 0$.
\end{proof}

What remains is to bound the Hausdorff dimension of $X$.

\begin{theorem}
    If $M_k = 2^{2^{k^2}}$, $X$ has Hausdorff dimension exceeding $m/(n-1)$.
\end{theorem}
\begin{proof}
    First, we construct the canonical measure $\mu$ on $X$. Property (B) of Corollary \ref{PramanikFraserBuildingBlockLemma} implies that for each $Q \in \DQ_{k+1}^d$, $\mu(Q) \leq 2/M_{k+1}^d \mu(Q^*)$, which implies that for $Q \in \DQ_k^d$, $\mu(Q) \leq 2^k / (M_k \dots M_1)^d$. By Lemma \ref{uniformMassFrostman}, it suffices to show that for each $\varepsilon > 0$,
    %
    \begin{equation}\label{inequality7} \frac{2^k}{(M_{k+1} \dots M_1)^d} \lesssim_\varepsilon \frac{1}{(N_1 \dots N_{k+1})^{m/(n-1)(1 - \varepsilon)}} \end{equation}
    %
    Set
    %
    \[ N_{k+1} = A (N_1 \dots N_k)^{2dn/m} M_{k+1}^{d(n-1)/m}, \]
    %
    where $A$ is an arbitrary constant not depending on $k$ so that $\eqref{equation903103513095}$ holds. Then inequality \eqref{inequality7} is then implied if
    %
    \[ M_{k+1} \gtrsim_\varepsilon \left( (N_1 \dots N_k)^{6dn} [2^k A^d] \right)^{1/\varepsilon}. \]
    %
    This equation is satisfied if $M_k = 2^{2^{k^2}}$.
\end{proof}

\section{Math\'{e}: Polynomial Configurations}

Math\'{e}'s result constructs sets avoiding algebraic varieties.

\begin{theorem}[Math\'{e}]
    Let $\{ f_k: \RR^{nd} \to \mathbf{R} \}$ be a countable family of rational coefficient polynomials with degree at most $m$. Then there exists a set $X \subset [0,1]^d$ with Hausdorff dimension $d/m$ which avoids the configurations
    %
    \[ \C = \bigcup_k \{ (x_1, \dots, x_n) \in \RR^{nd} : f(x_1, \dots, x_n) = 0 \}. \]
\end{theorem}

Originally, Math\'{e}'s result does not explicitly use a discretization method analogous to Keleti and Pramanik and Fraser, but his proof strategy can be reconfigured to work in this setting. For the purpose of brevity, we do not carry out the complete argument, merely giving the discretization method below. By first trying to avoid the zero sets of the derivatives of the function $f$, one can reduce to the case where a partial derivative of $f$ is non-vanishing on the $\DQ_k$ discretized sets we start with. The key technique is that $f$ maps discrete lattices of points to a discrete, `one dimensional lattice' in $\RR^d$, and the degree of the polynomial gives us the difference in lengths between the two lattices. We revisit this idea in Chapter BLAH.

\begin{theorem}
    Let $f: [0,1]^{dn} \to \RR$ be a rational coefficient polynomial of degree $m$, and disjoint, $\DQ_k$ discretized sets $T_1, \dots, T_n \subset [0,1]^d$, such that $|\partial_1 f| \neq 0$ on $T_1 \times \dots \times T_n$. Then there exists constants $C_0$ and $C_1$, depending only on $f$ and $T_1, \dots, T_n$, such that if
    %
    \begin{equation} \label{equation124426034990370935} N_{k+1} \geq \max(4C_1 \sqrt{nd}, 2C_0) \cdot (N_1 \dots N_k)^{m-1} \cdot M_{k+1}^m. \end{equation}
    %
    Then there exists $\DQ_{k+1}$ discretized sets $S_1 \subset T_1, \dots, S_n \subset T_n$ such that
    %
    \begin{enumerate}
        \item $f(x) \neq 0$ for $x \in S_1 \times \dots \times S_n$.
        \item For each $i$, and for each $R \in \DR_{k+1}^d(T_i)$, $\#(\DQ_{k+1}^d(R \cap S_i)) = 1$.
    \end{enumerate}
\end{theorem}
\begin{proof}
    Without loss of generality, by considering an appropriate integer multiple of $f$, we may assume $f$ has integer coefficients. Let $\mathbf{A} \subset (r_{k+1} \cdot \mathbf{Z})^d$. Since $f$ has degree $m$, $f(\mathbf{A}) \subset r_{k+1}^m \cdot \mathbf{Z}$. Suppose that $|\partial_1 f| \geq C_0$ on $T_1 \times \dots \times T_n$, and $\| \nabla f \|_{L^\infty[0,1]^{nd}} \leq C_1$. Then the mean value theorem guarantees that if $\delta \leq r_{k+1}$, then
    %
    \[ |f(a + \delta e_1) - f(a)| \geq C_0 \delta. \]
    %
    If $C_0 \delta \leq r_{k+1}^m/2$, then this implies
    %
    \[ d(f(a + \delta e_1), (r_{k+1}^m \cdot \ZZ)) \geq C_0 \delta. \]
    %
    In particular, we set $\delta = r_{k+1}^m/2C_0$, then
    %
    \[ d(f(a + \delta e_1), r_{k+1}^m \cdot \ZZ) \geq r_{k+1}^m/2. \]
    %
    Define
    %
    \[ S_i = \begin{cases} \bigcup [a_1, a_1 + l_{k+1}] \times \dots \times [a_d, a_d + l_{k+1}] &: i > 1 \\ \bigcup [a_1 + \delta e_1, a_1 + \delta e_1 + s] \times [a_2, a_2 + l_{k+1}] \times \dots \times [a_d, a_d + l_{k+1}] &: i = 1. \end{cases} \]
    %
    where $a$ ranges over all startpoints to intervals $[a_1,b_1] \times \dots \times [a_d,b_d] \in \DR_{k+1}^d(T_i)$. Then \eqref{equation124426034990370935} implies that
    %
    \[ d(f(S_1 \times \dots \times S_n), r_{k+1}^m \cdot \ZZ) \geq r_{k+1}^m/2 - C_1 \sqrt{nd} l_{k+1} \geq r_{k+1}^m/4. \]
    %
    In particular, this implies Property (A).
\end{proof}

\section{Schmerkin: Salem Sets Avoiding APs}

Schmerkin is the first of the papers to obtain a result involving Fourier dimension.

\begin{theorem}
    There exists a set $X \subset [0,1]$ with $\fordim(X) = 1$, which avoids
    %
    \[ \C = \{ (x,y,z) \in \C^3 : x < y < z, z - y = y - x \}. \]
    %
    In other words, $X$ contains no nontrivial three term arithmetic progressions.
\end{theorem}

Schmerkin implements two main ideas. First, working modulo $m$, Behrend's construction of subsets of $\mathbf{Z}/m\mathbf{Z}$ of cardinality $m^{1-\varepsilon}$ avoiding arithmetic progressions is exploited to find subsets of $\mathbf{Z}$.

\endinput